Dadas varias cuadrículas del mismo tamaño con números escritos en sus casillas, su suma se efectúa sumando las casillas que están en la misma posición, creando una cuadrícula del mismo tamaño.
Dado un entero positivo $N$, diremos que una cuadrícula es $N$-balanceada si tiene números enteros escritos en sus casillas y si la diferencia entre los números escritos en cualesquiera dos casillas que comparten un lado es menor o igual a $N$. Muestra que toda cuadrícula $2n$-balanceada (de cualquier tamaño) se puede escribir como suma de dos cuadrículas $n$-balanceadas. A demás, muestra que toda cuadrícula $3n$-balanceada (de cualquier tamaño) se puede escribir como suma de tres cuadrículas $n$-balanceadas.