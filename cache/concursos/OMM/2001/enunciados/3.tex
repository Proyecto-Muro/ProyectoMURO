En un cuadrilátero $ABCD$ inscrito en una circunferencia llamemos $P$ al punto de intersección de las diagonales $AC$ y $BD$, y sea $M$ el punto medio de $CD$. Una circunferencia que pasa por $P$ y que es tangente a $CD$ en $M$ corta a $BD$ y a $AC$ en los puntos $Q$ y $R$, respectivamente. Se toma un punto $S$ en el segmento $BD$ de tal manera que $BS=DQ$. Por $S$ se traza una paralela a $AB$ que corta a $AC$ en un punto $T$. Muestra que $AT=RC$.