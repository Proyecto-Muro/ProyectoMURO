Sea $n$ un entero positivo. María escribe en un pizarrón las $n^3$ ternas que se pueden formar
tomando tres enteros, no necesariamente distintos, entre $1$ y $n$, incluyéndolos. Después,
para cada una de las ternas, María determina el mayor (o los mayores, en caso de que
haya más de uno) y borra los demás. Por ejemplo, en la terna $(1, 3, 4)$ borrará los números
$1$ y $3$, mientras que en la terna $(1, 2, 2)$ borrará sólo el número $1$.
Muestra que, al terminar este proceso, la cantidad de números que quedan escritos en el
pizarrón no puede ser igual al cuadrado de un número entero.