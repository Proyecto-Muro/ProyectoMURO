Sea $n\ge 3$ un número entero. En un juego hay $n$ cajas en un círculo. Al principio, cada caja contiene un objeto que puede ser piedra, papel o tijeras, de forma que no hay dos cajas adyacentes con el mismo objeto, y cada objeto aparece en al menos una caja. <br>

Al igual que en el juego, piedra gana a tijera, tijera gana a papel y papel gana a piedra.<br> 

El juego consiste en mover objetos de una caja a otra según la siguiente regla: <br>

Se eligen dos cajas adyacentes y un objeto de cada una de ellas de forma que sean diferentes, y movemos el objeto perdedor a la caja que contiene el objeto ganador. Por ejemplo, si elegimos una piedra de la caja A y unas tijeras de la caja B, movemos las tijeras a la caja A.<br>

Demuestra que, aplicando la regla suficientes veces, es posible mover todos los objetos a la misma caja.
