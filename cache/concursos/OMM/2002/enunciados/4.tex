Una ficha de dominó tiene dos números (que pueden ser iguales) entre $0$ y $6$. Las fichas se pueden voltear, es decir, $\boxed 4\boxed 5$ es la misma ficha que $\boxed 5\boxed 4$. Se quiere formar una hilera de fichas de dominó distintas de manera que en cada momento de la construcción de la hilera, la suma de todos los números de fichas puestas hasta ese momento sea impar. Las fichas se pueden agregar de la manera usual a ambos lados de la hilera, es decir, de manera que en cualesquiera dos fichas consecutivas aparezca el mismo número en los extremos que se juntan. Por ejemplo, se podría hacer la hilera: $\boxed 1\boxed 3$, $\boxed 3\boxed 4$, $\boxed 4\boxed 4$, en la que se colocó primero la ficha del centro y luego la de la izquierda para mantener la suma impar.
¿Cuál es la mayor cantidad de fichas que de pueden colocar en una hilera?
¿Cuántas hileras de esa longitud máxima se pueden construir?