Sea $ABC$ un triángulo acutángulo cuyo ortocentro es el punto $H$. La circunferencia que
pasa por los puntos $B$, $H$ y $C$ vuelve a intersectar a las rectas $AB$ y $AC$ en los puntos $D$ y
$E$, respectivamente. Sean $P$ y $Q$ los puntos de intersección de $HB$ y $HC$ con el segmento
$DE$, respectivamente. Se consideran los puntos $X$ e $Y$ (distintos de $A$) que están sobre
las recta $AP$ y $AQ$, respectivamente, de manera que los puntos $X$, $A$, $H$ y $B$ están sobre
un círculo y los puntos $Y$, $A$, $H$ y $C$ están sobre un círculo. Muestra que las rectas $XY$
y $BC$ son paralelas.