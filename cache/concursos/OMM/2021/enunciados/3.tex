Sean $m,n\geq 2$ dos enteros. En una cuadrícula de $m\times n$, una hormiga empieza en el cuadrito inferior izquierdo y quiere caminar al camino superior derecho. Cada paso que da la hormiga debe ser a un cuadrito adyacente, y de acuerdo a las siguientes posibilidades: $\uparrow$, $\rightarrow$ y $\nearrow$. Sin embargo, un malvado mago ha dejado caer lava desde arriba de la cuadrícula y ha destruido algunos cuadritos, de forma tal que:
<ul>
<li>Si un cuadrito está destruido, entonces todos los cuadritos superiores a él también también están destruidos.</li>

<li>El número de cuadritos destruidos es mayor o igual a $0$.</li>

<li>Quedan suficientes cuadritos sin destruir para que la hormiga pueda llegar a la meta. </li>
</ul>
Sea $P$ el número de caminos de longitud par que puede seguir la hormiga. Sea $I$ el número de caminos de longitud impar que puede seguir la hormiga. Encuentra todos los posibles valores de $P-I$.<br>

Nota: La longitud de un camino es el número de pasos que da la hormiga. Por ejemplo, se muestra un posible camino de longitud $8$ en la figura de $6\times 7$ siguiente, en la que los cuadritos destruidos están sombreados y la meta está indicada con una estrella.