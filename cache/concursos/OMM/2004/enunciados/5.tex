Sean $\mathcal{A}$ y $\mathcal{B}$ dos circunferencias tales que el centro $O$ de $\mathcal{B}$ esté en $\mathcal{A}$. Sean $C$ y $D$ los dos puntos de intersección de las circunferencias. Se toman un punto $A$ en $\mathcal{A}$ y un punto $B$ en $\mathcal{B}$ tales que $AC$ es tangente a $\mathcal{B}$ en $C$ y $BC$ es tangente a $\mathcal{A}$ en el mismo punto $C$. El segmento $AB$ corta de nuevo a $\mathcal{B}$ en $E$ y ese mismo segmento corta de nuevo a $\mathcal{A}$ en $F$. La recta $CE$ vuelve a cortar a $\mathcal{A}$ en $G$ y la recta $CF$ corta a la recta $GD$ en $H$. Muestra que el punto de intersección de $GO$ y $EH$ es el centro de la circunferencia circunscrita al triángulo $DEF$.