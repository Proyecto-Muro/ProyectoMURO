Sean $\mathcal{C}_1$ una circunferencia con centro $O$, $P$ un punto sobre ella y $\ell$ la recta tangente a $\mathcal{C}_1$ en $P$. Considera un punto $Q$ sobre $\ell$, distinto de $P$, y sea $\mathcal{C}_2$ la circunferencia que pasa por $O$, $P$ y $Q$. El segmento $OQ$ intersecta a $\mathcal{C}_1$ en $S$ y la recta $PS$ intersecta a $\mathcal{C}_2$ en un punto $R$ distinto de $P$. Si $r_1$ y $r_2$ son las longitudes de los radios de $\mathcal{C}_1$ y $\mathcal{C}_2$, respectivamente, muestra que
\[\frac{PS}{SR}=\frac{r_1}{r_2}.\]