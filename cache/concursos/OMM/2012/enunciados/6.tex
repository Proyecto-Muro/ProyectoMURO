Considera un triángulo acutángulo $ABC$ con circuncírculo $\Omega$. Sean $H$ el ortocentro del
triángulo $ABC$ y $M$ el punto medio de $BC$. Las rectas $AH$, $BH$ y $CH$ cortan por
segunda vez a $\Omega$ en $D$, $E$ y $F$, respectivamente; y la recta $MH$ corta a $\Omega$ en $J$ de manera
que $H$ queda entre $M$ y $J$. Sean $K$ y $L$ los incentros de los triángulos $DEJ$ y $DFJ$,
respectivamente. Muestra que $KL$ es paralela a $BC$.