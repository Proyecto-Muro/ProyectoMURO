A cada entero positivo se le aplica el siguiente proceso: al número se le resta la suma de sus dígitos, y el resultado se divide entre $9$. Por ejemplo, el proceso aplicado al $938$ es $102$, ya que  $\frac{938-(9+3+8)}{9}=102$. Aplicado dos veces el proceso a $938$ se llega a $11$, aplicado tres veces se llega a $1$, y aplicado cuatro veces se llega al $0$. Cuando a un entero positivo $n$ se le aplica el proceso una o varias veces, se termina en $0$. Al número al que se llega antes de llegar al cero, lo llamamos la casa de $n$. ¿ Cuántos
números menores que $26000$ tienen la misma casa que el $2012$?