Sean $\mathcal{C}_1$ y $\mathcal{C}_2$ dos circunferencias que se cortan en los puntos $A$ y $B$. Consideremos un punto $C$ sobre la recta $AB$ de modo que $B$ queda entre $A$ y $C$. Sean $P$ y $Q$ puntos sobre $\mathcal{C}_1$ y $\mathcal{C}_2$, respectivamente, tales que que $CP$ es tangente a $\mathcal{C}_1$, $CQ$ es tangente a $\mathcal{C}_2$, $P$ no está dentro de $\mathcal{C}_2$ y $Q$ no está dentro de $\mathcal{C}_1$. La recta $PQ$ corta de nuevo a $\mathcal{C}_1$ en $R$ y a $\mathcal{C}_2$ en $S$, ambos puntos distintos de $B$. Supongamos que $CR$ corta de nuevo a $\mathcal{C}_1$ en $X$ y $CS$ corta de nuevo a $\mathcal{C}_1$ en $Y$ . Sea $Z$ un punto sobre la recta $XY$. Muestra que $SZ$ es paralela a $QX$ si y sólo si $PZ$ es paralela a $RX$.