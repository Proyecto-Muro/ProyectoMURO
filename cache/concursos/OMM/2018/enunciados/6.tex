Sea $ABC$ un triángulo acutángulo con circunferencia $\Omega$. Las bisectrices ángulos $B$ y $C$ intersectan a $\Omega$ en $M$ y $N$. Sea $I$ el punto de intersección de estas bisectrices. Sean $M'$ y $N'$ las respectivas reflexiones de $M$ y $N$ por $AC$ y $AB$. Muestra que el centro de la circunferencia que pasa por $I$, $M'$, $N'$ se encuentra en la altitud del triángulo $ABC$ desde $A$.
