Una pulga salta sobre puntos enteros de la recta numérica. En su primer movimiento salta desde el punto $0$ y cae en el punto $1$. Luego, si en un movimiento la pulga saltó desde el punto $a$ y cayó en el punto $b$, en el siguiente movimiento salta desde el punto $b$ y cae en uno de los puntos $b+(b-a)-1$, $b+(b-a)$, $b+(b-a)+1$.<br>
Demuestre que si la pulga ha caído dos veces sobre el punto $n$, para $n$ entero positivo, entonces ha debido hacer al menos $t$ movimientos, donde $t$ es el menor entero positivo mayor o igual que $2\sqrt{n}$.
