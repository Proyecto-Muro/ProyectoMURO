En cada casilla de un tablero de $n \times n$ hay una lámpara. Al ser tocada una lámpara cambian de estado ella misma y todas las lámparas situadas en la fila y la columna que ella determina (las que están encendidas se apagan y las apagadas se encienden). Inicialmente todas están apagadas. Demostrar que siempre es posible, con una sucesión adecuada de toques, que todo el tablero quede encendido y encontrar, en función de $n$, el número mínimo de toques para que se enciendan todas las lámparas.