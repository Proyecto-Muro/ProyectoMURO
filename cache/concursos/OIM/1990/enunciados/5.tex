Sean $A$ y $B$ dos vértices opuestos de un tablero de ajedrez de $n\times n$. En cada cuadrito de $1\times 1$ se traza la diagnonal paralela a $AB$, creando $2n^2$ triángulos iguales. Una ficha se mueve de $A$ a $B$ sobre las líneas del tablero, pasando a lo más una vez por cada segmento. Cada vez que la ficha pasa por un segmento, se pone una semilla dentro de los triángulos que tienen a ese segmento como lado. Después de llegar a $B$, hay exactamente dos semillas en cada uno de los $2n^2$ triángulos del tablero. ¿Para cuáles valores de $n$ es posible esto?