Con centro en el incentro $I$ de un triángulo $ABC$ se traza una circunferencia que corta en dos puntos a cada uno de los tres lados del triángulo: al segmento $BC$ en $D$ y $P$ (siendo $D$ el más cercano a $B$); al segmento $CA$ en $E$ y $Q$ (siendo $E$ el más cercano a $C$), y al segmento $AB$ en $F$ y $R$ (siendo $F$ el más cercano a $A$).<br>
Sea $S$ el punto de intersección de las diagonales del cuadrilátero $EQFR$. Sea $T$ el punto de intersección de las diagonales del cuadrilátero $FRDP$. Sea $U$ el punto de intersección de las diagonales del cuadrilátero $DPEQ$. Demostrar que las circunferencias circunscritas a los triángulos $FRT$, $DPU$ y $EQS$ tienen un único punto común.
