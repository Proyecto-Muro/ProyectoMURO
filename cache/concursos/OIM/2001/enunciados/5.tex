En un tablero de $2000 \times 2001$ las casillas tienen coordenadas $(x,y)$ con $x$, $y$ enteros, $0 \leq x \leq 1999$ y $0 \leq y \leq 2000$. Una nave en el tablero se mueve de la siguiente manera: antes de cada movimiento, la nave está en una posición $(x,y)$ y tiene una velocidad $(h,v)$ donde $h$ y $v$ son enteros. La nave escoge una nueva velocidad $(h',v')$ de forma que $h' - h$ sea igual a $-1$, $0$ ó $1$ y $v'- v$ sea igual a $-1$, $0$ ó $1$. La nueva posición de la nave será $(x',y')$ donde $x'$ es el resto de dividir $x+h'$ entre $2000$ e $y'$ es el resto de dividir $y + v'$ entre $2001$.<br>
Hay dos naves en el tablero: la marciana y la terrestre que quiere atrapar a la marciana. Inicialmente cada nave está en una casilla del tablero y tiene velocidad $(0,0)$. Primero se mueve la nave terrestre y continúan moviéndose alternadamente.<br>
¿Existe una estrategia que siempre le permita a la nave terrestre atrapar a la nave marciana, cualesquiera que sean las posiciones iniciales?<br>
<em>Nota: la nave terrestre, que siempre ve a la marciana, atrapa a la marciana si después de un movimiento suyo cae en la misma posición de la marciana.</em>
