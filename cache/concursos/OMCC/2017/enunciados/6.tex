La rana Tita se sienta en la recta numérica. Inicialmente está en el número entero $k>1$. Si está sentada en el número $n$, salta al número $f(n)+g(n)$, donde $f(n)$ y $g(n)$ son, respectivamente, los números primos positivos más grandes y más pequeños que dividen a $n$. Encuentre todos los valores de $k$ de manera que Tita pueda saltar a infinitos números enteros distintos.
