Multiplicando para deshacernos del denominador, 
la condición del problema se vuelve $r^2-5q^2=2p^2-2$, 
o $2=2p^2+5q^2-r^2$. Observemos que los residuos cuadráticos
mod $8$ son:
\begin{align*}
0^2&\equiv0\pmod 8
1^2&\equiv1\pmod 8
2^2&\equiv4\pmod 8
3^2&\equiv1\pmod 8
4^2&\equiv0\pmod 8
5^2&\equiv1\pmod 8
6^2&\equiv4\pmod 8
7^2&\equiv1\pmod 8
\end{align*}
Entonces, asumiendo que $p\neq 2$ y $q \neq 2$, 
\[2p^2+5q^2-r^2\equiv 2+5-r^2\equiv 7-r^2 \pmod 8\]
Pero esto es equivalente a $2\pmod 8$, por la condición del problema.
Entonces $r^2\equiv 5 \pmod 8$, que es imposible.
Esto significa que alguno de $p$ o $q$ es $2$.

Caso 1, $p=2$:

Si $p=2$, tenemos que \[0=r^2-5q^2-6\equiv r^2-11\equiv r^2-3 \pmod 8,\] 
a menos que $q=2$. Pero es imposible que $r^2-3\equiv 0 \pmod 8$, entonces 
llegamos a una contradicción, y $q=2$.

Caso 2, $q=2$:

Si $q=2$, tenemos que 
\[r^2-20=2p^2-2\implies r^2-2p^2=18.\]
Los residuos cuadráticos mod $3$ son $0$ o $1$, entonces si $p\neq 3$, 
$r^2-2\equiv 0 \pmod 3$, que es imposible. Por lo tanto, $p=3$.

Ahora, $r^2-18=18\implies r=6$.  

Entonces la única solución del problema es $(p,q,r)=(3,2,6)$.

