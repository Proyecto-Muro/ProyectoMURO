El banco de Oslo tiene dos tipos distintos de monedas: aluminio (denotado $A$), y bronce (denotado $B$). Marianne tiene $n$ monedas de aluminio y $n$ monedas de bronce en una fila, en orden aleatorio. Una cadena es cualquier subsecuencia de monedas consecutivas del mismo tipo. Dado un entero positivo $k\leq 2n$, Gilberty hace la siguiente operación varias veces: encuentra la cadena más larga que contiene a la $k$-ésima moneda (de izquierda a derecha) y mueve toda la cadena a el extremo izquierdo de la fila. Por ejemplo, si $n=4$ y $k=4$, el proceso empezando desde $AABBBABA$ sería 
\[AABBBABA \to BBBAAABA \to AAABBBBA \to BBBBAAAA \to ...\]
Encuentra todas las parejas $(n,k)$ con $1\leq k\leq2n$ tales que para cualquier orden inicial de la fila, en algún momento las $n$ monedas de la izquierda serán todas iguales.