Sea $n$ un número entero positivo. Un <i>cuadrado nórdico</i> es un tablero de $n\times n$ que contiene todos los números del $1$ al $n^2$ de modo que cada celda contiene exactamente un número. Dos celdas diferentes son adyacentes si comparten un mismo lado. Una celda que solamente es adyacente a celdas que contienen números mayores se llama un <i>valle</i>. Un <i>camino ascendente</i> es una sucesión de una o más celdas tales que:
<ol type="i">
<li> la primera celda de la sucesión es un valle,</li>
<li> cada celda subsiguiente de la sucesión es adyacente a la celda anteiror, y</li>
<li> los números escritos en las celdas de la sucesión están en orden creciente.</li>
</ol>
Hallar, como función de $n$, el menor número total de caminos ascendentes en un cuadrado nórdico.