Supongamos que el excírculo del triángulo $ABC$ opuesto al vértice $A$ es tangente al lado $BC$ en el punto $A_1$. Análogamente, se definen los puntos $B_1$ en $CA$ y $C_1$ en $AB$, utilizando los excírculos opuestos a $B$ y $C$ respectivamente. Supongamos que el circuncentro del triángulo $A_1B_1C_1$ pertenece a la circunferencia que pasa por los vértices $A$, $B$ y $C$. Demostrar que el triángulo $ABC$ es rectángulo.
