Sean $a_1, a_2, \dots a_n$ números reales. Para cada $i$, $ (1 \leq i \leq n )$, definimos 
\[ d_{i} = \max \{ a_{j}\mid 1 \leq j \leq i \} - \min \{ a_{j}\mid i \leq j \leq n \}, \]
y sea $d = \max \{d_{i}\mid 1 \leq i \leq n \}$.
<ul>
<li> Muestra que para cualesquiera números reales $x_1\leq x_2\leq\dots\leq x_n$, 
\[ \max \{ |x_{i} - a_{i}| \mid 1 \leq i \leq n \}\geq \frac {d}{2}. \quad \quad (*) \]</li>
<li> Muestra que existen números reales $x_1\leq x_2\leq\dots\leq x_n$ que cumplen la igualdad en $(*)$.</li>
</ul>