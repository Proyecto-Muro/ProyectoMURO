Un conejo invisible y un cazador juegan como sigue en el plano euclideano. El punto de partida $A_0$ del conejo, y el punto de partida $B_0$ del cazador son el mismo. Después de $n−1$ rondas del juego, el conejo se encuentra en el punto $A_{n−1}$ y el cazador se encuentra en el punto $B_{n−1}$. En la $n$-ésima ronda del juego, ocurren tres hechos en el siguiente orden: <br>
(i) El conejo se mueve de forma invisible a un punto $A_n$ tal que la distancia entre $A_{n−1}$ y $A_n$ es exactamente $1$. <br>
(ii) Un dispositivo de rastreo reporta un punto $P_n$ al cazador. La única información segura que da el dispositivo al cazador es que la distancia entre $P_n$ y $A_n$ es menor o igual que $1$. <br>
(iii) El cazador se mueve de forma visible a un punto $B_n$ tal que la distancia entre $B_{n−1}$ y $B_n$ es exactamente $1$. <br>
¿Es siempre posible que, cualquiera que sea la manera en que se mueva el conejo y cualesquiera que sean los puntos que reporte el dispositivo de rastreo, el cazador pueda escoger sus movimientos de modo que después de $10^9$ rondas el cazador pueda garantizar que la distancia entre él mismo y el conejo sea menor o igual que $100$?