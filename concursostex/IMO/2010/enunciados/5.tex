En cada una de las seis cajas $B_1, B_2, B_3, B_4, B_5, B_6$ hay inicialmente sólo una moneda. Se permiten dos tipos de operaciones: \newline 
Tipo 1: Elegir una caja no vacía $B_j$, con $1 \leq j \leq 5$. Retirar una moneda de $B_j$ y añadir dos monedas a $B_{j+1}$. \newline 
Tipo 2: Elegir una caja no vacía $B_k$, con $1 \leq k \leq 4$. Retirar una moneda de $B_k$ e intercambiar los contenidos de las cajas (posiblemente vacías) $B_{k+1}$ y $B_{k+2}$. \newline 
Determine si existe una sucesión finita de estas operaciones que deja a las cajas $B_1,B_2,B_3,B_4,B_5$ vacías y a la caja $B_6$ con exactamente $2010^{2010^{2010}}$ monedas. (Observe que $a^{b^{c}} = a^{(b^c)}$.)