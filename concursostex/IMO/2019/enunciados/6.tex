Sea $I$ el incentro del triángulo acutángulo $ABC$ con $AB\neq AC$. La circunferencia inscrita (o incírculo) $\omega$ de $ABC$ es tangente a los lados $BC$, $CA$ y $AB$ en $D$, $E$ y $F$, respectivamente. La recta que pasa por $D$ y es perpendicular a $EF$ corta a $\omega$ nuevamente en $R$. La recta $AR$ corta a $\omega$ nuevamente en $P$. Las circunferencias circunscritas (o circuncírculos) de los triángulos $PCE$ y $PBF$ se cortan nuevamente en $Q$.
Demostrar que las rectas $DI$ y $PQ$ se cortan en la recta que pasa por $A$ y es perpendicular a $AI.$