Sea $n\geq 2$ un entero positivo y $\lambda$ un real positivo. Inicialmente hay $n$ pulgas en una línea horizontal, no todas en el mismo punto. Definimos un movimiento como elegir dos pulgas en puntos $A$ y $B$ (con $A$ a la izquierda de $B$), y hacer que la pulga en $A$ salte sobre la pulga $B$ al punto $C$ tal que $ \frac {BC}{AB} = \lambda$. Encuentra todos los valores de $\lambda$ tales que para cualquier punto $M$ en la línea y cualquier posición inicial de las pulgas, existe una secuencia de movimientos que llevará a todas las pulgas a la derecha de $M$.