Un conjunto de rectas en el plano está en posición general si no hay dos que sean paralelas ni tres que pasen por el mismo punto. Un conjunto de rectas en posición general separa el plano en regiones, algunas de las cuales tienen área finita; a estas las llamamos sus regiones finitas. Demostrar que para cada $n$ suficientemente grande, en cualquier conjunto de n rectas en posición general es posible colorear de azul al menos $\sqrt{n}$ de ellas de tal manera que ninguna de sus regiones finitas tenga todos los lados de su frontera azules. \newline 
Nota: A las soluciones que reemplacen $\sqrt{n}$ por $c\sqrt{n}$ se les otorgarán puntos dependiendo del valor de $c$.