Sean $R$ y $S$ puntos distintos sobre la circunferencia $\Omega$ tales que $RS$ no es un diámetro de $\Omega$. Sea $\ell$ la recta tangente a $\Omega$ en $R$. El punto $T$ es tal que $S$ es el punto medio del segmento $RT$. El punto $J$ se elige en el menor arco $RS$ de $\Omega$ de manera que $\Gamma$, la circunferencia circunscrita al triángulo $JST$, intersecta a $\ell$ en dos puntos distintos. Sea $A$ el punto común de $\Gamma$ y $\ell$ más cercano a $R$. La recta $AJ$ corta por segunda vez a $\Omega$ en $K$. Demostrar que la recta $KT$ es tangente a $\Gamma$.