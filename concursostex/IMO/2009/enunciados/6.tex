Sean $a_1, a_2, \ldots , a_n$ enteros positivos distintos y $M$ un conjunto de $n-1$ enteros positivos que no contiene a $s=a_1 + a_2 + \ldots + a_n.$ Un saltamontes salta sobre la línea de los números reales, empezando en el $0$. Hace $n$ saltos a la derecha con longitudes $a_1, a_2, \ldots , a_n$ en algún orden. Muestra que el orden puede elegirse de manera que el saltamontes nunca pase por un número en $M$.