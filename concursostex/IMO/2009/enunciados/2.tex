Sea $ABC$ un triángulo con circuncentro $O$. Los puntos $P$ y $Q$ son puntos sobre los lados $CA$ y $AB$, respectivamente. Sean $K$, $L$, y $M$ los puntos medios de los segmentos $BP$, $CQ$, y $PQ$, respectivamente, y sea $\Gamma$ el circuncírculo que pasa por $K$, $L$ y $M$. Muestra que $OP=OQ$ si $PQ$ es tangente a $\Gamma$.