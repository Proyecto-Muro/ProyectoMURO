Sea $S$ un conjunto finito de dos o más puntos del plano. En $S$ no hay tres puntos colineales. Un remolino es un proceso que empieza con una recta $\ell$ que pasa por un único punto $P$ de $S$. Se rota $\ell$ en el sentido de las manecillas del reloj con centro en $P$ hasta que la recta encuentre por primera vez otro punto de $S$ al cual llamaremos $Q$. Con $Q$ como nuevo centro se sigue rotando la recta en el sentido de las manecillas del reloj hasta que la recta encuentre otro punto de $S$. Este proceso continúa indefinidamente. \newline 
Demostrar que se puede elegir un punto $P$ de $S$ y una recta $\ell$ que pasa por $P$ tales que el remolino que resulta usa cada punto de $S$ como centro de rotación un número infinito de veces.