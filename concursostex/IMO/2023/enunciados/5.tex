Sea $n$ un entero positivo. Un triángulo japonés consiste en $1 + 2 + \dots + n$ círculos iguales acomodados en forma de triángulo equilátero de modo que para cada $i = 1, 2, \dots , n$, la fila número $i$ contiene $i$ círculos, de los cuales exactamente uno de ellos se pinta de rojo. Un camino ninja en un triángulo japonés es una sucesión de $n$ círculos que comienza con el círculo de la fila superior y termina en la fila inferior, pasando sucesivamente de un círculo a uno de los dos círculos inmediatamente debajo de él. En el siguiente dibujo se muestra un ejemplo de un triángulo japonés con $n = 6$, junto con un camino ninja en ese triángulo que contiene dos círculos rojos. \newline 

En términos de $n$, determina el mayor $k$ tal que cada triángulo japonés tiene un camino ninja que contiene al menos $k$ círculos rojos.
