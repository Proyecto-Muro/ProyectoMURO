Sea $n$ un número entero positivo. Un \emph{cuadrado nórdico} es un tablero de $n\times n$ que contiene todos los números del $1$ al $n^2$ de modo que cada celda contiene exactamente un número. Dos celdas diferentes son adyacentes si comparten un mismo lado. Una celda que solamente es adyacente a celdas que contienen números mayores se llama un \emph{valle}. Un \emph{camino ascendente} es una sucesión de una o más celdas tales que:
\begin{enumerate}
 \item  la primera celda de la sucesión es un valle,
 \item  cada celda subsiguiente de la sucesión es adyacente a la celda anterior, y
 \item  los números escritos en las celdas de la sucesión están en orden creciente.
 \end{enumerate} 
Hallar, como función de $n$, el menor número total de caminos ascendentes en un cuadrado nórdico.