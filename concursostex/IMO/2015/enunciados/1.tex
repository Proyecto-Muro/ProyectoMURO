Decimos que un conjunto finito $S$ de puntos en el plano es "equilibrado" si para cada dos puntos distintos $A$ y $B$ en $S$ hay un punto $C$ en $S$ tal que $AC=BC$. Decimos que $S$ es "libre de centros" si para cada tre puntos distintos $A,B,C$ en $S$ no existe ningún punto $P$ en $S$ tal que $PA=PB=PC$. \newline 

(a) Demostrar que para todo $n\ge 3$ existe un conjunto de $n$ puntos equilibrado. \newline 
(b) Determinar todos los enteros $n\ge 3$ para los que existe un conjunto de $n$ puntos equilibrado y libre de centros.