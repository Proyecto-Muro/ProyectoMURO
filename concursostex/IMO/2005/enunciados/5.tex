Sea $ABCD$ un cuadrilátero convexo ctal que $BC=DA$ y $BC$ no es paralela a $DA$. Sean $E$ y $F$ dos puntos que varían sobre los lados $BC$ y $DA$, respectivamente, tales que $BE=DF$. Las líneas $AC$ y $BD$ se encuentran en $P$, las líneas $BD$ y $EF$ se encuentran en $Q$, y las líneas $EF$ y $AC$ se encuentran en $R$. Muestra que los circuncírculos de los triángulos $PQR$ tienen un punto común distinto a $P$ sin importar la elección de $E$ y $F$.