Alka encuentra escrito en un pizarrón un número $n$ que termina en 5. Realiza una secuencia de operaciones con el número en el pizarrón. En cada paso, decide realizar una de las dos operaciones siguientes:


 \begin{enumerate}  \item  Borrar el número escrito $m$ y escribir su cubo $m^3$.
 \item  Borrar el número escrito $m$ y escribir el producto $2023m$.
 \end{enumerate} 


Alka realiza cada operación un número par de veces en algún orden y al menos una vez, y obtiene finalmente el número $r$. Si la cifra de las decenas de $r$ es un número impar, encuentra todos los valores posibles que la cifra de las decenas de $n^3$ pudo haber tenido.