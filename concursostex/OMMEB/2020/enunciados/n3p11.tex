Los hexágonos regulares $A_0$ y $B_0$ tienen lados iguales a $1$. Para cada entero positivo $n$, el hexágono regular $A_n$ se construye uniendo los puntos medios de los lados del hexágono regular $A_{n-1}$. El hexágono regular $B_n$ se construye traslapando dos triángulos equiláteros con vértices de $B_{n-1}$. En la siguiente figura se muestran los hexágonos regulares $A_1$ y $B_1$. La razón del área de $A_{2020}$ entre el área de $B_{2020}$ puede escribirse de la forma $(\frac ab)^c$ donde $a,b,c$ son enteros positivos y $a$, $b$ son primos relativos. Si $c$ toma el valor máximo posible, encuentra el valor de $\frac{c}{a+b}$ .