Ana y Bety juegan un juego por turnos de manera alternada. Inicialmente Ana elige un entero positivo impar y compuesto $n$ tal que $2^j\lt n\lt 2^{j+1}$ con $2\lt j$. En su primer turno Bety elige un entero positivo impar y compuesto $n_1$ tal que \[n_1\leq \frac{1^n+2^n+\dots+(n-1)^n}{2(n-1)^{n-1}}.\]
Luego, en su turno, Ana elige un número primo $p_1$ que divida $n_1$. Si el primo que eligió Ana es $3$, $5$ o $7$, enctonces Ana gana. De lo contrario Bety elige un entero positivo impar y compuesto $n_2$ tal que 
\[n_2\leq \frac{1^{p_1}+2^{p_1}+\dots+(p_1-1)^{p_1}}{2(p_1-1)^{p_1-1}}.\]
Después de esto, en su turno, Ana elige un primo $p_2$ que divida a $n_2$. Si $p_2$ es $3$, $5$, o $7$, Ana gana. De lo contrario el proceso se repite. Además, Ana gana en cualquier momento si Bety no puede elegir un entero positivo impar y compuesto en el rango correspondiente. Bety gana si logra jugar al menos $j-1$ turnos. Encuentra cuál de las dos jugadoras tiene estrategia ganadora.