Nota: diremos que el área de la figura $\mathcal{P}$ es $[\mathcal{P}]$

Iniciamos prolongando $CM$ hasta que intersecta a la línea $AB$ en $N$. 
Como $M$ es el punto medio de $AD$, y $AB\parallel CD$, los triángulos 
$\bigtriangleup NAM$ y $\bigtriangleup CDM$ son congruentes, y 
$NM=MC$. 

Como comparten base y altura, los triángulos $\bigtriangleup NMB$ 
y $\bigtriangleup BMC$ tienen la misma área. A demás, como son congruentes, 
$[\bigtriangleup NAM]=[\bigtriangleup CDM]$. Entonces 
\begin{align*}
[ABCD]&=[ABM]+[MBC]+[CDM]\\
&=[ABM]+[MBC]+[NAM]\\
&=[NMB]+[MBC]\\
&=2\cdot[MBC].
\end{align*}
Entonces el área de $ABCD$ es el doble del área de 
$\bigtriangleup MBC$.

También, por la fórmula de área, $[MBC]=\frac12 ab \sin{150^{\circ}}$.
Entonces $[ABCD]=2\cdot\frac12 ab \sin{150^{\circ}}= \boxed{\frac12ab}$.
