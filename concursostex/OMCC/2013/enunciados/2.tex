Alrededor de una mesa redonda están sentadas en sentido horario las personas $P_1,P_2,\ldots,P_{2013}$. Cada una tiene cierta cantidad de monedas (posiblemente ninguna); entre todas tienen $10000$ monedas. Comenzando por $P_1$ y prosiguiendo en sentido horario, cada persona en su turno hace lo siguiente: Si tiene un número par de monedas, se las entrega todas a su vecino de la izquierda. Si en cambio tiene un número impar de monedas, le entrega a su vecino de la izquierda un número impar de monedas (al menos una y como máximo todas las que tiene), y conserva el resto.
Pruebe que, repitiendo este procedimiento, necesariamente llegará un momento en que todas las monedas estén en poder de una misma persona.
