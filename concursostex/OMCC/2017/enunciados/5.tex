Susana y Brenda juegan a escribir polinomios en la pizarra. Susana empieza y juegan por turnos.
 En el turno preparatorio (turno $0$), Susana elige un entero positivo $n_0$ y escribe el polinomio $P_0(x)=n_0$. En el turno $1$, Brenda elige un entero positivo $n_1$, distinto de $n_0$, y escribe el polinomio
\[ P_1(x)=n_1x+P_0(x) \textup{ o } P_1(x)=n_1x-P_0(x). \]
En general, en el turno $k$, el jugador respectivo elige un número entero $n_k$, diferente de $n_0, n_1, \ldots, n_{k-1}$, y escribe el polinomio
\[P_k(x)=n_kx^k+P_{k-1}(x) \textup{ o } P_k(x)=n_kx^k-P_{k-1}(x).\]

El primer jugador que escriba un polinomio con al menos una raíz entera gana. Encuentra y describe una estrategia ganadora.
