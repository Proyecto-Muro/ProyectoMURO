En La Habana se realiza un baile con $2018$ parejas. Para el baile, se dispone de una circunferencia donde inicialmente se marcan $2018$ puntos distintos, etiquetados con los números $0,1,\ldots,2017$. Las parejas son ubicadas sobre los puntos marcados, una en cada punto. Para $i\ge1$, se define $s_i$ como el residuo de dividir $i$ entre $2018$ y $r_i$ como el residuo de dividir $2i$ entre $2018$. El baile comienza en el minuto $0$. En el $i-$ésimo minuto después de haber inciado el baile, la pareja ubicada en el punto $s_i$ (si la hay) se mueve al punto $r_i$, la pareja que ocupaba el punto $r_i$ (si la hay) se retira, y el baile continúa con las parejas restantes. El baile termina después de $2018^2$ minutos. Determine cuantas parejas quedarán al terminar el baile.
