Dos jugadores $ A$, $ B$ y otras 2001 personas forman un círculo, de manera que $ A$ y $ B$ no están en posiciones consecutivas. $ A$ y $ B$ juegan en turnos alternos, empezando por $ A$. Una jugada consiste en tocar a una de las personas vecinas, la cual una vez tocada sale del círculo. El ganador es el último que queda en pie.
Demuestre que uno de los dos jugadores tiene una estrategia ganadora, y dé dicha estrategia.
Nota: Un jugador tiene una estrategia ganadora si es capaz de ganar sin importar lo que haga el adversario.
