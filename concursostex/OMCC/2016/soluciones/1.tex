Los únicos dígitos que son cuadrados perfectos son $0,1,4,9$. 
Como $n$ es múltiplo de $2$ y $5$, es múltiplo de $10$, entonces termina en $0$. 
Como $n$ es múltiplo de $3$, la suma de los dígitos debe ser múltiplo de $3$. Los dígitos $0,1,4,9$ 
son congruentes a $0,1,1,0$ módulo $3$, entonces deben sumar $0+0+0$ o $1+1+1$.
Esto significa que los dígitos deben ser todos $0$ o $9$, o todos $1$ o $4$. 

Entonces los primeros $3$ dígitos de $n$ son: 
\[(9,0,0),(9,9,0),(9,9,9),(1,1,1),(4,1,1),(4,4,1),(4,4,4),\]
o algunas de sus permutaciones. Estos números son 
\begin{align*}
&9000, 9900, 9090, \\
&9990, 1110, 4110, \\
&1410, 1140, 4410, \\
&4140, 1440, 4440
\end{align*}
Revisando uno por uno, vemos que el único de estos números 
que es múltiplo de $7$ es \boxed{4410}. 
