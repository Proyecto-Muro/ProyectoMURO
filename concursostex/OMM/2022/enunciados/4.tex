Sea $n$ un entero positivo. En un jardín de $n\times n$ cuyos lados dan al Norte, Sur, Este y Oeste, se va a construir una fuente usando plataformas de $1\times 1$ que cubran todo el jardín. Ana colocará las plataformas todas a diferentes alturas. Después, Beto pondrá salidas de agua en algunas de las plataformas. El agua de cada plataforma puede bajar a las plataformas contiguas (hacia el Norte, Sur, Este y Oeste) que tengan menor altura que la plataforma de donde viene el agua, siguiendo su flujo siempre que pueda dirigirse a plataformas de menor altura. El objetivo de Beto es que el agua llegue a todas las plataformas. ¿Cuál es el menor número de salidas de agua que Beto necesita tener disponibles a fin de garantizar que podrá lograr su objetivo, sin importar cómo Ana haya acomodado las plataformas?