En cada cuadrado de una cuadrícula de $6 \times 6$ hay una luciérnaga apagada o encendida.
Una movida es escoger tres cuadrados consecutivos ya sean los tres verticales o los tres
horizontales, y cambiar de estado a las tres luciérnagas que se encuentran en dichos
cuadrados. Cambiar de estado a una luciérnaga significa que si está apagada se enciende
y viceversa.
Muestra que si inicialmente hay una luciérnaga encendida y las demás apagadas, no es
posible hacer una serie de movidas tales que al final todas las luciérnagas estén apagadas.