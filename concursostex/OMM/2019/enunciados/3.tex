Sea $n\geq 2$ un número entero. Considera $2n$ puntos alrededor de un círculo. Cada vértice se ha marcado con un número entero desde $1$ hasta $n$, inclusive, y cada uno de estos enteros se ha utilizado exactamente dos veces. Isabel divide los puntos en $n$ pares, y dibuja los segmentos que los unen, con la condición de que estos segmentos no se crucen. A continuación, asigna a cada segmento el mayor número entero entre sus puntos extremos. Muestra que, independientemente de cómo se hayan marcado los puntos, Isabel siempre puede elegir los pares de forma que utilice exactamente $\lceil n/2\rceil$ números para marcar los segmentos. ¿Se pueden etiquetar los puntos de tal manera que, independientemente de cómo Isabel divida los puntos en pares, siempre utilice exactamente $\lceil n/2\rceil$ números para etiquetar los segmentos?

Nota: Para cada número real $x$, $\lceil x\rceil$ denota el menor número entero mayor o igual que $x$. Por ejemplo, $\lceil 3,6\rceil=4$ y $\lceil 2\rceil=2$.

