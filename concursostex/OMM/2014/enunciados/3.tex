Sean $\Gamma_1$ una circunferencia y $P$ un punto fuera de $\Gamma_1$. Las tangentes desde $P$ a $\Gamma_1$ tocan a la circunferencia en los puntos $A$ y $B$. Considera $M$ el punto medio del segmento $PA$ y $\Gamma_2$ la circunferencia que pasa por los puntos $P$, $A$ y $B$. La recta $BM$ intersecta de nuevo a $\Gamma_2$ en el punto $C$, la recta $CA$ intersecta de nuevo a $\Gamma_1$ en el punto $D$, el segmento DB intersecta de nuevo a $\Gamma_2$ en el punto $E$ y la recta $PE$ intersecta a $\Gamma_1$ en el punto $F$ (con $E$ entre $P$ y $F$). Muestra que las rectas $AF$, $BP$ y $CE$ concurren.