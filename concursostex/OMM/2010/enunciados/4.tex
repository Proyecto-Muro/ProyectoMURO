Sea $n$ un entero positivo. En una cuadrícula de $n\times 4$, cada región es igual a 
\[\boxed{2}\boxed{0}\boxed{1}\boxed{0}\]
Un "cambio" es tomar tres casillas consecutivas en el mismo renglón y con dígitos distintos escritos en ellas y cambiar los tres dígitos de estas casillas escritas de la siguiente manera
\[0\to 1, \quad 1\to 2, \quad 2\to 0.\]
Por ejemplo, un renglón $\boxed{2}\boxed{0}\boxed{1}\boxed{0}$ puede cambiarse al renglón $\boxed{0}\boxed{1}\boxed{2}\boxed{0}$ pero no al renglón $\boxed{2}\boxed{1}\boxed{2}\boxed{1}$ pues $0$, $1$ y $0$ no son distintos entre sí.
Los cambios se pueden aplicar cuantas veces se quiera, aún a renglones ya cambiados.
Muestra que para $n < 12$ no es posible hacer un número finito de cambios de forma que
la suma de los números en cada una de las cuatro columnas sea la misma.