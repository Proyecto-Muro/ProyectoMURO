Sean $\mathcal{C}_1$ y $\mathcal{C}_2$ dos circunferencias tangentes exteriormente en un punto $A$. Se traza una recta tangente a $\mathcal{C}_1$ en $B$ y secante a $\mathcal{C}_2$ en $C$ y $D$; luego se prolonga el segmento $AB$ hasta intersecar a $\mathcal{C}_2$ en un punto $E$. Sea $F$ el punto medio del arco $CD$ sobre $\mathcal{C}_2$ que no contiene a $E$ y sea $H$ la intersección de $BF$ con $\mathcal{C}_2$. Muestra que $CD$, $AF$ y $EH$ son concurrentes.