Sean $A$, $B$ y $C$ tres puntos colineales con $B$ entre $A$ y $C$. Sea $\mathcal{Y}$ una circunferencia tangente a $AC$ en $B$, sean $\mathcal{X}$ y $\mathcal{Z}$ las circunferencias de diámetros $AB$ y $BC$, respectivamente. Sea $P$ el punto (además de $B$) en el que se cortan las circunferencias $\mathcal{X}$ y $\mathcal{Y}$; sea $Q$ el otro punto (además de $B$) en el que se cortan las circunferencias $\mathcal{Y}$ y $\mathcal{Z}$. Supón que la recta $PQ$ corta a $\mathcal{X}$ en un punto $R$ distinto de $P$ y que esta misma recta $PQ$ corta a $\mathcal{Z}$ en un punto $S$ distinto de $Q$. Muestra que concurren $AR,$ $CS$ y la tangente común a $\mathcal{X}$ y $\mathcal{Z}$ por $B$.