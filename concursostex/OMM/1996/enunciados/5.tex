Los números de $1$ a $n^2$ se escriben en un papel cuadriculado de $n\times n$ en el orden normal. Cualquier secuencia de pasos hacia la derecha y hacia abajo de un cuadrado a otro adyacente (compartiendo lado) que comienza en el cuadrado $1$ y termina en el cuadrado $n^2$ se llama camino. Denotemos por $L(C)$ la suma de los números por los que pasa el camino $C$.

Para un $n$ fijo, sean $M$ y $m$ el mayor y menor valor de $L(C)$ posibles. Muestra que $M-m$ es un cubo perfecto.

Muestra que para ningún $n$ se puede encontrar un camino $C$ con $L(C) = 1996$.