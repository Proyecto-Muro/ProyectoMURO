Sobre una circunferencia $\Gamma$ se encuentran los puntos $A, B, N , C, D$ y $M$ colocados en el
sentido de las manecillas del reloj de manera que $M$ y $N$ son los puntos medios de los
arcos $DA$ y $BC$ (recorridos en el sentido de las manecillas del reloj). Sea $P$ la intersección
de los segmentos $AC$ y $BD$; y sea $Q$ un punto sobre $MB$ de manera que las rectas $PQ$
y $MN$ son perpendiculares. Sobre el segmento $MC$ se considera un punto $R$ de manera
que $QB = RC$. Muestra que $AC$ pasa por el punto medio del segmento $QR$.