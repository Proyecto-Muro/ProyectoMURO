Sea $ABCD$ un paralelogramo con ángulo obtuso en $A$. Sea $P$ un punto sobre el segmento $BD$ de manera que la circunferencia con centro en $P$ y que pasa por $A$, corte a la recta $AD$ en $A$ y $Y$, y corte a la recta $AB$ en $A$ y $X$. La recta $AP$ intersecta a $BC$ en $Q$ y a $CD$ en $R$, respectivamente. Muestra que $\angle XPY = \angle XQY + \angle XRY$.