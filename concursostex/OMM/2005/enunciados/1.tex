Sea $O$ el centro de la circunferencia circunscrita al triángulo $ABC$, y sea $P$ un punto cualquiera del segmento $BC$ ($P\neq B$ y $P\neq C$). Supón que la circunferencia circunscrita al triángulo $BPO$ corta al segmento $AB$ en $R$ ($R\neq A$ y $R\neq B$) y que la circunferencia circunscrita al triángulo $COP$ corta al segmento $CA$ en el punto $Q$ ($Q\neq C$ y $Q\neq A$). Muestra que el triángulo $PQR$ que es semejante al triángulo $ABC$ y su ortocentro es $O$. A demás, muestra que las circunferencias circunscritas a los triángulos $BPO$, $COP$, $PQR$ son todas del mismo tamaño.