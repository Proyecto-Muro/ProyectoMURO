Se tienen 25 focos distribuidos de la siguiente manera: los primeros 24 se disponen en una
circunferencia colocando un foco en cada uno de los vértices de un 24-ágono regular, y el
foco restante se coloca en el centro de dicha circunferencia.
Se permite aplicar cualquiera de las siguientes dos operaciones:

1) Tomar dos vértices sobre la circunferencia tales que hay una cantidad impar de
vértices en los arcos que definen, y cambiar el estado de los focos de estos dos
vértices, así como del foco del centro.

2)Tomar tres vértices sobre la circunferencia que formen un triángulo equilátero, y
cambiar el estado de los focos en estos tres vértices, así como del foco del centro.

Muestra que partiendo de cualquier configuración inicial de focos encendidos y apagados,
siempre es posible llegar a una configuración en la que todos los focos están encendidos.