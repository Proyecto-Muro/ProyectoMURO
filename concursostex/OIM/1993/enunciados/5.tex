 Sean $P$ y $Q$ dos puntos distintos del plano. Denotemos por $m(PQ)$ a la mediatriz del segmento $PQ$. Sea $S$ un subconjunto finito del plano, con más de un elemento que satisface las siguientes propiedades:
  \begin{enumerate} 
     \item  Si $P$ y $Q$ son puntos distintos de $S$, entonces $m(PQ)$ interseca a $S$.
     \item Si $P_1Q_1$, $P_2Q_2$ y $P_3Q_3$ son tres segmentos diferentes cuyos extremos son puntos de $S$, entonces ningún punto de $S$ pertenece simultáneamente a las tres rectas $m(P_1Q_1)$, $m(P_2Q_2)$ y $m(P_3Q_3)$.
 \end{enumerate} 
Determine el número de puntos que puede tener $S$. 