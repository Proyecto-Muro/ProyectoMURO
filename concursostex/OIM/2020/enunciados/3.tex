Sea $n \geq 2$ un entero. Una sucesión $\alpha = (a_1, a_2, \cdots , a_n)$ de $n$ números enteros se dice <em>limeña</em> si
\[ \textrm{mcd}\{ a_i - a_j \mid a_i > a_j, 1 \leq i,j \leq n \} = 1 \]
Una <em>operación</em> consiste en escoger dos elementos $a_k$ y $a_l$ de una sucesión, con $k \neq l$, y reemplazar $a_l$ por $a_l' = 2a_k - a_l$. \newline 
Demuestre que, dada una colección de $2^n-1$ sucesiones limeñas, cada una formada por $n$ números enteros, existen dos de ellas, digamos $\beta$ y $\gamma$, tales que es posible transformar $\beta$ en $\gamma$ mediante un número finito de operaciones. \newline 
<em>Aclaración:</em> Si todos los elementos de una sucesión son iguales, entonces esa sucesión no es limeña.
