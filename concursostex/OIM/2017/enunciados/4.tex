Sean $ABC$ un triángulo acutángulo con $AC > AB$ y $O$ su circuncentro. Sea $D$ un punto en el segmento $BC$ tal que $O$ está en el interior del triángulo $ADC$ y $\angle DAO + \angle ADB = \angle ADC$. Llamamos $P$ y $Q$ a los circuncentros de los triángulos $ABD$ y $ACD$, respectivamente, y $M$ al punto de intersección de las rectas $BP$ y $CQ$. Demostrar que las rectas $AM$, $PQ$ y $BC$ son concurrentes.
<em>Nota.</em> El circuncentro de un triángulo es el centro de la circunferencia que pasa por los tres vértices del triángulo.
