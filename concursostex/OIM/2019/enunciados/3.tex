Sea $\Gamma$ el circuncírculo del triángulo $ABC$. La paralela a $AC$ que pasa por $B$ corta a $\Gamma$ en $D$ ($D \neq B$) y la paralela a $AB$ que pasa por $C$ corta a $\Gamma$ en $E$ ($E \neq C$). Las rectas $AB$ y $CD$ se cortan en $P$, y las rectas $AC$ y $BE$ se cortan en $Q$. Sea $M$ el punto medio de $DE$. La recta $AM$ corta a $\Gamma$ en $Y$ ($Y \neq A$) y a la recta $PQ$ en $J$. La recta $PQ$ corta al circuncírculo del triángulo $BCJ$ en $Z$ ($Z \neq J$). Si las rectas $BQ$ y $CP$ se cortan en $X$, demuestra que $X$ pertenece a la recta $YZ$. \newline 
Nota: El circuncírculo de un triángulo es la circunferencia que pasa por los vértices del triángulo.
