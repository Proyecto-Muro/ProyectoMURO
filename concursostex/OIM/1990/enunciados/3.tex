Sea $f(x)=(x+b)^2-c$, con $b$ y $c$ enteros.
 \begin{itemize}  
 \item  Si $p$ es un primo tal que $p$ divide a $c$ pero $p^2$ no divide a $c$, muestra que para toda $n$, $p^2$ no divide a $f(n)$.
 \item   Sea $q$ un primo impar que no divide a $c$. Si $q$ divide a $f(n)$ para algún entero positivo $n$, muestra que para todo entero positivo $r$, existe un entero positivo $n'$ tal que $q^r$ divide a $f(n')$.
 \end{itemize} 