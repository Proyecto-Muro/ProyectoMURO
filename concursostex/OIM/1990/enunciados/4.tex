Sea $\mathcal C_1$ una circunferencia con diámetro $AB$, y sea $t$ la tangente a $\mathcal C_1$ por $B$. Sea $M$ otro punto en $\mathcal C_1$ distinto a $A$ y $B$. Se construye la circunferencia $\mathcal C_2$, tangente a $\mathcal C_1$ en $M$ y tangente a $t$. 
 \begin{itemize} 
 \item  Encuentra el punto de tangencia $P$ de $\mathcal C_2$ con $t$, y encuentra el lugar geométrico de el centro de $\mathcal C_2$ mientras $M$ varía sobre $\mathcal C_1$.
 \item  Muestra que existe una circunferencia ortogonal a todas las circunferencias $\mathcal C_2$.
 \end{itemize}  
Nota: dos circunferencias son ortogonales si las tangentes a uno de sus puntos de intersección son perpendiculares.