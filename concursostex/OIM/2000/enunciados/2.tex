Sean $S_1$ y $S_2$ dos circunferencias, de centros $O_1$ y $O_2$ respectivamente, secantes en $M$ y $N$. La recta $t$ es tangente común a $S_1$ y $S_2$, más cercana a $M$. Los puntos $A$ y $B$ son los respectivos puntos de contacto de $t$ con $S_1$ y $S_2$, $C$ el punto diametralmente opuesto a $B$ y $D$ el punto de intersección de la recta $O_1 O_2$ con la recta perpendicular a la recta $AM$ trazada por $B$. Demostrar que $M$, $D$ y $C$ están alineados.
