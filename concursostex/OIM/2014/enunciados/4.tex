Se tienen $N$ monedas, de las cuales $N-1$ son auténticas de igual peso y una es falsa, de peso diferente a las demás. El objetivo es, utilizando exclusivamente una balanza de dos platos, hallar la moneda falsa y determinar si es más pesada o más liviana que las auténticas. Cada vez que se pueda deducir que una o varias monedas son auténticas, entonces todas esas monedas se separan inmediatamente y no se pueden usar en las siguientes pesadas. Determine todos los $N$ para los que se puede lograr con certeza el objetivo. (Se pueden hacer tantas pesadas como se desee.)
