Dado un entero positivo $n$, en un plano se consideran $2n$ puntos alineados $A_1, A_2, \cdots, A_{2n}$. Cada punto se colorea de azul o de rojo mediante el siguiente procedimiento: \newline 
En el plano se trazan $n$ circunferencias con diámetros de extremos $A_i$ y $A_j$, disjuntas dos a dos. Cada $A_k$, $1 \leq k \leq 2n$, pertenece exactamente a una circunferencia. Se colorean los puntos de modo que los dos puntos de una misma circunferencia lleven el mismo color. \newline 
Determine cuántas coloraciones distintas de los $2n$ puntos se pueden obtener al variar las $n$ circunferencias y la distribución de los colores.
