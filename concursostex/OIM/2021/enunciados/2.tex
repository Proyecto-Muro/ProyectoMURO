Considere un triángulo acutángulo $ABC$, con $AC > AB$, y sea $\Gamma$ su circuncírculo. Sean $E$ y $F$ los puntos medios de los lados $AC$ y $AB$, respectivamente. El circuncírculo del triángulo $CEF$ y $\Gamma$ se cortan en $X$ y $C$, con $X \neq C$. La recta $BX$ y la tangente a $\Gamma$ por $A$ se cortan en $Y$. Sea $P$ el punto en el segmento $AB$ tal que$YP = YA$, con $P \neq A$, y sea $Q$ el punto donde se cortan $AB$ y la paralela a $BC$ que pasa por $Y$. Demuestre que $F$ es el punto medio de $PQ$. \newline 
<b>Nota:</b> El circuncírculo de un triángulo es la circunferencia que pasa por sus tres vértices.
