En un tablero cuadriculado de $19 \times 19$, una ficha llamada \emph{dragón} da saltos de la siguiente forma: se desplaza $4$ casillas en una dirección paralela a uno de los lados del tablero y $1$ casilla en dirección perpendicular a la anterior. \newline 
Se sabe que, con este tipo de saltos, el dragón puede moverse de cualquier casilla a cualquier otra. \newline 
La \emph{distancia dragoniana} entre dos casillas es el menor número de saltos que el dragón debe dar para moverse de una casilla a otra. \newline 
Sea $C$ una casilla situada en una esquina del tablero y sea $V$ la casilla vecina a $C$ que le toca en un único punto. \newline 
Demostrar que existe alguna casilla $X$ del tablero tal que la distancia dragoniana de $C$ a $X$ es mayor que la distancia dragoniana de $C$ a $V$.
