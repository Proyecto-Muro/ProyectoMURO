En un triángulo acutángulo $ABC$ sean $AE$ y $BF$ dos alturas, y sea $H$ el ortocentro. La recta simétrica de $AE$ respecto de la bisectriz (interior) del ángulo en $A$ y la recta simétrica de $BF$ respecto de la bisectriz (interior) del ángulo en $B$ se intersecan en un punto $O$. Las rectas $AE$ y $AO$ cortan por segunda vez a la circunferencia circunscrita al triángulo $ABC$ en los puntos $M$ y $N$, respectivamente. \newline 
Sean: $P$, la intersección de $BC$ con $HN$; $R$, la intersección de $BC$ con $OM$; y $S$ la intersección de $HR$ con $OP$. \newline 
Demostrar que $AHSO$ es un paralelogramo.
