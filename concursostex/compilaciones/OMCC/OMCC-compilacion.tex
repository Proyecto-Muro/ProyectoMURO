\documentclass[11pt]{scrartcl}
\usepackage{muro}
\usepackage[inline]{asymptote}

\usepackage{pgffor} % para voltear lista de años

%%fancyhdr
\pagestyle{fancy}
\lfoot{\sffamily\bfseries\hyperlink{tabla}{Índice}}
\rfoot{\thepage}
\cfoot{}
\lhead{\color{jampmDate}{\sffamily\today}}
\rhead{\rmfamily{Proyecto MURO}}
%name of the handout, create variable with that string package on the macros page in wikibook
\chead{{\sffamily\large Compilación OMCC}}

%HyperSETUP
\hypersetup{
colorlinks= true,
urlcolor= cyan,
linkcolor= jampmLinks,
citecolor=red,
pdftitle={Compilación OMCC},
bookmarks = true,
pdfpagemode = FullScreen,
}


\begin{document}

\title{Compilación OMCC}
\author{Proyecto MURO}
\date{\today}
\maketitle
\noindent{\bfseries\Large\sffamily Introducción}

\noindent 
Somos el proyecto MURO, y nuestro objetivo es conformar un Movimiento Unificador de Recursos Olímpicos.

Esta es una compilación con todos los problemas de la Olimpiada de Matemáticas de Centroamérica y el Caribe, (en un futuro agregaremos pistas y soluciones). Esperamos la disfrutes! y no dudes en compartirnos cualquier comentario o crítica constructiva. 

Nos puedes contactar por aquí: \href{https://proyectomuro.com}{proyectomuro.com}. 

\hypertarget{tabla}{\tableofcontents}
\vfill
\eject

\section{Problemas}

\let\mylist\empty
\foreach\x in {1999,2000,...,2021} {
  \ifx\mylist\empty
    \xdef\mylist{\x}%
  \else
    \xdef\mylist{\x,\mylist}%
  \fi
}
\foreach \j in \mylist {%
    \addsubsec{OMCC \j}
    \foreach \i in {1,2,...,6} {
        \begin{problema}
            \enunciado{\j}{OMCC}{\i}
        \end{problema}
    }
    \eject
}




\end{document}
