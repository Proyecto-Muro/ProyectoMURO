Sea $n \ge 2$ un entero. Una $n$-tupla $(a_1, a_2, \dots , a_n)$ de enteros positivos no necesariamente distintos es costosa si existe un entero positivo $k$ tal que

\[(a_1+a_2)(a_2+a_3)\dots(a_{n-1}+a_n)(a_n+a_1)=2^{2k-1}.\]

a) Encuentra todos los enteros $n \geq 2$ para los cuales existe una $n$-tupla costosa.
b) Demuestra que para todo entero positivo impar $m$ existe un entero $n \geq 2$ tal que $m$ pertenece a
una $n$-tupla costosa.