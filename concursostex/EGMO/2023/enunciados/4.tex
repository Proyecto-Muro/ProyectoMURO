El caracolito Turbo está sobre un punto de una circunferencia de longitud 1. Sea $c_1,c_2,c_3,\dots$ una sucesión infinita de números reales positivos. Turbo se arrastra sucesivamente distancias $c_1,c_2,c_3,\dots$ sobre la circunferencia, eligiendo cada vez el sentido horario o antihorario. Por ejemplo, si la sucesión $c_1,c_2,c_3,\dots$ es $0.4,0.6,0.3\dots$, entonces Turbo podría haber elegido arrastrarse como en la figura.

Determine la mayor constante $C\gt 0$ con la propiedad siguiente: para toda sucesión de números reales positivos $c_1,c_2,c_3,\dots$ tales que $c_1\lt C$ para todo $i$, Turbo puede asegurar (tras haber estudiado la sucesión) que hay un punto de la circunferencia por el que siempre puede evitar pasar.