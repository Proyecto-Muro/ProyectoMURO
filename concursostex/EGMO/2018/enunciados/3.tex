Las $n$ concursantes de cierta EGMO se llaman $C_1, \dots , C_n$. Después de la competencia,
se ponen en fila fuera del restaurante de acuerdo a las siguientes reglas:
El Jurado escoge el orden inicial de las concursantes en la fila.
Cada minuto, el Jurado escoge un entero $i$ con $1 \le i \le n$.
– Si la concursante $C_i$ tiene al menos otras $i$ concursantes delante de ella, le paga una moneda
al Jurado y se mueve exactamente $i$ posiciones adelante en la fila.
– Si la concursante $C_i$ tiene menos de $i$ concursantes delante de ella, el restaurante se abre y
el proceso termina.

(a) Muestra que el proceso no puede continuar indefinidamente, sin importar las elecciones del
Jurado.
(b) Determine para cada $n$ el máximo número de monedas que el Jurado puede recolectar escogiendo
el orden inicial y la secuencia de movimientos astutamente.