Un conjunto $A$ de enteros se denomina de "suma completa" si $A \subseteq A + A$, es decir, cada elemento $a \in A$ es la suma de algún par de elementos $b,c \in A$ (no necesariamente diferentes). Se dice que un conjunto $A$ de enteros es "libre de suma cero" si $0$ es el único entero que no puede expresarse como la suma de los elementos de un subconjunto finito no vacío de $A$.
¿Existe un conjunto de suma completa libre de suma cero?
