Denotamos por $d(m)$ el número de divisores positivos de un entero positivo $m$, y por
$\omega(m)$ el número de primos distintos que dividen a $m$. Sea k un entero positivo. Demuestra que hay
una infinidad de enteros positivos $n$ tales que $\omega(n) = k$ y $d(n)$ no divide a $d(a^2 + b^2)$ para todos los enteros positivos $a$ y $b$ tales que $a + b = n$.