
OIM 2021 

OIM 2021 problem 1:  Let $P = \{p_1,p_2,\ldots, p_{10}\}$ be a set of $10$ different prime numbers and let $A$ be the set of all the integers greater than $1$ so that their prime decomposition only contains primes of $P$. The elements of $A$ are colored in such a way that:
\begin{itemize}
  \item each element of $P$ has a different color,
  \item if $m,n \in A$, then $mn$ is the same color of $m$ or $n$,
  \item for any pair of different colors $\mathcal{R}$ and $\mathcal{S}$, there are no $j,k,m,n\in A$ (not necessarily distinct from one another), with $j,k$ colored $\mathcal{R}$ and $m,n$ colored $\mathcal{S}$, so that $j$ is a divisor of $m$ and $n$ is a divisor of $k$, simultaneously.
\end{itemize}
Prove that there exists a prime of $P$ so that all its multiples in $A$ are the same color. 
OIM 2021 problem 2:  Consider an acute-angled triangle $ABC$, with $AC>AB$, and let $\Gamma$ be its circumcircle. Let $E$ and $F$ be the midpoints of the sides $AC$ and $AB$, respectively. The circumcircle of the triangle $CEF$ and $\Gamma$ meet at $X$ and $C$, with $X\neq C$. The line $BX$ and the tangent to $\Gamma$ through $A$ meet at $Y$. Let $P$ be the point on segment $AB$ so that $YP = YA$, with $P\neq A$, and let $Q$ be the point where $AB$ and the parallel to $BC$ through $Y$ meet each other. Show that $F$ is the midpoint of $PQ$. 
OIM 2021 problem 3:  Let $a_1,a_2,a_3, \ldots$ be a sequence of positive integers and let $b_1,b_2,b_3,\ldots$ be the sequence of real numbers given by
\[ b_n = \dfrac{a_1a_2\cdots a_n}{a_1+a_2+\cdots + a_n},\ \mbox{for}\ n\geq 1 \]
Show that, if there exists at least one term among every million consecutive terms of the sequence $b_1,b_2,b_3,\ldots$ that is an integer, then there exists some $k$ such that $b_k > 2021^{2021}$. 
OIM 2021 problem 4:  Let $a,b,c,x,y,z$ be real numbers such that
\[ a^2+x^2=b^2+y^2=c^2+z^2=(a+b)^2+(x+y)^2=(b+c)^2+(y+z)^2=(c+a)^2+(z+x)^2 \]
Show that $a^2+b^2+c^2=x^2+y^2+z^2$. 
OIM 2021 problem 5:  For a finite set $C$ of integer numbers, we define $S(C)$ as the sum of the elements of $C$. Find two non-empty sets $A$ and $B$ whose intersection is empty, whose union is the set $\{1,2,\ldots, 2021\}$ and such that the product $S(A)S(B)$ is a perfect square. 
OIM 2021 problem 6:  Consider a $n$-sided regular polygon, $n \geq 4$, and let $V$ be a subset of $r$ vertices of the polygon. Show that if $r(r-3) \geq n$, then there exist at least two congruent triangles whose vertices belong to $V$. 

OIM 2020 

OIM 2020 problem 1:  Let $ABC$ be an acute scalene triangle such that $AB <AC$. The midpoints of sides $AB$ and $AC$ are $M$ and $N$, respectively. Let $P$ and $Q$ be points on the line $MN$ such that $\angle CBP = \angle ACB$ and $\angle QCB = \angle CBA$. The circumscribed circle of triangle $ABP$ intersects line $AC$ at $D$ ($D\ne A$) and the circumscribed circle of triangle $AQC$ intersects line $AB$ at $E$ ($E \ne A$). Show that lines $BC, DP,$ and $EQ$ are concurrent. \\\\
Nicolás De la Hoz, Colombia 
OIM 2020 problem 2:  Let $T_n$ denotes the least natural such that
\[ n\mid 1+2+3+\cdots +T_n=\sum_{i=1}^{T_n} i \]
Find all naturals $m$ such that $m\ge T_m$. \\\\
\textit{Proposed by Nicolás De la Hoz } 
OIM 2020 problem 3:  Let $n\ge 2$ be an integer. A sequence $\alpha = (a_1, a_2,..., a_n)$ of $n$ integers is called \textit{Lima } if $\gcd \{a_i - a_j \text{ such that } a_i> a_j \text{ and } 1\le i, j\le n\} = 1$, that is, if the greatest common divisor of all the differences $a_i - a_j$  with $a_i> a_j$ is $1$. One operation consists of choosing two elements $a_k$ and $a_{\ell}$ from a sequence, with $k\ne \ell $ , and replacing $a_{\ell}$  by $a'_{\ell}  = 2a_k - a_{\ell}$ . \\
Show that, given a collection of $2^n - 1$ Lima sequences, each one formed by $n$ integers, there are two of them, say $\beta$ and $\gamma$, such that it is possible to transform  $\beta$ into  $\gamma$ through a finite number of operations. \\\\
Notes. \\
The sequences $(1,2,2,7)$ and $(2,7,2,1)$ have the same elements but are different. \\
If all the elements of a sequence are equal, then that sequence is not Lima. 
OIM 2020 problem 4:  Show that there exists a set $\mathcal{C}$ of $2020$ distinct, positive integers that satisfies simultaneously the following properties: \\
$\bullet$ When one computes the greatest common divisor of each pair of elements of $\mathcal{C}$, one gets a list of numbers that are all distinct. \\
$\bullet$ When one computes the least common multiple of each pair of elements of $\mathcal{C}$, one gets a list of numbers that are all distinct. 
OIM 2020 problem 5:  Determine all functions $f: \mathbb{R} \rightarrow \mathbb{R}$ such that
\[ f(xf(x-y))+yf(x)=x+y+f(x^2), \]
for all real numbers $x$ and $y.$ 
OIM 2020 problem 6:  Let $ABC$ be an acute, scalene triangle. Let $H$ be the orthocenter and $O$ be the circumcenter of triangle $ABC$, and let $P$ be a point interior to the segment $HO.$ The circle with center $P$ and radius $PA$ intersects the lines $AB$ and $AC$ again at $R$ and $S$, respectively. Denote by $Q$ the symmetric point of $P$ with respect to the perpendicular bisector of $BC$. Prove that points $P$, $Q$, $R$ and $S$ lie on the same circle. 

OIM 2019 

OIM 2019 problem 1:  For each positive integer $n$, let $s(n)$ be the sum of the squares of the digits of $n$. For example, $s(15)=1^2+5^2=26$. Determine all integers $n\geq 1$ such that $s(n)=n$. 
OIM 2019 problem 2:  Determine all polynomials $P(x)$ with degree $n\geq 1$ and integer coefficients so that for every real number $x$ the following condition is satisfied
\[ P(x)=(x-P(0))(x-P(1))(x-P(2))\cdots (x-P(n-1)) \] 
OIM 2019 problem 3:  Let $\Gamma$ be the circumcircle of triangle $ABC$. The line parallel to $AC$ passing through $B$ meets $\Gamma$ at $D$ ($D\neq B$), and the line parallel to $AB$ passing through $C$ intersects $\Gamma$ to $E$ ($E\neq C$). Lines $AB$ and $CD$ meet at $P$, and lines $AC$ and $BE$ meet at $Q$. Let $M$ be the midpoint of $DE$. Line $AM$ meets $\Gamma$ at $Y$ ($Y\neq A$) and line $PQ$ at $J$. Line $PQ$ intersects the circumcircle of triangle $BCJ$ at $Z$ ($Z\neq J$). If lines $BQ$ and $CP$ meet each other at $X$, show that $X$ lies on the line $YZ$. 
OIM 2019 problem 4:  Let $ABCD$ be a trapezoid with $AB\parallel CD$ and inscribed in a circumference $\Gamma$. Let $P$ and $Q$ be two points on segment $AB$ ($A$, $P$, $Q$, $B$ appear in that order and are distinct) such that $AP=QB$. Let $E$ and $F$ be the second intersection points of lines $CP$ and $CQ$ with $\Gamma$, respectively. Lines $AB$ and $EF$ intersect at $G$. Prove that line $DG$ is tangent to $\Gamma$. 
OIM 2019 problem 5:  Don Miguel places a token in one of the $(n+1)^2$ vertices determined by an $n \times n$ board. A \textit{move} consists of moving the token from the vertex on which it is placed to an adjacent vertex which is at most $\sqrt2$ away, as long as it stays on the board. A \textit{path} is a sequence of moves such that the token was in each one of the $(n+1)^2$ vertices exactly once. What is the maximum number of diagonal moves (those of length $\sqrt2$) that a path can have in total? 
OIM 2019 problem 6:  Let $a_1, a_2, \dots, a_{2019}$ be positive integers and $P$ a polynomial with integer coefficients such that, for every positive integer $n$,
\[ P(n) \text{ divides  } a_1^n+a_2^n+\dots+a_{2019}^n. \]
Prove that $P$ is a constant polynomial. 

OIM 2018 

OIM 2018 problem 1:  For each integer $n \geq 2$, find all integer solutions of the following system of equations:
\begin{align*}
x_1 \&= (x_2 + x_3 + x_4 + \dots + x_n)^{2018} \\
x_2 \&= (x_1 + x_3 + x_4 + \dots + x_n)^{2018} \\
\& \vdotswithin{ = } \\
x_n \&= (x_1 + x_2 + x_3 + \dots + x_{n - 1})^{2018}
\end{align*} 
OIM 2018 problem 2:  Let $ABC$ be a triangle such that $\angle BAC = 90^{\circ}$ and $AB = AC$. Let $M$ be the midpoint of $BC$. A point $D \neq A$ is chosen on the semicircle with diameter $BC$ that contains $A$. The circumcircle of triangle $DAM$ cuts lines $DB$ and $DC$ at $E$ and $F$ respectively. Show that $BE = CF$. 
OIM 2018 problem 3:  In a plane we have $n$ lines, no two of which are parallel or perpendicular, and no three of which are concurrent. A cartesian system of coordinates is chosen for the plane with one of the lines as the $x$-axis. A point $P$ is located at the origin of the coordinate system and starts moving along the positive $x$-axis with constant velocity. Whenever $P$ reaches the intersection of two lines, it continues along the line it just reached in the direction that increases its $x$-coordinate. Show that it is possible to choose the system of coordinates in such a way that $P$ visits points from all $n$ lines. 
OIM 2018 problem 4:  A set $X$ of positive integers is said to be \textit{iberic} if $X$ is a subset of $\{2, 3, \dots, 2018\}$, and whenever $m, n$ are both in $X$, $\gcd(m, n)$ is also in $X$. An iberic set is said to be \textit{olympic} if it is not properly contained in any other iberic set. Find all olympic iberic sets that contain the number $33$. 
OIM 2018 problem 5:  Let $n$ be a positive integer. For a permutation $a_1, a_2, \dots, a_n$ of the numbers $1, 2, \dots, n$ we define
\[ b_k = \min_{1 \leq i \leq k} a_i + \max_{1 \leq j \leq k} a_j \]
We say that the permutation $a_1, a_2, \dots, a_n$ is \textit{guadiana} if the sequence $b_1, b_2, \dots, b_n$ does not contain two consecutive equal terms. How many guadiana permutations exist? 
OIM 2018 problem 6:  Let $ABC$ be an acute triangle with $AC > AB > BC$. The perpendicular bisectors of $AC$ and $AB$ cut line $BC$ at $D$ and $E$ respectively. Let $P$ and $Q$ be points on lines $AC$ and $AB$ respectively, both different from $A$, such that $AB = BP$ and $AC = CQ$, and let $K$ be the intersection of lines $EP$ and $DQ$. Let $M$ be the midpoint of $BC$. Show that $\angle DKA = \angle EKM$. 

OIM 2017 

OIM 2017 problem 1:  For every positive integer $n$ let $S(n)$ be the sum of its digits. We say $n$ has a property $P$ if all terms in the infinite secuence $n, S(n), S(S(n)),...$ are even numbers, and we say $n$ has a property $I$ if all terms in this secuence are odd. Show that for, $1 \le n \le 2017$ there are more $n$ that have property $I$ than those who have $P$. 
OIM 2017 problem 2:  Let $ABC$ be an acute angled triangle and $\Gamma$ its circumcircle. Led $D$ be a point on segment $BC$, different from $B$ and $C$, and let $M$ be the midpoint of $AD$. The line perpendicular to $AB$ that passes through $D$ intersects $AB$ in $E$ and $\Gamma$ in $F$, with point $D$ between $E$ and $F$. Lines $FC$ and $EM$ intersect at point $X$. If $\angle DAE = \angle AFE$, show that line $AX$ is tangent to $\Gamma$. 
OIM 2017 problem 3:  Consider the configurations of integers \\
$a_{1,1}$ \\
$a_{2,1} \quad a_{2,2}$ \\
$a_{3,1} \quad a_{3,2} \quad a_{3,3}$ \\
$\dots \quad \dots \quad \dots$ \\
$a_{2017,1} \quad a_{2017,2} \quad a_{2017,3} \quad \dots \quad a_{2017,2017}$ \\
Where $a_{i,j} = a_{i+1,j} + a_{i+1,j+1}$ for all $i,j$ such that $1 \leq j \leq i \leq 2016$. \\
Determine the maximum amount of odd integers that such configuration can contain. 
OIM 2017 problem 4:  Let $ABC$ be an acute triangle with $AC > AB$ and $O$ its circumcenter. Let $D$ be a point on segment $BC$ such that $O$ lies inside triangle $ADC$ and $\angle DAO + \angle ADB = \angle ADC$. Let $P$ and $Q$ be the circumcenters of triangles $ABD$ and $ACD$ respectively, and let $M$ be the intersection of lines $BP$ and $CQ$. Show that lines $AM, PQ$ and $BC$ are concurrent. \\\\
\textit{Pablo Jaén, Panama} 
OIM 2017 problem 5:  Given a positive integer $n$, all of its positive integer divisors are written on a board. Two players $A$ and $B$ play the following game: \\\\
Each turn, each player colors one of these divisors either red or blue. They may choose whichever color they wish, but they may only color numbers that have not been colored before. The game ends once every number has been colored. $A$ wins if the product of all of the red numbers is a perfect square, or if no number has been colored red, $B$ wins otherwise. If $A$ goes first, determine who has a winning strategy for each $n$. 
OIM 2017 problem 6:  Let $n > 2$ be an even positive integer and let $a_1 < a_2 < \dots < a_n$ be real numbers such that $a_{k + 1} - a_k \leq 1$ for each $1 \leq k \leq n - 1$. Let $A$ be the set of ordered pairs $(i, j)$ with $1 \leq i < j \leq n$ such that $j - i$ is even, and let $B$ the set of ordered pairs $(i, j)$ with $1 \leq i < j \leq n$ such that $j - i$ is odd. Show that
\[ \prod_{(i, j) \in A} (a_j - a_i) > \prod_{(i, j) \in B} (a_j - a_i) \] 

OIM 2016 

OIM 2016 problem 1:  Find all prime numbers $p,q,r,k$ such that $pq+qr+rp = 12k+1$ 
OIM 2016 problem 2:  Find all positive real numbers $(x,y,z)$ such that:
\[ x = \frac{1}{y^2+y-1} \]
\[ y = \frac{1}{z^2+z-1} \]
\[ z = \frac{1}{x^2+x-1} \] 
OIM 2016 problem 3:  Let $ABC$ be an acute triangle and $\Gamma$ its circumcircle. The lines tangent to $\Gamma$ through $B$ and $C$ meet at $P$. Let $M$ be a point on the arc $AC$ that does not contain $B$ such that $M \neq A$ and $M \neq C$, and $K$ be the point where the lines $BC$ and $AM$ meet. Let $R$ be the point symmetrical to $P$ with respect to the line $AM$ and $Q$ the point of intersection of lines $RA$ and $PM$. Let $J$ be the midpoint of $BC$ and $L$ be the intersection point of the line $PJ$ and the line through $A$ parallel to $PR$. Prove that $L, J, A, Q,$ and $K$ all lie on a circle. 
OIM 2016 problem 4:  Determine the maximum number of bishops that we can place in a $8 \times 8$ chessboard such that there are not two bishops in the same cell, and each bishop is threatened by at most one bishop. \\\\
Note: A bishop threatens another one, if both are placed in different cells, in the same diagonal. A board has as diagonals the $2$ main diagonals and the ones parallel to those ones. 
OIM 2016 problem 5:  The circumferences $C_1$ and $C_2$ cut each other at different points $A$ and $K$. The common tangent to $C_1$ and $C_2$ nearer to $K$ touches $C_1$ at $B$ and $C_2$ at $C$. Let $P$ be the foot of the perpendicular from $B$ to $AC$, and let $Q$ be the foot of the perpendicular from $C$ to $AB$. If $E$ and $F$ are the symmetric points of $K$ with respect  to the lines $PQ$ and $BC$, respectively, prove that $A, E$ and $F$ are collinear. 
OIM 2016 problem 6:  Let $k$ be a positive integer and $a_1, a_2,$ $\cdot \cdot \cdot$ $, a_k$ digits. Prove that there exists a positive integer $n$ such that the last $2k$ digits of $2^n$ are, in the following order, $a_1, a_2,$ $\cdot \cdot \cdot$ $, a_k , b_1, b_2,$ $\cdot \cdot \cdot$ $, b_k$, for certain digits $b_1, b_2,$ $\cdot \cdot \cdot$ $, b_k$ 

OIM 2015 

OIM 2015 problem 1:  The number $125$ can be written as a sum of some pairwise coprime integers larger than $1$. Determine the largest number of terms that the sum may have. 
OIM 2015 problem 2:  A line $r$ contains the points $A$, $B$, $C$, $D$ in that order. Let $P$ be a point not in $r$ such that $\angle{APB} = \angle{CPD}$. Prove that the angle bisector of $\angle{APD}$ intersects the line $r$ at a point $G$ such that: \\\\
$\frac{1}{GA} + \frac{1}{GC} = \frac{1}{GB} + \frac{1}{GD}$ 
OIM 2015 problem 3:  Let $\alpha$ and $\beta$ be the roots of $x^2 - qx + 1$, where $q$ is a rational number larger than $2$. Let $s_1 = \alpha + \beta$, $t_1 = 1$, and for all integers $n \geq 2$: \\\\
$s_n = \alpha^n + \beta^n$ \\\\
$t_n = s_{n-1} + 2s_{n-2} + \cdot \cdot \cdot + (n - 1)s_1 + n$ \\\\
Prove that, for all odd integers $n$, $t_n$ is the square of a rational number. 
OIM 2015 problem 4:  Let $ABC$ be an acute triangle and let $D$ be the foot of the perpendicular from $A$ to side $BC$. Let $P$ be a point on segment $AD$. Lines $BP$ and $CP$ intersect sides $AC$ and $AB$ at $E$ and $F$, respectively. Let $J$ and $K$ be the feet of the peroendiculars from $E$ and $F$ to $AD$, respectively. Show that \\\\
$\frac{FK}{KD}=\frac{EJ}{JD}$. 
OIM 2015 problem 5:  Find all pairs of integers $(a,b)$ such that \\\\
$(b^2+7(a-b))^2=a^3b$. 
OIM 2015 problem 6:  Beto plays the following game with his computer: initially the computer randomly picks $30$ integers from $1$ to $2015$, and Beto writes them on a chalkboard (there may be repeated numbers). On each turn, Beto chooses a positive integer $k$ and some if the numbers written on the chalkboard, and subtracts $k$ from each of the chosen numbers, with the condition that the resulting numbers remain non-negative. The objective of the game is to reduce all $30$ numbers to $0$, in which case the game ends. Find the minimal number $n$ such that, regardless of which numbers the computer chooses, Beto can end the game in at most $n$ turns. 

OIM 2014 

OIM 2014 problem 1:  For each positive integer $n$, let $s(n)$ be the sum of the digits of $n$. Find the smallest positive integer $k$ such that
\[ s(k) = s(2k) = s(3k) = \cdots = s(2013k) = s(2014k). \] 
OIM 2014 problem 2:  Find all polynomials $P(x)$ with real coefficients such that $P(2014) = 1$ and, for some integer $c$: \\\\
$xP(x-c) = (x - 2014)P(x)$ 
OIM 2014 problem 3:  $2014$ points are placed on a circumference. On each of the segments with end points on two of the $2014$ points is written a non-negative real number. For any convex polygon with vertices on some of the $2014$ points, the sum of the numbers written on their sides is less or equal than $1$. Find the maximum possible value for the sum of all the written numbers. 
OIM 2014 problem 4:  $N$ coins are placed on a table, $N - 1$ are genuine and have the same weight, and one is fake, with a different weight. Using a two pan balance, the goal is to determine with certainty the fake coin, and whether it is lighter or heavier than a genuine coin. Whenever one can deduce that one or more coins are genuine, they will be inmediately discarded and may no longer be used in subsequent weighings. Determine all $N$ for which the goal is achievable. (There are no limits regarding how many times one may use the balance). \\\\
Note: the only difference between genuine and fake coins is their weight; otherwise, they are identical. 
OIM 2014 problem 5:  Let $ABC$ be an acute triangle and $H$ its orthocenter. Let $D$ be the intersection of the altitude from $A$ to $BC$. Let $M$ and $N$ be the midpoints of $BH$ and $CH$, respectively. Let the lines $DM$ and $DN$ intersect $AB$ and $AC$ at points $X$ and $Y$ respectively. If $P$ is the intersection of $XY$ with $BH$ and $Q$ the intersection of $XY$ with $CH$, show that $H, P, D, Q$ lie on a circumference. 
OIM 2014 problem 6:  Given a set $X$ and a function $f: X \rightarrow X$, for each $x \in X$ we define $f^1(x)=f(x)$ and, for each $j \ge 1$, $f^{j+1}(x)=f(f^j(x))$. We say that $a \in X$ is a fixed point of $f$ if $f(a)=a$. For each $x \in \mathbb{R}$, let $\pi (x)$ be the quantity of positive primes lesser or equal to $x$. \\\\
Given an positive integer $n$, we say that $f: \{1,2, \dots, n\} \rightarrow \{1,2, \dots, n\}$ is \textit{catracha} if $f^{f(k)}(k)=k$, for every $k=1, 2, \dots n$. Prove that:
\begin{enumerate}[(a)]
  \item If $f$ is catracha, $f$ has at least $\pi (n) -\pi (\sqrt{n}) +1$ fixed points.
  \item If $n \ge 36$, there exists a catracha function $f$ with exactly $ \pi (n) -\pi (\sqrt{n}) + 1$ fixed points.
\end{enumerate} 

OIM 2013 

OIM 2013 problem 1:  A set $S$ of positive integers is said to be \textit{channeler} if for any three distinct numbers $a,b,c \in S$, we have $a\mid bc$, $b\mid ca$, $c\mid ab$.
\begin{enumerate}[a)]
  \item Prove that for any finite set of positive integers $ \{ c_1, c_2, \ldots, c_n \} $ there exist infinitely many positive integers $k$, such that the set $ \{ kc_1, kc_2, \ldots, kc_n \} $ is a channeler set.
  \item Prove that for any integer $n \ge 3$ there is a channeler set who has exactly $n$ elements, and such that no integer greater than $1$ divides all of its elements.
\end{enumerate} 
OIM 2013 problem 2:  Let $X$ and $Y$ be the diameter's extremes of a circunference $\Gamma$ and $N$ be the midpoint of one of the arcs $XY$ of $\Gamma$. Let $A$ and $B$ be two points on the segment $XY$. The lines $NA$ and $NB$ cuts $\Gamma$ again in $C$ and $D$, respectively.  The tangents to $\Gamma$ at $C$ and at $D$ meets in $P$. Let $M$ the the intersection point between $XY$ and $NP$. Prove that $M$ is the midpoint of the segment $AB$. 
OIM 2013 problem 3:  Let $A = \{1,...,n\}$ with $n \textgreater 5$. Prove that one can find $B$ a finite set of positive integers such that $A$ is a subset of $B$ and \\\\
$\displaystyle\sum_{x \in B} x^2 = \displaystyle\prod_{x \in B} x$ 
OIM 2013 problem 4:  Let $\Gamma$ be a circunference and $O$ its center. $AE$ is a diameter of $\Gamma$ and $B$ the midpoint of one of the arcs $AE$ of $\Gamma$. The point $D \ne E$ in on the segment $OE$. The point $C$ is such that the quadrilateral $ABCD$ is a parallelogram, with $AB$ parallel to $CD$ and $BC$ parallel to $AD$. The lines $EB$ and $CD$ meets at point $F$. The line $OF$ cuts the minor arc $EB$ of $\Gamma$ at $I$. \\\\
Prove that the line $EI$ is the angle bissector of $\angle BEC$. 
OIM 2013 problem 5:  Let $A$ and $B$ be two sets such that $A \cup B$ is the set of the positive integers, and $A \cap B$ is the empty set. It is known that if two positive integers have a prime larger than $2013$ as their difference, then one of them is in $A$ and the other is in $B$. Find all the possibilities for the sets $A$ and $B$. 
OIM 2013 problem 6:  A \textit{beautiful configuration} of points is a set of $n$ colored points, such that if a triangle with vertices in the set has an angle of at least $120$ degrees, then exactly 2 of its vertices are colored with the same color. Determine the maximum possible value of $n$. 

OIM 2012 

OIM 2012 problem 1:  Let $ABCD$ be a rectangle. Construct equilateral triangles $BCX$ and $DCY$, in such a way that both of these triangles share some of their interior points with some interior points of the rectangle. Line $AX$ intersects line $CD$ on $P$, and line $AY$ intersects line $BC$ on $Q$. Prove that triangle $APQ$ is equilateral. 
OIM 2012 problem 2:  A positive integer is called \textit{shiny} if it can be written as the sum of two not necessarily distinct integers $a$ and $b$ which have the same sum of their digits. For instance, $2012$ is \textit{shiny}, because $2012 = 2005 + 7$, and both $2005$ and $7$ have the same sum of their digits. Find all positive integers which are \textbf{not} \textit{shiny} (the dark integers). 
OIM 2012 problem 3:  Let $n$ to be a positive integer. Given a set $\{ a_1, a_2, \ldots, a_n \} $ of integers, where $a_i \in \{ 0, 1, 2, 3, \ldots, 2^n -1 \},$ $\forall i$, we associate to each of its subsets the sum of its elements; particularly, the empty subset has sum of its elements equal to $0$. If all of these sums have different remainders when divided by $2^n$, we say that $\{ a_1, a_2, \ldots, a_n \} $ is \textit{$n$-complete}. \\\\
For each $n$, find the number of \textit{$n$-complete} sets. 
OIM 2012 problem 4:  Let $a,b,c,d$ be integers such that the number $a-b+c-d$ is odd and it divides the number $a^2-b^2+c^2-d^2$. Show that, for every positive integer $n$, $a-b+c-d$ divides $a^n-b^n+c^n-d^n$. 
OIM 2012 problem 5:  Let $ABC$ be a triangle, $P$ and $Q$ the intersections of the parallel line to $BC$ that passes through $A$ with the external angle bisectors of angles $B$ and $C$, respectively. The perpendicular to $BP$ at $P$ and the perpendicular to $CQ$ at $Q$ meet at $R$. Let $I$ be the incenter of $ABC$. Show that $AI  = AR$. 
OIM 2012 problem 6:  Show that, for every positive integer $n$, there exist $n$ consecutive positive integers such that none is divisible by the sum of its digits. \\\\
(Alternative Formulation: Call a number good if it's not divisible by the sum of its digits. Show that for every positive integer $n$ there are $n$ consecutive good numbers.) 

OIM 2011 

OIM 2011 problem 1:  The number $2$ is written on the board. Ana and Bruno play alternately. Ana begins. Each one, in their turn, replaces the number written by the one obtained by applying exactly one of these operations: multiply the number by $2$, multiply the number by $3$ or add $1$ to the number. The first player to get a number greater than or equal to $2011$ wins. Find which of the two players has a winning strategy and describe it. 
OIM 2011 problem 2:  Find all positive integers $n$ for which exist three nonzero integers $x, y, z$ such that $x+y+z=0$ and:
\[ \frac{1}{x}+\frac{1}{y}+\frac{1}{z}=\frac{1}{n} \] 
OIM 2011 problem 3:  Let $ABC$ be a triangle and $X,Y,Z$ be the tangency points of its inscribed circle with the sides $BC, CA, AB$, respectively. Suppose that $C_1, C_2, C_3$ are circle with chords $YZ, ZX, XY$, respectively, such that $C_1$ and $C_2$ intersect on the line $CZ$ and that $C_1$ and $C_3$ intersect on the line $BY$. Suppose that $C_1$ intersects the chords $XY$ and $ZX$ at $J$ and $M$, respectively; that $C_2$ intersects the chords $YZ$ and $XY$ at $L$ and $I$, respectively; and that $C_3$ intersects the chords $YZ$ and $ZX$ at $K$ and $N$, respectively. Show that $I, J, K, L, M, N$ lie on the same circle. 
OIM 2011 problem 4:  Let $ABC$ be an acute-angled triangle, with $AC \neq BC$ and let $O$ be its circumcenter. Let $P$ and $Q$ be points such that $BOAP$ and $COPQ$ are parallelograms. Show that $Q$ is the orthocenter of $ABC$. 
OIM 2011 problem 5:  Let $x_1,\ldots ,x_n$ be positive real numbers. Show that there exist $a_1,\ldots ,a_n\in\{-1,1\}$ such that:
\[ a_1x_1^2+a_2x_2^2+\ldots +a_nx_n^2\ge (a_1x_1+a_2x_2+\ldots + a_n x_n)^2 \] 
OIM 2011 problem 6:  Let $k$ and $n$ be positive integers, with $k \geq 2$. In a straight line there are $kn$ stones of $k$ colours, such that there are $n$ stones of each colour. A \textit{step} consists of exchanging the position of two adjacent stones. Find the smallest positive integer $m$ such that it is always possible to achieve, with at most $m$ steps, that the $n$ stones are together, if:
\begin{enumerate}[a)]
  \item $n$ is even.
  \item $n$ is odd and $k=3$
\end{enumerate} 

OIM 2010 

OIM 2010 problem 1:  There are ten coins a line, which are indistinguishable. It is known that two of them are false and have consecutive positions on the line. For each set of positions, you may ask how many false coins it contains. Is it possible to identify the false coins by making only two of those questions, without knowing the answer to the first question before making the second? 
OIM 2010 problem 2:  Determine if there are positive integers $a, b$ such that all terms of the sequence defined by
\[ x_1= 2010,x_2= 2011\\  x_{n+2}= x_n+ x_{n+1}+a\sqrt{x_nx_{n+1}+b}\quad (n\ge 1) \]
are integers. 
OIM 2010 problem 3:  The circle $ \Gamma $ is inscribed to the scalene triangle  $ABC$. $ \Gamma $ is tangent to the sides $BC, CA$ and $AB$ at $D, E$ and $F$ respectively. The line $EF$ intersects the line $BC$ at $G$. The circle of diameter $GD$ intersects $ \Gamma $  in $R$ ($ R\neq D $). Let $P$, $Q$ ($ P\neq R , Q\neq R $)  be the intersections of $ \Gamma $  with $BR$ and $CR$, respectively. The lines $BQ$ and $CP$ intersects at $X$. The circumcircle of $CDE$ meets $QR$ at $M$, and the circumcircle of $BDF$ meet $PR$ at $N$. Prove that $PM$, $QN$ and $RX$ are concurrent. \\\\
\textit{Author: Arnoldo Aguilar, El Salvador} 
OIM 2010 problem 4:  The arithmetic, geometric and harmonic mean of two distinct positive integers are different numbers. Find the smallest possible value for the arithmetic mean. 
OIM 2010 problem 5:  Let $ABCD$ be a cyclic quadrilateral whose diagonals $AC$ and $BD$ are perpendicular. Let $O$ be the circumcenter of $ABCD$, $K$ the intersection of the diagonals,  $ L\neq O $ the intersection of the circles circumscribed to $OAC$ and $OBD$, and $G$ the intersection of the diagonals of the quadrilateral whose vertices are the midpoints of the sides of $ABCD$. Prove that $O, K, L$ and $G$ are collinear 
OIM 2010 problem 6:  Around a circular table sit $12$ people, and on the table there are $28$ vases. Two people can see each other, if and only if there is no vase lined with them. Prove that there are at least two people who can be seen. 

OIM 2009 

OIM 2009 problem 1:  Given a positive integer $ n\geq 2$, consider a set of $ n$ islands arranged in a circle. Between every two neigboring islands two bridges are built as shown in the figure. \\\\
Starting at the island $ X_1$, in how many ways one can one can cross the $ 2n$ bridges so that no bridge is used more than once? 
OIM 2009 problem 2:  Define the succession $ a_n$, $ n>0$ as $ n+m$, where $ m$ is the largest integer such that $ 2^{2^m}\leq n2^n$. Find all numbers that are not in the succession. 
OIM 2009 problem 3:  Let $ C_1$ and $ C_2$ be two congruent circles centered at $ O_1$ and $ O_2$, which intersect at $ A$ and $ B$. Take a point $ P$ on the arc $ AB$ of $ C_2$ which is contained in $ C_1$. $ AP$ meets $ C_1$ at $ C$, $ CB$ meets $ C_2$ at $ D$ and the bisector of $ \angle CAD$ intersects $ C_1$ and $ C_2$ at $ E$ and $ L$, respectively. Let $ F$ be the symmetric point of $ D$ with respect to the midpoint of $ PE$. Prove that there exists a point $ X$ satisfying $ \angle XFL = \angle XDC = 30^\circ$ and $ CX = O_1O_2$. \\
\textit{
Author: Arnoldo Aguilar (El Salvador)} 
OIM 2009 problem 4:  Given a triangle $ ABC$ of incenter $ I$, let $ P$ be the intersection of the external bisector of angle $ A$ and the circumcircle of $ ABC$, and $ J$ the second intersection of $ PI$ and the circumcircle of $ ABC$. Show that the circumcircles of triangles $ JIB$ and $ JIC$ are respectively tangent to $ IC$ and $ IB$. 
OIM 2009 problem 5:  Consider the sequence $ \{a_n\}_{n\geq1}$ defined as follows: $ a_1 = 1$, $ a_{2k} = 1 + a_k$ and $ a_{2k + 1} = \frac {1}{a_{2k}}$ for every $ k\geq 1$. Prove that every positive rational number appears on the sequence $ \{a_n\}$ exactly once. 
OIM 2009 problem 6:  Six thousand points are marked on a circle, and they are colored using 10 colors in such a way that within every group of 100 consecutive points all the colors are used. Determine the least positive integer $ k$ with the following property: In every coloring satisfying the condition above, it is possible to find a group of $ k$ consecutive points in which all the colors are used. 

OIM 2008 

OIM 2008 problem 1:  The integers from 1 to $ 2008^2$ are written on each square of a $ 2008 \times 2008$ board. For every row and column the difference between the maximum and minimum numbers is computed. Let $ S$ be the sum of these 4016 numbers. Find the greatest possible value of $ S$. 
OIM 2008 problem 2:  Given a triangle $ ABC$, let $ r$ be the external bisector of $ \angle ABC$. $ P$ and $ Q$ are the feet of the perpendiculars from $ A$ and $ C$ to $ r$. If $ CP \cap BA = M$ and $ AQ \cap BC=N$, show that $ MN$, $ r$ and $ AC$ concur. 
OIM 2008 problem 3:  Let $ P(x) = x^3 + mx + n$ be an integer polynomial satisfying that if $ P(x) - P(y)$ is divisible by 107, then $ x - y$ is divisible by 107 as well, where $ x$ and $ y$ are integers. Prove that 107 divides $ m$. 
OIM 2008 problem 4:  Prove that the equation
\[ x^{2008}+ 2008!= 21^y \]
doesn't have solutions in integers. 
OIM 2008 problem 5:  Let $ ABC$ a triangle and $ X$, $ Y$ and $ Z$ points at the segments $ BC$, $ AC$ and $ AB$, respectively.Let $ A'$, $ B'$ and $ C'$ the circuncenters of triangles $ AZY$,$ BXZ$,$ CYX$, respectively.Prove that $ 4(A'B'C')\geq(ABC)$ with equality if and only if $ AA'$, $ BB'$ and $ CC'$ are concurrents. \\\\
Note: $ (XYZ)$ denotes the area of $ XYZ$ 
OIM 2008 problem 6:  \textit{Biribol} is a game played between two teams of 4 people each (teams are not fixed). Find all the possible values of $ n$ for which it is possible to arrange a tournament with $ n$ players in such a way that every couple of people plays a match in opposite teams exactly once. 

OIM 2007 

OIM 2007 problem 1:  Given an integer $ m$, define the sequence $ \left\{a_n\right\}$ as follows:
\[ a_1=\frac{m}{2},\ a_{n+1}=a_n\left\lceil a_n\right\rceil,\textnormal{ if }n\geq 1 \]
Find all values of $ m$ for which $ a_{2007}$ is the first integer appearing in the sequence. \\\\
Note: For a real number $ x$, $ \left\lceil x\right\rceil$ is defined as the smallest integer greater or equal to $ x$. For example, $ \left\lceil\pi\right\rceil=4$, $ \left\lceil 2007\right\rceil=2007$. 
OIM 2007 problem 2:  Let $ ABC$ be a triangle with incenter $ I$ and let $ \Gamma$ be a circle centered at $ I$, whose radius is greater than the inradius and does not pass through any vertex. Let $ X_1$ be the intersection point of $ \Gamma$ and line $ AB$, closer to $ B$; $ X_2$, $ X_3$ the points of intersection of $ \Gamma$ and line $ BC$, with $ X_2$ closer to $ B$; and let $ X_4$ be the point of intersection of $ \Gamma$ with line $ CA$ closer to $ C$. Let $ K$ be the intersection point of lines $ X_1X_2$ and $ X_3X_4$. Prove that $ AK$ bisects segment $ X_2X_3$. 
OIM 2007 problem 3:  Two teams, $ A$ and $ B$, fight for a territory limited by a circumference. \\\\
$ A$ has $ n$ blue flags and $ B$ has $ n$ white flags ($ n\geq 2$, fixed). They play alternatively and $ A$ begins the game. Each team, in its turn, places one of his flags in a point of the circumference that has not been used in a previous play. Each flag, once placed, cannot be moved. \\\\
Once all $ 2n$ flags have been placed, territory is divided between the two teams. A point of the territory belongs to $ A$ if the closest flag to it is blue, and it belongs to $ B$ if the closest flag to it is white. If the closest blue flag to a point is at the same distance than the closest white flag to that point, the point is neutral (not from $ A$ nor from $ B$). A team wins the game is their points cover a greater area that that covered by the points of the other team. There is a draw if both cover equal areas. \\
Prove that, for every $ n$, team $ B$ has a winning strategy. 
OIM 2007 problem 4:  In a $ 19\times 19$ board, a piece called \textit{dragon} moves as follows: It travels by four squares (either horizontally or vertically) and then it moves one square more in a direction perpendicular to its previous direction. It is known that, moving so, a dragon can reach every square of the board. \\\\
The \textit{draconian distance} between two squares is defined as the least number of moves a dragon needs to move from one square to the other. \\\\
Let $ C$ be a corner square, and $ V$ the square neighbor of $ C$ that has only a point in common with $ C$. Show that there exists a square $ X$ of the board, such that the draconian distance between $ C$ and $ X$ is greater than the draconian distance between $ C$ and $ V$. 
OIM 2007 problem 5:  Let's say a positive integer $ n$ is \textit{atresvido} if the set of its divisors (including 1 and $ n$) can be split in in 3 subsets such that the sum of the elements of each is the same. Determine the least number of divisors an atresvido number can have. 
OIM 2007 problem 6:  Let $ \mathcal{F}$ be a family of hexagons $ H$ satisfying the following properties:
\begin{enumerate}[i)]
  \item $ H$ has parallel opposite sides.
  \item Any 3 vertices of $ H$ can be covered with a strip of width 1.
\end{enumerate}
Determine the least $ \ell\in\mathbb{R}$ such that every hexagon belonging to $ \mathcal{F}$ can be covered with a strip of width $ \ell$. \\\\
Note: A strip is the area bounded by two parallel lines separated by a distance $ \ell$. The lines belong to the strip, too. 

OIM 2006 

OIM 2006 problem 1:  In a scalene triangle $ABC$ with $\angle A = 90^\circ,$ the tangent line at $A$ to its circumcircle meets line $BC$ at $M$ and the incircle touches $AC$ at $S$ and $AB$ at $R.$ \\
The lines $RS$ and $BC$ intersect at $N,$ while the lines $AM$ and $SR$ intersect at $U.$ \\
Prove that the triangle $UMN$ is isosceles. 
OIM 2006 problem 2:  For n real numbers $a_1,\, a_2,\, \ldots\, , a_n,$ let $d$ denote the difference between the greatest and smallest of them and $S = \sum_{i<j}\left |a_i-a_j \right|.$ Prove that
\[ (n-1)d\le S\le\frac{n^2}{4}d \]
and find when each equality holds. 
OIM 2006 problem 3:  The numbers $1,\, 2,\, \ldots\, , n^2$ are written in the squares of an $n \times n$ board in some order. Initially there is a token on the square labelled with $n^2.$ In each step, the token can be moved to any adjacent square (by side). At the beginning, the token is moved to the square labelled with the number $1$ along a path with the minimum number of steps. Then it is moved to the square labelled with $2,$ then to square $3,$ etc, always taking the shortest path, until it returns to the initial square. If the total trip takes $N$ steps, find the smallest and greatest possible values of $N.$ 
OIM 2006 problem 4:  Find all pairs $(a,\, b)$ of positive integers such that $2a-1$ and $2b+1$ are coprime and $a+b$ divides $4ab+1.$ 
OIM 2006 problem 5:  The sides $AD$ and $CD$ of a tangent quadrilateral $ABCD$ touch the incircle $\varphi$ at $P$ and $Q,$ respectively. If $M$ is the midpoint of the chord $XY$ determined by $\varphi$ on the diagonal $BD,$ prove that $\angle AMP = \angle CMQ.$ 
OIM 2006 problem 6:  Consider a regular $n$-gon with $n$ odd. Given two adjacent vertices $A_1$ and $A_2,$ define the sequence $(A_k)$ of vertices of the $n$-gon as follows: For $k\ge 3,\, A_k$ is the vertex lying on the perpendicular bisector of $A_{k-2}A_{k-1}.$ Find all $n$ for which each vertex of the $n$-gon occurs in this sequence. 

OIM 2005 

OIM 2005 problem 1:  Determine all triples of real numbers $(x,y,z))$ such that
\begin{eqnarray*} xyz \&=\& 8 \\ x^2y + y^2z + z^2x \&=\& 73 \\ x(y-z)^2 + y(z-x)^2 + z(x-y)^2 \&=\& 98 . \end{eqnarray*} 
OIM 2005 problem 2:  A flea jumps in a straight numbered line. It jumps first from point $0$ to point $1$. Afterwards, if its last jump was from $A$ to $B$, then the next jump is from $B$ to one of the points $B + (B - A) - 1$, $B + (B - A)$, $B + (B-A) + 1$. \\\\
Prove that if the flea arrived twice at the point $n$, $n$ positive integer, then it performed at least $\lceil 2\sqrt n\rceil$ jumps. 
OIM 2005 problem 3:  Let $p > 3$ be a prime. Prove that if
\[ \sum_{i=1 }^{p-1}{1\over i^p} = {n\over m}, \]
with $\gdc(n,m) = 1$, then $p^3$ divides $n$. 
OIM 2005 problem 4:  Denote by $a \bmod b$ the remainder of the euclidean division of $a$ by $b$. Determine all pairs of positive integers $(a,p)$ such that $p$ is prime and
\[ a \bmod p + a\bmod 2p + a\bmod 3p + a\bmod 4p = a + p. \] 
OIM 2005 problem 5:  Let $O$ be the circumcenter of acutangle triangle $ABC$ and let $A_1$ be some point in the smallest arc $BC$ of the circumcircle of $ABC$. Let $A_2$ and $A_3$ points on sides $AB$ and $AC$, respectively, such that $\angle BA_1A_2 = \angle OAC$ and $\angle CA_1A_3 = \angle OAB$. \\\\
Prove that the line $A_2A_3$ passes through the orthocenter of $ABC$. 
OIM 2005 problem 6:  Let $n$ be a fixed positive integer. The points $A_1$, $A_2$, $\ldots$, $A_{2n}$ are on a straight line. Color each point blue or red according to the following procedure: draw $n$ pairwise disjoint circumferences, each with diameter $A_iA_j$ for some $i \neq j$ and such that every point $A_k$ belongs to exactly one circumference. Points in the same circumference must be of the same color. \\\\
Determine the number of ways of coloring these $2n$ points when we vary the $n$ circumferences and the distribution of the colors. 


OIM 2004 

OIM 2004 problem 1:  It is given a 1001*1001 board divided in 1*1 squares. We want to amrk m squares in such a way that: \\
1: if 2 squares are adjacent then one of them is marked. \\
2: if 6 squares lie consecutively in a row or column then two adjacent squares from them are marked. \\\\
Find the minimun number of squares we most mark. 
OIM 2004 problem 2:  In the plane are given a circle with center $ O$ and radius $ r$ and a point $ A$ outside the circle. For any point $ M$ on the circle, let $ N$ be the diametrically opposite point. Find the locus of the circumcenter of triangle $ AMN$ when $ M$ describes the circle. 
OIM 2004 problem 3:  Let $ n$ and $ k$ be positive integers such as either $ n$ is odd or both $ n$ and $ k$ are even. Prove that exists integers $ a$ and $ b$ such as $ GCD(a,n) = GCD(b,n) = 1$ and $ k = a + b$ 
OIM 2004 problem 4:  Determine all pairs $ (a,b)$ of positive integers, each integer having two decimal digits, such that $ 100a+b$ and $ 201a+b$ are both perfect squares. 
OIM 2004 problem 5:  Given a scalene triangle $ ABC$. Let $ A'$, $ B'$, $ C'$ be the points where the internal bisectors of the angles $ CAB$, $ ABC$, $ BCA$ meet the sides $ BC$, $ CA$, $ AB$, respectively. Let the line $ BC$ meet the perpendicular bisector of $ AA'$ at $ A''$. Let the line $ CA$ meet the perpendicular bisector of $ BB'$ at $ B'$. Let the line $ AB$ meet the perpendicular bisector of $ CC'$ at $ C''$. Prove that $ A''$, $ B''$ and $ C''$ are collinear. 
OIM 2004 problem 6:  Given a set $ \mathcal{H}$ of points in the plane, $ P$ is called an ``intersection point of $ \mathcal{H}$" if distinct points $ A,B,C,D$ exist in $ \mathcal{H}$ such that lines $ AB$ and $ CD$ are distinct and intersect in $ P$. \\
Given a finite set $ \mathcal{A}_0$ of points in the plane, a sequence of sets is defined as follows: for any $ j\geq0$, $ \mathcal{A}_{j+1}$ is the union of $ \mathcal{A}_j$ and the intersection points of $ \mathcal{A}_j$. \\
Prove that, if the union of all the sets in the sequence is finite, then $ \mathcal{A}_i=\mathcal{A}_1$ for any $ i\geq1$. 

OIM 2003 

OIM 2003 problem 1:  $(a)$There are two sequences of numbers, with $2003$ consecutive integers each, and a table of $2$ rows and $2003$ columns \\
$
\begin{array}{|c|c|c|c|c|c|} \hline\ \ \&\ \&\ \&\cdots\cdots\&\ \&\ \\ \hline \ \&\ \&\ \&\cdots\cdots\&\ \&\ \\ \hline \end{array}
$ \\
Is it always possible to arrange the numbers in the first sequence in the first row and the second sequence in the second row, such that the sequence obtained of the $2003$ column-wise sums form a new sequence of $2003$ consecutive integers? \\
$(b)$ What if $2003$ is replaced with $2004$? 
OIM 2003 problem 2:  Let $C$ and $D$ be two points on the semicricle with diameter $AB$ such that $B$ and $C$ are on distinct sides of the line $AD$. Denote by $M$, $N$ and $P$ the midpoints of $AC$, $BD$ and $CD$ respectively. Let $O_A$ and $O_B$ the circumcentres of the triangles $ACP$ and $BDP$. Show that the lines $O_AO_B$ and $MN$ are parallel. 
OIM 2003 problem 3:  Pablo copied from the blackboard the problem:
\begin{itemize}
Consider all the sequences of\end{itemize}
$2004$ real numbers $(x_0,x_1,x_2,\dots, x_{2003})$ such that: $x_0=1, 0\le x_1\le 2x_0,0\le x_2\le 2x_1\ldots ,0\le x_{2003}\le 2x_{2002}$. From all these sequences, determine the sequence which minimizes $S=\cdots$ \\\\
As Pablo was copying the expression, it was erased from the board. The only thing that he could remember was that $S$ was of the form $S=\pm x_1\pm x_2\pm\cdots\pm x_{2002}+x_{2003}$. Show that, even when Pablo does not have the complete statement, he can determine the solution of the problem. 
OIM 2003 problem 4:  Let $M=\{1,2,\dots,49\}$ be the set of the first $49$ positive integers. Determine the maximum integer $k$ such that the set $M$ has a subset of $k$ elements such that there is no $6$ consecutive integers in such subset. For this value of $k$, find the number of subsets of $M$ with $k$ elements with the given property. 
OIM 2003 problem 5:  In a square $ABCD$, let $P$ and $Q$ be points on the sides $BC$ and $CD$ respectively, different from its endpoints, such that $BP=CQ$. Consider points $X$ and $Y$ such that $X\neq Y$, in the segments $AP$ and $AQ$ respectively. Show that, for every $X$ and $Y$ chosen, there exists a triangle whose sides have lengths $BX$, $XY$ and $DY$. 
OIM 2003 problem 6:  The sequences $(a_n),(b_n)$ are defined by $a_0=1,b_0=4$ and for $n\ge 0$
\[ a_{n+1}=a_n^{2001}+b_n,\ \ b_{n+1}=b_n^{2001}+a_n \]
Show that $2003$ is not divisor of any of the terms in these two sequences. 

OIM 2002 

OIM 2002 problem 1:  The integer numbers from $1$ to $2002$ are written in a blackboard in increasing order $1,2,\ldots, 2001,2002$. After that, somebody erases the numbers in the $ (3k+1)-th$ places i.e. $(1,4,7,\dots)$. After that, the same person erases the numbers in the $(3k+1)-th$ positions of the new list (in this case, $2,5,9,\ldots$). This process is repeated until one number remains. What is this number? 
OIM 2002 problem 2:  Given any set of $9$ points in the plane such that there is no $3$ of them collinear, show that for each point $P$ of the set, the number of triangles with its vertices on the other $8$ points and that contain $P$ on its interior is even. 
OIM 2002 problem 3:  Let $P$ be a point in the interior of the equilateral triangle $\triangle ABC$ such that $\sphericalangle{APC}=120^\circ$. Let $M$ be the intersection of $CP$ with $AB$, and $N$ the intersection of $AP$ and $BC$. Find the locus of the circumcentre of the triangle $MBN$ as $P$ varies. 
OIM 2002 problem 4:  In a triangle $\triangle{ABC}$ with all its sides of different length, $D$ is on the side $AC$, such that $BD$ is the angle bisector of $\sphericalangle{ABC}$. Let $E$ and $F$, respectively, be the feet of the perpendicular drawn from $A$ and $C$ to the line $BD$ and let $M$ be the point on $BC$ such that $DM$ is perpendicular to $BC$. Show that $\sphericalangle{EMD}=\sphericalangle{DMF}$. 
OIM 2002 problem 5:  The sequence of real numbers $a_1,a_2,\dots$ is defined as follows: $a_1=56$ and $a_{n+1}=a_n-\frac{1}{a_n}$ for $n\ge 1$. Show that there is an integer $1\leq{k}\leq2002$ such that $a_k<0$. 
OIM 2002 problem 6:  A policeman is trying to catch a robber on a board of $2001\times2001$ squares. They play alternately, and the player whose trun it is moves to a space in one of the following directions: $\downarrow,\rightarrow,\nwarrow$. \\\\
If the policeman is on the square in the bottom-right corner, he can go directly to the square in the upper-left corner (the robber can not do this). Initially the policeman is in the central square, and the robber is in the upper-left adjacent square. Show that: \\\\
$a)$ The robber may move at least $10000$ times before the being captured. \\
$b)$ The policeman has an strategy such that he will eventually catch the robber. \\\\
Note: The policeman can catch the robber if he reaches the square where the robber is, but not if the robber enters the square occupied by the policeman. 

OIM 2001 

OIM 2001 problem 1:  We say that a natural number $n$ is \textit{charrua} if it satisfy simultaneously the following conditions: \\
- Every digit of $n$ is greater than 1. \\
- Every time that four digits of $n$ are multiplied, it is obtained a divisor of $n$ \\\\
Show that every natural number $k$ there exists a \textit{charrua} number with more than $k$ digits. 
OIM 2001 problem 2:  The incircle of the triangle $\triangle{ABC}$ has center at $O$ and it is tangent to the sides $BC$, $AC$ and $AB$ at the points $X$, $Y$ and $Z$, respectively. The lines $BO$ and $CO$ intersect the line $YZ$ at the points $P$ and $Q$, respectively. \\\\
Show that if the segments $XP$ and $XQ$ has the same length, then the triangle $\triangle ABC$ is isosceles. 
OIM 2001 problem 3:  Let $S$ be a set of $n$ elements and $S_1,\ S_2,\dots,\ S_k$ are subsets of $S$ ($k\geq2$), such that every one of them has at least $r$ elements. \\\\
Show that there exists $i$ and $j$, with $1\leq{i}<j\leq{k}$, such that the number of common elements of $S_i$ and $S_j$ is greater or equal to: $r-\frac{nk}{4(k-1)}$ 
OIM 2001 problem 4:  Find the maximum number of increasing arithmetic progressions that can have a finite sequence of real numbers $a_1<a_2<\cdots<a_n$ of $n\ge 3$ real numbers. 
OIM 2001 problem 5:  In a board of $2000\times2001$ squares with integer coordinates $(x,y)$, $0\leq{x}\leq1999$ and $0\leq{y}\leq2000$. A ship in the table moves in the following way: before a move, the ship is in position $(x,y)$ and has a velocity of $(h,v)$ where $x,y,h,v$ are integers. The ship chooses new velocity $(h^\prime,v^\prime)$ such that $h^\prime-h,v^\prime-v\in\{-1,0,1\}$. The new position of the ship will be $(x^\prime,y^\prime)$ where $x^\prime$ is the remainder of the division of $x+h^\prime$ by $2000$ and $y^\prime$ is the remainder of the division of $y+v^\prime$ by $2001$. \\\\
There are two ships on the board: The Martian ship and the Human trying to capture it. Initially each ship is in a different square and has velocity $(0,0)$. The Human is the first to move; thereafter they continue moving alternatively. \\\\
Is there a strategy for the Human to capture the Martian, independent of the initial positions and the Martian’s moves? \\\\
\textit{Note}: The Human catches the Martian ship by reaching the same position as the Martian ship after the same move. 
OIM 2001 problem 6:  Show that it is impossible to cover a unit square with five equal squares with side $s<\frac{1}{2}$. 

OIM 2000 

OIM 2000 problem 1:  A regular polygon of $ n$ sides ($ n\geq3$) has its vertex numbered from 1 to $ n$. One draws all the diagonals of the polygon. Show that if $ n$ is odd, it is possible to assign to each side and to each diagonal an integer number between 1 and $ n$, such that the next two conditions are simultaneously satisfied:
\begin{enumerate}[(a)]
  \item The number assigned to each side or diagonal is different to the number assigned to any of the vertices that is endpoint of it.
  \item For each vertex, all the sides and diagonals that have it as an endpoint, have different number assigned.
\end{enumerate} 
OIM 2000 problem 2:  Let $S_1$ and $S_2$ be two circumferences, with centers $O_1$ and $O_2$ respectively, and secants on $M$ and $N$. The line \\
$t$ is the common tangent to  $S_1$ and $S_2$ closer to $M$. The points $A$ and $B$ are the intersection points of $t$ with $S_1$ and $S_2$, $C$ is the point such that $BC$ is a diameter of $S_2$, and $D$ the intersection point of the line $O_1O_2$ with the perpendicular line to $AM$ through $B$. Show that $M$, $D$ and $C$ are collinear. 
OIM 2000 problem 3:  Find all the solutions of the equation
\[ \left(x+1\right)^y-x^z=1 \]
For $x,y,z$ integers greater than 1. 
OIM 2000 problem 4:  From an infinite arithmetic progression $ 1,a_1,a_2,\dots$ of real numbers some terms are deleted, obtaining an infinite geometric progression $ 1,b_1,b_2,\dots$ whose ratio is $ q$. Find all the possible values of $ q$. 
OIM 2000 problem 5:  There are a buch of 2000 stones. Two players play alternatively, following the next rules: \\\\
($a$)On each turn, the player can take 1, 2, 3, 4 or 5 stones \textbf{of} the bunch. \\
($b$) On each turn, the player has forbidden to take the exact same amount of stones that the other player took just before of him in the last play. \\\\
The loser is the player who can't make a valid play. Determine which player has winning strategy and give such strategy. 
OIM 2000 problem 6:  A convex hexagon is called \textit{pretty} if it has four diagonals of length 1, such that their endpoints are all the vertex of the hexagon. \\
($a$) Given any real number $k$ with $0<k<1$ find a \textit{pretty} hexagon with area equal to $k$ \\
($b$) Show that the area of any \textit{pretty} hexagon is less than 1. 

OIM 1999 

OIM 1999 problem 1:  Find all the positive integers less than 1000 such that the cube of the sum of its digits  is equal to the square of such integer. 
OIM 1999 problem 2:  Given two circle $M$ and $N$, we say that $M$ bisects $N$ if they intersect in two points and the common chord is a diameter of $N$. Consider two fixed non-concentric circles $C_1$ and $C_2$.
\begin{enumerate}[a)]
  \item Show that there exists infinitely many circles $B$ such that $B$ bisects both $C_1$ and $C_2$.
  \item Find the locus of the centres of such circles $B$.
\end{enumerate} 
OIM 1999 problem 3:  Let $P_1,P_2,\dots,P_n$ be $n$ distinct points over a line in the plane ($n\geq2$). Consider all the circumferences with diameters $P_iP_j$ ($1\leq{i,j}\leq{n}$) and they are painted with $k$ given colors. Lets call this configuration a ($n,k$)-cloud. \\\\
For each positive integer $k$, find all the positive integers $n$ such that every possible ($n,k$)-cloud has two mutually exterior tangent circumferences of the same color. 
OIM 1999 problem 4:  Let $B$ be an integer greater than 10 such that everyone of its digits belongs to the set $\{1,3,7,9\}$. Show that $B$ has a \textbf{prime divisor}  greater than or equal to 11. 
OIM 1999 problem 5:  An acute triangle $\triangle{ABC}$ is inscribed in a circle with centre $O$. The altitudes of the triangle are $AD,BE$ and $CF$. The line $EF$ cut the circumference on $P$ and $Q$.
\begin{enumerate}[a)]
  \item Show that $OA$ is perpendicular to $PQ$.
  \item If $M$ is the midpoint of $BC$, show that $AP^2=2AD\cdot{OM}$.
\end{enumerate} 
OIM 1999 problem 6:  Let $A$ and $B$ points in the plane and $C$ a point in the perpendiclar bisector of $AB$. It is constructed a sequence of points $C_1,C_2,\dots, C_n,\dots$ in the following way: $C_1=C$ and for $n\geq1$, if $C_n$ does not belongs to $AB$, then $C_{n+1}$ is the circumcentre of the triangle $\triangle{ABC_n}$. \\\\
Find all the points $C$ such that the sequence $C_1,C_2,\dots$ is defined for all $n$ and turns eventually periodic. \\\\
Note: A sequence $C_1,C_2, \dots$ is called eventually periodic if there exist positive integers $k$ and $p$ such that $C_{n+p}=c_n$ for all $n\geq{k}$. 

OIM 1998 

OIM 1998 problem 1:  Given 98 points in a circle. Mary and Joseph play alternatively in the next way: \\\\
- Each one draw a segment joining two points that have not been  joined before. \\\\
The game ends when the 98 points have been used as end points of a segments at least once.  The winner is the person that draw the last segment. If Joseph starts the game, who can assure that is going to win the game. 
OIM 1998 problem 2:  The circumference inscribed on the triangle $ABC$ is tangent to the sides $BC$, $CA$ and $AB$ on the points $D$, $E$ and $F$, respectively. $AD$ intersect the circumference on the point $Q$. Show that the line $EQ$ meet the segment $AF$ at its midpoint if and only if $AC=BC$. 
OIM 1998 problem 3:  Find the minimum natural number $n$ with the following property: between any collection of $n$ distinct natural numbers in the set $\{1,2, \dots,999\}$ it is possible to choose four different $a,\ b,\ c,\ d$ such that: $a + 2b + 3c = d$. 
OIM 1998 problem 4:  There are representants from $n$ different countries sit around a circular table ($n\geq2$), in such way that if two representants are from the same country, then, their neighbors to the right are not from the same country. Find, for every $n$, the maximal number of people that can be sit around the table. 
OIM 1998 problem 5:  Find the maximal possible value of $n$ such that there exist points $P_1,P_2,P_3,\ldots,P_n$ in the plane and real numbers $r_1,r_2,\ldots,r_n$ such that the distance between any two different points $P_i$ and $P_j$ is $r_i+r_j$. 
OIM 1998 problem 6:  Let $\lambda$ the positive root of the equation $t^2-1998t-1=0$.  It is defined the sequence $x_0,x_1,x_2,\ldots,x_n,\ldots$ by $x_0=1,\ x_{n+1}=\lfloor\lambda{x_n}\rfloor\mbox{ for }n=1,2\ldots$ Find the remainder of the division of $x_{1998}$ by $1998$. \\\\
Note: $\lfloor{x}\rfloor$ is the greatest integer less than or equal to $x$. 

OIM 1997 

OIM 1997 problem 1:  Let $r\geq1$ be areal number that holds with the property that for each pair of positive integer numbers $m$ and $n$, with $n$ a multiple of $m$, it is true that $\lfloor{nr}\rfloor$ is multiple of $\lfloor{mr}\rfloor$. Show that $r$ has to be an integer number. \\\\
\textbf{Note: }\textit{If $x$ is a real number, $\lfloor{x}\rfloor$ is the greatest integer lower than or equal to $x$}.} 
OIM 1997 problem 2:  In a triangle $ABC$, it is drawn a circumference with center in the incenter $I$ and that meet twice each of the sides of the triangle: the segment $BC$ on $D$ and $P$ (where $D$ is nearer two $B$); the segment $CA$ on $E$ and $Q$ (where $E$ is nearer to $C$); and the segment $AB$ on $F$ and $R$ ( where $F$ is nearer to $A$). \\\\
Let $S$ be the point of intersection of the diagonals of the quadrilateral $EQFR$. Let $T$ be the point of intersection of the diagonals of the quadrilateral $FRDP$. Let $U$ be the point of intersection of the diagonals of the quadrilateral $DPEQ$. \\\\
Show that the circumcircle to the triangle $\triangle{FRT}$, $\triangle{DPU}$ and $\triangle{EQS}$ have a unique point in common. 
OIM 1997 problem 3:  Let $n \geq2$ be an integer number and $D_n$ the set of all the points $(x,y)$ in the plane such that its coordinates are integer numbers with: $-n \le x \le n$ and $-n \le y \le n$.
\begin{enumerate}[(a)]
  \item There are three possible colors in which the points of $D_n$ are painted with (each point has a unique color). Show that with
\end{enumerate}
any distribution of the colors, there are always two points of $D_n$ with the same color such that the line that contains them does not go through any other point of $D_n$. \\\\
(b) Find a way to paint the points of $D_n$ with 4 colors such that if a line contains  exactly two points of $D_n$, then, this points have different colors. 
OIM 1997 problem 4:  Let $n$ be a positive integer. Consider the sum $x_1y_1 + x_2y_2 +\cdots + x_ny_n$, where that values of the variables $x_1, x_2,\ldots, x_n, y_1, y_2,\ldots, y_n$ are either 0 or 1. \\\\
Let $I(n)$ be the number of 2$n$-tuples $(x_1, x_2,\ldots, x_n, y_1, y_2,\ldots, y_n)$ such that the sum of the number is odd, and let $P(n)$ be the number of 2$n$-tuples  $(x_1, x_2,\ldots, x_n, y_1, y_2,\ldots, y_n)$ such that the sum is an even number. Show that:
\[ \frac{P(n)}{I(n)}=\frac{2^n+1}{2^n-1} \] 
OIM 1997 problem 5:  In an acute triangle $\triangle{ABC}$, let $AE$ and $BF$ be highs of it, and $H$ its orthocenter. The symmetric line of $AE$ with respect to the angle bisector of $\sphericalangle{A}$ and the symmetric line of $BF$ with respect to the angle bisector of $\sphericalangle{B}$ intersect each other on the point $O$. The lines $AE$ and $AO$ intersect again the circuncircle to $\triangle{ABC}$ on the points $M$ and $N$ respectively. \\\\
Let $P$ be the intersection of $BC$ with $HN$; $R$ the intersection of $BC$ with $OM$; and $S$ the intersection of $HR$ with $OP$. Show that $AHSO$ is a paralelogram. 
OIM 1997 problem 6:  Let $P = \{P_1, P_2, ..., P_{1997}\}$ be a set of $1997$ points in the interior of a circle of radius 1, where $P_1$ is the center of the circle. For each $k=1.\ldots,1997$, let $x_k$ be the distance of $P_k$ to the point of $P$ closer to $P_k$, but different from it. Show that $(x_1)^2 + (x_2)^2 + ... + (x_{1997})^2 \le 9.$ 

OIM 1996 

OIM 1996 problem 1:  Let $ n$ be a natural number. A cube of edge $ n$ may be divided in 1996 cubes whose edges length are also natural numbers. Find the minimum possible value for $ n$. 
OIM 1996 problem 2:  Let $\triangle{ABC}$ be a triangle, $D$ the midpoint of $BC$, and $M$ be the midpoint of $AD$. The line $BM$ intersects the side $AC$ on the point $N$. Show that $AB$ is tangent to the circuncircle to the triangle $\triangle{NBC}$ if and only if the following equality is true:
\[ \frac{{BM}}{{MN}} =\frac{({BC})^2}{({BN})^2}. \] 
OIM 1996 problem 3:  We have a grid of $k^2-k+1$ rows and $k^2-k+1$ columns, where $k=p+1$ and $p$ is prime. For each prime $p$, give a method to put the numbers 0 and 1, one number for each square in the grid, such that on each row there are exactly $k$ 0's, on each column there are exactly $k$ 0's, and there is no rectangle with sides parallel to the sides of the grid with 0s on each four vertices. 
OIM 1996 problem 4:  Given a natural number $n \geq 2$, consider all the fractions of the form $\frac{1}{ab}$, where $a$ and $b$ are natural numbers, relative primes and such that: \\
$a < b \leq n$, \\
$a+b>n$. \\
Show that for each $n$, the sum of all this fractions are $\frac12$. 
OIM 1996 problem 5:  Three tokens $A$, $B$, $C$ are, each one in a vertex of an equilateral triangle of side $n$. Its divided on equilateral triangles of side 1, such as it is shown in the figure for the case $n=3$ \\\\
Initially, all the lines of the figure are painted blue. The tokens are moving along the lines painting them of red, following the next two rules: \\\\
\textbf{(1) }First $A$ moves, after that $B$ moves, and then $C$, by turns. On each turn, the token moves over exactly one line of one of the little triangles, form one side to the other. \\
\textbf{(2)} Non token moves over a line that is already painted red, but it can rest on one endpoint of a side that is already red, even if there is another token there waiting its turn. \\\\
Show that for every positive integer $n$ it is possible to paint red all the sides of the little triangles. 
OIM 1996 problem 6:  There are $n$ different points  $A_1, \ldots , A_n$ in the plain and each point $A_i$ it is assigned a real number $\lambda_i$ distinct from zero in such way that $(\overline{A_i A_j})^2 = \lambda_i + \lambda_j$ for all the $i$,$j$ with $i\neq{}j$} \\
Show that:
\begin{enumerate}[(1)]
  \item $n \leq 4$
  \item If $n=4$, then $\frac{1}{\lambda_1} + \frac{1}{\lambda_2} + \frac{1}{\lambda_3}+ \frac{1}{\lambda_4} = 0$
\end{enumerate} 

OIM 1995 

OIM 1995 problem 1:  Find all the possible values of the sum of the digits of all the perfect squares. \\\\
[Commented by djimenez] \\
\textbf{Comment: }I would rewrite it as follows: \\
Let $f: \mathbb{N}\rightarrow \mathbb{N}$ such that $f(n)$ is the sum of all the digits  of the number $n^2$. Find the image of $f$ (where, by image it is understood the set of all $x$ such that exists an $n$ with $f(n)=x$). 
OIM 1995 problem 2:  Let $n$ be a positive integer greater than 1. Determine all the collections of real numbers  $x_1,\ x_2,\dots,\ x_n\geq1\mbox{ and }x_{n+1}\leq0$ such that the next two conditions hold:
\begin{enumerate}[(i)]
  \item $x_1^{\frac12}+x_2^{\frac32}+\cdots+x_n^{n-\frac12}= nx_{n+1}^\frac12$
  \item $\frac{x_1+x_2+\cdots+x_n}{n}=x_{n+1}$
\end{enumerate} 
OIM 1995 problem 3:  Let $ r$ and $ s$ two orthogonal lines that does not lay on the same plane. Let $ AB$ be their common perpendicular, where $ A\in{}r$ and $ B\in{}s$(*).Consider the sphere of diameter $ AB$. The points $ M\in{r}$ and $ N\in{s}$ varies with the condition that $ MN$ is tangent to the sphere on the point $ T$. Find the locus of $ T$. \\\\
Note: The plane that contains $ B$ and $ r$ is perpendicular to $ s$. 
OIM 1995 problem 4:  In a $m\times{n}$ grid are there are token. Every  token \textit{dominates } every square on its same row ($\leftrightarrow$), its same column ($\updownarrow$), and diagonal ($\searrow\hspace{-4.45mm}\nwarrow$)(Note that the token does not \emph{dominate} the diagonal ($\nearrow\hspace{-4.45mm}\swarrow$), determine the lowest number of tokens that must be on the board to \textit{dominate } all the squares on the board. 
OIM 1995 problem 5:  The incircle of a triangle $ABC$ touches the sides $BC$, $CA$, $AB$ at the points $D$, $E$, $F$ respectively. Let the line $AD$ intersect this incircle of triangle $ABC$ at a point $X$ (apart from $D$). Assume that this point $X$ is the midpoint of the segment $AD$, this means, $AX = XD$. Let the line $BX$ meet the incircle of triangle $ABC$ at a point $Y$ (apart from $X$), and let the line $CX$ meet the incircle of triangle $ABC$ at a point $Z$ (apart from $X$). Show that $EY = FZ$. 
OIM 1995 problem 6:  A function $f: \N\rightarrow\N$ is circular if for every $p\in\N$ there exists $n\in\N,\ n\leq{p}$ such that $f^n(p)=p$ ($f$ composed with itself $n$ times) The function $f$ has repulsion degree $k>0$ if for every $p\in\N$ $f^i(p)\neq{p}$ for every $i=1,2,\dots,\lfloor{kp}\rfloor$. Determine the maximum repulsion degree can have a circular function. \\\\
\textbf{Note:} Here $\lfloor{x}\rfloor$ is the integer part of $x$. 

OIM 1994 

OIM 1994 problem 1:  A number $n$ is said to be \textit{nice} if it exists an integer $r>0$ such that the expression of $n$ in base $r$ has all \\
its digits equal. For example, 62 and 15 are $\emph{nice}$ because 62 is 222 in base 5, and 15 is 33 in base 4. Show that 1993 is not \textit{nice}, but 1994 is. 
OIM 1994 problem 2:  Let $ ABCD$ a cuadrilateral inscribed in a circumference. Suppose that there is a semicircle with its center on $ AB$, that \\
is tangent to the other three sides of the cuadrilateral.
\begin{enumerate}[(i)]
  \item Show that $ AB = AD + BC$.
  \item Calculate, in term of $ x = AB$ and $ y = CD$, the maximal area that can be reached for such quadrilateral.
\end{enumerate} 
OIM 1994 problem 3:  In each square of an $n\times{n}$ grid there is a lamp. If the lamp is touched it changes its state every lamp in the same row and every lamp in the same column (the one that are on are turned off and viceversa). At the begin, all the lamps are off. Show that  lways is possible, with an appropriated sequence of touches, that all the the lamps on the board end on and find, in function of $n$ the minimal number of touches that are necessary to turn on every lamp. 
OIM 1994 problem 4:  Let $A,\ B$ and $C$ be given points on a circumference $K$ such that the triangle $\triangle{ABC}$ is acute. Let $P$ be a point in the interior of $K$. $X,\ Y$ and $Z$ be the other intersection of $AP, BP$ and $CP$  with the circumference. Determine the position of $P$ such that $\triangle{XYZ}$ is equilateral. 
OIM 1994 problem 5:  Let $n$ and $r$ two positive integers. It is wanted to make $r$ subsets $A_1,\ A_2,\dots,A_r$ from the set $\{0,1,\cdots,n-1\}$ such that all those subsets contain exactly $k$ elements and such that, for all integer $x$ with $0\leq{x}\leq{n-1}$ there exist $x_1\in{}A_1,\ x_2\in{}A_2 \dots,x_r\in{}A_r$ (an element of each set) with $x=x_1+x_2+\cdots+x_r$. \\\\
Find the minimum value of $k$ in terms of $n$ and $r$. 
OIM 1994 problem 6:  Show that every natural number $n\leq2^{1\;000\;000}$ can be obtained first with 1 doing less than $1\;100\;000$ sums; more precisely, there is a finite sequence of natural numbers $x_0,\ x_1,\dots,\ x_k\mbox{ with }k\leq1\;100\;000,\ x_0=1,\ x_k=n$ such that for all $i=1,\ 2,\dots,\ k$ there exist $r,\ s$ with $0\leq{r}\leq{s}<i$ such that $x_i=x_r+x_s$. 

OIM 1993 

OIM 1993 problem 1:  A number is called \textit{capicua} if when it is written in decimal notation, it can be read equal from left to right as from right to left; for example: $8, 23432, 6446$. Let $x_1<x_2<\cdots<x_i<x_{i+1},\cdots$ be the sequence of all capicua numbers. For each $i$ define $y_i=x_{i+1}-x_i$. How many distinct primes contains the set $\{y_1,y_2, \ldots\}$? 
OIM 1993 problem 2:  Show that for every convex polygon whose area is less than or equal to $1$, there exists a parallelogram with area $2$ containing the polygon. 
OIM 1993 problem 3:  Let $\mathbb{N}^*=\{1,2,\ldots\}$. Find al the functions $f: \mathbb{N}^*\rightarrow \mathbb{N}^*$ such that:
\begin{enumerate}[(1)]
  \item If $x<y$ then $f(x)<f(y)$.
  \item $f\left(yf(x)\right)=x^2f(xy)$ for all $x,y \in\mathbb{N}^*$.
\end{enumerate} 
OIM 1993 problem 4:  Let $ABC$ be an equilateral triangle and $\Gamma$ its incircle. If $D$ and $E$ are points on the segments $AB$ and $AC$ such that $DE$ is tangent to $\Gamma$, show that $\frac{AD}{DB}+\frac{AE}{EC}=1$. 
OIM 1993 problem 5:  Let $P$ and $Q$ be two distinct points in the plane. Let us denote by $m(PQ)$ the segment bisector of $PQ$. Let $S$ be a finite subset of the plane, with more than one element, that satisfies the following properties:
\begin{enumerate}[(i)]
  \item If $P$ and $Q$ are in $S$, then $m(PQ)$ intersects $S$.
  \item If $P_1Q_1, P_2Q_2, P_3Q_3$ are three diferent segments such that its endpoints are points of $S$, then, there is non point in $S$ such that it intersects the three lines $m(P_1Q_1)$, $m(P_2Q_2)$, and $m(P_3Q_3)$.
\end{enumerate}
Find the number of points that $S$ may contain. 
OIM 1993 problem 6:  Two nonnegative integers $a$ and $b$ are \textit{tuanis} if the decimal expression of $a+b$ contains only $0$ and $1$ as digits. Let $A$ and $B$ be two infinite sets of non negative integers such that $B$ is the set of all the \textit{tuanis} numbers to elements of the set $A$ and $A$ the set of all the \textit{tuanis} numbers to elements of the set $B$. Show that in at least one of the sets $A$ and $B$ there is an infinite number of pairs $(x,y)$ such that $x-y=1$. 

OIM 1992 

OIM 1992 problem 1:  For every positive integer $n$ we define $a_n$ as the last digit of the sum $1+2+\cdots+n$. Compute $a_1+a_2+\cdots+a_{1992}$. 
OIM 1992 problem 2:  Given the positive real numbers $a_1<a_2<\cdots<a_n$, consider the function
\[ f(x)=\frac{a_1}{x+a_1}+\frac{a_2}{x+a_2}+\cdots+\frac{a_n}{x+a_n} \]
Determine the sum of the lengths of the disjoint intervals formed by all the values of $x$ such that $f(x)>1$. 
OIM 1992 problem 3:  Let $ABC$ be an equilateral triangle of sidelength 2 and let $\omega$ be its incircle.
\begin{enumerate}[a)]
  \item Show that for every point $P$ on $\omega$ the sum of the squares of its distances to $A$, $B$, $C$ is 5.
  \item Show that for every point $P$ on $\omega$ it is possible to construct a triangle of sidelengths $AP$, $BP$, $CP$. Also, the area of such triangle is $\frac{\sqrt{3}}{4}$.
\end{enumerate} 
OIM 1992 problem 4:  Let $\{a_n\}_{n \geq 0}$ and $\{b_n\}_{n \geq 0}$ be two sequences of integer numbers such that:
\begin{enumerate}[i.]
  \item $a_0=0$, $b_0=8$.
  \item For every $n \geq 0$, $a_{n+2}=2a_{n+1}-a_n+2$, $b_{n+2}=2b_{n+1}-b_n$.
  \item $a_n^2+b_n^2$ is a perfect square for every $n \geq 0$.
\end{enumerate}
Find at least two values of the pair $(a_{1992},\, b_{1992})$. 
OIM 1992 problem 5:  Given a circle $\Gamma$ and the positive numbers $h$ and $m$,  construct with straight edge and compass a trapezoid inscribed in $\Gamma$, such that it has altitude  $h$ and the sum of its parallel sides is $m$. 
OIM 1992 problem 6:  In a triangle $ABC$, points $A_1$ and $A_2$ are chosen in the prolongations beyond $A$ of segments $AB$ and $AC$, such that $AA_1=AA_2=BC$. Define analogously points $B_1$, $B_2$, $C_1$, $C_2$. If $[ABC]$ denotes the area of triangle $ABC$, show that $[A_1A_2B_1B_2C_1C_2] \geq 13 [ABC]$. 

OIM 1991 

OIM 1991 problem 1:  Each vertex of a cube is assigned an 1 or a -1, and each face is assigned the product of the numbers assigned to its vertices. Determine the possible values the sum of these 14 numbers can attain. 
OIM 1991 problem 2:  A square is divided in four parts by two perpendicular lines, in such a way that three of these parts have areas equal to 1. Show that the square has area equal to 4. 
OIM 1991 problem 3:  Let $f: \ [0,\ 1] \rightarrow \mathbb{R}$ be an increasing function satisfying the following conditions:
\begin{enumerate}[a)]
  \item $f(0)=0$;
  \item $f\left(\frac{x}{3}\right)=\frac{f(x)}{2}$;
  \item $f(1-x)=1-f(x)$.
\end{enumerate}
Determine $f\left(\frac{18}{1991}\right)$. 
OIM 1991 problem 4:  Find a positive integer $n$ with five non-zero different digits, which satisfies to be equal to the sum of all the three-digit numbers that can be formed using the digits of $n$. 
OIM 1991 problem 5:  Let $P(x,\, y)=2x^2-6xy+5y^2$. Let us say an integer number $a$ is a value of $P$ if there exist integer numbers $b$, $c$ such that $P(b,\, c)=a$.
\begin{enumerate}[a)]
  \item Find all values of $P$ lying between 1 and 100.
  \item Show that if $r$ and $s$ are values of $P$, then so is $rs$.
\end{enumerate} 
OIM 1991 problem 6:  Let $M$, $N$ and $P$ be three non-collinear points. Construct using straight edge and compass a triangle for which $M$ and $N$ are the midpoints of two of its sides, and $P$ is its orthocenter. 

OIM 1990 

OIM 1990 problem 1:  Let $f$ be a function defined for the non-negative integers, such that:
\begin{enumerate}[a)]
  \item $f(n)=0$ if $n=2^j-1$ for some $j \geq 0$.
  \item $f(n+1)=f(n)-1$ otherwise.
  \item Show that for every $n \geq 0$ there exists $k \geq 0$ such that $f(n)+n=2^k-1$.
  \item Find $f(2^{1990})$.
\end{enumerate} 
OIM 1990 problem 2:  Let $ABC$ be a triangle. $I$ is the incenter, and the incircle is tangent to $BC$, $CA$, $AB$ at $D$, $E$, $F$, respectively. $P$ is the second point of intersection of $AD$ and the incircle. If $M$ is the midpoint of $EF$, show that $P$, $I$, $M$, $D$ are concyclic. 
OIM 1990 problem 3:  Let $b$, $c$ be integer numbers, and define $f(x)=(x+b)^2-c$.
\begin{enumerate}[i)]
  \item If $p$ is a prime number such that $c$ is divisible by $p$ but not by $p^2$, show that for every integer $n$, $f(n)$ is not divisible by $p^2$.
  \item Let $q \neq 2$ be a  prime divisor of $c$. If $q$ divides $f(n)$ for some integer $n$, show that for every integer $r$ there exists an integer $n'$ such that $f(n')$ is divisible by $qr$.
\end{enumerate} 
OIM 1990 problem 4:  Let $\Gamma_1$ be a circle. $AB$ is a diameter, $\ell$ is the tangent at $B$, and $M$ is a point on $\Gamma_1$ other than $A$. $\Gamma_2$ is a circle tangent to $\ell$, and also to $\Gamma_1$ at $M$.
\begin{enumerate}[a)]
  \item Determine the point of tangency $P$ of $\ell$ and $\Gamma_2$ and find the locus of the center of $\Gamma_2$ as $M$ varies.
  \item Show that there exists a circle that is always orthogonal to $\Gamma_2$, regardless of the position of $M$.
\end{enumerate} 
OIM 1990 problem 5:  $A$ and $B$ are two opposite vertices of an $n \times n$ board. Within each small square of the board, the diagonal parallel to $AB$ is drawn, so that the board is divided in $2n^2$ equal triangles. A coin moves from $A$ to $B$ along the grid, and for every segment of the grid that it visits, a seed is put in each triangle that contains the segment as a side. The path followed by the coin is such that no segment is visited more than once, and after the coins arrives at $B$, there are exactly two seeds in each of the $2n^2$ triangles of the board. Determine all the values of $n$ for which such scenario is possible. 
OIM 1990 problem 6:  Let $f(x)$ be a cubic polynomial with rational coefficients. If the graph of $f(x)$ is tangent to the $x$ axis, prove that the roots of $f(x)$ are all rational. 

OIM 1989 

OIM 1989 problem 1:  Determine all triples of real numbers that satisfy the following system of equations:
\[ x+y-z=-1\\ x^2-y^2+z^2=1\\ -x^3+y^3+z^3=-1 \] 
OIM 1989 problem 2:  Let $x,y,z$ be real numbers such that $0\le x,y,z\le\frac{\pi}{2}$. Prove the inequality
\[ \frac{\pi}{2}+2\sin x\cos y+2\sin y\cos z\ge\sin 2x+\sin 2y+\sin 2z. \] 
OIM 1989 problem 3:  Let $a,b$ and $c$ be the side lengths of a triangle. Prove that:
\[ \frac{a-b}{a+b}+\frac{b-c}{b+c}+\frac{c-a}{c+a}<\frac{1}{16} \] 
OIM 1989 problem 4:  The incircle of the triangle $ABC$ is tangent to sides $AC$ and $BC$ at $M$ and $N$, respectively. The bisectors of the angles at $A$ and $B$ intersect $MN$ at points $P$ and $Q$, respectively. Let $O$ be the incentre of $\triangle ABC$. Prove that $MP\cdot OA=BC\cdot OQ$. 
OIM 1989 problem 5:  Let the function $f$ be defined on the set $\mathbb{N}$ such that \\\\
$\text{(i)}\ \ \quad f(1)=1$ \\
$\text{(ii)}\ \quad f(2n+1)=f(2n)+1$ \\
$\text{(iii)}\quad f(2n)=3f(n)$ \\\\
Determine the set of values taken $f$. 
OIM 1989 problem 6:  Show that the equation $2x^2-3x=3y^2$ has infinitely many solutions in positive integers. 

OIM 1988 

OIM 1988 problem 1:  The measure of the angles of a triangle are in arithmetic progression and the lengths of its altitudes are as well. Show that such a triangle is equilateral. 
OIM 1988 problem 2:  Let $a,b,c,d,p$ and $q$ be positive integers satisfying $ad-bc=1$ and $\frac{a}{b}>\frac{p}{q}>\frac{c}{d}$. \\\\
Prove that: \\\\
$(a)$ $q\ge b+d$ \\\\
$(b)$ If $q=b+d$, then $p=a+c$. 
OIM 1988 problem 3:  Prove that among all possible triangles whose vertices are $3,5$ and $7$ apart from a given point $P$, the ones with the largest perimeter have $P$ as incentre. 
OIM 1988 problem 4:  $\triangle ABC$ is a triangle with sides $a,b,c$. Each side of $\triangle ABC$ is divided in $n$ equal segments. Let $S$ be the sum of the squares of the distances from each vertex to each of the points of division on its opposite side. Show that $\frac{S}{a^2+b^2+c^2}$ is a rational number. 
OIM 1988 problem 5:  Consider all the numbers of the form $x+yt+zt^2$, with $x,y,z$ rational numbers and $t=\sqrt[3]{2}$. Prove that if $x+yt+zt^2\not= 0$, then there exist rational numbers $u,v,w$ such that
\[ (x+yt+z^2)(u+vt+wt^2)=1 \] 
OIM 1988 problem 6:  Consider all sets of $n$ distinct positive integers, no three of which form an arithmetic progression. Prove that among all such sets there is one which has the largest sum of the reciprocals of its elements. 

OIM 1987 

OIM 1987 problem 1:  Find the function $f(x)$ such that
\[ f(x)^2f\left(\frac{1-x}{x+1}\right) =64x \]
for $x\not=0,x\not=1,x\not=-1$. 
OIM 1987 problem 2:  In a triangle $ABC$, $M$ and $N$ are the respective midpoints of the sides $AC$ and $AB$, and $P$ is the point of intersection of $BM$ and $CN$. Prove that, if it is possible to inscribe a circle in the quadrilateral $AMPN$, then the triangle $ABC$ is isosceles. 
OIM 1987 problem 3:  Prove that if $m,n,r$ are positive integers, and:
\[ 1+m+n\sqrt{3}=(2+\sqrt{3})^{2r-1} \]
then $m$ is a perfect square. 
OIM 1987 problem 4:  The sequence $(p_n)$ is defined as follows: $p_1=2$ and for all $n$ greater than or equal to $2$, $p_n$ is the largest prime divisor of the expression $p_1p_2p_3\ldots p_{n-1}+1$. \\
Prove that every $p_n$ is different from $5$. 
OIM 1987 problem 5:  Let $r,s,t$ be the roots of the equation $x(x-2)(3x-7)=2$. Show that $r,s,t$ are real and positive and determine $\arctan r+\arctan s +\arctan t$. 
OIM 1987 problem 6:  Let $ABCD$ be a convex quadrilateral and let $P$ and $Q$ be the points on the sides $AD$ and $BC$ respectively such that $\frac{AP}{PD}=\frac{BQ}{QC}=\frac{AB}{CD}$. \\
Prove that the line $PQ$ forms equal angles with the lines $AB$ and $CD$. 

OIM 1985 

OIM 1985 problem 1:  Find all the triples of integers $ (a, b,c)$ such that:
\[
\begin{array}{ccc}a+b+c \&=\& 24\\ a^2+b^2+c^2\&=\& 210\\ abc \&=\& 440\end{array}
\] 
OIM 1985 problem 2:  Let $ P$ be a point in the interior of the equilateral triangle $ \triangle{}ABC$ such that $ PA = 5$, $ PB = 7$, $ PC = 8$. Find the length of the side of the triangle $ ABC$. 
OIM 1985 problem 3:  Find all the roots $ r_1$, $ r_2$, $ r_3$ y $ r_4$ of the equation $ 4x^4-ax^3+bx^2-cx+5 = 0$, knowing that they are real, positive and that:
\[ \frac{r_1}{2}+\frac{r_2}{4}+\frac{r_3}{5}+\frac{r_4}{8}= 1. \] 
OIM 1985 problem 4:  If $ x\neq1$, $ y\neq1$, $ x\neq y$ and
\[ \frac{yz-x^2}{1-x}=\frac{xz-y^2}{1-y} \]
show that both fractions are equal to $ x+y+z$. 
OIM 1985 problem 5:  To each positive integer $ n$ it is assigned a non-negative integer $f(n)$ such that the following conditions are satisfied:
\begin{enumerate}[(1)]
  \item $ f(rs) = f(r)+f(s)$
  \item $ f(n) = 0$, if the first digit (from right to left) of $ n$ is 3.
  \item $ f(10) = 0$.
\end{enumerate}
Find $f(1985)$. Justify your answer. 
OIM 1985 problem 6:  Given an acute triangle $ABC$, let $D$, $E$ and $F$ be points in the lines $BC$, $AC$ and $AB$ respectively. If the lines $AD$, $BE$ and $CF$ pass through $O$ the centre of the circumcircle of the triangle $ABC$, whose radius is $R$, show that:
\[ \frac{1}{AD}+\frac{1}{BE}+\frac{1}{CF}=\frac{2}{R} \] 
