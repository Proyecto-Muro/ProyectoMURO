
RMM 2021 

RMM 2021 problem 1:  Let $T_1, T_2, T_3, T_4$ be pairwise distinct collinear points such that $T_2$ lies between $T_1$ and $T_3$, and $T_3$ lies between $T_2$ and $T_4$. Let $\omega_1$ be a circle through $T_1$ and $T_4$; let $\omega_2$ be the circle through $T_2$ and internally tangent to $\omega_1$ at $T_1$; let $\omega_3$ be the circle through $T_3$ and externally tangent to $\omega_2$ at $T_2$; and let $\omega_4$ be the circle through $T_4$ and externally tangent to $\omega_3$ at $T_3$. A line crosses $\omega_1$ at $P$ and $W$, $\omega_2$ at $Q$ and $R$, $\omega_3$ at $S$ and $T$, and $\omega_4$ at $U$ and $V$, the order of these points along the line being $P,Q,R,S,T,U,V,W$. Prove that $PQ + TU = RS + VW$ \\\\
\textit{Geza Kos, Hungary} 
RMM 2021 problem 2:  Xenia and Sergey play the following game. Xenia thinks of a positive integer $N$ not exceeding $5000$. Then she fixes $20$ distinct positive integers $a_1, a_2, \cdots, a_{20}$ such that, for each $k = 1,2,\cdots,20$, the numbers $N$ and $a_k$ are congruent modulo $k$. By a move, Sergey tells Xenia a set $S$ of positive integers not exceeding $20$, and she tells him back the set $\{a_k : k \in S\}$ without spelling out which number corresponds to which index. How many moves does Sergey need to determine for sure the number Xenia thought of? \\\\
\textit{Sergey Kudrya, Russia} 
RMM 2021 problem 3:  A number of $17$ workers stand in a row. Every contiguous group of at least $2$ workers is a $brigade$. The chief wants to assign each brigade a leader (which is a member of the brigade) so that each worker’s number of assignments is divisible by $4$. Prove that the number of such ways to assign the leaders is divisible by $17$. \\
$Mikhail$ $Antipov, Russia$ 
RMM 2021 problem 4:  Consider an integer \(n \ge 2\) and write the numbers \(1, 2,  \ldots, n\) down on a board. A move consists in erasing any two numbers \(a\) and \(b\), then writing down the numbers \(a+b\) and \(\vert a-b \vert\) on the board, and then removing repetitions (e.g., if the board contained the numbers \(2, 5, 7, 8\), then one could choose the numbers \(a = 5\) and \(b = 7\), obtaining the board with numbers \(2, 8, 12\)). For all integers \(n \ge 2\), determine whether it is possible to be left with exactly two numbers on the board after a finite number of moves. \\\\
\textit{Proposed by China} 
RMM 2021 problem 5:  Let \(n\) be a positive integer. The kingdom of Zoomtopia is a convex polygon with integer sides, perimeter \(6n\), and \(60^\circ\) rotational symmetry (that is, there is a point \(O\) such that a \(60^\circ\) rotation about \(O\) maps the polygon to itself). In light of the pandemic, the government of Zoomtopia would like to relocate its \(3n^2+3n+1\) citizens at \(3n^2+3n+1\) points in the kingdom so that every two citizens have a distance of at least \(1\) for proper social distancing. Prove that this is possible. (The kingdom is assumed to contain its boundary.) \\\\
\textit{Proposed by Ankan Bhattacharya, USA} 
RMM 2021 problem 6:  Initially, a non-constant polynomial $S(x)$ with real coefficients is written down on a board. Whenever the board contains a polynomial $P(x)$, not necessarily alone, one can write down on the board any polynomial of the form $P(C + x)$ or $C + P(x)$ where $C$ is a real constant. Moreover, if the board contains two (not necessarily distinct) polynomials $P(x)$ and $Q(x)$, one can write $P(Q(x))$ and $P(x) + Q(x)$ down on the board. No polynomial is ever erased from the board. \\\\
Given two sets of real numbers, $A = \{ a_1, a_2, \dots, a_n \}$ and $B = \{ b_1, \dots, b_n \}$, a polynomial $f(x)$ with real coefficients is $(A,B)$-\textit{nice} if $f(A) = B$, where $f(A) = \{ f(a_i) : i = 1, 2, \dots, n \}$. \\\\
Determine all polynomials $S(x)$ that can initially be written down on the board such that, for any two finite sets $A$ and $B$ of real numbers, with $|A| = |B|$, one can produce an $(A,B)$-\textit{nice} polynomial in a finite number of steps. \\\\
\textit{Proposed by Navid Safaei, Iran} 

RMM 2020 

RMM 2020 problem 1:  Let $ABC$ be a triangle with a right angle at $C$. Let $I$ be the incentre of triangle $ABC$, and let $D$ be the foot of the altitude from $C$ to $AB$. The incircle $\omega$ of triangle $ABC$ is tangent to sides $BC$, $CA$, and $AB$ at $A_1$, $B_1$, and $C_1$, respectively. Let $E$ and $F$ be the reflections of $C$ in lines $C_1A_1$ and $C_1B_1$, respectively. Let $K$ and $L$ be the reflections of $D$ in lines $C_1A_1$ and $C_1B_1$, respectively. \\\\
Prove that the circumcircles of triangles $A_1EI$, $B_1FI$, and $C_1KL$ have a common point. 
RMM 2020 problem 2:  Let $N \geq 2$ be an integer, and let $\mathbf a$ $= (a_1, \ldots, a_N)$ and $\mathbf b$ $= (b_1, \ldots b_N)$ be sequences of non-negative integers. For each integer $i \not \in \{1, \ldots, N\}$, let $a_i = a_k$ and $b_i = b_k$, where $k \in \{1, \ldots, N\}$ is the integer such that $i-k$ is divisible by $n$. We say $\mathbf a$ is $\mathbf b$-\textit{harmonic} if each $a_i$ equals the following arithmetic mean:
\[ a_i = \frac{1}{2b_i+1} \sum_{s=-b_i}^{b_i} a_{i+s}. \]
Suppose that neither $\mathbf a $ nor $\mathbf b$ is a constant sequence, and that both $\mathbf a$ is $\mathbf b$-\textit{harmonic} and $\mathbf b$ is $\mathbf a$-\textit{harmonic}. \\\\
Prove that at least $N+1$ of the numbers $a_1, \ldots, a_N,b_1, \ldots, b_N$ are zero. 
RMM 2020 problem 3:  Let $n\ge 3$ be an integer. In a country there are $n$ airports and $n$ airlines operating two-way flights. For each airline, there is an odd integer $m\ge 3$, and $m$ distinct airports $c_1, \dots, c_m$, where the flights offered by the airline are exactly those between the following pairs of airports: $c_1$ and $c_2$; $c_2$ and $c_3$; $\dots$ ; $c_{m-1}$ and $c_m$; $c_m$ and $c_1$. \\\\
Prove that there is a closed route consisting of an odd number of flights where no two flights are operated by the same airline. 
RMM 2020 problem 4:  Let $\mathbb N$ be the set of all positive integers. A subset $A$ of $\mathbb N$ is \textit{sum-free} if, whenever $x$ and $y$ are (not necessarily distinct) members of $A$,  their sum $x+y$ does not belong to $A$. Determine all surjective functions $f:\mathbb N\to\mathbb N$ such that, for each sum-free subset $A$ of $\mathbb N$, the image $\{f(a):a\in A\}$ is also sum-free. \\\\
\textit{Note: a function $f:\mathbb N\to\mathbb N$ is surjective if, for every positive integer $n$, there exists a positive integer $m$ such that $f(m)=n$.} 
RMM 2020 problem 5:  A \textit{lattice point} in the Cartesian plane is a point whose coordinates are both integers. A \textit{lattice polygon} is a polygon all of whose vertices are lattice points. \\\\
Let $\Gamma$ be a convex lattice polygon. Prove that $\Gamma$ is contained in a convex lattice polygon $\Omega$ such that the vertices of $\Gamma$ all lie on the boundary of $\Omega$, and exactly one vertex of $\Omega$ is not a vertex of $\Gamma$. 
RMM 2020 problem 6:  For each integer $n \geq 2$, let $F(n)$ denote the greatest prime factor of $n$. A \textit{strange pair} is a pair of distinct primes $p$ and $q$ such that there is no integer $n \geq 2$ for which $F(n)F(n+1)=pq$. \\\\
Prove that there exist infinitely many strange pairs. 

RMM 2019 

RMM 2019 problem 1:  Amy and Bob play the game. At the beginning, Amy writes down a positive integer on the board. Then the players take moves in turn, Bob moves first. On any move of his, Bob replaces the number $n$ on the blackboard with a number of the form $n-a^2$, where $a$ is a positive integer. On any move of hers, Amy replaces the number $n$ on the blackboard with a number of the form $n^k$, where $k$ is a positive integer. Bob wins if the number on the board becomes zero. \\
Can Amy prevent Bob’s win? \\\\
\textit{Maxim Didin, Russia} 
RMM 2019 problem 2:  Let $ABCD$ be an isosceles trapezoid with $AB\parallel CD$. Let $E$ be the midpoint of $AC$. Denote by $\omega$ and $\Omega$ the circumcircles of the triangles $ABE$ and $CDE$, respectively. Let $P$ be the crossing point of the tangent to $\omega$ at $A$ with the tangent to $\Omega$ at $D$. Prove that $PE$ is tangent to $\Omega$. \\\\
\textit{Jakob Jurij Snoj, Slovenia} 
RMM 2019 problem 3:  Given any positive real number $\varepsilon$, prove that, for all but finitely many positive integers $v$, any graph on $v$ vertices with at least  $(1+\varepsilon)v$ edges has two distinct simple cycles of equal lengths. \\
(Recall that the notion of a simple cycle does not allow repetition of vertices in a cycle.) \\\\
\textit{Fedor Petrov, Russia} 
RMM 2019 problem 4:  Prove that for every positive integer $n$ there exists a (not necessarily convex) polygon with no three collinear vertices, which admits exactly $n$ diffferent triangulations. \\\\
(A \textit{triangulation} is a dissection of the polygon into triangles by interior diagonals which have no common interior points with each other nor with the sides of the polygon) 
RMM 2019 problem 5:  Determine all functions $f: \mathbb{R} \to \mathbb{R}$ satisfying
\[ f(x + yf(x)) + f(xy) = f(x) + f(2019y), \]
for all real numbers $x$ and $y$. 
RMM 2019 problem 6:  Find all pairs of integers $(c, d)$, both greater than 1, such that the following holds: \\\\
For any monic polynomial $Q$ of degree $d$ with integer coefficients and for any prime $p > c(2c+1)$, there exists a set $S$ of at most $\big(\tfrac{2c-1}{2c+1}\big)p$ integers, such that
\[ \bigcup_{s \in S} \{s,\; Q(s),\; Q(Q(s)),\; Q(Q(Q(s))),\; \dots\} \]
contains a complete residue system modulo $p$ (i.e., intersects with every residue class modulo $p$). 

RMM 2018 

RMM 2018 problem 1:  Let $ABCD$ be a cyclic quadrilateral an let $P$ be a point on the side $AB.$ The diagonals $AC$ meets the segments $DP$ at $Q.$ The line through $P$ parallel to $CD$ mmets the extension of the side $CB$ beyond $B$ at $K.$ The line through $Q$ parallel to $BD$ meets the extension of the side $CB$ beyond $B$ at $L.$ Prove that the circumcircles of the triangles $BKP$ and $CLQ$ are tangent . 
RMM 2018 problem 2:  Determine whether there exist non-constant polynomials $P(x)$ and $Q(x)$ with real coefficients satisfying
\[ P(x)^{10}+P(x)^9 = Q(x)^{21}+Q(x)^{20}. \] 
RMM 2018 problem 3:  Ann and Bob play a game on the edges of an infinite square grid, playing in turns. Ann plays the first move. A move consists of orienting any edge that has not yet been given an orientation. Bob wins if at any point a cycle has been created. Does Bob have a winning strategy? 
RMM 2018 problem 4:  Let $a,b,c,d$ be positive integers such that $ad \neq bc$ and $gcd(a,b,c,d)=1$. Let $S$ be the set of values attained by $\gcd(an+b,cn+d)$ as $n$ runs through the positive integers. Show that $S$ is the set of all positive divisors of some positive integer. 
RMM 2018 problem 5:  Let $n$ be positive integer and fix $2n$ distinct points on a circle. Determine the number of ways to connect the points with $n$ arrows (oriented line segments) such that all of the following conditions hold:
\begin{itemize}
  \item each of the $2n$ points is a startpoint or endpoint 	of an arrow;
  \item no two arrows intersect; and
  \item there are no two arrows $\overrightarrow{AB}$ and 	$\overrightarrow{CD}$ such that $A$, $B$, $C$ and $D$ 	appear in clockwise order around the circle 	(not necessarily consecutively).
\end{itemize} 
RMM 2018 problem 6:  Fix a circle $\Gamma$, a line $\ell$ to tangent $\Gamma$, and another circle $\Omega$ disjoint from $\ell$ such that $\Gamma$ and $\Omega$ lie on opposite sides of $\ell$. The tangents to $\Gamma$ from a variable point $X$ on $\Omega$ meet $\ell$ at $Y$ and $Z$. Prove that, as $X$ varies over $\Omega$, the circumcircle of $XYZ$ is tangent to two fixed circles. 

RMM 2017 

RMM 2017 problem 1:  \textbf{(a)} Prove that every positive integer $n$ can be written uniquely in the form
\[ n=\sum_{j=1}^{2k+1}(-1)^{j-1}2^{m_j}, \]
where $k\geq 0$ and $0\le m_1<m_2\cdots <m_{2k+1}$ are integers. \\
This number $k$ is called \textit{weight} of $n$. \\\\
\textbf{(b)} Find (in closed form) the difference between the number of positive integers at most $2^{2017}$ with even weight and the number of positive integers at most $2^{2017}$ with odd weight. 
RMM 2017 problem 2:  Determine all positive integers $n$ satisfying the following condition: for every monic polynomial $P$ of degree at most $n$ with integer coefficients, there exists a positive integer $k\le n$ and $k+1$ distinct integers $x_1,x_2,\cdots ,x_{k+1}$ such that
\[ P(x_1)+P(x_2)+\cdots +P(x_k)=P(x_{k+1}) .\] \\\\
\textit{Note.} A polynomial is \textit{monic} if the coefficient of the highest power is one. 
RMM 2017 problem 3:  Let $n$ be an integer greater than $1$ and let $X$ be an $n$-element set. A non-empty collection of subsets $A_1, ..., A_k$ of $X$ is tight if the union $A_1 \cup \cdots  \cup A_k$ is a proper subset of $X$ and no element of $X$ lies in exactly one of the $A_i$s. Find the largest cardinality of a collection of proper non-empty subsets of $X$, no non-empty subcollection of which is tight. \\\\
\textit{Note}. A subset $A$ of $X$ is proper if $A\neq X$. The sets in a collection are assumed to be distinct. The whole collection is assumed to be a subcollection. 
RMM 2017 problem 4:  In the Cartesian plane, let $G_1$ and $G_2$ be the graphs of the quadratic functions $f_1(x) = p_1x^2 + q_1x + r_1$ and $f_2(x) = p_2x^2 + q_2x + r_2$, where $p_1 > 0 > p_2$. The graphs $G_1$ and $G_2$ cross at distinct points $A$ and $B$. The four tangents to $G_1$ and $G_2$ at $A$ and $B$ form a convex quadrilateral which has an inscribed circle. Prove that the graphs $G_1$ and $G_2$ have the same axis of symmetry. 
RMM 2017 problem 5:  Fix an integer $n \geq 2$. An $n\times n$ sieve is an $n\times n$ array with $n$ cells removed so that exactly one cell is removed from every row and every column. A stick is a $1\times k$ or $k\times 1$ array for any positive integer $k$. For any sieve $A$, let $m(A)$ be the minimal number of sticks required to partition $A$. Find all possible values of $m(A)$, as $A$ varies over all possible $n\times n$ sieves. \\\\
\textit{Palmer Mebane} 
RMM 2017 problem 6:  Let $ABCD$ be any convex quadrilateral and let $P, Q, R, S$ be points on the segments $AB, BC, CD$, and $DA$, respectively. It is given that the segments $PR$ and $QS$ dissect $ABCD$ into four quadrilaterals, each of which has perpendicular diagonals. Show that the points $P, Q, R, S$ are concyclic. 

RMM 2016 

RMM 2016 problem 1:  Let $ABC$ be a triangle and let $D$ be a point on the segment $BC, D\neq B$ and $D\neq C$. The circle $ABD$ meets the segment $AC$ again at an interior point $E$. The circle $ACD$ meets the segment $AB$ again at an interior point $F$. Let $A'$ be the reflection of $A$ in the line $BC$. The lines $A'C$ and $DE$ meet at $P$, and the lines $A'B$ and $DF$ meet at $Q$. Prove that the lines $AD, BP$ and $CQ$ are concurrent (or all parallel). 
RMM 2016 problem 2:  Given positive integers $m$ and $n \ge m$, determine the largest number of dominoes ($1\times2$ or $2 \times 1$ rectangles) that can be placed on a rectangular board with $m$ rows and $2n$ columns consisting of cells ($1 \times 1$ \\
squares) so that:
\begin{enumerate}[(i)]
  \item each domino covers exactly two adjacent cells of the board;
  \item no two dominoes overlap;
  \item no two form a $2 \times 2$ square; and
  \item the bottom row of the board is completely covered by $n$ dominoes.
\end{enumerate} 
RMM 2016 problem 3:  A $\textit{cubic sequence}$ is a sequence of integers given by $a_n =n^3 + bn^2 + cn + d$, where $b, c$ and $d$ are integer constants and $n$ ranges over all integers, including negative integers. \\
$\textbf{(a)}$ Show that there exists a cubic sequence such that the only terms \\
of the sequence which are squares of integers are $a_{2015}$ and $a_{2016}$. \\
$\textbf{(b)}$ Determine the possible values of $a_{2015} \cdot a_{2016}$ for a cubic sequence \\
satisfying the condition in part $\textbf{(a)}$. 
RMM 2016 problem 4:  Let $x$ and $y$ be positive real numbers such that: $x+y^{2016}\geq 1$. Prove that $x^{2016}+y> 1-\frac{1}{100}$ 
RMM 2016 problem 5:  A hexagon convex $A_1B_1A_2B_2A_3B_3$ it is inscribed in a circumference $\Omega$ with radius $R$. The diagonals $A_1B_2$, $A_2B_3$, $A_3B_1$ are concurrent in $X$. For each $i=1,2,3$ let $\omega_i$ tangent to the segments $XA_i$ and $XB_i$ and tangent to the arc $A_iB_i$ of $\Omega$ that does not contain the other vertices of the hexagon; let $r_i$ the radius of $\omega_i$. \\\\
$(a)$ Prove that $R\geq r_1+r_2+r_3$ \\
$(b)$ If $R= r_1+r_2+r_3$, prove that the six points of tangency of the circumferences $\omega_i$ with the diagonals $A_1B_2$, $A_2B_3$, $A_3B_1$ are concyclic 
RMM 2016 problem 6:  A set of $n$ points in Euclidean 3-dimensional space, no four of which are coplanar, is partitioned into two subsets $\mathcal{A}$ and $\mathcal{B}$. An $\mathcal{AB}$-tree is a configuration of $n-1$ segments, each of which has an endpoint in $\mathcal{A}$ and an endpoint in $\mathcal{B}$, and such that no segments form a closed polyline. An $\mathcal{AB}$-tree is transformed into another as follows: choose three distinct segments $A_1B_1$, $B_1A_2$, and $A_2B_2$ in the $\mathcal{AB}$-tree such that $A_1$ is in $\mathcal{A}$ and $|A_1B_1|+|A_2B_2|>|A_1B_2|+|A_2B_1|$, and remove the segment $A_1B_1$ to replace it by the segment $A_1B_2$. Given any $\mathcal{AB}$-tree, prove that every sequence of successive transformations comes to an end (no further transformation is possible) after finitely many steps. 

RMM 2015 

RMM 2015 problem 1:  Does there exist an infinite sequence of positive integers $a_1, a_2, a_3, . . .$  such that $a_m$ and $a_n$ are coprime if and only if $|m - n| = 1$? 
RMM 2015 problem 2:  For an integer $n \geq 5,$ two players play the following game on a regular $n$-gon. Initially, three consecutive vertices are chosen, and one counter is placed on each. A move consists of one player sliding one counter along any number of edges to another vertex of the $n$-gon without jumping over another counter. A move is legal if the area of the triangle formed by the counters is strictly greater after the move than before. The players take turns to make legal moves, and if a player cannot make a legal move, that player loses. For which values of $n$ does the player making the first move have a winning strategy? 
RMM 2015 problem 3:  A finite list of rational numbers is written on a blackboard. In an \textit{operation}, we choose any two numbers $a$, $b$, erase them, and write down one of the numbers
\[
a + b, \; a - b, \; b - a, \; a \times b, \; a/b \text{ (if $b \neq 0$)}, \; b/a \text{ (if $a \neq 0$)}.
\]
Prove that, for every integer $n > 100$, there are only finitely many integers $k \ge 0$, such that, starting from the list
\[ k + 1, \; k + 2, \; \dots, \; k + n, \]
it is possible to obtain, after $n - 1$ operations, the value $n!$. 
RMM 2015 problem 4:  Let $ABC$ be a triangle, and let $D$ be the point where the incircle meets side $BC$. Let $J_b$ and $J_c$ be the incentres of the triangles $ABD$ and $ACD$, respectively. Prove that the circumcentre of the triangle $AJ_bJ_c$ lies on the angle bisector of $\angle BAC$. 
RMM 2015 problem 5:  Let $p \ge 5$ be a prime number. For a positive integer $k$, let $R(k)$ be the remainder when $k$ is divided by $p$, with $0 \le R(k) \le p-1$. Determine all positive integers $a < p$ such that, for every $m = 1, 2, \cdots, p-1$,
\[ m + R(ma) > a. \] 
RMM 2015 problem 6:  Given a positive integer $n$, determine the largest real number $\mu$ satisfying the following condition: for every set $C$ of $4n$ points in the interior of the unit square $U$, there exists a rectangle $T$ contained in $U$ such that \\\\
$\bullet$ the sides of $T$ are parallel to the sides of $U$; \\\\
$\bullet$ the interior of $T$ contains exactly one point of $C$; \\\\
$\bullet$ the area of $T$ is at least $\mu$. 

RMM 2013 

RMM 2013 problem 1:  For a positive integer $a$, define a sequence of integers $x_1,x_2,\ldots$ by letting $x_1=a$ and $x_{n+1}=2x_n+1$ for $n\geq 1$. Let $y_n=2^{x_n}-1$. Determine the largest possible $k$ such that, for some positive integer $a$, the numbers $y_1,\ldots,y_k$ are all prime. 
RMM 2013 problem 2:  Does there exist a pair $(g,h)$ of functions $g,h:\mathbb{R}\rightarrow\mathbb{R}$ such that the only function $f:\mathbb{R}\rightarrow\mathbb{R}$ satisfying $f(g(x))=g(f(x))$ and $f(h(x))=h(f(x))$ for all $x\in\mathbb{R}$ is identity function $f(x)\equiv x$? 
RMM 2013 problem 3:  Let $ABCD$ be a quadrilateral inscribed in a circle $\omega$. The lines $AB$ and $CD$ meet at $P$, the lines $AD$ and $BC$ meet at $Q$, and the diagonals $AC$ and $BD$ meet at $R$. Let $M$ be the midpoint of the segment $PQ$, and let $K$ be the common point of the segment $MR$ and the circle $\omega$. Prove that the circumcircle of the triangle $KPQ$ and $\omega$ are tangent to one another. 
RMM 2013 problem 4:  Suppose two convex quadrangles in the plane $P$ and $P'$, share a point $O$ such that, for every line $l$ trough $O$, the segment along which $l$ and $P$ meet is longer then the segment along which $l$ and $P'$ meet. Is it possible that the ratio of the area of $P'$ to the area of $P$ is greater then $1.9$? 
RMM 2013 problem 5:  Given a positive integer $k\geq2$, set $a_1=1$ and, for every integer $n\geq 2$, let $a_n$ be the smallest solution of equation
\[ x=1+\sum_{i=1}^{n-1}\left\lfloor\sqrt[k]{\frac{x}{a_i}}\right\rfloor \]
that exceeds $a_{n-1}$. Prove that all primes are among the terms of the sequence $a_1,a_2,\ldots$ 
RMM 2013 problem 6:  A token is placed at each vertex of a regular $2n$-gon. A \textit{move} consists in choosing an edge of the $2n$-gon and swapping the two tokens placed at the endpoints of that edge. After a finite number of moves have been performed, it turns out that every two tokens have been swapped exactly once. Prove that some edge has never been chosen. 

RMM 2012 

RMM 2012 problem 1:  Given a finite number of boys and girls, a \textit{sociable set of boys} is a set of boys such that every girl knows at least one boy in that set; and a \textit{sociable set of girls} is a set of girls such that every boy knows at least one girl in that set. Prove that the number of sociable sets of boys and the number of sociable sets of girls have the same parity. (Acquaintance is assumed to be mutual.) \\\\
\textit{(Poland) Marek Cygan} 
RMM 2012 problem 2:  Given a non-isosceles triangle $ABC$, let $D,E$, and $F$ denote the midpoints of the sides $BC,CA$, and $AB$ respectively. The circle $BCF$ and the line $BE$ meet again at $P$, and the circle $ABE$ and the line $AD$ meet again at $Q$. Finally, the lines $DP$ and $FQ$ meet at $R$. Prove that the centroid $G$ of the triangle $ABC$ lies on the circle $PQR$. \\\\
\textit{(United Kingdom) David Monk} 
RMM 2012 problem 3:  Each positive integer is coloured red or blue. A function $f$ from the set of positive integers to itself has the following two properties:
\begin{enumerate}[(a)]
  \item if $x\le y$, then $f(x)\le f(y)$; and
  \item if $x,y$ and $z$ are (not necessarily distinct) positive integers of the same colour and $x+y=z$, then $f(x)+f(y)=f(z)$.
\end{enumerate}
Prove that there exists a positive number $a$ such that $f(x)\le ax$ for all positive integers $x$. \\\\
\textit{(United Kingdom) Ben Elliott} 
RMM 2012 problem 4:  Prove that there are infinitely many positive integers $n$ such that $2^{2^n+1}+1$ is divisible by $n$ but $2^n+1$ is not. \\\\
\textit{(Russia) Valery Senderov} 
RMM 2012 problem 5:  Given a positive integer $n\ge 3$, colour each cell of an $n\times n$ square array with one of $\lfloor (n+2)^2/3\rfloor$ colours, each colour being used at least once. Prove that there is some $1\times 3$ or $3\times 1$ rectangular subarray whose three cells are coloured with three different colours. \\\\
\textit{(Russia) Ilya Bogdanov, Grigory Chelnokov, Dmitry Khramtsov} 
RMM 2012 problem 6:  Let $ABC$ be a triangle and let $I$ and $O$ denote its incentre and circumcentre respectively. Let $\omega_A$ be the circle through $B$ and $C$ which is tangent to the incircle of the triangle $ABC$; the circles $\omega_B$ and $\omega_C$ are defined similarly. The circles $\omega_B$ and $\omega_C$ meet at a point $A'$ distinct from $A$; the points $B'$ and $C'$ are defined similarly. Prove that the lines $AA',BB'$ and $CC'$ are concurrent at a point on the line $IO$. \\\\
\textit{(Russia) Fedor Ivlev} 

RMM 2011 

RMM 2011 problem 1:  Prove that there exist two functions $f,g \colon \mathbb{R} \to \mathbb{R}$, such that $f\circ g$ is strictly decreasing and $g\circ f$ is strictly increasing. \\\\
\textit{(Poland) Andrzej Komisarski and Marcin Kuczma} 
RMM 2011 problem 2:  Determine all positive integers $n$ for which there exists a polynomial $f(x)$ with real coefficients, with the following properties:
\begin{enumerate}[(1)]
  \item for each integer $k$, the number $f(k)$ is an integer if and only if $k$ is not divisible by $n$;
  \item the degree of $f$ is less than $n$.
\end{enumerate}
\textit{(Hungary) Géza Kós} 
RMM 2011 problem 3:  A triangle $ABC$ is inscribed in a circle $\omega$. \\
A variable line $\ell$ chosen parallel to $BC$ meets segments $AB$, $AC$ at points $D$, $E$ respectively, and meets $\omega$ at points $K$, $L$ (where $D$ lies between $K$ and $E$). \\
Circle $\gamma_1$ is tangent to the segments $KD$ and $BD$ and also tangent to $\omega$, while circle $\gamma_2$ is tangent to the segments $LE$ and $CE$ and also tangent to $\omega$. \\
Determine the locus, as $\ell$ varies, of the meeting point of the common inner tangents to $\gamma_1$ and $\gamma_2$. \\\\
\textit{(Russia) Vasily Mokin and Fedor Ivlev} 
RMM 2011 problem 4:  Given a positive integer $\displaystyle n = \prod_{i=1}^s p_i^{\alpha_i}$, we write $\Omega(n)$ for the total number $\displaystyle \sum_{i=1}^s \alpha_i$ of prime factors of $n$, counted with multiplicity. Let $\lambda(n) = (-1)^{\Omega(n)}$ (so, for example, $\lambda(12)=\lambda(2^2\cdot3^1)=(-1)^{2+1}=-1$). \\
Prove the following two claims:
\begin{enumerate}[i)]
  \item There are infinitely many positive integers $n$ such that $\lambda(n) = \lambda(n+1) = +1$;
  \item There are infinitely many positive integers $n$ such that $\lambda(n) = \lambda(n+1) = -1$.
\end{enumerate}
\textit{(Romania) Dan Schwarz} 
RMM 2011 problem 5:  For every $n\geq 3$, determine all the configurations of $n$ distinct points $X_1,X_2,\ldots,X_n$ in the plane, with the property that for any pair of distinct points $X_i$, $X_j$ there exists a permutation $\sigma$ of the integers $\{1,\ldots,n\}$, such that $\textrm{d}(X_i,X_k) = \textrm{d}(X_j,X_{\sigma(k)})$ for all $1\leq k \leq n$. \\
(We write $\textrm{d}(X,Y)$ to denote the distance between points $X$ and $Y$.) \\\\
\textit{(United Kingdom) Luke Betts} 
RMM 2011 problem 6:  The cells of a square $2011 \times 2011$ array are labelled with the integers $1,2,\ldots, 2011^2$, in such a way that every label is used exactly once. We then identify the left-hand and right-hand edges, and then the top and bottom, in the normal way to form a torus (the surface of a doughnut). \\
Determine the largest positive integer $M$ such that, no matter which labelling we choose, there exist two neighbouring cells with the difference of their labels at least $M$. \\
(Cells with coordinates $(x,y)$ and $(x',y')$ are considered to be neighbours if $x=x'$ and $y-y'\equiv\pm1\pmod{2011}$, or if $y=y'$ and $x-x'\equiv\pm1\pmod{2011}$.) \\\\
\textit{(Romania) Dan Schwarz} 

RMM 2010 

RMM 2010 problem 1:  For a finite non empty set of primes $P$, let $m(P)$ denote the largest possible number of consecutive positive integers, each of which is divisible by at least one member of $P$.
\begin{enumerate}[(i)]
  \item Show that $|P|\le m(P)$, with equality if and only if $\min(P)>|P|$.
  \item Show that $m(P)<(|P|+1)(2^{|P|}-1)$.
\end{enumerate}
(The number $|P|$ is the size of set $P$) \\\\
\textit{Dan Schwarz, Romania} 
RMM 2010 problem 2:  For each positive integer $n$, find the largest real number $C_n$ with the following property. Given any $n$ real-valued functions $f_1(x), f_2(x), \cdots, f_n(x)$ defined on the closed interval $0 \le x \le 1$, one can find numbers $x_1, x_2, \cdots x_n$, such that $0 \le x_i \le 1$ satisfying
\[ |f_1(x_1)+f_2(x_2)+\cdots f_n(x_n)-x_1x_2\cdots x_n| \ge C_n \]
\textit{Marko Radovanović, Serbia} 
RMM 2010 problem 3:  Let $A_1A_2A_3A_4$ be a quadrilateral with no pair of parallel sides. For each $i=1, 2, 3, 4$, define $\omega_1$ to be the circle touching the quadrilateral externally, and which is tangent to the lines $A_{i-1}A_i, A_iA_{i+1}$ and $A_{i+1}A_{i+2}$ (indices are considered modulo $4$ so $A_0=A_4, A_5=A_1$ and $A_6=A_2$). Let $T_i$ be the point of tangency of $\omega_i$ with the side $A_iA_{i+1}$. Prove that the lines $A_1A_2, A_3A_4$ and $T_2T_4$ are concurrent if and only if the lines $A_2A_3, A_4A_1$ and $T_1T_3$ are concurrent. \\\\
\textit{Pavel Kozhevnikov, Russia} 
RMM 2010 problem 4:  Determine whether there exists a polynomial $f(x_1, x_2)$ with two variables, with integer coefficients, and two points $A=(a_1, a_2)$ and $B=(b_1, b_2)$ in the plane, satisfying the following conditions:
\begin{enumerate}[(i)]
  \item $A$ is an integer point (i.e $a_1$ and $a_2$ are integers);
  \item $|a_1-b_1|+|a_2-b_2|=2010$;
  \item $f(n_1, n_2)>f(a_1, a_2)$ for all integer points $(n_1, n_2)$ in the plane other than $A$;
  \item $f(x_1, x_2)>f(b_1, b_2)$ for all integer points $(x_1, x_2)$ in the plane other than $B$.
\end{enumerate}
\textit{Massimo Gobbino, Italy} 
RMM 2010 problem 5:  Let $n$ be a given positive integer. Say that a set $K$ of points with integer coordinates in the plane is connected if for every pair of points $R, S\in K$, there exists a positive integer $\ell$ and a sequence $R=T_0,T_1, T_2,\ldots ,T_{\ell}=S$ of points in $K$, where each $T_i$ is distance $1$ away from $T_{i+1}$. For such a set $K$, we define the set of vectors
\[ \Delta(K)=\{\overrightarrow{RS}\mid R, S\in K\} \]
What is the maximum value of $|\Delta(K)|$ over all connected sets $K$ of $2n+1$ points with integer coordinates in the plane? \\\\
\textit{Grigory Chelnokov, Russia} 
RMM 2010 problem 6:  Given a polynomial $f(x)$ with rational coefficients, of degree $d \ge 2$, we define the sequence of sets $f^0(\mathbb{Q}), f^1(\mathbb{Q}), \ldots$ as $f^0(\mathbb{Q})=\mathbb{Q}$, $f^{n+1}(\mathbb{Q})=f(f^n(\mathbb{Q}))$ for $n\ge 0$. (Given a set $S$, we write $f(S)$ for the set $\{f(x)\mid x\in S\})$. \\
Let $f^{\omega}(\mathbb{Q})=\bigcap_{n=0}^{\infty} f^n(\mathbb{Q})$ be the set of numbers that are in all of the sets $f^n(\mathbb{Q})$, $n\geq 0$. Prove that $f^{\omega}(\mathbb{Q})$ is a finite set. \\\\
\textit{Dan Schwarz, Romania} 

RMM 2009 

RMM 2009 problem 1:  For $ a_i \in \mathbb{Z}^ +$, $ i = 1, \ldots, k$, and $ n = \sum^k_{i = 1} a_i$, let $ d = \gcd(a_1, \ldots, a_k)$ denote the greatest common divisor of $ a_1, \ldots, a_k$. \\
Prove that $ \frac {d} {n} \cdot \frac {n!}{\prod\limits^k_{i = 1} (a_i!)}$ is an integer. \\\\
\textit{Dan Schwarz, Romania} 
RMM 2009 problem 2:  A set $ S$ of points in space satisfies the property that all pairwise distances between points in $ S$ are distinct. Given that all points in $ S$ have integer coordinates $ (x,y,z)$ where $ 1 \leq x,y, z \leq n,$ show that the number of points in $ S$ is less than $ \min \Big((n + 2)\sqrt {\frac {n}{3}}, n \sqrt {6}\Big).$ \\\\
\textit{Dan Schwarz, Romania} 
RMM 2009 problem 3:  Given four points $ A_1, A_2, A_3, A_4$ in the plane, no three collinear, such that
\[ A_1A_2 \cdot A_3 A_4 = A_1 A_3 \cdot A_2 A_4 = A_1 A_4 \cdot A_2 A_3, \]
denote by $ O_i$ the circumcenter of $ \triangle A_j A_k A_l$ with $ \{i,j,k,l\} = \{1,2,3,4\}.$ Assuming $ \forall i A_i \neq O_i ,$ prove that the four lines $ A_iO_i$ are concurrent or parallel. \\\\
\textit{Nikolai Ivanov Beluhov, Bulgaria} 
RMM 2009 problem 4:  For a finite set $ X$ of positive integers, let $ \Sigma(X) = \sum_{x \in X} \arctan \frac{1}{x}.$ Given a finite set $ S$ of positive integers for which $ \Sigma(S) < \frac{\pi}{2},$ show that there exists at least one finite set $ T$ of positive integers for which $ S \subset T$ and $ \Sigma(S) = \frac{\pi}{2}.$ \\\\
\textit{Kevin Buzzard, United Kingdom} 

RMM 2008 

RMM 2008 problem 1:  Let $ ABC$ be an equilateral triangle and $ P$ in its interior. The distances from $ P$ to the triangle's sides are denoted by $ a^2, b^2,c^2$ respectively, where $ a,b,c>0$. Find the locus of the points $ P$ for which $ a,b,c$ can be the sides of a non-degenerate triangle. 
RMM 2008 problem 2:  Prove that every bijective function $ f: \mathbb{Z}\rightarrow\mathbb{Z}$ can be written in the way $ f=u+v$ where $ u,v: \mathbb{Z}\rightarrow\mathbb{Z}$ are bijective functions. 
RMM 2008 problem 3:  Let $ a>1$ be a positive integer. Prove that every non-zero positive integer $ N$ has a multiple in the sequence $ (a_n)_{n\ge1}$, $ a_n=\left\lfloor\frac{a^n}n\right\rfloor$. 
RMM 2008 problem 4:  Strange problem. Let $ N=N_1N_2$, where $ N_2$ and $ a$ are coprime, and $ N_1$ divides $ a^C$ for suitable positive integer $ C$. Without loss of generality, $ CN_1$ divides $ a^C$ (we may take $ C$ being large power of $ a$). Take large prime $ p$ such that $ p-1$ is divisible by $ \varphi(n)$. The existence of such $ p$ is well-known elementary case of the Dirichlet theorem, which may be obtained using the cyclotomic polynomials. Then take $ n=pC$. Then by Fermat's little theorem we have $ [a^{Cp}/(Cp)]=(a^{Cp}-a^C)/(Cp)$, which is divisible by $ N=N_1N_2$. 
RMM 2008 problem 5:  Consider a square of sidelength $ n$ and $ (n+1)^2$ interior points. Prove that we can choose $ 3$ of these points so that they determine a triangle (eventually degenerated) of area at most $ \frac12$. 
