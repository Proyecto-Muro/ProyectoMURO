Consideramos un tablero de $8 \times  8$. El Batab es una pieza que puede moverse de una casilla a otra vecina (que comparte un lado). Un camino del Mayab es un camino que va de una casilla inicial a una final tal que:
<ul>
<li> Consta exclusivamente de movimientos del Batab. </li>
<li> En cada paso se aleja del punto inicial y se acerca al punto final.</li>
</ul>
Se coloca una ficha verde en una casilla y una ficha naranja en otra distinta, luego se coloca una ficha blanca en una casilla que está dentro de un camino del Mayab que va de la ficha verde a la ficha naranja. Llamamos $T$ al número total de caminos del Mayab que van de la ficha verde a la naranja pasando por la ficha blanca. Encuentra el número total de formas distintas en que se pueden colocar las tres fichas de modo que $49$ divida a $T$.