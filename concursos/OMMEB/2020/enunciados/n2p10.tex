Se consideran todos los enteros positivos que se construyen escribiendo consecutivamente los primeros números enteros positivos. Por ejemplo, el que aparece en el primer lugar es el $1$, en segundo lugar aparece $12$, en el lugar $3$ aparece $123$, y así sucesivamente (en el lugar $12$ aparece $123456789101112$). De los números anteriores, ¿cuántos dígitos tiene el número más pequeño en el que aparece la cadena $2022$? Por ejemplo, el número de dígitos del número más pequeño en el que aparece la cadena $91$ es $11$ pues aparece por primera vez en el décimo número $12345678910$, el cual tiene $11$ dígitos.