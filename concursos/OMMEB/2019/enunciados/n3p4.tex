En un rectángulo $ABCD$ de área $40cm^2$, considera a $E$ y $G$, puntos en la diagonal $AC$ de manera que $AE = 2cm$, $EG = 6cm$ y $GC = 2cm$; considera también los puntos $F$ y $H$ en la diagonal $BD$, de manera que $BF = 2cm$, $FH = 6cm$ y $HD = 2cm$. Se construye una estrella de $4$ puntas $APBQCRDSA$, de manera que $P, Q, R, S$ son los puntos medios de los segmentos $EF, FG, GH y HE$, respectivamente. ¿Cuál es, en $cm^2$, el área de la estrella?