En una carrera de atletismo los corredores $A$, $B$ y $C$ fueron los primeros en llegar a la meta. Cuando $A$ llegó a la meta, los corredores $B$ y $C$ se encontraban a $2$ metros y a $2.98$ metros de $A$, respectivamente. Cuando llego $B$ a la meta el corredor $C$ estaba a $1$ metro de la meta. Suponga que cada una de las velocidades de los corredores son constantes durante la carrera. ¿De cuántos metros es la carrera?