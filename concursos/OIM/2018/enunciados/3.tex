En un plano tenemos $n$ rectas sin que haya dos paralelas, ni dos perpendiculares, ni tres concurrentes. Se elige un sistema de ejes cartesianos con una de las $n$ rectas como eje de las abscisas. Un punto $P$ se sitúa en el origen de coordenadas del sistema elegido y comienza a moverse a velocidad constante por la parte positiva del eje de las abscisas. Cada vez que $P$ llega a la intersección de dos rectas, sigue por la recta recién alcanzada en el sentido que permite que el valor de la abscisa de $P$ sea siempre creciente. Demostrar que se puede elegir el sistema de ejes cartesianos de modo que $P$ pase por puntos de las $n$ rectas.<br>
Nota: El eje de las abscisas de un sistema de coordenadas del plano es el eje de la primera coordenada o eje de las $x$.
