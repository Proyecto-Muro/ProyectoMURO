Se construye un polígono regular de $n$ lados ($n \geq 3$) y se enumeran sus vértices de $1$ a $n$. Se trazan todas las diagonales del polígono. Demostrar que si $n$ es impar, se puede asignar a cada lado y a cada diagonal un número entero de $1$ a $n$, tal que se cumplan simultáneamente las siguientes condiciones:
<ol>
  <li> El número asignado a cada lado o diagonal sea distinto a los asignados a los vértices que une.</li>
  <li> Para cada vértice, todos los lados y diagonales que compartan dicho vértice tengan números diferentes.</li>
</ol>
