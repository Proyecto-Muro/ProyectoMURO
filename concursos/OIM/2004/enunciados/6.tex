Para un conjunto $H$ de puntos en el plano, se dice que un punto $P$ del plano es un <em>punto de corte</em> de $H$ si existen cuatro puntos distintos $A$, $B$, $C$ y $D$ en $H$ tales que las rectas $AB$ y $CD$ son distintas y se cortan en $P$.<br>
Dado un conjunto finito $A_0$ de puntos en el plano, se construye una sucesión de conjuntos $A_1, A_2, A_3, \cdots$ de la siguiente manera: para cualquier $j \geq 0$, $A_{j+1}$ es la unión de $A_j$ con el conjunto de todos los puntos de corte de $A_j$.<br>
Demostrar que si la unión de todos los conjuntos de la sucesión es un conjunto finito, entonces para cualquier $j \geq 1$ se tiene que $A_j = A_1$.

