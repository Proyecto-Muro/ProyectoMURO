Tres fichas $A$, $B$ y $C$ están situadas cada una en un vértice de un triángulo equilátero de lado $n$. Se ha dividido el triángulo en $n^2$ triangulitos equiláteros de lado $1$.<br>
Inicialmente todas las líneas de la figura están pintadas de azul. Las fichas se desplazan por las líneas, pintando de rojo su trayectoria, de acuerdo con las dos reglas siguientes:
<ol type="i">
  <li> Primero se mueve $A$, después $B$, después $C$, después $A$ y así sucesivamente, por turnos. En cada turno, cada ficha recorre exactamente un lado de un triángulito, de un extremo a otro.</li>
  <li> Ninguna ficha puede recorrer un lado de un triangulito que ya esté pintado de rojo; pero puede descansar en un extremo pintado, incluso si ya hay otra ficha esperando ahí su turno.
</ol>
Demuestra que para todo entero $n>0$ es posible pintar de rojo todos los lados de los triangulitos.
