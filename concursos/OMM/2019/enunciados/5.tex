Sean $a > b$ enteros positivos relativamente primos. Un saltamontes se sitúa en el punto $0$ de una recta numérica. Cada minuto, el saltamontes salta de acuerdo con las siguientes reglas:

    Si el minuto actual es un múltiplo de $a$ y no un múltiplo de $b$, salta $a$ unidades hacia adelante.
    Si el minuto actual es un múltiplo de $b$ y no un múltiplo de $a$, salta $b$ unidades hacia atrás.
    Si el minuto actual es tanto un múltiplo de $b$ como un múltiplo de $a$, salta $a - b$ unidades hacia adelante.
    Si el minuto actual no es ni un múltiplo de $a$ ni un múltiplo de $b$, no se mueve.

Encuentra todas las posiciones de la recta numérica que el saltamontes acabará alcanzando.