Sea $n$ un entero positivo y sean $d_1, d_2, \dots , d_k$ todos sus divisores positivos ordenados de
menor a mayor. Considera el número
\[f (n) = (−1)^{d_1} d_1 + (−1)^{d_2} d_2 + \dots + (−1)^{d_k} d_k\]
Por ejemplo, los divisores positivos de $10$ son $1$, $2$, $5$ y $10$, así que
\[f (10) = (−1)^1 \cdot 1 + (−1)^2 \cdot 2 + (−1)^5 \cdot 5 + (−1)^{10} \cdot 10 = 6.\]
Supón que $f (n)$ es una potencia de $2$. Muestra que si $m$ es un entero mayor que $1$, entonces
$m^2$ no divide a $n$.