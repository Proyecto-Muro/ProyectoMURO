Algunas ranas, unas de ellas rojas y otras verdes, se van a mover en un tablero de $11 \times 11$,
de acuerdo a las siguientes reglas. Si una rana está ubicada, digamos, en la casilla marcada
con $\#$ en la siguiente figura, entonces si es roja, puede saltar a cualquiera de las casillas marcadas con $\times$. Si es verde, puede saltar a cualquiera de las casillas marcadas con $\circ$.
Diremos que dos ranas (de cualquier color) se pueden encontrar en una casilla si ambas
pueden llegar hasta ella saltando una o más veces, no necesariamente con el mismo número
de saltos.
(a) Muestra que si ponemos $6$ ranas (de cualquier color), entonces hay al menos $2$ que se
pueden encontrar.
(b) ¿Para qué valores de $k$ es posible poner una rana roja y una rana verde de manera
que haya exactamente $k$ casillas en las que estas ranas se pueden encontrar?