Sea $n \geq 4$ un número par. Considera una cuadrícula de $n \times n$. Dos celdas (cuadritos
de $1 \times 1$) son vecinas si comparten un lado, si están en extremos opuestos de un mismo
renglón o si están en extremos opuestos de una misma columna. De esta forma, toda celda
en la cuadrícula tiene exactamente cuatro celdas vecinas.
En cada celda está escrito un número del $1$ al $4$ de acuerdo con las siguientes reglas:

Si en una celda está escrito un $2$ entonces en dos o más celdas vecinas está escrito
un $1$.

Si en una celda está escrito un $3$ entonces en tres o más celdas vecinas está escrito
un $1$.

Si en una celda está escrito un $4$ entonces en las cuatro celdas vecinas está escrito
un $1$.

Entre los acomodos que cumplan las condiciones anteriores, ¿cual el máximo número que
se puede obtener sumando los números escritos en todas las celdas?