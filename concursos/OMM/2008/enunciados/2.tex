Considera una circunferencia $\Gamma$, un punto $A$ fuera de $\Gamma$ y las tangentes $AB$, $AC$ a $\Gamma$ desde
$A$, con $B$ y $C$ los puntos de tangencia. Sea $P$ un punto sobre el segmento $AB$, distinto de
$A$ y de $B$. Considera el punto $Q$ sobre el segmento $AC$ tal que $P Q$ es tangente a $\Gamma$, y a los
puntos $R$ y $S$ que están sobre las rectas $AB$ y $AC$, respectivamente, de manera que $RS$
es paralela a $P Q$ y tangente a $\Gamma$. Muestra que el producto de las áreas de los triángulos
$APQ$ y $ARS$ no depende de la elección del punto $P$ .