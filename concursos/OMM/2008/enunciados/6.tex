Las bisectrices internas de los ángulos $A, B$ y $C$ de un triángulo $ABC$ concurren en $I$
y cortan al circuncírculo de $ABC$ en $L, M, N$, respectivamente. La circunferencia de
diámetro $IL$ corta al lado $BC$, en $D$ y $E$; la circunferencia de diámetro $IM$ corta al lado
$CA$ en $F$ y $G$; la circunferencia de diámetro $IN$ corta al lado $AB$ en $H$ y $J$. Muestra que
$D, E, F , G, H, J$ están sobre una misma circunferencia.