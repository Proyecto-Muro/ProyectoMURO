Sean $\mathcal{C}_1$ y $\mathcal{C}_2$ dos circunferencias tangentes externamente en $S$ tales que el radio de $\mathcal{C}_2$ es el triple del radio de $\mathcal{C}_1$. Sea $\ell$ una recta que es tangente a $\mathcal{C}_1$ en $P$ y tangente a $\mathcal{C}_2$ en $Q$, con $P$ y $Q$ distintos de $S$. Sea $T$ el punto en $\mathcal{C}_2$ tal que $TQ$ es diámetro de $\mathcal{C}_2$ y sea $R$ la
intersección de la bisectriz de $\angle SQT$ con el segmento $ST$ . Demuestra que $QR = RT$.