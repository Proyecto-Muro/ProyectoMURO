Dado un conjunto $A$ de enteros positivos, el conjunto $A'$ se compone de los elementos de $A$ y de todos los enteros positivos que se pueden obtener de la siguiente manera: \\
Se escriben algunos elementos de $A$ uno tras otro sin repetir, se escribe un signo $+ $ o $-$ antes de cada uno de ellos, y se evalúa la expresión obtenida. El resultado se incluye en $A'$. \\
Por ejemplo, si $A = \{2,8,13,20\}$, los números $8$ y $14 = 20-2+8$ son elementos de $A'$. \\
El conjunto $A''$ se construye a partir de $A'$ de la misma manera. \\
Hallar el menor número posible de elementos de $A$, si $A''$ contiene todos los enteros de $1$ a $40$. 