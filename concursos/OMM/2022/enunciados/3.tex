Sea $n>1$ un entero y sean $d_1\lt d_2\lt \dots\lt d_m$ la lista completa de sus divisores positivos, incluyendo $1$ y $n$. Los $m$ instrumentos de una orquesta matemática se disponen a tocar una pieza musical de $m$ segundos, donde el instrumento $i$ tocará una nota de tono $d_i$ durante $s_i$ segundos (no necesariamente consecutivos), donde $d_i$ y $s_i$ son enteros positivos. Decidimos que esta pieza tiene sonoridad $S=s_1+\dots+s_m$. Un par de notas de tonos $a$ y $b$ son armónicas si $\frac ab$ o $\frac ba$ es un entero. Si cada instrumento toca al menos un segundo y cada par de notas que suenan al mismo tiempo son armónicas, demuestra que la máxima sonoridad posible de la pieza es un número compuesto.