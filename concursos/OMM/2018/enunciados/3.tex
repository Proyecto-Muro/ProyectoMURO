Una secuencia $a_2, a_3, \dots, a_n$ de enteros positivos se dice que es campechana, si para cada $i$ tal que $2 \leq i \leq n$ se cumple que exactamente $a_i$ términos de la secuencia son relativamente primos de $i$. Decimos que el tamaño de dicha sucesión es $n - 1$. Sea $m = p_1p_2 \dots p_k$, donde $p_1, p_2, \dots, p_k$ son primos distintos por parejas y $k \geq 2$. Muestra que existen al menos dos secuencias campechanas diferentes de tamaño $m$.