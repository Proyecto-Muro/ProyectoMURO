Sea $ABC$ un triángulo acutángulo escálelo con $\angle BAC=60^{\circ}$ y ortocentro $H$. Sean $\omega_b$ la circunferencia que pasa por $H$ y es tangente a $AB$ en $B$, y $\omega_c$ la circunferencia que pasa por $H$ y es tangente a $AC$ en $C$. Prueba que $\omega_b$ y $\omega_c$ solamente tienen a $H$ como punto común. Prueba que la recta que pasa por $H$ y el circuncentro $O$ del triángulo $ABC$ es una tangente común a $\omega_b$ y $\omega_c$.