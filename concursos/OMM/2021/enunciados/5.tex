Para cada entero $n\lt 0$ con expansión decimal $\overline{a_1a_2\cdots a_{k-1}a_k}$, definimos a $s(n)$ como sigue: 

Si $n$ es par, $s(n)=\overline{a_1a_2}+\overline{a_3a_4}\cdots+\overline{a_{k-1}a_k}$.

Si $n$ es impar, $s(n)=a_1+\overline{a_2a_3}\cdots+\overline{a_{k-1}a_k}$

Por ejemplo, si $n=123$, entonces $s(n)=1+23=24$ y si $n=2021$ entonces $s(n)=20+21=41$.

Decimos que $n$ es "digital" si $n$ es múltiplo de $s(n)$. Muestra que entre cualesquiera $198$ enteros positivos consecutivos, todos ellos menoress a $2000021$, hay uno de ellos que es digital.