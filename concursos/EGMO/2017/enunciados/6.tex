Sea ABC un triángulo acutángulo que no tiene dos lados con la misma longitud.
Las reflexiones del gravicentro $G$ y el circuncentro $O$ de $ABC$ con respecto a los lados $BC$, $CA$, $AB$
se denotan como $G_1, G_2, G_3$, y $O_1, O_2, O_3$, respectivamente. Demuestra que los circuncírculos de los
triángulos $G_1G_2C, G_1G_3B, G_2G_3A, O_1O_2C, O_1O_3B, O_2O_3A$ y $ABC$ tienen un punto en común.