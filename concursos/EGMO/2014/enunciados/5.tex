Sea $n$ un entero positivo. Se tienen $n$ cajas y cada caja contiene un número no negativo
de fıchas. Un movimiento consiste en tomar dos fıchas de una de las cajas, dejar una fuera de las
cajas y poner la otra en otra caja. Decimos que una configuración de fıchas es resoluble si es posible
aplicar un número finito de movimientos (que puede ser igual a cero) para obtener una configuración
en la que no haya cajas vacías. Determinar todas las configuraciones iniciales de fıchas que no son
resolubles y se vuelven resolubles al agregar una fıcha en cualquiera de las cajas (sin importar en cual
caja se pone la fıcha).