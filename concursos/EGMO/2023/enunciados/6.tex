Sea $ABC$ un triángulo con circunferencia circunscrita $\Omega$. Se denota por $S_b$ y $S_c$ los puntos medios de los arcos $AC$ y $AB$ que no contienen el tercer vértice del triángulo, respectivamente. Sea $N_a$ el punto medio del arco $BAC$ (el arco $BC$ que contiene a $A$). Sea $I$ el incentro de $ABC$. Sea $\omega_b$ el círculo que es tangente a $AB$ y tangente internamente a $\Omega$ en $S_b$, y sea $\omega_c$ el círculo que es tangente a $AB$ y tangente internamente a $\Omega$ en $S_c$. Demuestre que la recta $IN_a$ y la recta que pasa por las intersecciones de $\omega_b$ y $\omega_c$, se intersectan sobre $\Omega$.