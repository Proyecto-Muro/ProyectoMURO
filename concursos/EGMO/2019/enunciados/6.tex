Alina traza $2019$ cuerdas en una circunferencia. Los puntos extremos de éstas son
todos diferentes. Un punto se considera marcado si es de uno de los siguientes tipos:
(i) uno de los $4038$ puntos extremos de las cuerdas; o
(ii) un punto de intersección de al menos dos de las cuerdas.
Alina etiqueta con un número cada punto marcado. De los $4038$ puntos del tipo (i), $2019$ son etiquetados con un $0$ y los otros $2019$ puntos con un $1$. Ella etiqueta cada punto del tipo (ii) con un entero
arbitrario, no necesariamente positivo.
En cada cuerda, Alina considera todos los segmentos entre puntos marcados consecutivos (si una
cuerda tiene $k$ puntos marcados, entonces tiene $k − 1$ de estos segmentos). Sobre cada uno de estos
segmentos, Alina escribe dos números: en amarillo escribe la suma de las etiquetas de los puntos
extremos del segmento, mientras que en azul escribe el valor absoluto de su diferencia.
Alina se da cuenta que los $N + 1$ números amarillos son exactamente los números $0, 1,\dots , N$.
Muestre que al menos uno de los números azules es múltiplo de tres.
Nota: Una cuerda es el segmento de recta que une dos puntos distintos en una circunferencia.