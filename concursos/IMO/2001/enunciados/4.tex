Sea $n$ un entero impar mayor a $1$ y sean $c_1, c_2, \ldots, c_n$ enteros. Para cada permutación $a = (a_1, a_2, \ldots, a_n)$ de $\{1,2,\ldots,n\}$, definimos $S(a) = \sum_{i=1}^n c_i a_i$. Muestra que existen permutaciones distintas $a$ y $b$ de $\{1,2,\ldots,n\}$ tales que $n!$ es divisor de $S(a)-S(b)$.