Sea $ABC$ un triángulo acutángulo con $AB\neq AC$. El círculo con diámetro $BC$ corta a los lados $AB$ y $AC$ en $M$ y $N$, respectivamente. Sea $O$ el punto medio del lado $BC$. Las bisectrices de los ángulos $\angle BAC$ y $\angle MON$ se intersecan en $R$. Muestra que los ciruncírculos de los triángulos $BMR$ y $CNR$ tienen un punto en común sobre el lado $BC$.