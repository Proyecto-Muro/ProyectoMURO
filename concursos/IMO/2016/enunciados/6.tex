Se tienen $n \ge 2$ segmentos en el plano tales que cada par de segmentos se intersectan en un punto interior a ambos, y no hay tres segmentos que tengan un punto en común. Mafalda debe elegir uno de los extremos de cada segmento y colocar sobre él una rana mirando hacia el otro extremo. Luego silbará $n−1$ veces. En cada silbido, cada rana saltará inmediatamente hacia adelante hasta el siguiente punto de intersección sobre su segmento. Las ranas nunca cambian las direcciones de sus saltos. Mafalda quiere colocar las ranas de tal forma que nunca dos de ellas ocupen al mismo tiempo el mismo punto de intersección.<br>
(a) Demostrar que si $n$ es impar, Mafalda siempre puede lograr su objetivo. <br>
(b) Demostrar que si $n$ es par, Mafalda nunca logrará su objetivo.