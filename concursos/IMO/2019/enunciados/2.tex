En el triángulo $ABC$, el punto $A_1$ está en el lado $BC$ y el punto $B_1$ está en el lado $AC$. Sean $P$ y $Q$ puntos en los segmentos $AA_1$ y $BB_1$, respectivamente, tales que $PQ$ es paralelo a $AB$. Sea $P_1$ un punto en la recta $PB_1$ distinto de $B_1$, con $B_1$ entre $P$ y $P_1$, y $\angle PP_1C = \angle BAC$. Análogamente, sea $Q_1$ un punto en la recta $QA_1$ distinto de $A_1$, con $A_1$ entre $Q$ y $Q_1$, y $\angle CQ_1Q = \angle CBA$.
Demostrar que los puntos $P$, $Q$, $P_1$, y $Q_1$ son concíclicos.
