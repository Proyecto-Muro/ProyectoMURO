Sean $ AH_1, BH_2, CH_3$ las alturas de un triángulo acutángulo $ABC$. Su incírculo toca a los lados $BC$, $AC$ y $AB$ en $T_1$, $T_2$ y $T_3$, respectivamente. Considera las imágenes simétricas de las líneas $ H_1H_2, H_2H_3$ y $ H_3H_1$ con respecto a las líneas $ T_1T_2, T_2T_3$ y $ T_3T_1$. Muestra que estas imágenes crean un triángulo cuyos vértices caen sobre el incírculo de $ABC$.