Sea $ABC$ un triángulo acutángulo con $AB \lt AC$. Sea $\Omega$ el circuncírculo de $ABC$. Sea $S$ el punto medio del arco $CB$ de $\Omega$ que contiene a $A$. La perpendicular por $A$ a $BC$ corta al segmento $BS$ en $D$ y a $\Omega$ de nuevo en $E \neq A$. La paralela a $BC$ por $D$ corta a la recta $BE$ en $L$. Sea $\omega$ el circuncírculo del triángulo $BDL$. Las circunferencias $\omega$ y $\Omega$ se cortan de nuevo en $P \neq B$. Demuestra que la recta tangente a $\omega$ en $P$ corta a la recta $BS$ en un punto de la bisectriz interior del ángulo $\angle BAC$.