Se tienen 9 palitos de madera: 3 azules de longitud $a$ cada uno, 3 rojos de longitud $r$ cada uno y 3 verdes de longitud $v$ cada uno, tales que es posible formar un triángulo $T$ con palitos de colores todos distintos. Dana puede formar dos arreglos, comenzando con $T$ y utilizando los otros seis palitos para prolongar los lados de $T$, como se muestra en la figura. De esta manera, se pueden formar dos hexágonos cuyos vértices son los extremos de dichos seis palitos. Demuestra que ambos hexágonos tienen la misma área.