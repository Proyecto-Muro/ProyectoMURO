Leticia tiene un tablero de $9\times 9$ casillas. Se dice que dos casillas son amigas si comparten un lado o si están en una misma columna, pero en extremos opuestos, o si están en una misma fila, pero en extremos opuestos. De esta forma, cada casilla tiene exactamente $4$ casillas amigas. Leticia va a pintar cada casilla de uno de tres colores: verde, azul o rojo. Una vez que todas las casillas estñen pintadas, en cada casilla se va a escribir un número, siguiendo las siguientes reglas:
<ul>
<li> Si la casilla es verde, se escribe la cantidad de casillas rojas amigas más dos veces la cantidad de casillas azules amigas.</li>
<li> Si la casilla es roja, se escribe la cantidad de casillas azules amigas más dos veces la cantidad de casillas verdes amigas.</li>
<li> Si la casilla es azul, se escribe la cantidad de casillas verdes amigas más dos veces la cantidad de casillas rojas amigas.</li>
</ul>
Encuentra el máximo valor posible de la suma de los números asignados a las casillas que Leticia puede obtener, sabiendo que ella puede escoger la coloración de las casillas del tablero.
