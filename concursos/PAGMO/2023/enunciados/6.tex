Sea $n \geq 2$ un entero. Lucía escoge $n$ números reales $x_1, x_2, \dots, x_n$ tales que $|x_i - x_j|\geq 1$ para todo $i \neq j.$ Luego, en cada una de las casillas de una cuadrícula $n\times n$ ella escribe alguno de estos números, de modo que no se repite ningún número en una misma fila o una misma columna. Finalmente, para cada casilla, ella calcula el valor absoluto de la diferencia del número en la casilla y el número en la primera casilla de su misma fula. Determinar el menor valor que puede tomar la suma de los $n^2$ números que Lucía clculó.