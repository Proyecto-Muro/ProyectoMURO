La figura siguiente muestra una red hexagonal formada por muchos triángulos equiláteros congruentes. Por turnos, Gabriel y Arnaldo juegan de la siguiente manera. En su turno, el jugador colorea un segmento, incluyendo los puntos extremos, siguiendo estas tres reglas: Los puntos finales deben coincidir con los vértices de los triángulos equiláteros marcados. El segmento debe estar formado por uno o varios de los lados de los triángulos. El segmento no puede contener ningún punto (puntos finales incluidos) de un segmento previamente coloreado.

Gabriel juega primero, y el jugador que no puede hacer un movimiento legal pierde. Encuentra una estrategia ganadora y descríbela.
