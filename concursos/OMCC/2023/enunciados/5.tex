Sean $A B C$ un triángulo acutángulo con $A B\lt A C$ y $\Gamma$ la circunferencia que pasa por $A, B$ y $C$. Sean $D$ el punto diametralmente opuesto a $A$ en $\Gamma$ y $\ell$ la recta tangente en $D$ a $\Gamma$. Sean $P, Q$ y $R$ las intersecciones de $B C$ con $\ell$, de $A P$ con $\Gamma$ tal que $Q \neq A$ y de $Q D$ con la altura del triángulo $A B C$ por $A$, respectivamente. Se definen el punto $S$ como la intersección de $A B$ con la recta $\ell$ y el punto $T$ como la intersección de $A C$ con la recta $\ell$. Probar que $S$ y $T$ pertenecen a la circunferencia que pasa por $A, Q$ y $R$.