Con perlas de diferentes colores formando collares, se dice que un collar es {\itshape primo} si no se puede descomponer en hilos de perlas de la misma longitud, e iguales entre sí. 
Sean $n$ y $q$ enteros positivos. Demostrar que el número de collares primos con $n$ perlas, cada una de las cuales tiene uno de los $q^n$ colores posibles, es igual a $n$ veces el número de collares primos con $n^2$ perlas, cada una de las cuales tiene uno de los $q$ colores posibles.

Nota: dos collares se consideran iguales si tienen el mismo número de perlas y se puede conseguir el mismo color en ambos collares, girando uno de ellos para que coincida con el otro.
