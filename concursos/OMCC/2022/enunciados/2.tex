Ana, Beto, Carlos, Diana, Elena y Fabián se encuentran en un círculo, ubicados en ese orden. Cada uno de Ana, Beto, Carlos, Diana, Elena y Fabián tiene un papel, en los cuales están escritos inicialmente los números reales $a, b, c, d, e, f$, respectivamente. Al final de cada minuto, todas las personas reemplazan simultáneamente el número de su papel por la suma de tres números: los que había al principio del minuto en su papel y en los papeles de sus dos vecinos, el de la derecha y el de la izquierda. Al final del minuto 2022 se han hecho 2022 reemplazos y cada persona tiene escrito en su papel su número inicial. Determine todos los posibles valores de $abc + def$.
vamente, entonces al final del minuto $N$, Beto va a tener el número $x + y + z$.<br>

Nota: Si al inicio del minuto $N$ Ana, Beto y Carlos tienen los números $x,y,z$, respectivamente, entonces al final del minuto $N$, Beto va a tener el número $x + y + z$.