Notemos que si $N$ termina en $0$, como es cuadrado perfecto termina en una cantidad par de ceros. Entonces, el $N'$ obtenido por quitarle todos los ceros al final a $N$ también es cuadrado perfecto. Por ello, es suficiente encontrar todos los $N$ que satisfacen las condiciones y no terminan en cero.\\
Veamos que, por módulo $10$, el último dígito solo puede ser $1,4,6,9$. Entonces, nuestro número se ve como 
\[ 3\underbrace{0\ldots0}_{k\,0\text{'s}}d \]
Si $k=0$, entonces tenemos $d=6$ por fuerza, dando una solución. De lo contrario, $k>0$.\\
Si $d=9$, vemos que $3\mid N$ pero $9\nmid N$ por los criterios de divisibilidad. Esto contradice que $N$ es cuadrado perfecto.\\
Si $d=6$, módulo $4$, $N\equiv2\pmod 4$, pero los residuos cuadráticos módulo $4$ son $0,1$.\\
Si $d=5$, encontramos que $5\mid N$ pero $25\nmid N$, contradiciendo que $N$ es cuadrado perfecto.\\
Si $d=4$, entonces $4\mid N$. Por ello, $N/4$ es cuadrado perfecto. Si $k=1$, $N/4=76$ que no es cuadrado perfecto. De lo contrario, vemos que
\[ N/4=75\underbrace{0\ldots0}_{k-2\,0\text{'s}}1. \]
Sea $x^2=N/4$. Entonces, $75\cdot10^{k-1}=(x-1)(x+1)$. Notemos que el único factor que comparten $x-1$ y $x+1$ es $2$. Entonces, uno de ellos es al menos $5^{k+1}$ y el otro es a lo sumo $3\cdot 2^{k-1}$. Pero $5^{k+1}-3\cdot 2^{k-1}>2$ así que esto es imposible.\\
Si $d=1$, sea $x^2=N$. Entonces, $3\cdot10^{k+1}=(x-1)(x+1)$. Igual que antes, concluímos que uno de los factores es $5^{k+1}$ y el otro es $3\cdot 2^{k+1}$. Pero $5^{k+1}-2\cdot3\cdot2^{k+1}>2$ así que esto es imposible.\\
En conclusión, las únicas posibilidades para $N$ son $36\cdot10^{2k}$, con $k$ en los enteros no-negativos. $\blacksquare$
