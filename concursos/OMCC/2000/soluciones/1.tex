Digamos que $n=abc$. $a^2+b^2+c^2$ divide a $26$, entonces debe ser igual a algún divisor de $26$. Los divisores de $26$ son 
$1,2,13,$ y $26$. Haremos los 4 casos para el valor de $a^2+b^2c^2$.

Caso 1, $a^2+b^2+c^2=1$:

Como $a>0$, la única solución es $n=100$.

Caso 2, $a^2+b^2+c^2=2$:

Si algún dígito es mayor o igual a $2$, la suma de los cuadrados será mayor a $2$. Entonces las 
soluciones son $n=110$ y $n=101$.

Caso 3, $a^2+b^2+c^2=13$:

Todos los dígitos deben ser menores o iguales a $3$. De lo contrario, la suma de cuadrados será 
mayor a $13$. Los únicos cuadrados perfectos que suman 13 son $3^2+2^2$, entonces las soluciones son $n=320,302,230,203$. 

Caso 4, $a^2+b^2+c^2=26$:

Todos los dígitos deben ser menores o iguales a $5$. De lo contrario, la suma de cuadrados será 
mayor a $26$. Los únicos cuadrados perfectos que suman $26$ son $5^2+1^2$ o $4^2+3^2+1^2$. Entonces las soluciones son 
\[n=510,501,150,105,431,413,341,314,143,134\].

Finalmente, juntando todas las soluciones obtenemos:
\[(110,101,320,302,230,203,510,501,150,105,431,413,341,314,143,134).\]
