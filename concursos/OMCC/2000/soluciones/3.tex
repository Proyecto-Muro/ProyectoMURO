Sean $X, Y, Z, W$ los puntos medios de $AB$, $BC$, $CD$ y $AD$ respectivamente. Observa que hay una homotecia de factor 3 centrada en $E$ que transforma $PQRS$ en $XYZW$\footnote{El cuadrilátero $XYZW$ suele ser conocido como el cuadrilátero medial de $ABCD$.}. 
\begin{lemma}
    El cuadrilátero $XYZW$ es un paralelogramo.\footnotemark
\end{lemma}\footnotetext{Este lema es bastante conocido, y es relativamente útil (en un concurso no sería necesario demostrarlo). Como dato curioso resolví el problema 2.3 de los selectivos del 2022 de México con él.}
\begin{proof}
    Observa que por tales, $XY \parallel AC \parallel ZW$, y analogamente $XW \parallel BD \parallel YZ$.
    
\end{proof}
De aquí se sigue por la homotecia que observamos, que $PQRS$ es un paralelogramo. Ahora, por la homotecia también tenemos que el ratio entre $[PQRS]$ y $[XYZW]$ es $\frac{1}{9}$, pero es claro que el ratio entre un cuadrilátero y su cuadrilátero medial es un medio, por lo que el ratio entre $[ABCD]$ y $[XYZW]$ es $\frac{1}{2}$, y substituyendo obtenemos el valor deseado.
