
OMCC 2021 

OMCC 2021 problem 1:  An ordered triple $(p, q, r)$ of prime numbers is called \textit{parcera} if $p$ divides $q^2-4$, $q$ divides $r^2-4$ and $r$ divides $p^2-4$. Find all parcera triples. 
OMCC 2021 problem 2:  Let $ABC$ be a triangle and let $\Gamma$ be its circumcircle. Let $D$ be a point on $AB$ such that $CD$ is parallel to the line tangent to $\Gamma$ at $A$. Let $E$ be the intersection of $CD$ with $\Gamma$ distinct from $C$, and $F$ the intersection of $BC$ with the circumcircle of $\bigtriangleup ADC$ distinct from $C$. Finally, let $G$ be the intersection of the line $AB$ and the internal bisector of $\angle DCF$. Show that $E,\ G,\ F$ and $C$ lie on the same circle. 
OMCC 2021 problem 3:  In a table consisting of $2021\times 2021$ unit squares, some unit squares are colored black in such a way that if we place a mouse in the center of any square on the table it can walk in a straight line (up, down, left or right along a column or row) and leave the table without walking on any black square (other than the initial one if it is black). What is the maximum number of squares that can be colored black? 
OMCC 2021 problem 4:  There are $2021$ people at a meeting. It is known that one person at the meeting doesn't have any friends there and another person has only one friend there. In addition, it is true that, given any $4$ people, at least $2$ of them are friends. Show that there are $2018$ people at the meeting that are all friends with each other. \\
\textit{Note. }If $A$ is friend of $B$ then $B$ is a friend of $A$. 
OMCC 2021 problem 5:  Let $n \geq 3$ be an integer and $a_1,a_2,...,a_n$ be positive real numbers such that $m$ is the smallest and $M$ is the largest of these numbers. It is known that for any distinct integers $1 \leq i,j,k \leq n$, if $a_i \leq a_j \leq a_k$ then $a_ia_k \leq a_j^2$. Show that
\[ a_1a_2 \cdots a_n \geq m^2M^{n-2} \]
and determine when equality holds 
OMCC 2021 problem 6:  Let $ABC$ be a triangle with $AB<AC$ and let $M$ be the midpoint of $AC$. A point $P$ (other than $B$) is chosen on the segment $BC$ in such a way that $AB=AP$. Let $D$ be the intersection of $AC$ with the circumcircle of $\bigtriangleup ABP$ distinct from $A$, and $E$ be the intersection of $PM$ with the circumcircle of $\bigtriangleup ABP$ distinct from $P$. Let $K$ be the intersection of lines $AP$ and $DE$. Let $F$ be a point on $BC$ (other than $P$) such that $KP=KF$. Show that $C,\ D,\ E$ and $F$ lie on the same circle. 

OMCC 2020 

OMCC 2020 problem 1:  A four-digit positive integer is called \textit{virtual} if it has the form $\overline{abab}$, where $a$ and $b$ are digits and $a \neq 0$. For example 2020, 2121 and 2222 are virtual numbers, while 2002 and 0202 are not. Find all virtual numbers of the form $n^2+1$, for some positive integer $n$. 
OMCC 2020 problem 2:  Suppose you have identical coins distributed in several piles with one or more coins in each pile. An action consists of taking two piles, which have an even total of coins among them, and redistribute their coins in two piles so that they end up with the same number of coins. \\\\
A distribution is \textit{levelable} if it is possible, by means of 0 or more operations, to end up with all the piles having the same number of coins. \\\\
Determine all positive integers $n$ such that, for all positive integers $k$, any distribution of $nk$ coins in $n$ piles is levelable. 
OMCC 2020 problem 3:  Find all the functions $f: \mathbb{Z}\to \mathbb{Z}$ satisfying the following property: if $a$, $b$ and $c$ are integers such that $a+b+c=0$, then
\[ f(a)+f(b)+f(c)=a^2+b^2+c^2. \] 
OMCC 2020 problem 4:  Consider a triangle $ABC$ with $BC>AC$. The circle with center $C$ and radius $AC$ intersects the segment $BC$ in $D$. Let $I$ be the incenter of triangle $ABC$ and $\Gamma$ be the circle that passes through $I$ and is tangent to the line $CA$ at $A$. The line $AB$ and $\Gamma$ intersect at a point $F$ with $F \neq A$. Prove that $BF=BD$. 
OMCC 2020 problem 5:  Let $P(x)$ be a polynomial with real non-negative coefficients. Let $k$ be a positive integer and $x_1, x_2, \dots, x_k$ positive real numbers such that $x_1x_2\cdots x_k=1$. Prove that
\[ P(x_1)+P(x_2)+\cdots+P(x_k)\geq kP(1). \] 
OMCC 2020 problem 6:  A positive integer $N$ is \textit{interoceanic} if its prime factorization
\[ N=p_1^{x_1}p_2^{x_2}\cdots p_k^{x_k} \]
satisfies
\[ x_1+x_2+\dots +x_k=p_1+p_2+\cdots +p_k. \]
Find all interoceanic numbers less than 2020. 

OMCC 2019 

OMCC 2019 problem 1:  Let $N=\overline{abcd}$ be a positive integer with four digits. We name \textit{plátano power} to the smallest positive integer $p(N)=\overline{\alpha_1\alpha_2\ldots\alpha_k}$ that can be inserted between  the numbers $\overline{ab}$ and $\overline{cd}$ in such a way the new number $\overline{ab\alpha_1\alpha_2\ldots\alpha_kcd}$ is divisible by $N$. Determine the value of $p(2025)$. 
OMCC 2019 problem 2:  We have a regular polygon $P$ with 2019 vertices, and in each vertex there is a coin. Two players \textit{Azul} and \textit{Rojo} take turns alternately, beginning with Azul, in the following way: first, Azul chooses a triangle with vertices in $P$ and colors its interior with blue, then Rojo selects a triangle with vertices in $P$ and colors its interior with red, so that the triangles formed in each move don't intersect internally the previous colored triangles. They continue playing until it's not possible to choose another triangle to be colored. Then, a player wins the coin of a vertex if he colored the greater quantity of triangles incident to that vertex (if the quantities of triangles colored with blue or red incident to the vertex are the same, then no one wins that coin and the coin is deleted). The player with the greater quantity of coins wins the game.  Find a winning strategy for one of the players. \\\\
\textit{Note.} Two triangles can share vertices or sides. 
OMCC 2019 problem 3:  Let $ABC$ be a triangle and $\Gamma$ its circumcircle. Let $D$ be the foot of the altitude from $A$ to the side $BC$, $M$ and $N$ the midpoints of $AB$ and $AC$, and $Q$ the point on $\Gamma$ diametrically opposite to $A$. Let $E$ be the midpoint of $DQ$. Show that the lines perpendicular to $EM$ and $EN$ passing through $M$ and $N$ respectively, meet on $AD$. 
OMCC 2019 problem 4:  Let $ABC$ be a triangle, $\Gamma$ its circumcircle and $l$ the tangent to $\Gamma$ through $A$. The altitudes from $B$ and $C$ are extended and meet $l$ at $D$ and $E$, respectively. The lines $DC$ and $EB$ meet $\Gamma$ again at $P$ and $Q$, respectively. Show that the triangle $APQ$ is isosceles. 
OMCC 2019 problem 5:  Let $a,\ b$ and $c$ be positive real numbers so that $a+b+c=1$. Show that
\[ a\sqrt{a^2+6bc}+b\sqrt{b^2+6ac}+c\sqrt{c^2+6ab}\leq\frac{3\sqrt{2}}{4} \] 
OMCC 2019 problem 6:  A \textit{triminó} is a rectangular tile of $1\times 3$. Is it possible to cover a $8\times8$ chessboard using $21$ triminós, in such a way there remains exactly one $1\times 1$ square without covering? In case the answer is in the affirmative, determine all the possible locations of such a unit square in the chessboard. 

OMCC 2018 

OMCC 2018 problem 1:  There are 2018 cards numbered from 1 to 2018. The numbers of the cards are visible at all times. Tito and Pepe play a game. Starting with Tito, they take turns picking cards until they're finished. Then each player sums the numbers on his cards and whoever has an even sum wins. Determine which player has a winning strategy and describe it. \\\\\\\\
P.S. Proposed by yours truly  :-D 
OMCC 2018 problem 2:  Let $\Delta ABC$ be a triangle inscribed in the circumference $\omega$ of center $O$. Let $T$ be the symmetric of $C$ respect to $O$ and $T'$ be the reflection of $T$ respect to line $AB$. Line $BT'$ intersects $\omega$ again at $R$. The perpendicular to $CT$ through $O$ intersects line $AC$ at $L$. Let $N$ be the intersection of lines $TR$ and $AC$. Prove that $\overline{CN}=2\overline{AL}$. 
OMCC 2018 problem 3:  Let $x, y$ be real numbers such that $x-y, x^2-y^2, x^3-y^3$ are all prime numbers. Prove that $x-y=3$. \\\\
EDIT: Problem submitted by Leonel Castillo, Panama. 
OMCC 2018 problem 4:  Determine all triples $(p, q, r)$ of positive integers, where $p, q$ are also primes, such that $\frac{r^2-5q^2}{p^2-1}=2$. 
OMCC 2018 problem 5:  Let $n$ be a positive integer, $1<n<2018$. For each $i=1, 2, \ldots ,n$ we define the polynomial $S_i(x)=x^2-2018x+l_i$, where $l_1, l_2, \ldots, l_n$ are distinct positive integers. If the polynomial $S_1(x)+S_2(x)+\cdots+S_n(x)$ has at least an integer root, prove that at least one of the $l_i$ is greater or equal than $2018$. 
OMCC 2018 problem 6:  A dance with 2018 couples takes place in Havana. For the dance, 2018 distinct points labeled $0, 1,\ldots, 2017$ are marked in a circumference and each couple is placed on a different point. For $i\geq1$, let $s_i=i\ (\textrm{mod}\ 2018)$ and $r_i=2i\ (\textrm{mod}\ 2018)$. The dance begins at minute $0$. On the $i$-th minute, the couple at point $s_i$ (if there's any) moves to point $r_i$, the couple on point $r_i$ (if there's any) drops out, and the dance continues with the remaining couples. The dance ends after $2018^2$ minutes. Determine how many couples remain at the end. \\\\
Note: If $r_i=s_i$, the couple on $s_i$ stays there and does not drop out. 

OMCC 2017 

OMCC 2017 problem 1:  The figure below shows a hexagonal net formed by many congruent equilateral triangles. Taking turns, Gabriel and Arnaldo play a game as follows. On his turn, the player colors in a segment, including the endpoints, following these three rules:
\begin{enumerate}[i)]
  \item The endpoints must coincide with vertices of the marked equilateral triangles.
  \item The segment must be made up of one or more of the sides of the triangles.
  \item The segment cannot contain any point (endpoints included) of a previously colored segment.
\end{enumerate}
Gabriel plays first, and the player that cannot make a legal move loses. Find a winning strategy and describe it. 
OMCC 2017 problem 3:  We call a pair $(a,b)$ of positive integers, $a<391$, \textit{pupusa} if
\[ \textup{lcm}(a,b)>\textup{lcm}(a,391) \]
Find the minimum value of $b$ across all \textit{pupusa} pairs. \\\\
Fun Fact: OMCC 2017 was held in El Salvador. \textit{Pupusa} is their national dish. It is a corn tortilla filled with cheese, meat, etc. 
OMCC 2017 problem 4:  Let $ABC$ be a triangle and $D$ be the foot of the altitude from $A$. Let $l$ be the line that passes through the midpoints of $BC$ and $AC$. $E$ is the reflection of $D$ over $l$. Prove that the circumcentre of $\triangle ABC$ lies on the line $AE$. 
OMCC 2017 problem 5:  $ABC$ is a right-angled triangle, with $\angle ABC = 90^{\circ}$. $B'$ is the reflection of $B$ over $AC$. $M$ is the midpoint of $AC$. We choose $D$ on $\overrightarrow{BM}$, such that $BD = AC$. Prove that $B'C$ is the angle bisector of $\angle MB'D$. \\\\
NOTE: An important condition not mentioned in the original problem is $AB<BC$. Otherwise, $\angle MB'D$ is not defined or $B'C$ is the external bisector. 
OMCC 2017 problem 6:  Susana and Brenda play a game writing polynomials on the board. Susana starts and they play taking turns.
\begin{enumerate}[1)]
  \item On the preparatory turn (turn 0), Susana choose a positive integer $n_0$ and writes the polynomial $P_0(x)=n_0$.
  \item On turn 1, Brenda choose a positive integer $n_1$, different from $n_0$, and either writes the polynomial
\end{enumerate}
\[ P_1(x)=n_1x+P_0(x) \textup{  or  } P_1(x)=n_1x-P_0(x) \]
3) In general, on turn $k$, the respective player chooses an integer $n_k$, different from $n_0, n_1, \ldots, n_{k-1}$, and either writes the polynomial
\[ P_k(x)=n_kx^k+P_{k-1}(x) \textup{  or  } P_k(x)=n_kx^k-P_{k-1}(x) \]
The first player to write a polynomial with at least one whole whole number root wins. Find and describe a winning strategy. 
OMCC 2017 problem 8:  Tita the Frog sits on the number line. She is initially on the integer number $k>1$. If she is sitting on the number $n$, she hops to the number $f(n)+g(n)$, where $f(n)$ and $g(n)$ are, respectively, the biggest and smallest positive prime numbers that divide $n$. Find all values of $k$ such that Tita can hop to infinitely many distinct integers. 

OMCC 2016 

OMCC 2016 problem 1:  Find all positive integers $n$ that have 4 digits, all of them perfect squares, and such that $n$ is divisible by 2, 3, 5 and 7. 
OMCC 2016 problem 2:  Let $ABC$ be an acute-angled triangle, $\Gamma$ its circumcircle and $M$ the midpoint of $BC$. Let $N$ be a point in the arc $BC$ of $\Gamma$ not containing $A$ such that $\angle NAC= \angle BAM$. Let $R$ be the midpoint of $AM$, $S$ the midpoint of $AN$ and $T$ the foot of the altitude through $A$. Prove that $R$, $S$ and $T$ are collinear. 
OMCC 2016 problem 3:  The polynomial $Q(x)=x^3-21x+35$ has three different real roots. Find real numbers $a$ and $b$ such that the polynomial $x^2+ax+b$ cyclically permutes the roots of $Q$, that is, if $r$, $s$ and $t$ are the roots of $Q$ (in some order) then $P(r)=s$, $P(s)=t$ and $P(t)=r$. 
OMCC 2016 problem 4:  The number ``3" is written on a board. Ana and Bernardo take turns, starting with Ana, to play the following game. If the number written on the board is $n$, the player in his/her turn must replace it by an integer $m$ coprime with $n$ and such that $n<m<n^2$. The first player that reaches a number greater or equal than 2016 loses. Determine which of the players has a winning strategy and describe it. 
OMCC 2016 problem 5:  We say a number is irie if it can be written in the form $1+\dfrac{1}{k}$ for some positive integer $k$. Prove that every integer $n \geq 2$ can be written as the product of $r$ distinct irie numbers for every integer $r \geq n-1$. 
OMCC 2016 problem 6:  Let $\triangle ABC$ be triangle with incenter $I$ and circumcircle $\Gamma$. Let $M=BI\cap \Gamma$ and $N=CI\cap \Gamma$, the line parallel to $MN$ through $I$ cuts $AB$, $AC$ in $P$ and $Q$. Prove that the circumradius of $\odot (BNP)$ and $\odot (CMQ)$ are equal. 

OMCC 2015 

OMCC 2015 problem 1:  We wish to write $n$ distinct real numbers $(n\geq3)$ on the circumference of a circle in such a way that each number is equal to the product of its immediate neighbors to the left and right. Determine all of the values of $n$ such that this is possible. 
OMCC 2015 problem 2:  A sequence $(a_n)$ of real numbers is defined by $a_0=1$, $a_1=2015$ and for all $n\geq1$, we have
\[ a_{n+1}=\frac{n-1}{n+1}a_n-\frac{n-2}{n^2+n}a_{n-1}. \]
Calculate the value of $\frac{a_1}{a_2}-\frac{a_2}{a_3}+\frac{a_3}{a_4}-\frac{a_4}{a_5}+\ldots+\frac{a_{2013}}{a_{2014}}-\frac{a_{2014}}{a_{2015}}$. 
OMCC 2015 problem 3:  Let $ABCD$ be a cyclic quadrilateral with $AB<CD$, and let $P$ be the point of intersection of the lines $AD$ and $BC$.The circumcircle of the triangle $PCD$ intersects the line $AB$ at the points $Q$ and $R$. Let $S$ and $T$ be the points where the tangents from $P$ to the circumcircle of $ABCD$ touch that circle.
\begin{enumerate}[(a)]
  \item Prove that $PQ=PR$.
  \item Prove that $QRST$ is a cyclic quadrilateral.
\end{enumerate} 
OMCC 2015 problem 4:  Anselmo and Bonifacio start a game where they alternatively substitute a number written on a board. In each turn, a player can substitute the written number by either the number of divisors of the written number or by the difference between the written number and the number of divisors it has. Anselmo is the first player to play, and whichever player is the first player to write the number $0$ is the winner. Given that the initial number is $1036$, determine which player has a winning strategy and describe that strategy. \\\\
Note: For example, the number of divisors of $14$ is $4$, since its divisors are $1$, $2$, $7$, and $14$. 
OMCC 2015 problem 5:  Let $ABC$ be a triangle such that $AC=2AB$. Let $D$ be the point of intersection of the angle bisector of the angle $CAB$ with $BC$. Let $F$ be the point of intersection of the line parallel to $AB$ passing through $C$ with the perpendicular line to $AD$ passing through $A$. Prove that $FD$ passes through the midpoint of $AC$. 
OMCC 2015 problem 6:  $39$ students participated in a math competition. The exam consisted of $6$ problems and each problem was worth $1$ point for a correct solution and $0$ points for an incorrect solution. For any $3$ students, there is at most $1$ problem that was not solved by any of the three. Let $B$ be the sum of all of the scores of the $39$ students. Find the smallest possible value of $B$. 

OMCC 2014 

OMCC 2014 problem 1:  A positive integer is called \textit{tico} if it is the product of three different prime numbers that add up to 74. Verify that 2014 is tico. Which year will be the next tico year? Which one will be the last tico year in history? 
OMCC 2014 problem 2:  Let $ABCD$ be a trapezoid with bases $AB$ and $CD$, inscribed in a circle of center $O$. Let $P$ be the intersection of the lines $BC$ and $AD$. A circle through $O$ and $P$ intersects the segments $BC$ and $AD$ at interior points $F$ and $G$, respectively. Show that $BF=DG$. 
OMCC 2014 problem 3:  Let $a$, $b$, $c$ and $d$ be real numbers such that no two of them are equal,
\[ \frac{a}{b}+\frac{b}{c}+\frac{c}{d}+\frac{d}{a}=4 \]
and $ac=bd$. Find the maximum possible value of
\[ \frac{a}{c}+\frac{b}{d}+\frac{c}{a}+\frac{d}{b}. \] 
OMCC 2014 problem 4:  Using squares of side 1, a stair-like figure is formed in stages following the pattern of the drawing. \\\\
For example, the first stage uses 1 square, the second uses 5, etc. Determine the last stage for which the corresponding figure uses less than 2014 squares. \\\\
http://www.artofproblemsolving.com/Forum/download/file.php?id=49934 
OMCC 2014 problem 5:  Points $A$, $B$, $C$ and $D$ are chosen on a line in that order, with $AB$ and $CD$ greater than $BC$. Equilateral triangles $APB$, $BCQ$ and $CDR$ are constructed so that $P$, $Q$ and $R$ are on the same side with respect to $AD$. If $\angle PQR=120^\circ$, show that
\[ \frac{1}{AB}+\frac{1}{CD}=\frac{1}{BC}. \] 
OMCC 2014 problem 6:  A positive integer $n$ is \textit{funny} if for all positive divisors $d$ of $n$, $d+2$ is a prime number. Find all funny numbers with the largest possible number of divisors. 

OMCC 2013 

OMCC 2013 problem 1:  Juan writes the list of pairs $(n, 3^n)$, with $n=1, 2, 3,...$ on a chalkboard. As he writes the list, he underlines the pairs $(n, 3^n)$ when $n$ and $3^n$ have the same units digit. What is the $2013^{th}$ underlined pair? 
OMCC 2013 problem 2:  Around a round table the people $P_1, P_2,..., P_{2013}$ are seated in a clockwise order. Each person starts with a certain amount of coins (possibly none); there are a total of $10000$ coins. Starting with $P_1$ and proceeding in clockwise order, each person does the following on their turn:
\begin{itemize}
  \item If they have an even number of coins, they give all of their coins to their neighbor to the left.

  \item If they have an odd number of coins, they give their neighbor to the left an odd number of coins (at least $1$ and at most all of their coins) and keep the rest.
\end{itemize}
Prove that, repeating this procedure, there will necessarily be a point where one person has all of the coins. 
OMCC 2013 problem 3:  Let $ABCD$ be a convex quadrilateral and let $M$ be the midpoint of side $AB$. The circle passing through $D$ and tangent to $AB$ at $A$ intersects the segment $DM$ at $E$. The circle passing through $C$ and tangent to $AB$ at $B$ intersects the segment $CM$ at $F$. Suppose that the lines $AF$ and $BE$ intersect at a point which belongs to the perpendicular bisector of side $AB$. Prove that $A$, $E$, and $C$ are collinear if and only if $B$, $F$, and $D$ are collinear. 
OMCC 2013 problem 4:  Ana and Beatriz take turns in a game that starts with a square of side $1$ drawn on an infinite grid. Each turn consists of drawing a square that does not overlap with the rectangle already drawn, in such a way that one of its sides is a (complete) side of the figure already drawn. A player wins if she completes a rectangle whose area is a multiple of $5$. If Ana goes first, does either player have a winning strategy? 
OMCC 2013 problem 5:  Let $ABC$ be an acute triangle and let $\Gamma$ be its circumcircle. The bisector of $\angle{A}$ intersects $BC$ at $D$, $\Gamma$ at $K$ (different from $A$), and the line through $B$ tangent to $\Gamma$ at $X$. Show that $K$ is the midpoint of $AX$ if and only if  $\frac{AD}{DC}=\sqrt{2}$. 
OMCC 2013 problem 6:  Determine all pairs of non-constant polynomials $p(x)$ and $q(x)$, each with leading coefficient $1$, degree $n$, and $n$ roots which are non-negative integers, that satisfy $p(x)-q(x)=1$. 

OMCC 2012 

OMCC 2012 problem 1:  Find all positive integers that are equal to $700$ times the sum of its digits. 
OMCC 2012 problem 2:  Let $\gamma$ be the circumcircle of the acute triangle $ABC$. Let $P$ be the midpoint of the minor arc $BC$. The parallel to $AB$ through $P$ cuts $BC, AC$ and $\gamma$ at points $R,S$ and $T$, respectively. Let $K \equiv AP \cap BT$ and $L \equiv BS \cap AR$.  Show that $KL$ passes through the midpoint of $AB$ if and only if $CS = PR$. 
OMCC 2012 problem 3:  Let $a,b,c$ be real numbers that satisfy $\frac{1}{a+b}+\frac{1}{b+c}+\frac{1}{a+c} =1$ and $ab+bc+ac >0$. \\\\
Show that
\[ a+b+c - \frac{abc}{ab+bc+ac} \ge 4 \] 
OMCC 2012 problem 4:  Trilandia is a very unusual city. The city has the shape of an equilateral triangle of side lenght 2012. The streets divide the city into several blocks that are shaped like equilateral triangles of side lenght 1. There are streets at the border of Trilandia too. There are 6036 streets in total. The mayor wants to put sentinel sites at some intersections of the city to monitor the streets. A sentinel site can monitor every street on which it is located. What is the smallest number of sentinel sites that are required to monitor every street of Trilandia? 
OMCC 2012 problem 5:  Alexander and Louise are a pair of burglars. Every morning, Louise steals one third of Alexander's money, but feels remorse later in the afternoon and gives him half of all the money she has. If Louise has no money at the beginning and starts stealing on the first day, what is the least positive integer amount of money Alexander must have so that at the end of the 2012th day they both have an integer amount of money? 
OMCC 2012 problem 6:  Let $ABC$ be a triangle with $AB < BC$, and let $E$ and $F$ be points in $AC$ and $AB$ such that $BF = BC = CE$, both on the same halfplane as $A$ with respect to  $BC$. \\\\
Let $G$ be the intersection of $BE$ and $CF$. Let $H$ be a point in the parallel through $G$ to $AC$ such that $HG = AF$ (with $H$ and $C$ in opposite halfplanes with respect to $BG$). Show that $\angle EHG = \frac{\angle BAC}{2}$. 

OMCC 2011 

OMCC 2011 problem 1:  Consider a cube with a fly standing at each of its vertices. When a whistle blows, each fly moves to a vertex in the same face as the previous one but diagonally opposite to it. After the whistle blows, in how many ways can the flies change position so that there is no vertex with 2 or more flies? 
OMCC 2011 problem 2:  In a scalene triangle $ABC$, $D$ is the foot of the altitude through $A$, $E$ is the intersection of $AC$ with the bisector of $\angle ABC$ and $F$ is a point on $AB$. Let $O$ the circumcenter of $ABC$ and $X=AD\cap BE$, $Y=BE\cap CF$, $Z=CF \cap AD$. If $XYZ$ is an equilateral triangle, prove that one of the triangles $OXY$, $OYZ$, $OZX$ must be equilateral. 
OMCC 2011 problem 3:  A \textit{slip} on an integer $n\geq 2$ is an operation that consists in choosing a prime divisor $p$ of $n$ and replacing $n$ by $\frac{n+p^2}{p}.$ \\\\
Starting with an arbitrary integer $n\geq 5$, we successively apply the slip operation on it. Show that one eventually reaches $5$, no matter the slips applied. 
OMCC 2011 problem 4:  Find all positive integers $p$, $q$, $r$ such that $p$ and $q$ are prime numbers and  $\frac{1}{p+1}+\frac{1}{q+1}-\frac{1}{(p+1)(q+1)} = \frac{1}{r}.$ 
OMCC 2011 problem 5:  If $x$, $y$, $z$ are positive numbers satisfying
\[ x+\frac{y}{z}=y+\frac{z}{x}=z+\frac{x}{y}=2. \]
Find all the possible values of $x+y+z$. 
OMCC 2011 problem 6:  Let $ABC$ be an acute triangle and $D$, $E$, $F$ be the feet of the altitudes through $A$, $B$, $C$ respectively.  Call $Y$ and $Z$ the feet of the perpendicular lines from $B$ and $C$ to $FD$ and $DE$, respectively. Let $F_1$ be the symmetric of $F$ with respect to $E$ and $E_1$ be the symmetric of $E$ with respect to $F$. If $3EF=FD+DE$, prove that $\angle BZF_1=\angle CYE_1$. 

OMCC 2010 

OMCC 2010 problem 1:  Denote by $S(n)$ the sum of the digits of the positive integer $n$. Find all the solutions of the equation \\\\
$n(S(n)-1)=2010.$ 
OMCC 2010 problem 2:  Let $ABC$ be a triangle and $L$, $M$, $N$ be the midpoints of $BC$, $CA$ and $AB$, respectively. The tangent to the circumcircle of $ABC$ at $A$ intersects $LM$ and $LN$ at $P$ and $Q$, respectively. Show that $CP$ is parallel to $BQ$. 
OMCC 2010 problem 3:  A token is placed in one square of a $m\times n$ board, and is moved according to the following rules:
\begin{itemize}
  \item In each turn, the token can be moved to a square sharing a side with the one currently occupied.

  \item The token cannot be placed in a square that has already been occupied.

  \item Any two consecutive moves cannot have the same direction.
\end{itemize}
The game ends when the token cannot be moved. Determine the values of $m$ and $n$ for which, by placing the token in some square, all the squares of the board will have been occupied in the end of the game. 
OMCC 2010 problem 4:  Find all positive integers $N$ such that an $N\times N$ board can be tiled using tiles of size $5\times 5$ or $1\times 3$. \\\\
Note: The tiles must completely cover all the board, with no overlappings. 
OMCC 2010 problem 5:  If $p$, $q$ and $r$ are nonzero rational numbers such that $\sqrt[3]{pq^2}+\sqrt[3]{qr^2}+\sqrt[3]{rp^2}$ is a nonzero rational number, prove that \\\\
$\frac{1}{\sqrt[3]{pq^2}}+\frac{1}{\sqrt[3]{qr^2}}+\frac{1}{\sqrt[3]{rp^2}}$ \\\\
is also a rational number. 
OMCC 2010 problem 6:  Let $\Gamma$ and $\Gamma_1$ be two circles internally tangent at $A$, with centers $O$ and  $O_1$ and radii $r$ and $r_1$, respectively ($r>r_1$). $B$ is a point diametrically opposed to $A$ in $\Gamma$, and $C$ is a point on $\Gamma$ such that $BC$ is tangent to $\Gamma_1$ at $P$. Let $A'$ the midpoint of $BC$. Given that $O_1A'$ is parallel to $AP$, find the ratio $r/r_1$. 

OMCC 2009 

OMCC 2009 problem 1:  Let $ P$ be the product of all non-zero digits of the positive integer $ n$. For example, $ P(4) = 4$, $ P(50) = 5$, $ P(123) = 6$, $ P(2009) = 18$. \\
Find the value of the sum: P(1) + P(2) + ... + P(2008) + P(2009). 
OMCC 2009 problem 2:  \item Two circles $ \Gamma_1$ and $ \Gamma_2$ intersect at points $ A$ and $ B$. Consider a circle $ \Gamma$ contained in $ \Gamma_1$ and $ \Gamma_2$, which is tangent to both of them at $ D$ and $ E$ respectively. Let $ C$ be one of the intersection points of line $ AB$ with $ \Gamma$, $ F$ be the intersection of line $ EC$ with $ \Gamma_2$ and $ G$ be the intersection of line $ DC$ with $ \Gamma_1$. Let $ H$ and $ I$ be the intersection points of line $ ED$ with $ \Gamma_1$ and $ \Gamma_2$ respectively. Prove that $ F$, $ G$, $ H$ and $ I$ are on the same circle. 
OMCC 2009 problem 3:  There are 2009 boxes numbered from 1 to 2009, some of which contain stones. Two players, $ A$ and $ B$, play alternately, starting with $ A$. A move consists in selecting a non-empty box $ i$, taking one or more stones from that box and putting them in box $ i + 1$. If $ i = 2009$, the selected stones are eliminated. The player who removes the last stone wins
\begin{enumerate}[a)]
  \item If there are 2009 stones in the box 2 and the others are empty, find a winning strategy for either player.
  \item If there is exactly one stone in each box, find a winning strategy for either player.
\end{enumerate} 
OMCC 2009 problem 4:  We wish to place natural numbers around a circle such that the following property is satisfied: the absolute values of the differences of each pair of neighboring numbers are all different.
\begin{enumerate}[a)]
  \item Is it possible to place the numbers from 1 to 2009 satisfying this property
  \item Is it possible to suppress one of the numbers from 1 to 2009 in such a way that the remaining 2008 numbers can be placed satisfying the property
\end{enumerate} 
OMCC 2009 problem 5:  Given an acute and scalene triangle $ ABC$, let $ H$ be its orthocenter, $ O$ its circumcenter, $ E$ and $ F$ the feet of the altitudes drawn from $ B$ and $ C$, respectively. Line $ AO$ intersects the circumcircle of the triangle again at point $ G$ and segments $ FE$ and $ BC$ at points $ X$ and $ Y$ respectively. Let $ Z$ be the point of intersection of line $ AH$ and the tangent line to the circumcircle at $ G$. Prove that $ HX$ is parallel to $ YZ$. 
OMCC 2009 problem 6:  Find all prime numbers $ p$ and $ q$ such that $ p^3 - q^5 = (p + q)^2$. 

OMCC 2008 

OMCC 2008 problem 1:  Find the least positive integer $ N$ such that the sum of its digits is 100 and the sum of the digits of $ 2N$ is 110. 
OMCC 2008 problem 2:  Let $ ABCD$ be a convex cuadrilateral inscribed in a circumference centered at $ O$ such that $ AC$ is a diameter. Pararellograms $ DAOE$ and $ BCOF$ are constructed. Show that if $ E$ and $ F$ lie on the circumference then $ ABCD$ is a rectangle. 
OMCC 2008 problem 3:  There are 2008 bags numbered from 1 to 2008, with 2008 frogs in each one of them. Two people play in turns. A play consists in selecting  a bag and taking out of it any number of frongs (at least one), leaving $ x$ frogs in it ($ x\geq 0$). After each play, from each bag with a number higher than the selected one and having more than $ x$ frogs, some frogs scape until there are $ x$ frogs in the bag. The player that takes out the last frog from bag number 1 looses. Find and explain a winning strategy. 
OMCC 2008 problem 4:  Five girls have a little store that opens from Monday through Friday. Since two people are always enough for taking care of it, they decide to do a work plan for the week, specifying who will work each day, and fulfilling the following conditions:
\begin{enumerate}[a)]
  \item Each girl will work exactly two days a week
  \item The 5 assigned couples for the week must be different
\end{enumerate}
In how many ways can the girls do the work plan? 
OMCC 2008 problem 5:  Find a polynomial $ p\left(x\right)$ with real coefficients such that \\\\
$ \left(x+10\right)p\left(2x\right)=\left(8x-32\right)p\left(x+6\right)$ \\\\
for all real $ x$ and $ p\left(1\right)=210$. 
OMCC 2008 problem 6:  Let $ ABC$ be an acute triangle. Take points $ P$ and $ Q$ inside $ AB$ and $ AC$, respectively, such that $ BPQC$ is cyclic. The circumcircle of $ ABQ$ intersects $ BC$ again in $ S$ and the circumcircle of $ APC$ intersects $ BC$ again in $ R$, $ PR$ and $ QS$ intersect again in $ L$. Prove that the intersection of $ AL$ and $ BC$ does not depend on the selection of $ P$ and $ Q$. 

OMCC 2007 

OMCC 2007 problem 1:  The Central American Olympiad is an annual competition. The ninth Olympiad is held in 2007. Find all the positive integers $n$ such that $n$ divides the number of the year in which the $n$-th Olympiad takes place. 
OMCC 2007 problem 2:  In a triangle $ABC$, the angle bisector of $A$ and the cevians $BD$ and $CE$ concur at a point $P$ inside the triangle. Show that the quadrilateral $ADPE$ has an incircle if and only if $AB=AC$. 
OMCC 2007 problem 3:  Let $S$ be a finite set of integers. Suppose that for every two different elements of $S$, $p$ and $q$, there exist not necessarily distinct integers $a \neq 0$, $b$, $c$ belonging to $S$, such that $p$ and $q$ are the roots of the polynomial $ax^2+bx+c$. Determine the maximum number of elements that $S$ can have. 
OMCC 2007 problem 4:  In a remote island, a language in which every word can be written using only the letters $a$, $b$, $c$, $d$, $e$, $f$, $g$ is spoken. Let's say two words are \textit{synonymous} if we can transform one into the other according to the following rules:
\begin{enumerate}[i)]
  \item Change a letter by another two in the following way:
\end{enumerate}
\[
a \rightarrow bc,\ b \rightarrow cd,\ c \rightarrow de,\ d \rightarrow ef,\ e \rightarrow fg,\ f\rightarrow ga,\ g\rightarrow ab
\]
ii) If a letter is between other two equal letters, these can be removed. For example, $dfd \rightarrow f$. \\\\
Show that all words in this language are synonymous. 
OMCC 2007 problem 5:  Given two non-negative integers $m>n$, let's say that $m$ \textit{ends in} $n$ if we can get $n$ by erasing some digits (from left to right) in the decimal representation of $m$. For example, 329 ends in 29, and also in 9. \\\\
Determine how many three-digit numbers end in the product of their digits. 
OMCC 2007 problem 6:  Consider a circle $S$, and a point $P$ outside it. The tangent lines from $P$ meet $S$ at $A$ and $B$, respectively. Let $M$ be the midpoint of $AB$. The perpendicular bisector of $AM$ meets $S$ in a point $C$ lying inside the triangle $ABP$. $AC$ intersects $PM$ at $G$, and $PM$ meets $S$ in a point $D$ lying outside the triangle $ABP$. If $BD$ is parallel to $AC$, show that $G$ is the centroid of the triangle $ABP$. \\\\
\textit{Arnoldo Aguilar (El Salvador)} 

OMCC 2006 

OMCC 2006 problem 1:  August 1st 
OMCC 2006 problem 2:  For $0 \leq d \leq 9$, we define the numbers
\[ S_d=1+d+d^2+\cdots+d^{2006} \]
Find the last digit of the number
\[ S_0+S_1+\cdots+S_9. \] 
OMCC 2006 problem 3:  Let $\Gamma$ and $\Gamma'$ be two congruent circles centered at $O$ and $O'$, respectively, and let $A$ be one of their two points of intersection. $B$ is a point on $\Gamma$, $C$ is the second point of intersection of $AB$ and $\Gamma'$, and $D$ is a point on $\Gamma'$ such that $OBDO'$ is a parallelogram. Show that the length of $CD$ does not depend on the position of $B$. 
OMCC 2006 problem 4:  For every natural number $n$ we define
\[ f(n)=\left\lfloor n+\sqrt{n}+\frac{1}{2}\right\rfloor \]
Show that for every integer $k \geq 1$ the equation
\[ f(f(n))-f(n)=k \]
has exactly $2k-1$ solutions. 
OMCC 2006 problem 5:  August 2nd 
OMCC 2006 problem 6:  The product of several distinct positive integers is divisible by ${2006}^2$. Determine the minimum value the sum of such numbers can take. 
OMCC 2006 problem 7:  The \textit{Olimpia} country is formed by $n$ islands. The most populated one is called \textit{Panacenter}, and every island has a different number of inhabitants. We want to build bridges between these islands, which we'll be able to travel in both directions, under the following conditions:
\begin{enumerate}[a)]
  \item No pair of islands is joined by more than one bridge.
  \item Using the bridges we can reach every island from Panacenter.
  \item If we want to travel from Panacenter to every other island, in such a way that we use each bridge at most once, the number of inhabitants of the islands we visit is strictly decreasing.
\end{enumerate}
Determine the number of ways we can build the bridges. 
OMCC 2006 problem 8:  Let $ABCD$ be a convex quadrilateral. $I=AC\cap BD$, and $E$, $H$, $F$ and $G$ are points on $AB$, $BC$, $CD$ and $DA$ respectively, such that $EF \cap GH= I$. If $M=EG \cap AC$, $N=HF \cap AC$, show that
\[ \frac{AM}{IM}\cdot \frac{IN}{CN}=\frac{IA}{IC}. \] 

OMCC 2005 

OMCC 2005 problem 1: Among the positive integers that can be expressed as the sum of 2005 consecutive integers, which occupies the 2005th position when arranged in order?
OMCC 2005 problem 2:  Show that the equation $a^2b^2+b^2c^2+3b^2-c^2-a^2=2005$ has no integer solutions. \\\\
\textit{Arnoldo Aguilar, El Salvador} 
OMCC 2005 problem 3:  Let $ABC$ be a triangle. $P$, $Q$ and $R$ are the points of contact of the incircle with sides $AB$, $BC$ and $CA$, respectively. Let $L$, $M$ and $N$ be the feet of the altitudes of the triangle $PQR$ from $R$, $P$ and $Q$, respectively.
\begin{enumerate}[a)]
  \item Show that the lines $AN$, $BL$ and $CM$ meet at a point.
  \item Prove that this points belongs to the line joining the orthocenter and the circumcenter of triangle $PQR$.
\end{enumerate}
\textit{Aarón Ramírez, El Salvador} 
OMCC 2005 problem 4:  Two players, Red and Blue, play in alternating turns on a 10x10 board. Blue goes first. In his turn, a player picks a row or column (not chosen by any player yet) and color all its squares with his own color. If any of these squares was already colored, the new color substitutes the old one. \\\\
The game ends after 20 turns, when all rows and column were chosen. Red wins if the number of red squares in the board exceeds at least by 10 the number of blue squares; otherwise Blue wins. \\\\
Determine which player has a winning strategy and describe this strategy. 
OMCC 2005 problem 5:  Let $ABC$ be a triangle, $H$ the orthocenter and $M$ the midpoint of $AC$. Let $\ell$ be the parallel through $M$ to the bisector of $\angle AHC$. Prove that $\ell$ divides the triangle in two parts of equal perimeters. \\\\
\textit{Pedro Marrone, Panamá} 
OMCC 2005 problem 6:  Let $n$ be a positive integer and $p$ a fixed prime. We have a deck of $n$ cards, numbered $1,\ 2,\ldots,\ n$ and $p$ boxes for put the cards on them. Determine all posible integers $n$ for which is possible to distribute the cards in the boxes in such a way the sum of the numbers of the cards in each box is the same. 


OMCC 2004 

OMCC 2004 problem 1:  On a whiteboard, the numbers $1$ to $9$ are written. Players $A$ and $B$ take turns, and $A$ is first. Each player in turn chooses one of the numbers on the whiteboard and removes it, along with all multiples (if any). The player who removes the last number loses. \\
Determine whether any of the players has a winning strategy, and explain why. 
OMCC 2004 problem 2:  Define the sequence $(a_n)$ as follows: $a_0=a_1=1$ and for $k\ge 2$, $a_k=a_{k-1}+a_{k-2}+1$. \\
Determine how many integers between $1$ and $2004$ inclusive can be expressed as $a_m+a_n$ with $m$ and $n$ positive integers and $m\not= n$. 
OMCC 2004 problem 3:  $ABC$ is a triangle, and $E$ and $F$ are points on the segments $BC$ and $CA$ respectively, such that $\frac{CE}{CB}+\frac{CF}{CA}=1$ and $\angle CEF=\angle CAB$. Suppose that $M$ is the midpoint of $EF$ and $G$ is the point of intersection between $CM$ and $AB$. Prove that triangle $FEG$ is similar to triangle $ABC$. 
OMCC 2004 problem 4:  In a $10\times 10$ square board, half of the squares are coloured white and half black. One side common to two squares on the board side is called a \textit{border} if the two squares have different colours. Determine the minimum and maximum possible number of borders that can be on the board. 
OMCC 2004 problem 5:  Let $ABCD$ be a trapezium such that $AB||CD$ and $AB+CD=AD$. Let $P$ be the point on $AD$ such that $AP=AB$ and $PD=CD$. \\\\
$a)$ Prove that $\angle BPC=90^{\circ}$. \\
$b)$ $Q$ is the midpoint of $BC$ and $R$ is the point of intersection between the line $AD$ and the circle passing through the points $B,A$ and $Q$. Show that the points $B,P,R$ and $C$ are concyclic. 
OMCC 2004 problem 6:  With pearls of different colours form necklaces, it is said that a necklace is \textit{prime} if it cannot be decomposed into strings of pearls of the same length, and equal to each other. \\
Let $n$ and $q$ be positive integers. Prove that the number of prime necklaces with $n$ beads, each of which has one of the $q^n$ possible colours, is equal to $n$ times the number of prime necklaces with $n^2$ pearls, each of which has one of $q$ possible colours. \\\\
Note: two necklaces are considered equal if they have the same number of pearls and you can get the same colour on both necklaces, rotating one of them to match it to the other. 

OMCC 2003 

OMCC 2003 problem 1:  Two players $A$ and $B$ take turns playing the following game: There is a pile of $2003$ stones. In his first turn, $A$ selects a divisor of $2003$ and removes this number of stones from the pile. $B$ then chooses a divisor of the number of remaining stones, and removes that number of stones from the new pile, and so on. The player who has to remove the last stone loses. Show that one of the two players has a winning strategy and describe the strategy. 
OMCC 2003 problem 2:  $S$ is a circle with $AB$ a diameter and $t$ is the tangent line to $S$ at $B$. Consider the two points $C$ and $D$ on $t$ such that $B$ is between $C$ and $D$. Suppose $E$ and $F$ are the intersections of $S$ with $AC$ and $AD$ and $G$ and $H$ are the intersections of $S$ with $CF$ and $DE$. Show that $AH=AG$. 
OMCC 2003 problem 3:  Let $a$ and $b$ be positive integers with $a>1$ and $b>2$. Prove that $a^b+1\ge b(a+1)$ and determine when there is inequality. 
OMCC 2003 problem 4:  $S_1$ and $S_2$ are two circles that intersect at two different points $P$ and $Q$. Let $\ell_1$ and $\ell_2$ be two parallel lines such that $\ell_1$ passes through the point $P$ and intersects $S_1,S_2$ at $A_1,A_2$ respectively (both distinct from $P$), and $\ell_2$ passes through the point $Q$ and intersects $S_1,S_2$ at $B_1,B_2$ respectively (both distinct from $Q$). \\
Show that the triangles $A_1QA_2$ and $B_1PB_2$ have the same perimeter. 
OMCC 2003 problem 5:  A square board with $8\text{cm}$ sides is divided into $64$ squares square with each side $1\text{cm}$. Each box can be painted white or black. Find the total number of ways to colour the board so that each square of side $2\text{cm}$ formed by four squares with a common vertex contains two white and two black squares. 
OMCC 2003 problem 6:  Say a number is \textit{tico} if the sum of it's digits is a multiple of $2003$. \\\\
$\text{(i)}$ Show that there exists a positive integer $N$ such that the first $2003$ multiples, $N,2N,3N,\ldots 2003N$ are all tico. \\\\
$\text{(ii)}$ Does there exist a positive integer $N$ such that all it's multiples are tico? 

OMCC 2002 

OMCC 2002 problem 1:  For what integers $ n\ge 3$ is it possible to accommodate, in some order, the numbers $ 1,2,\cdots, n$ in a circular form such that every number divides the sum of the next two numbers, in a clockwise direction? 
OMCC 2002 problem 2:  Let $ ABC$ be an acute triangle, and let $ D$ and $ E$ be the feet of the altitudes drawn from vertexes $ A$ and $ B$, respectively. Show that if,
\[ Area[BDE]\le Area[DEA]\le Area[EAB]\le Area[ABD] \]
then, the triangle is isosceles. 
OMCC 2002 problem 3:  For every integer $ a>1$ an infinite list of integers is constructed $ L(a)$, as follows:
\begin{itemize}
\end{itemize}
$ a$ is the first number in the list $ L(a)$.
\begin{itemize}
Given a number\end{itemize}
$ b$ in $ L(a)$, the next number in the list is $ b+c$, where $ c$ is the largest integer that divides $ b$ and is smaller than $ b$. \\
Find all the integers $ a>1$ such that $ 2002$ is in the list $ L(a)$. 
OMCC 2002 problem 4:  Let $ ABC$ be a triangle, $ D$ be the midpoint of $ BC$, $ E$ be a point on segment $ AC$ such that $ BE=2AD$ and $ F$ is the intersection point of $ AD$ with $ BE$. If $ \angle DAC=60^{\circ}$, find the measure of the angle $ FEA$. 
OMCC 2002 problem 5:  Find a set of infinite positive integers $ S$ such that for every $ n\ge 1$ and whichever $ n$ distinct elements $ x_1,x_2,\cdots, x_n$ of S, the number $ x_1+x_2+\cdots +x_n$ is not a perfect square. 
OMCC 2002 problem 6:  A path from $ (0,0)$ to $ (n,n)$ on the lattice is made up of unit moves upward or rightward. It is balanced if the sum of the x-coordinates of its $ 2n+1$ vertices equals the sum of their y-coordinates. Show that a balanced path divides the square with vertices $ (0,0)$, $ (n,0)$, $ (n,n)$, $ (0,n)$ into two parts with equal area. 

OMCC 2001 

OMCC 2001 problem 1:  Two players $ A$, $ B$ and another 2001 people form a circle, such that $ A$ and $ B$ are not in consecutive  positions. $ A$ and $ B$ play in alternating turns, starting with $ A$. A play consists of touching one of the people neighboring you, which such person once touched leaves the circle. The winner is the last man standing. \\\\
Show that one of the two players has a winning strategy, and give such strategy. \\\\
Note: A player has a winning strategy if he/she is able to win no matter what the opponent does. 
OMCC 2001 problem 2:  Let $ AB$ be the diameter of a circle with a center $ O$ and radius $ 1$. Let $ C$ and $ D$ be two points on the circle such that $ AC$ and $ BD$ intersect at a point $ Q$ situated inside of the circle, and $ \angle AQB= 2 \angle COD$. Let $ P$ be a point that intersects the tangents to the circle that pass through the points $ C$ and $ D$. \\\\
Determine the length of segment $ OP$. 
OMCC 2001 problem 3:  Find all the real numbers $ N$ that satisfy these requirements:
\begin{enumerate}
  \item Only two of the digits of $ N$ are distinct from $ 0$, and one of them is $ 3$.
  \item $ N$ is a perfect square.
\end{enumerate} 
OMCC 2001 problem 4:  Determine the smallest positive integer $ n$ such that there exists positive integers $ a_1,a_2,\cdots,a_n$, that smaller than or equal to $ 15$ and are not necessarily distinct, such that the last four digits of the sum,
\[ a_1!+a_2!+\cdots+a_n! \]
Is $ 2001$. 
OMCC 2001 problem 5:  Let $ a,b$ and $ c$ real numbers such that the equation $ ax^2+bx+c=0$ has two distinct real solutions $ p_1,p_2$ and the equation $ cx^2+bx+a=0$ has two distinct real solutions $ q_1,q_2$. We know that the numbers $ p_1,q_1,p_2,q_2$ in that order, form an arithmetic progression. Show that $ a+c=0$. 
OMCC 2001 problem 6:  In a circumference of a circle, $ 10000$ points are marked, and they are numbered from $ 1$ to $ 10000$ in a clockwise manner. $ 5000$ segments are drawn in such a way so that the following conditions are met:
\begin{enumerate}
  \item Each segment joins two marked points.
  \item Each marked point belongs to one and only one segment.
  \item Each segment intersects exactly one of the remaining segments.
  \item A number is assigned to each segment that is the product of the number assigned to each end point of the segment.
\end{enumerate}
Let $ S$ be the sum of the products assigned to all the segments. \\\\
Show that $ S$ is a multiple of $ 4$. 

OMCC 2000 

OMCC 2000 problem 1:  Find all three-digit numbers $ abc$ (with $ a \neq 0$) such that $ a^2+b^2+c^2$ is a divisor of 26. 
OMCC 2000 problem 2:  Determine all positive integers $ n$ such that it is possible to tile a $ 15 \times n$ board with pieces shaped like this:
\begin{center}
\begin{asy}[width=100pt]
size(100, 45);
size(100); draw((0,0)--(3,0)); draw((0,1)--(3,1)); draw((0,2)--(1,2)); draw((2,2)--(3,2)); draw((0,0)--(0,2)); draw((1,0)--(1,2)); draw((2,0)--(2,2)); draw((3,0)--(3,2)); draw((5,0)--(6,0)); draw((4,1)--(7,1)); draw((4,2)--(7,2)); draw((5,3)--(6,3)); draw((4,1)--(4,2)); draw((5,0)--(5,3)); draw((6,0)--(6,3)); draw((7,1)--(7,2));
\end{asy}
\end{center} 
OMCC 2000 problem 3:  Let $ ABCDE$ be a convex pentagon. If $ P$, $ Q$, $ R$ and $ S$ are the respective centroids of the triangles $ ABE$, $ BCE$, $ CDE$ and $ DAE$, show that $ PQRS$ is a parallelogram and its area is $ 2/9$ of that of $ ABCD$. 
OMCC 2000 problem 4:  Write an integer on each of the 16 small triangles in such a way that every number having at least two neighbors is equal to the difference of two of its neighbors. \\\\
Note: Two triangles are said to be neighbors if they have a common side.
\begin{center}
\begin{asy}[width=100pt]
size(100, 87);
size(100); pair P=(0,0); pair Q=(2, 2*sqrt(3)); pair R=(4,0); draw(P--Q--R--cycle); pair B=midpoint(P--Q); pair A=midpoint(P--B); pair C=midpoint(B--Q); pair E=midpoint(Q--R); pair D=midpoint(Q--E); pair F=midpoint(E--R); pair H=midpoint(R--P); pair G=midpoint(R--H); pair I=midpoint(H--P); draw(A--I); draw(B--H); draw(C--G); draw(I--D); draw(H--E); draw(G--F); draw(C--D); draw(B--E); draw(A--F);
\end{asy}
\end{center} 
OMCC 2000 problem 5:  Let $ ABC$ be an acute-angled triangle. $ C_1$ and $ C_2$ are two circles of diameters $ AB$ and $ AC$, respectively. $ C_2$ and $ AB$ intersect again at $ F$, and $ C_1$ and $ AC$ intersect again at $ E$. Also,  $ BE$ meets $ C_2$ at $ P$ and $ CF$ meets $ C_1$ at $ Q$. Prove that $ AP=AQ$. 
OMCC 2000 problem 6:  Let's say we have a \textit{nice} representation of the positive integer $ n$ if we write it as a sum of powers of 2 in such a way that there are at most two equal powers in the sum (representations differing only in the order of their summands are considered to be the same).
\begin{enumerate}[a)]
  \item Write down the 5 nice representations of 10.
  \item Find all positive integers with an even number of nice representations.
\end{enumerate} 

OMCC 1999 

OMCC 1999 problem 1:  Suppose that each of the 5 persons knows a piece of information, each piece is different, about a certain event. Each time person $A$ calls person $B$, $A$ gives $B$ all the information that $A$ knows at that moment about the event, while $B$ does not say to $A$ anything that he knew.
\begin{enumerate}[(a)]
  \item What is the minimum number of calls are necessary so that everyone knows about the event?
  \item How many calls are necessary if there were $n$ persons?
\end{enumerate} 
OMCC 1999 problem 2:  Find a positive integer $n$ with 1000 digits, all distinct from zero, with the following property: it's possible to group the digits of $n$ into 500 pairs in such a way that if the two digits of each pair are multiplied and then add the 500 products, it results a number $m$ that is a divisor of $n$. 
OMCC 1999 problem 3:  The digits of a calculator (with the exception of 0) are shown in the form indicated by the figure below, where there is also a button ``+": \\
Invalid URL \\
Two players $A$ and $B$ play in the following manner: $A$ turns on the calculator and presses a digit, and then presses the button ``+". $A$ passes the calculator to $B$, which presses a digit in the same row or column with the one pressed by $A$ that is not the same as the last one pressed by $A$; and then presses + and returns the calculator to $A$, repeating the operation in this manner successively. The first player that reaches or exceeds the sum of 31 loses the game. Which of the two players have a winning strategy and what is it? 
OMCC 1999 problem 4:  In the trapezoid $ABCD$ with bases $AB$ and $CD$, let $M$ be the midpoint of side $DA$. If $BC=a$, $MC=b$ and $\angle MCB=150^\circ$, what is the area of trapezoid $ABCD$ as a function of $a$ and $b$? 
OMCC 1999 problem 5:  Let $a$ be an odd positive integer greater than 17 such that $3a-2$ is a perfect square. Show that there exist distinct positive integers $b$ and $c$ such that $a+b,a+c,b+c$ and $a+b+c$ are four perfect squares. 
OMCC 1999 problem 6:  Denote $S$ as the subset of $\{1,2,3,\dots,1000\}$ with the property that none of the sums of two different elements in $S$ is in $S$. Find the maximum number of elements in $S$. 
