
EGMO 2021 

EGMO 2021 problem 1:  The number 2021 is fantabulous. For any positive integer $m$, if any element of the set $\{m, 2m+1, 3m\}$ is fantabulous, then all the elements are fantabulous. Does it follow that the number $2021^{2021}$ is fantabulous? 
EGMO 2021 problem 2:  Find all functions $f:\mathbb{Q}\to\mathbb{Q}$ such that the equation
\[ f(xf(x)+y) = f(y) + x^2 \]
holds for all rational numbers $x$ and $y$. \\\\
Here, $\mathbb{Q}$ denotes the set of rational numbers. 
EGMO 2021 problem 3:  Let $ABC$ be a triangle with an obtuse angle at $A$. Let $E$ and $F$ be the intersections of the external bisector of angle $A$ with the altitudes of $ABC$ through $B$ and $C$ respectively. Let $M$ and $N$ be the points on the segments $EC$ and $FB$ respectively such that $\angle EMA = \angle BCA$ and $\angle ANF = \angle ABC$. Prove that the points $E, F, N, M$ lie on a circle. 
EGMO 2021 problem 4:  Let $ABC$ be a triangle with incenter $I$ and let $D$ be an arbitrary point on the side $BC$. Let the line through $D$ perpendicular to $BI$ intersect $CI$ at $E$. Let the line through $D$ perpendicular to $CI$ intersect $BI$ at $F$. Prove that the reflection of $A$ across the line $EF$ lies on the line $BC$. 
EGMO 2021 problem 5:  A plane has a special point $O$ called the origin. Let $P$ be a set of 2021 points in the plane such that
\begin{itemize}
  \item no three points in $P$ lie on a line and
  \item no two points in $P$ lie on a line through the origin.
\end{itemize}
A triangle with vertices in $P$ is \textit{fat} if $O$ is strictly inside the triangle. Find the maximum number of fat triangles. 
EGMO 2021 problem 6:  Does there exist a nonnegative integer $a$ for which the equation
\[
\left\lfloor\frac{m}{1}\right\rfloor + \left\lfloor\frac{m}{2}\right\rfloor + \left\lfloor\frac{m}{3}\right\rfloor + \cdots + \left\lfloor\frac{m}{m}\right\rfloor = n^2 + a
\]
has more than one million different solutions $(m, n)$ where $m$ and $n$ are positive integers? \\\\
\textit{The expression $\lfloor x\rfloor$ denotes the integer part (or floor) of the real number $x$. Thus $\lfloor\sqrt{2}\rfloor = 1, \lfloor\pi\rfloor =\lfloor 22/7 \rfloor = 3, \lfloor 42\rfloor = 42,$ and $\lfloor 0 \rfloor = 0$.} 

EGMO 2020 

EGMO 2020 problem 1:  The positive integers $a_0, a_1, a_2, \ldots, a_{3030}$ satisfy
\[ 2a_{n + 2} = a_{n + 1} + 4a_n \text{ for } n = 0, 1, 2, \ldots, 3028. \]
Prove that at least one of the numbers $a_0, a_1, a_2, \ldots, a_{3030}$ is divisible by $2^{2020}$. 
EGMO 2020 problem 2:  Find all lists $(x_1, x_2, \ldots, x_{2020})$ of non-negative real numbers such that the following three conditions are all satisfied:
\begin{itemize}
  \item $x_1 \le x_2 \le \ldots \le x_{2020}$;
  \item $x_{2020} \le x_1  + 1$;
  \item there is a permutation $(y_1, y_2, \ldots, y_{2020})$ of $(x_1, x_2, \ldots, x_{2020})$ such that
\[ \sum_{i = 1}^{2020} ((x_i + 1)(y_i + 1))^2 = 8 \sum_{i = 1}^{2020} x_i^3. \]
\end{itemize}
\textit{A permutation of a list is a list of the same length, with the same entries, but the entries are allowed to be in any order.  For example, $(2, 1, 2)$ is a permutation of $(1, 2, 2)$, and they are both permutations of $(2, 2, 1)$.  Note that any list is a permutation of itself.} 
EGMO 2020 problem 3:  Let $ABCDEF$ be a convex hexagon such that $\angle A = \angle C = \angle E$ and $\angle B = \angle D = \angle F$ and the (interior) angle bisectors of $\angle A, ~\angle C,$ and $\angle E$ are concurrent. \\\\
Prove that the (interior) angle bisectors of $\angle B, ~\angle D, $ and $\angle F$ must also be concurrent. \\\\
\textit{Note that $\angle A = \angle FAB$.  The other interior angles of the hexagon are similarly described.} 
EGMO 2020 problem 4:  A permutation of the integers $1, 2, \ldots, m$ is called \textit{fresh} if there exists no positive integer $k < m$ such that the first $k$ numbers in the permutation are $1, 2, \ldots, k$ in some order.  Let $f_m$ be the number of fresh permutations of the integers $1, 2, \ldots, m$. \\\\
Prove that $f_n \ge n \cdot f_{n - 1}$ for all $n \ge 3$. \\\\
\textit{For example, if $m = 4$, then the permutation $(3, 1, 4, 2)$ is fresh, whereas the permutation $(2, 3, 1, 4)$ is not.} 
EGMO 2020 problem 5:  Consider the triangle $ABC$ with $\angle BCA > 90^{\circ}$.  The circumcircle $\Gamma$ of $ABC$ has radius $R$.  There is a point $P$ in the interior of the line segment $AB$ such that $PB = PC$ and the length of $PA$ is $R$.  The perpendicular bisector of $PB$ intersects $\Gamma$ at the points $D$ and $E$. \\\\
Prove $P$ is the incentre of triangle $CDE$. 
EGMO 2020 problem 6:  Let $m > 1$ be an integer.  A sequence $a_1, a_2, a_3, \ldots$ is defined by $a_1 = a_2 = 1$, $a_3 = 4$, and for all $n \ge 4$,
\[ a_n = m(a_{n - 1} + a_{n - 2}) - a_{n - 3}. \]
Determine all integers $m$ such that every term of the sequence is a square. 

EGMO 2019 

EGMO 2019 problem 1:  Find all triples $(a, b, c)$ of real numbers such that $ab + bc + ca = 1$ and
\[ a^2b + c = b^2c + a = c^2a + b. \] 
EGMO 2019 problem 2:  Let $n$ be a positive integer. Dominoes are placed on a $2n \times 2n$ board in such a way that every cell of the board is adjacent to exactly one cell covered by a domino. For each $n$, determine the largest number of dominoes that can be placed in this way. \\
(A domino is a tile of size $2 \times 1$ or $1 \times 2$. Dominoes are placed on the board in such a way that each domino covers exactly two cells of the board, and dominoes do not overlap. Two cells are said to be adjacent if they are different and share a common side.) 
EGMO 2019 problem 3:  Let $ABC$ be a triangle such that $\angle CAB > \angle ABC$, and let $I$ be its incentre. Let $D$ be the point on segment $BC$ such that $\angle CAD = \angle ABC$. Let $\omega$ be the circle tangent to $AC$ at $A$ and passing through $I$. Let $X$ be the second point of intersection of $\omega$ and the circumcircle of $ABC$. Prove that the angle bisectors of $\angle DAB$ and $\angle CXB$ intersect at a point on line $BC$. 
EGMO 2019 problem 4:  Let $ABC$ be a triangle with incentre $I$. The circle through $B$ tangent to $AI$ at $I$ meets side $AB$ again at $P$. The circle through $C$ tangent to $AI$ at $I$ meets side $AC$ again at $Q$. Prove that $PQ$ is tangent to the incircle of $ABC.$ 
EGMO 2019 problem 5:  Let $n\ge 2$ be an integer, and let $a_1, a_2, \cdots , a_n$ be positive integers. Show that there exist positive integers $b_1, b_2, \cdots, b_n$ satisfying the following three conditions: \\\\
$\text{(A)} \ a_i\le b_i$ for $i=1, 2, \cdots , n;$ \\\\
$\text{(B)} \ $ the remainders of $b_1, b_2, \cdots, b_n$ on division by $n$ are pairwise different; and \\\\
$\text{(C)} \ $ $b_1+b_2+\cdots b_n \le n\left(\frac{n-1}{2}+\left\lfloor \frac{a_1+a_2+\cdots a_n}{n}\right \rfloor \right)$ \\\\
(Here, $\lfloor x \rfloor$ denotes the integer part of real number $x$, that is, the largest integer that does not exceed $x$.) 
EGMO 2019 problem 6:  On a circle, Alina draws $2019$ chords, the endpoints of which are all different. A point is considered \textit{marked} if it is either \\\\
$\text{(i)}$ one of the $4038$ endpoints of a chord; or \\\\
$\text{(ii)}$ an intersection point of at least two chords. \\\\
Alina labels each marked point. Of the $4038$ points meeting criterion $\text{(i)}$, Alina labels $2019$ points with a $0$ and the other $2019$ points with a $1$. She labels each point meeting criterion $\text{(ii)}$ with an arbitrary integer (not necessarily positive). \\
Along each chord, Alina considers the segments connecting two consecutive marked points. (A chord with $k$ marked points has $k-1$ such segments.) She labels each such segment in yellow with the sum of the labels of its two endpoints and in blue with the absolute value of their difference. \\
Alina finds that the $N + 1$ yellow labels take each value $0, 1, . . . , N$ exactly once. Show that at least one blue label is a multiple of $3$. \\
(A chord is a line segment joining two different points on a circle.) 

EGMO 2018 

EGMO 2018 problem 1:  Let $ABC$ be a triangle with $CA=CB$ and $\angle{ACB}=120^\circ$, and let $M$ be the midpoint of $AB$. Let $P$ be a variable point of the circumcircle of $ABC$, and let $Q$ be the point on the segment $CP$ such that $QP = 2QC$. It is given that the line through $P$ and perpendicular to $AB$ intersects the line $MQ$ at a unique point $N$. \\
Prove that there exists a fixed circle such that $N$ lies on this circle for all possible positions of $P$. 
EGMO 2018 problem 2:  Consider the set
\[ A = \left\{1+\frac{1}{k} : k=1,2,3,4,\cdots \right\}. \]
\begin{enumerate}[a.]
  \item Prove that every integer $x \geq 2$ can be written as the product of one or more elements of $A$, which are not necessarily different.

  \item For every integer $x \geq 2$ let $f(x)$ denote the minimum integer such that $x$ can be written as the
product of $f(x)$ elements of $A$, which are not necessarily different.
Prove that there exist infinitely many pairs $(x,y)$ of integers with $x\geq 2$, $y \geq 2$, and
\[ f(xy)<f(x)+f(y). \]
(Pairs $(x_1,y_1)$ and $(x_2,y_2)$ are different if $x_1 \neq x_2$ or $y_1 \neq y_2$).
\end{enumerate} 
EGMO 2018 problem 3:  The $n$ contestant of EGMO are named $C_1, C_2, \cdots C_n$. After the competition, they queue in front of the restaurant according to the following rules.
\begin{itemize}
  \item The Jury chooses the initial order of the contestants in the queue.
  \item Every minute, the Jury chooses an integer $i$ with $1 \leq i \leq n$.

If contestant $C_i$ has at least $i$ other contestants in front of her, she pays one euro to the Jury and moves forward in the queue by exactly $i$ positions.
If contestant $C_i$ has fewer than $i$ other contestants in front of her, the restaurant opens and process ends.


\end{itemize}
\begin{enumerate}[a.]
  \item Prove that the process cannot continue indefinitely, regardless of the Jury’s choices.
  \item Determine for every $n$ the maximum number of euros that the Jury can collect by cunningly choosing the initial order and the sequence of moves.
\end{enumerate} 
EGMO 2018 problem 4:  A domino is a $ 1 \times 2 $ or $ 2 \times 1 $ tile. \\
Let $n \ge 3 $ be an integer. Dominoes are placed on an $n \times n$ board in such a way that each domino covers exactly two cells of the board, and dominoes do not overlap. The value of a row or column is the number of dominoes that cover at least one cell of this row or column. The configuration is called balanced if there exists some $k \ge 1 $ such that each row and each column has a value of $k$. Prove that a balanced configuration exists for every $n \ge 3 $, and find the minimum number of dominoes needed in such a configuration. 
EGMO 2018 problem 5:  Let $\Gamma $ be the circumcircle of triangle $ABC$. A circle $\Omega$ is tangent to the line segment $AB$ and is tangent to $\Gamma$ at a point lying on the same side of the line $AB$ as $C$. The angle bisector of $\angle BCA$ intersects $\Omega$ at two different points $P$ and $Q$. \\
Prove that $\angle ABP = \angle QBC$. 
EGMO 2018 problem 6:  \begin{enumerate}[a.]
  \item Prove that for every real number $t$ such that $0 < t < \tfrac{1}{2}$ there exists a positive integer $n$ with the following property: for every set $S$ of $n$ positive integers there exist two different elements $x$ and $y$ of $S$, and a non-negative integer $m$ (i.e. $m \ge 0 $), such that
\[ |x-my|\leq ty. \]
\item Determine whether for every real number $t$ such that $0 < t < \tfrac{1}{2} $ there exists an infinite set $S$ of positive integers such that
\[ |x-my| > ty \]
for every pair of different elements $x$ and $y$ of $S$ and every positive integer $m$ (i.e. $m > 0$).
\end{enumerate} 

EGMO 2017 

EGMO 2017 problem 1:  Let $ABCD$ be a convex quadrilateral with $\angle DAB=\angle BCD=90^{\circ}$ and $\angle ABC> \angle CDA$. Let $Q$ and $R$ be points on segments $BC$ and $CD$, respectively, such that line $QR$ intersects lines $AB$ and $AD$ at points $P$ and $S$, respectively. It is given that $PQ=RS$.Let the midpoint of $BD$ be $M$ and the midpoint of $QR$ be $N$.Prove that the points $M,N,A$ and $C$ lie on a circle. 
EGMO 2017 problem 2:  Find the smallest positive integer $k$ for which there exists a colouring of the positive integers $\mathbb{Z}_{>0}$ with $k$ colours and a function $f:\mathbb{Z}_{>0}\to \mathbb{Z}_{>0}$ with the following two properties: \\\\
$(i)$ For all positive integers $m,n$ of the same colour, $f(m+n)=f(m)+f(n).$ \\\\
$(ii)$ There are positive integers $m,n$ such that $f(m+n)\ne f(m)+f(n).$ \\\\
\textit{In a colouring of $\mathbb{Z}_{>0}$ with $k$ colours, every integer is coloured in exactly one of the $k$ colours. In both $(i)$ and $(ii)$ the positive integers $m,n$ are not necessarily distinct.} 
EGMO 2017 problem 3:  There are $2017$ lines in the plane such that no three of them go through the same point. Turbo the snail sits on a point on exactly one of the lines and starts sliding along the lines in the following fashion: she moves on a given line until she reaches an intersection of two lines. At the intersection, she follows her journey on the other line turning left or right, alternating her choice at each intersection point she reaches. She can only change direction at an intersection point. Can there exist a line segment through which she passes in both directions during her journey? 
EGMO 2017 problem 4:  Let $n\geq1$ be an integer and let $t_1<t_2<\dots<t_n$ be positive integers. In a group of $t_n+1$ people, some games of chess are played. Two people can play each other at most once. Prove that it is possible for the following two conditions to hold at the same time:
\begin{enumerate}[(i)]
  \item The number of games played by each person is one of $t_1,t_2,\dots,t_n$.
  \item For every $i$ with $1\leq i\leq n$, there is someone who has played exactly $t_i$ games of chess.
\end{enumerate} 
EGMO 2017 problem 5:  Let $n\geq2$ be an integer. An $n$-tuple $(a_1,a_2,\dots,a_n)$ of not necessarily different positive integers is \textit{expensive} if there exists a positive integer $k$ such that
\[ (a_1+a_2)(a_2+a_3)\dots(a_{n-1}+a_n)(a_n+a_1)=2^{2k-1}. \]
\begin{enumerate}[a)]
  \item Find all integers $n\geq2$ for which there exists an expensive $n$-tuple.
  \item Prove that for every odd positive integer $m$ there exists an integer $n\geq2$ such that $m$ belongs to an expensive $n$-tuple.
\end{enumerate}
\textit{There are exactly $n$ factors in the product on the left hand side.} 
EGMO 2017 problem 6:  Let $ABC$ be an acute-angled triangle in which no two sides have the same length. The reflections of the centroid $G$ and the circumcentre $O$ of $ABC$ in its sides $BC,CA,AB$ are denoted by $G_1,G_2,G_3$ and $O_1,O_2,O_3$, respectively. Show that the circumcircles of triangles $G_1G_2C$, $G_1G_3B$, $G_2G_3A$, $O_1O_2C$, $O_1O_3B$, $O_2O_3A$ and $ABC$ have a common point. \\\\
\textit{The centroid of a triangle is the intersection point of the three medians. A median is a line connecting a vertex of the triangle to the midpoint of the opposite side.} 

EGMO 2016 

EGMO 2016 problem 1:  Let $n$ be an odd positive integer, and let $x_1,x_2,\cdots ,x_n$ be non-negative real numbers. Show that
\[ \min_{i=1,\ldots,n} (x_i^2+x_{i+1}^2) \leq \max_{j=1,\ldots,n} (2x_jx_{j+1}) \]
where $x_{n+1}=x_1$. 
EGMO 2016 problem 2:  Let $ABCD$ be a cyclic quadrilateral, and let diagonals $AC$ and $BD$ intersect at $X$.Let $C_1,D_1$ and $M$ be the midpoints of segments $CX,DX$ and $CD$, respecctively. Lines $AD_1$ and $BC_1$ intersect at $Y$, and line $MY$ intersects diagonals $AC$ and $BD$ at different points $E$ and $F$, respectively. Prove that line $XY$ is tangent to the circle through $E,F$ and $X$. 
EGMO 2016 problem 3:  Let $m$ be a positive integer. Consider a $4m\times 4m$ array of square unit cells. Two different cells are \textit{related} to each other if they are in either the same row or in the same column.No cell is related to itself.Some cells are coloured blue, such that every cell is related to at lest two blue cells.Determine the minimum number of blue cells. 
EGMO 2016 problem 4:  Two circles $\omega_1$ and $\omega_2$, of equal radius intersect at different points $X_1$ and $X_2$. Consider a circle $\omega$ externally tangent to $\omega_1$ at $T_1$ and internally tangent to $\omega_2$ at point $T_2$. Prove that lines $X_1T_1$ and $X_2T_2$ intersect at a point lying on $\omega$. 
EGMO 2016 problem 5:  Let $k$ and $n$ be integers such that $k\ge 2$ and $k \le n \le 2k-1$. Place rectangular tiles, each of size $1 \times k$, or $k \times 1$ on a $n \times n$ chessboard so that each tile covers exactly $k$ cells and no two tiles overlap. Do this until no further tile can be placed in this way. For each such $k$ and $n$, determine the minimum number of tiles that such an arrangement may contain. 
EGMO 2016 problem 6:  Let $S$ be the set of all positive integers $n$ such that $n^4$ has a divisor in the range $n^2 +1, n^2 + 2,...,n^2 + 2n$. Prove that there are infinitely many elements of $S$ of each of the forms $7m, 7m+1, 7m+2, 7m+5, 7m+6$ and no elements of $S$ of the form $7m+3$ and $7m+4$, where $m$ is an integer. 

EGMO 2015 

EGMO 2015 problem 1:  Let $\triangle ABC$ be an acute-angled triangle, and let $D$ be the foot of the altitude from $C.$ The angle bisector of $\angle ABC$ intersects $CD$ at $E$ and meets the circumcircle $\omega$ of triangle $\triangle ADE$ again at $F.$ \\
If $\angle ADF = 45^{\circ}$, show that $CF$ is tangent to $\omega .$ 
EGMO 2015 problem 2:  A \textit{domino} is a $2 \times 1$ or $1 \times  2$ tile. Determine in how many ways exactly $n^2$ dominoes can be placed without overlapping on a $2n \times  2n$ chessboard so that every $2 \times  2$ square contains at least two uncovered unit squares which lie in the same row or column. 
EGMO 2015 problem 3:  Let $n, m$ be integers greater than $1$, and let $a_1, a_2, \dots, a_m$ be positive integers not greater than $n^m$. Prove that there exist positive integers $b_1, b_2, \dots, b_m$ not greater than $n$, such that
\[ \gcd(a_1 + b_1, a_2 + b_2, \dots, a_m + b_m) < n, \]
where $\gcd(x_1, x_2, \dots, x_m)$ denotes the greatest common divisor of $x_1, x_2, \dots, x_m$. 
EGMO 2015 problem 4:  Determine whether there exists an infinite sequence $a_1, a_2, a_3, \dots$ of positive integers \\
which satisfies the equality
\[ a_{n+2}=a_{n+1}+\sqrt{a_{n+1}+a_n} \]
for every positive integer $n$. 
EGMO 2015 problem 5:  Let $m, n$ be positive integers with $m > 1$. Anastasia partitions the integers $1, 2, \dots , 2m$ into $m$ pairs. Boris then chooses one integer from each pair and finds the sum of these chosen integers. \\
Prove that Anastasia can select the pairs so that Boris cannot make his sum equal to $n$. 
EGMO 2015 problem 6:  Let $H$ be the orthocentre and $G$ be the centroid of acute-angled triangle $ABC$ with $AB\ne AC$. The line $AG$ intersects the circumcircle of $ABC$ at $A$ and $P$. Let $P'$ be the reflection of $P$ in the line $BC$. Prove that $\angle CAB = 60$ if and only if $HG = GP'$ 

EGMO 2014 

EGMO 2014 problem 1:  Determine all real constants $t$ such that whenever $a$, $b$ and $c$ are the lengths of sides of a triangle, then so are $a^2+bct$, $b^2+cat$, $c^2+abt$. 
EGMO 2014 problem 2:  Let $D$ and $E$ be points in the interiors of sides $AB$ and $AC$, respectively, of a triangle $ABC$, such that $DB = BC = CE$. Let the lines $CD$ and $BE$ meet at $F$. Prove that the incentre $I$ of triangle $ABC$, the orthocentre $H$ of triangle $DEF$ and the midpoint $M$ of the arc $BAC$ of the circumcircle of triangle $ABC$ are collinear. 
EGMO 2014 problem 3:  We denote the number of positive divisors of a positive integer $m$ by $d(m)$ and the number of distinct prime divisors of $m$ by $\omega(m)$. Let $k$ be a positive integer. Prove that there exist infinitely many positive integers $n$ such that $\omega(n) = k$ and $d(n)$ does not divide $d(a^2+b^2)$ for any positive integers $a, b$ satisfying $a + b = n$. 
EGMO 2014 problem 4:  Determine all positive integers $n\geq 2$ for which there exist integers $x_1,x_2,\ldots ,x_{n-1}$ satisfying the condition that if $0<i<n,0<j<n, i\neq j$ and $n$ divides $2i+j$, then $x_i<x_j$. 
EGMO 2014 problem 5:  Let $n$ be a positive integer. We have $n$ boxes where each box contains a non-negative number of pebbles. In each move we are allowed to take two pebbles from a box we choose, throw away one of the pebbles and put the other pebble in another box we choose. An initial configuration of pebbles is called \textit{solvable} if it is possible to reach a configuration with no empty box, in a finite (possibly zero) number of moves. Determine all initial configurations of pebbles which are not solvable, but become solvable when an additional pebble is added to a box, no matter which box is chosen. 
EGMO 2014 problem 6:  Determine all functions $f:\mathbb R\rightarrow\mathbb R$ satisfying the condition
\[ f(y^2+2xf(y)+f(x)^2)=(y+f(x))(x+f(y)) \]
for all real numbers $x$ and $y$. 

EGMO 2013 

EGMO 2013 problem 1:  The side $BC$ of the triangle $ABC$ is extended beyond $C$ to $D$ so that $CD = BC$.  The side $CA$ is extended beyond $A$ to $E$ so that $AE = 2CA$.  Prove that, if $AD=BE$, then the triangle $ABC$ is right-angled. 
EGMO 2013 problem 2:  Determine all integers $m$ for which the $m \times m$ square can be dissected into five rectangles, the side lengths of which are the integers $1,2,3,\ldots,10$ in some order. 
EGMO 2013 problem 3:  Let $n$ be a positive integer.
\begin{enumerate}[(a)]
  \item Prove that there exists a set $S$ of $6n$ pairwise different positive integers, such that the least common multiple of any two elements of $S$ is no larger than $32n^2$.
  \item Prove that every set $T$ of $6n$ pairwise different positive integers contains two elements the least common multiple of which is larger than $9n^2$.
\end{enumerate} 
EGMO 2013 problem 4:  Find all positive integers $a$ and $b$ for which there are three consecutive integers at which the polynomial
\[ P(n) = \frac{n^5+a}{b} \]
takes integer values. 
EGMO 2013 problem 5:  Let $\Omega$ be the circumcircle of the triangle $ABC$. The circle $\omega$ is tangent to the sides $AC$ and $BC$, and it is internally tangent to the circle $\Omega$ at the point $P$.  A line parallel to $AB$ intersecting the interior of triangle $ABC$ is tangent to $\omega$ at $Q$. \\\\
Prove that $\angle ACP = \angle QCB$. 
EGMO 2013 problem 6:  Snow White and the Seven Dwarves are living in their house in the forest.  On each of $16$ consecutive days, some of the dwarves worked in the diamond mine while the remaining dwarves collected berries in the forest.  No dwarf performed both types of work on the same day.  On any two different (not necessarily consecutive) days, at least three dwarves each performed both types of work.  Further, on the first day, all seven dwarves worked in the diamond mine. \\\\
Prove that, on one of these $16$ days, all seven dwarves were collecting berries. 

EGMO 2012 

EGMO 2012 problem 1:  Let $ABC$ be a triangle with circumcentre $O$. The points $D,E,F$ lie in the interiors of the sides $BC,CA,AB$ respectively, such that $DE$ is perpendicular to $CO$ and $DF$ is perpendicular to $BO$. (By interior we mean, for example, that the point $D$ lies on the line $BC$ and $D$ is between $B$ and $C$ on that line.) \\
Let $K$ be the circumcentre of triangle $AFE$. Prove that the lines $DK$ and $BC$ are perpendicular. \\\\
\textit{Netherlands (Merlijn Staps)} 
EGMO 2012 problem 2:  Let $n$ be a positive integer. Find the greatest possible integer $m$, in terms of $n$, with the following property: a table with $m$ rows and $n$ columns can be filled with real numbers in such a manner that for any two different rows $\left[ {{a_1},{a_2},\ldots,{a_n}}\right]$ and $\left[ {{b_1},{b_2},\ldots,{b_n}} \right]$ the following holds:
\[
\max\left( {\left| {{a_1} - {b_1}} \right|,\left| {{a_2} - {b_2}} \right|,...,\left| {{a_n} - {b_n}} \right|} \right) = 1
\]
\textit{Poland (Tomasz Kobos)} 
EGMO 2012 problem 3:  Find all functions $f:\mathbb{R}\to\mathbb{R}$ such that
\[ f\left( {yf(x + y) + f(x)} \right) = 4x + 2yf(x + y) \]
for all $x,y\in\mathbb{R}$. \\\\
\textit{Netherlands (Birgit van Dalen)} 
EGMO 2012 problem 4:  A set $A$ of integers is called \textit{sum-full} if $A \subseteq A + A$, i.e. each element $a \in A$ is the sum of some pair of (not necessarily different) elements $b,c \in A$. A set $A$ of integers is said to be \textit{zero-sum-free} if $0$ is the only integer that cannot be expressed as the sum of the elements of a finite nonempty subset of $A$. \\
Does there exist a sum-full zero-sum-free set of integers? \\\\
\textit{Romania (Dan Schwarz)} 
EGMO 2012 problem 5:  The numbers $p$ and $q$ are prime and satisfy
\[ \frac{p}{{p + 1}} + \frac{{q + 1}}{q} = \frac{{2n}}{{n + 2}} \]
for some positive integer $n$. Find all possible values of $q-p$. \\\\
\textit{Luxembourg (Pierre Haas)} 
EGMO 2012 problem 6:  There are infinitely many people registered on the social network Mugbook. Some pairs of (different) users are registered as friends, but each person has only finitely many friends. Every user has at least one friend. (Friendship is symmetric; that is, if $A$ is a friend of $B$, then $B$ is a friend of $A$.) \\
Each person is required to designate one of their friends as their best friend. If $A$ designates $B$ as her best friend, then (unfortunately) it does not follow that $B$ necessarily designates $A$ as her best friend. Someone designated as a best friend is called a $1$-best friend. More generally, if $n> 1$ is a positive integer, then a user is an $n$-best friend provided that they have been designated the best friend of someone who is an $(n-1)$-best friend. Someone who is a $k$-best friend for every positive integer $k$ is called popular.
\begin{enumerate}[(a)]
  \item Prove that every popular person is the best friend of a popular person.
  \item Show that if people can have infinitely many friends, then it is possible that a popular person is not the best friend of a popular person.
\end{enumerate}
\textit{Romania (Dan Schwarz)} 
EGMO 2012 problem 7:  Let $ABC$ be an acute-angled triangle with circumcircle $\Gamma$ and orthocentre $H$. Let $K$ be a point of $\Gamma$ on the other side of $BC$ from $A$. Let $L$ be the reflection of $K$ in the line $AB$, and let $M$ be the reflection of $K$ in the line $BC$. Let $E$ be the second point of intersection of $\Gamma $ with the circumcircle of triangle $BLM$. \\
Show that the lines $KH$, $EM$ and $BC$ are concurrent. (The orthocentre of a triangle is the point on all three of its altitudes.) \\\\
\textit{Luxembourg (Pierre Haas)} 
EGMO 2012 problem 8:  A \textit{word} is a finite sequence of letters from some alphabet. A word is \textit{repetitive} if it is a concatenation of at least two identical subwords (for example, $ababab$ and $abcabc$ are repetitive, but $ababa$ and $aabb$ are not). Prove that if a word has the property that swapping any two adjacent letters makes the word repetitive, then all its letters are identical. (Note that one may swap two adjacent identical letters, leaving a word unchanged.) \\\\
\textit{Romania (Dan Schwarz)} 
