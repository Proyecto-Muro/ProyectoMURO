
IMO 2021 

IMO 2021 problem 1:  Let $n \geqslant 100$ be an integer. Ivan writes the numbers $n, n+1, \ldots, 2 n$ each on different cards. He then shuffles these $n+1$ cards, and divides them into two piles. Prove that at least one of the piles contains two cards such that the sum of their numbers is a perfect square. 
IMO 2021 problem 2:  Show that the inequality
\[
\sum_{i=1}^n \sum_{j=1}^n \sqrt{|x_i-x_j|}\leqslant \sum_{i=1}^n \sum_{j=1}^n \sqrt{|x_i+x_j|}
\]
holds for all real numbers $x_1,\ldots x_n.$ 
IMO 2021 problem 3:  Let $D$ be an interior point of the acute triangle $ABC$ with $AB > AC$ so that $\angle DAB = \angle CAD.$ The point $E$ on the segment $AC$ satisfies $\angle ADE =\angle BCD,$ the point $F$ on the segment $AB$ satisfies $\angle FDA =\angle DBC,$ and the point $X$ on the line $AC$ satisfies $CX = BX.$ Let $O_1$ and $O_2$ be the circumcenters of the triangles $ADC$ and $EXD,$ respectively. Prove that the lines $BC, EF,$ and $O_1O_2$ are concurrent. 
IMO 2021 problem 4:  Let $\Gamma$ be a circle with centre $I$, and $A B C D$ a convex quadrilateral such that each of the segments $A B, B C, C D$ and $D A$ is tangent to $\Gamma$. Let $\Omega$ be the circumcircle of the triangle $A I C$. The extension of $B A$ beyond $A$ meets $\Omega$ at $X$, and the extension of $B C$ beyond $C$ meets $\Omega$ at $Z$. The extensions of $A D$ and $C D$ beyond $D$ meet $\Omega$ at $Y$ and $T$, respectively. Prove that
\[ A D+D T+T X+X A=C D+D Y+Y Z+Z C. \]
\textit{Proposed by Dominik Burek, Poland and Tomasz Ciesla, Poland} 
IMO 2021 problem 5:  Two squirrels, Bushy and Jumpy, have collected 2021 walnuts for the winter. Jumpy numbers the walnuts from 1 through 2021, and digs 2021 little holes in a circular pattern in the ground around their favourite tree. The next morning Jumpy notices that Bushy had placed one walnut into each hole, but had paid no attention to the numbering. Unhappy, Jumpy decides to reorder the walnuts by performing a sequence of 2021 moves. In the $k$-th move, Jumpy swaps the positions of the two walnuts adjacent to walnut $k$. \\\\
Prove that there exists a value of $k$ such that, on the $k$-th move, Jumpy swaps some walnuts $a$ and $b$ such that $a<k<b$. 
IMO 2021 problem 6:  Let $m\ge 2$ be an integer, $A$ a finite set of integers (not necessarily positive) and $B_1,B_2,...,B_m$ subsets of $A$. Suppose that, for every $k=1,2,...,m$, the sum of the elements of $B_k$ is $m^k$. Prove that $A$ contains at least $\dfrac{m}{2}$ elements. 

IMO 2020 

IMO 2020 problem 1:  Consider the convex quadrilateral $ABCD$. The point $P$ is in the interior of $ABCD$. The following ratio equalities hold:
\[ \angle PAD:\angle PBA:\angle DPA=1:2:3=\angle CBP:\angle BAP:\angle BPC \]
Prove that the following three lines meet in a point: the internal bisectors of angles $\angle ADP$ and $\angle PCB$ and the perpendicular bisector of segment $AB$. \\\\
\textit{Proposed by Dominik Burek, Poland} 
IMO 2020 problem 2:  The real numbers $a, b, c, d$ are such that $a\geq b\geq c\geq d>0$ and $a+b+c+d=1$. Prove that
\[ (a+2b+3c+4d)a^ab^bc^cd^d<1 \]
\textit{Proposed by Stijn Cambie, Belgium} 
IMO 2020 problem 3:  There are $4n$ pebbles of weights $1, 2, 3, \dots, 4n.$ Each pebble is coloured in one of $n$ colours and there are four pebbles of each colour. Show that we can arrange the pebbles into two piles so that the following two conditions are both satisfied:
\begin{itemize}
  \item The total weights of both piles are the same.
  \item Each pile contains two pebbles of each colour.
\end{itemize}
\textit{Proposed by Milan Haiman, Hungary and Carl Schildkraut, USA} 
IMO 2020 problem 4:  There is an integer $n > 1$. There are $n^2$ stations on a slope of a mountain, all at different altitudes. Each of two cable car companies, $A$ and $B$, operates $k$ cable cars; each cable car provides a transfer from one of the stations to a higher one (with no intermediate stops). The $k$ cable cars of $A$ have $k$ different starting points and $k$ different finishing points, and a cable car which starts higher also finishes higher. The same conditions hold for $B$. We say that two stations are linked by a company if one can start from the lower station and reach the higher one by using one or more cars of that company (no other movements between stations are allowed). Determine the smallest positive integer $k$ for which one can guarantee that there are two stations that are linked by both companies. \\\\
\textit{Proposed by Tejaswi Navilarekallu, India} 
IMO 2020 problem 5:  A deck of $n > 1$ cards is given. A positive integer is written on each card. The deck has the property that the arithmetic mean of the numbers on each pair of cards is also the geometric mean of the numbers on some collection of one or more cards. \\
For which $n$ does it follow that the numbers on the cards are all equal? \\\\
\textit{Proposed by Oleg Košik, Estonia} 
IMO 2020 problem 6:  Prove that there exists a positive constant $c$ such that the following statement is true: \\
Consider an integer $n > 1$, and a set $\mathcal S$ of $n$ points in the plane such that the distance between any two different points in $\mathcal S$ is at least 1. It follows that there is a line $\ell$ separating $\mathcal S$ such that the distance from any point of $\mathcal S$ to $\ell$ is at least $cn^{-1/3}$. \\\\
(A line $\ell$ separates a set of points S if some segment joining two points in $\mathcal S$ crosses $\ell$.) \\\\
\textit{Note. Weaker results with $cn^{-1/3}$ replaced by $cn^{-\alpha}$ may be awarded points depending on the value of the constant $\alpha > 1/3$.} \\\\
\textit{Proposed by Ting-Feng Lin and Hung-Hsun Hans Yu, Taiwan} 

IMO 2019 

IMO 2019 problem 1:  Let $\mathbb{Z}$ be the set of integers. Determine all functions $f: \mathbb{Z} \rightarrow \mathbb{Z}$ such that, for all integers $a$ and $b$,
\[ f(2a)+2f(b)=f(f(a+b)). \]
\textit{Proposed by Liam Baker, South Africa} 
IMO 2019 problem 2:  In triangle $ABC$, point $A_1$ lies on side $BC$ and point $B_1$ lies on side $AC$. Let $P$ and $Q$ be points on segments $AA_1$ and $BB_1$, respectively, such that $PQ$ is parallel to $AB$. Let $P_1$ be a point on line $PB_1$, such that $B_1$ lies strictly between $P$ and $P_1$, and $\angle PP_1C=\angle BAC$. Similarly, let $Q_1$ be the point on line $QA_1$, such that $A_1$ lies strictly between $Q$ and $Q_1$, and $\angle CQ_1Q=\angle CBA$. \\\\
Prove that points $P,Q,P_1$, and $Q_1$ are concyclic. \\\\
\textit{Proposed by Anton Trygub, Ukraine} 
IMO 2019 problem 3:  A social network has $2019$ users, some pairs of whom are friends. Whenever user $A$ is friends with user $B$, user $B$ is also friends with user $A$. Events of the following kind may happen repeatedly, one at a time:
\begin{itemize}
  \item Three users $A$, $B$, and $C$ such that $A$ is friends with both $B$ and $C$, but $B$ and $C$ are not friends, change their friendship statuses such that $B$ and $C$ are now friends, but $A$ is no longer friends with $B$, and no longer friends with $C$. All other friendship statuses are unchanged.
\end{itemize}
Initially, $1010$ users have $1009$ friends each, and $1009$ users have $1010$ friends each. Prove that there exists a sequence of such events after which each user is friends with at most one other user. \\\\
\textit{Proposed by Adrian Beker, Croatia} 
IMO 2019 problem 4:  Find all pairs $(k,n)$ of positive integers such that
\[ k!=(2^n-1)(2^n-2)(2^n-4)\cdots(2^n-2^{n-1}). \]
\textit{Proposed by Gabriel Chicas Reyes, El Salvador} 
IMO 2019 problem 5:  The Bank of Bath issues coins with an $H$ on one side and a $T$ on the other. Harry has $n$ of these coins arranged in a line from left to right. He repeatedly performs the following operation: if there are exactly $k>0$ coins showing $H$, then he turns over the $k$th coin from the left; otherwise, all coins show $T$ and he stops. For example, if $n=3$ the process starting with the configuration $THT$ would be $THT \to HHT  \to HTT \to TTT$, which stops after three operations.
\begin{enumerate}[(a)]
  \item Show that, for each initial configuration, Harry stops after a finite number of operations.
  \item For each initial configuration $C$, let $L(C)$ be the number of operations before Harry stops. For example, $L(THT) = 3$ and $L(TTT) = 0$. Determine the average value of $L(C)$ over all $2^n$ possible initial configurations $C$.
\end{enumerate}
\textit{Proposed by David Altizio, USA} 
IMO 2019 problem 6:  Let $I$ be the incentre of acute triangle $ABC$ with $AB\neq AC$. The incircle $\omega$ of $ABC$ is tangent to sides $BC, CA$, and $AB$ at $D, E,$ and $F$, respectively. The line through $D$ perpendicular to $EF$ meets $\omega$ at $R$. Line $AR$ meets $\omega$ again at $P$. The circumcircles of triangle $PCE$ and $PBF$ meet again at $Q$. \\\\
Prove that lines $DI$ and $PQ$ meet on the line through $A$ perpendicular to $AI$. \\\\
\textit{Proposed by Anant Mudgal, India} 

IMO 2018 

IMO 2018 problem 1:  Let $\Gamma$ be the circumcircle of acute triangle $ABC$. Points $D$ and $E$ are on segments $AB$ and $AC$ respectively such that $AD = AE$. The perpendicular bisectors of $BD$ and $CE$ intersect minor arcs $AB$ and $AC$ of $\Gamma$ at points $F$ and $G$ respectively. Prove that lines $DE$ and $FG$ are either parallel or they are the same line. \\\\
\textit{Proposed by Silouanos Brazitikos, Evangelos Psychas and Michael Sarantis, Greece} 
IMO 2018 problem 2:  Find all integers $n \geq 3$ for which there exist real numbers $a_1, a_2, \dots a_{n + 2}$ satisfying $a_{n + 1} = a_1$, $a_{n + 2} = a_2$ and
\[ a_ia_{i + 1} + 1 = a_{i + 2}, \]
for $i = 1, 2, \dots, n$. \\\\
\textit{Proposed by Patrik Bak, Slovakia} 
IMO 2018 problem 3:  An \textit{anti-Pascal} triangle is an equilateral triangular array of numbers such that, except for the numbers in the bottom row, each number is the absolute value of the difference of the two numbers immediately below it. For example, the following is an anti-Pascal triangle with four rows which contains every integer from $1$ to $10$.
\[
\begin{array}{
c@{\hspace{4pt}}c@{\hspace{4pt}}
c@{\hspace{4pt}}c@{\hspace{2pt}}c@{\hspace{2pt}}c@{\hspace{4pt}}c
} \vspace{4pt}
 \& \& \& 4 \& \& \&  \\\vspace{4pt}
 \& \& 2 \& \& 6 \& \&  \\\vspace{4pt}
 \& 5 \& \& 7 \& \& 1 \& \\\vspace{4pt}
 8 \& \& 3 \& \& 10 \& \& 9 \\\vspace{4pt}
\end{array}
\]
Does there exist an anti-Pascal triangle with $2018$ rows which contains every integer from $1$ to $1 + 2 + 3 + \dots + 2018$? \\\\
\textit{Proposed by Morteza Saghafian, Iran} 
IMO 2018 problem 4:  A \textit{site} is any point $(x, y)$ in the plane such that $x$ and $y$ are both positive integers less than or equal to 20. \\\\
Initially, each of the 400 sites is unoccupied. Amy and Ben take turns placing stones with Amy going first. On her turn, Amy places a new red stone on an unoccupied site such that the distance between any two sites occupied by red stones is not equal to $\sqrt{5}$. On his turn, Ben places a new blue stone on any unoccupied site. (A site occupied by a blue stone is allowed to be at any distance from any other occupied site.) They stop as soon as a player cannot place a stone. \\\\
Find the greatest $K$ such that Amy can ensure that she places at least $K$ red stones, no matter how Ben places his blue stones. \\\\
\textit{Proposed by Gurgen Asatryan, Armenia} 
IMO 2018 problem 5:  Let $a_1$, $a_2$, $\ldots$ be an infinite sequence of positive integers. Suppose that there is an integer $N > 1$ such that, for each $n \geq N$, the number
\[ \frac{a_1}{a_2} + \frac{a_2}{a_3} + \cdots + \frac{a_{n-1}}{a_n} + \frac{a_n}{a_1} \]
is an integer. Prove that there is a positive integer $M$ such that $a_m = a_{m+1}$ for all $m \geq M$. \\\\
\textit{Proposed by Bayarmagnai Gombodorj, Mongolia} 
IMO 2018 problem 6:  A convex quadrilateral $ABCD$ satisfies $AB\cdot CD = BC\cdot DA$. Point $X$ lies inside $ABCD$ so that
\[ \angle{XAB} = \angle{XCD}\quad\,\,\text{and}\quad\,\,\angle{XBC} = \angle{XDA}. \]
Prove that $\angle{BXA} + \angle{DXC} = 180^\circ$. \\\\
\textit{Proposed by Tomasz Ciesla, Poland} 

IMO 2017 

IMO 2017 problem 1:  For each integer $a_0 > 1$, define the sequence $a_0, a_1, a_2, \ldots$ for $n \geq 0$ as
\[
a_{n+1} =
\begin{cases}
\sqrt{a_n} \& \text{if } \sqrt{a_n} \text{ is an integer,} \\
a_n + 3 \& \text{otherwise.}
\end{cases}
\]
Determine all values of $a_0$ such that there exists a number $A$ such that $a_n = A$ for infinitely many values of $n$. \\\\
\textit{Proposed by Stephan Wagner, South Africa} 
IMO 2017 problem 2:  Let $\mathbb{R}$ be the set of real numbers. Determine all functions $f: \mathbb{R} \rightarrow \mathbb{R}$ such that, for any real numbers $x$ and $y$,
\[ f(f(x)f(y)) + f(x+y) = f(xy). \]
\textit{Proposed by Dorlir Ahmeti, Albania} 
IMO 2017 problem 3:  A hunter and an invisible rabbit play a game in the Euclidean plane. The rabbit's starting point, $A_0,$ and the hunter's starting point, $B_0$ are the same. After $n-1$ rounds of the game, the rabbit is at point $A_{n-1}$ and the hunter is at point $B_{n-1}.$ In the $n^{\text{th}}$ round of the game, three things occur in order:
\begin{enumerate}[i.]
  \item The rabbit moves invisibly to a point $A_n$ such that the distance between $A_{n-1}$ and $A_n$ is exactly $1.$

  \item A tracking device reports a point $P_n$ to the hunter. The only guarantee provided by the tracking device to the hunter is that the distance between $P_n$ and $A_n$ is at most $1.$

  \item The hunter moves visibly to a point $B_n$ such that the distance between $B_{n-1}$ and $B_n$ is exactly $1.$
\end{enumerate}
Is it always possible, no matter how the rabbit moves, and no matter what points are reported by the tracking device, for the hunter to choose her moves so that after $10^9$ rounds, she can ensure that the distance between her and the rabbit is at most $100?$ \\\\
\textit{Proposed by Gerhard Woeginger, Austria} 
IMO 2017 problem 4:  Let $R$ and $S$ be different points on a circle $\Omega$ such that $RS$ is not a diameter. Let $\ell$ be the tangent line to $\Omega$ at $R$. Point $T$ is such that $S$ is the midpoint of the line segment $RT$. Point $J$ is chosen on the shorter arc $RS$ of $\Omega$ so that the circumcircle $\Gamma$ of triangle $JST$ intersects $\ell$ at two distinct points. Let $A$ be the common point of $\Gamma$ and $\ell$ that is closer to $R$. Line $AJ$ meets $\Omega$ again at $K$. Prove that the line $KT$ is tangent to $\Gamma$. \\\\
\textit{Proposed by Charles Leytem, Luxembourg} 
IMO 2017 problem 5:  An integer $N \ge 2$ is given. A collection of $N(N + 1)$ soccer players, no two of whom are of the same height, stand in a row. Sir Alex wants to remove $N(N - 1)$ players from this row leaving a new row of $2N$ players in which the following $N$ conditions hold: \\
($1$) no one stands between the two tallest players, \\
($2$) no one stands between the third and fourth tallest players, \\
$\;\;\vdots$ \\
($N$) no one stands between the two shortest players. \\\\
Show that this is always possible. \\\\
\textit{Proposed by Grigory Chelnokov, Russia} 
IMO 2017 problem 6:  An ordered pair $(x, y)$ of integers is a primitive point if the greatest common divisor of $x$ and $y$ is $1$. Given a finite set $S$ of primitive points, prove that there exist a positive integer $n$ and integers $a_0, a_1, \ldots , a_n$ such that, for each $(x, y)$ in $S$, we have:
\[ a_0x^n + a_1x^{n-1} y + a_2x^{n-2}y^2 + \cdots + a_{n-1}xy^{n-1} + a_ny^n = 1. \]
\textit{Proposed by John Berman, United States} 

IMO 2016 

IMO 2016 problem 1:  Triangle $BCF$ has a right angle at $B$. Let $A$ be the point on line $CF$ such that $FA=FB$ and $F$ lies between $A$ and $C$. Point $D$ is chosen so that $DA=DC$ and $AC$ is the bisector of $\angle{DAB}$. Point $E$ is chosen so that $EA=ED$ and $AD$ is the bisector of $\angle{EAC}$. Let $M$ be the midpoint of $CF$. Let $X$ be the point such that $AMXE$ is a parallelogram. Prove that $BD,FX$ and $ME$ are concurrent. 
IMO 2016 problem 2:  Find all integers $n$ for which each cell of $n \times n$ table can be filled with one of the letters $I,M$ and $O$ in such a way that:
\begin{itemize}
  \item in each row and each column, one third of the entries are $I$, one third are $M$ and one third are $O$; and 
  \item in any diagonal, if the number of entries on the diagonal is a multiple of three, then one third of the entries are $I$, one third are $M$ and one third are $O$.
\end{itemize}
\textbf{Note.} The rows and columns of an $n \times n$ table are each labelled $1$ to $n$ in a natural order. Thus each cell corresponds to a pair of positive integer $(i,j)$ with $1 \le i,j \le n$. For $n>1$, the table has $4n-2$ diagonals of two types. A diagonal of first type consists all cells $(i,j)$  for which $i+j$ is a constant, and the diagonal of this second type consists all cells $(i,j)$ for which $i-j$ is constant. 
IMO 2016 problem 3:  Let $P=A_1A_2\cdots A_k$ be a convex polygon in the plane. The vertices $A_1, A_2, \ldots, A_k$ have integral coordinates and lie on a circle. Let $S$ be the area of $P$. An odd positive integer $n$ is given such that the squares of the side lengths of $P$ are integers divisible by $n$. Prove that $2S$ is an integer divisible by $n$. 
IMO 2016 problem 4:  A set of positive integers is called \textit{fragrant} if it contains at least two elements and each of its elements has a prime factor in common with at least one of the other elements.  Let $P(n)=n^2+n+1$.  What is the least possible positive integer value of $b$ such that there exists a non-negative integer $a$ for which the set
\[ \{P(a+1),P(a+2),\ldots,P(a+b)\} \]
is fragrant? 
IMO 2016 problem 5:  The equation
\[ (x-1)(x-2)\cdots(x-2016)=(x-1)(x-2)\cdots (x-2016) \]
is written on the board, with $2016$ linear factors on each side. What is the least possible value of $k$ for which it is possible to erase exactly $k$ of these $4032$ linear factors so that at least one factor remains on each side and the resulting equation has no real solutions? 
IMO 2016 problem 6:  There are $n\ge 2$ line segments in the plane such that every two segments cross and no three segments meet at a point. Geoff has to choose an endpoint of each segment and place a frog on it facing the other endpoint. Then he will clap his hands $n-1$ times. Every time he claps,each frog will immediately jump forward to the next intersection point on its segment. Frogs never change the direction of their jumps. Geoff wishes to place the frogs in such a way that no two of them will ever occupy the same intersection point at the same time.
\begin{enumerate}[(a)]
  \item Prove that Geoff can always fulfill his wish if $n$ is odd.
  \item Prove that Geoff can never fulfill his wish if $n$ is even.
\end{enumerate} 

IMO 2015 

IMO 2015 problem 1:  We say that a finite set $\mathcal{S}$ of points in the plane is \textit{balanced} if, for any two different points $A$ and $B$ in $\mathcal{S}$, there is a point $C$ in $\mathcal{S}$ such that $AC=BC$. We say that $\mathcal{S}$ is \textit{centre-free} if for any three different points $A$, $B$ and $C$ in $\mathcal{S}$, there is no points $P$ in $\mathcal{S}$ such that $PA=PB=PC$.
\begin{enumerate}[(a)]
  \item Show that for all integers $n\ge 3$, there exists a balanced set consisting of $n$ points.
  \item Determine all integers $n\ge 3$ for which there exists a balanced centre-free set consisting of $n$ points.
\end{enumerate}
Proposed by Netherlands 
IMO 2015 problem 2:  Find all positive integers $(a,b,c)$ such that
\[ ab-c,\quad bc-a,\quad ca-b \]
are all powers of $2$. \\\\
\textit{Proposed by Serbia} 
IMO 2015 problem 3:  Let $ABC$ be an acute triangle with $AB > AC$. Let $\Gamma $ be its circumcircle, $H$ its orthocenter, and $F$ the foot of the altitude from $A$. Let $M$ be the midpoint of $BC$. Let $Q$ be the point on $\Gamma$ such that $\angle HQA = 90^{\circ}$ and let $K$ be the point on $\Gamma$ such that $\angle HKQ = 90^{\circ}$. Assume that the points $A$, $B$, $C$, $K$ and $Q$ are all different and lie on $\Gamma$ in this order. \\\\
Prove that the circumcircles of triangles $KQH$ and $FKM$ are tangent to each other. \\\\
Proposed by Ukraine 
IMO 2015 problem 4:  Triangle $ABC$ has circumcircle $\Omega$ and circumcenter $O$. A circle $\Gamma$ with center $A$ intersects the segment $BC$ at points $D$ and $E$, such that $B$, $D$, $E$, and $C$ are all different and lie on line $BC$ in this order. Let $F$ and $G$ be the points of intersection of $\Gamma$ and $\Omega$, such that $A$, $F$, $B$, $C$, and $G$ lie on $\Omega$ in this order. Let $K$ be the second point of intersection of the circumcircle of triangle $BDF$ and the segment $AB$. Let $L$ be the second point of intersection of the circumcircle of triangle $CGE$ and the segment $CA$. \\\\
Suppose that the lines $FK$ and $GL$ are different and intersect at the point $X$. Prove that $X$ lies on the line $AO$. \\\\
\textit{Proposed by Greece} 
IMO 2015 problem 5:  Let $\mathbb R$ be the set of real numbers. Determine all functions $f:\mathbb R\to\mathbb R$ that satisfy the equation
\[ f(x+f(x+y))+f(xy)=x+f(x+y)+yf(x) \]
for all real numbers $x$ and $y$. \\\\
\textit{Proposed by Dorlir Ahmeti, Albania} 
IMO 2015 problem 6:  The sequence $a_1,a_2,\dots$ of integers satisfies the conditions:
\begin{enumerate}[(i)]
  \item $1\le a_j\le2015$ for all $j\ge1$,
  \item $k+a_k\neq \ell+a_\ell$ for all $1\le k<\ell$.
\end{enumerate}
Prove that there exist two positive integers $b$ and $N$ for which
\[ \left\vert\sum_{j=m+1}^n(a_j-b)\right\vert\le1007^2 \]
for all integers $m$ and $n$ such that $n>m\ge N$. \\\\
\textit{Proposed by Ivan Guo and Ross Atkins, Australia} 

IMO 2014 

IMO 2014 problem 1:  Let $a_0 < a_1 < a_2 \ldots$ be an infinite sequence of positive integers. Prove that there exists a unique integer $n\geq 1$ such that
\[ a_n < \frac{a_0+a_1+a_2+\cdots+a_n}{n} \leq a_{n+1}. \]
\textit{Proposed by Gerhard Wöginger, Austria.} 
IMO 2014 problem 2:  Let $n \ge 2$ be an integer. Consider an $n \times n$ chessboard consisting of $n^2$ unit squares. A configuration of $n$ rooks on this board is \textit{peaceful} if every row and every column contains exactly one rook. Find the greatest positive integer $k$ such that, for each peaceful configuration of $n$ rooks, there is a $k \times k$ square which does not contain a rook on any of its $k^2$ unit squares. 
IMO 2014 problem 3:  Convex quadrilateral $ABCD$ has $\angle ABC = \angle CDA = 90^{\circ}$. Point $H$ is the foot of the perpendicular from $A$ to $BD$. Points $S$ and $T$ lie on sides $AB$ and $AD$, respectively, such that $H$ lies inside triangle $SCT$ and
\[ \angle CHS - \angle CSB = 90^{\circ}, \quad \angle THC - \angle DTC = 90^{\circ}. \]
Prove that line $BD$ is tangent to the circumcircle of triangle $TSH$. 
IMO 2014 problem 4:  Let $P$ and $Q$ be on segment $BC$ of an acute triangle $ABC$ such that $\angle PAB=\angle BCA$ and $\angle CAQ=\angle ABC$. Let $M$ and $N$ be the points on $AP$ and $AQ$, respectively, such that $P$ is the midpoint of $AM$ and $Q$ is the midpoint of $AN$. Prove that the intersection of $BM$ and $CN$ is on the circumference of triangle $ABC$. \\\\
\textit{Proposed by Giorgi Arabidze, Georgia.} 
IMO 2014 problem 5:  For each positive integer $n$, the Bank of Cape Town issues coins of denomination $\frac1n$. Given a finite collection of such coins (of not necessarily different denominations) with total value at most most $99+\frac12$, prove that it is possible to split this collection into $100$ or fewer groups, such that each group has total value at most $1$. 
IMO 2014 problem 6:  A set of lines in the plane is in \textit{general position} if no two are parallel and no three pass through the same point. A set of lines in general position cuts the plane into regions, some of which have finite area; we call these its \textit{finite regions}. Prove that for all sufficiently large $n$, in any set of $n$ lines in general position it is possible to colour at least $\sqrt{n}$ lines blue in such a way that none of its finite regions has a completely blue boundary. \\\\
\textit{Note}: Results with $\sqrt{n}$ replaced by $c\sqrt{n}$ will be awarded points depending on the value of the constant $c$. 

IMO 2013 

IMO 2013 problem 1:  Assume that $k$ and $n$ are two positive integers. Prove that there exist positive integers $m_1 , \dots , m_k$ such that
\[ 1+\frac{2^k-1}{n}=\left(1+\frac1{m_1}\right)\cdots \left(1+\frac1{m_k}\right). \]
\textit{Proposed by Japan} 
IMO 2013 problem 2:  A configuration of $4027$ points in the plane is called Colombian if it consists of $2013$ red points and $2014$ blue points, and no three of the points of the configuration are collinear. By drawing some lines, the plane is divided into several regions. An arrangement of lines is good for a Colombian configuration if the following two conditions are satisfied:
\begin{enumerate}[i)]
  \item No line passes through any point of the configuration.
  \item No region contains points of both colors.
\end{enumerate}
Find the least value of $k$ such that for any Colombian configuration of $4027$ points, there is a good arrangement of $k$ lines. \\\\
Proposed by \textit{Ivan Guo} from \textit{Australia.} 
IMO 2013 problem 3:  Let the excircle of triangle $ABC$ opposite the vertex $A$ be tangent to the side $BC$ at the point $A_1$. Define the points $B_1$ on $CA$ and $C_1$ on $AB$ analogously, using the excircles opposite $B$ and $C$, respectively. Suppose that the circumcentre of triangle $A_1B_1C_1$ lies on the circumcircle of triangle $ABC$. Prove that triangle $ABC$ is right-angled. \\\\
\textit{Proposed by Alexander A. Polyansky, Russia} 
IMO 2013 problem 4:  Let $ABC$ be an acute triangle with orthocenter $H$, and let $W$ be a point on the side $BC$, lying strictly between $B$ and $C$. The points $M$ and $N$ are the feet of the altitudes from $B$ and $C$, respectively. Denote by $\omega_1$ is the circumcircle of $BWN$, and let $X$ be the point on $\omega_1$ such that $WX$ is a diameter of $\omega_1$. Analogously, denote by $\omega_2$ the circumcircle of triangle $CWM$, and let $Y$ be the point such that $WY$ is a diameter of $\omega_2$. Prove that $X,Y$ and $H$ are collinear. \\\\
\textit{Proposed by Warut Suksompong and Potcharapol Suteparuk, Thailand} 
IMO 2013 problem 5:  Let $\mathbb Q_{>0}$ be the set of all positive rational numbers. Let $f:\mathbb Q_{>0}\to\mathbb R$ be a function satisfying the following three conditions:
\begin{enumerate}[(i)]
  \item for all $x,y\in\mathbb Q_{>0}$, we have $f(x)f(y)\geq f(xy)$;
  \item for all $x,y\in\mathbb Q_{>0}$, we have $f(x+y)\geq f(x)+f(y)$;
  \item there exists a rational number $a>1$ such that $f(a)=a$.
\end{enumerate}
Prove that $f(x)=x$ for all $x\in\mathbb Q_{>0}$. \\\\
\textit{Proposed by Bulgaria} 
IMO 2013 problem 6:  Let $n \ge 3$ be an integer, and consider a circle with $n + 1$ equally spaced points marked on it. Consider all labellings of these points with the numbers $0, 1, ... , n$ such that each label is used exactly once; two such labellings are considered to be the same if one can be obtained from the other by a rotation of the circle. A labelling is called \textit{beautiful} if, for any four labels $a < b < c < d$ with $a + d = b + c$, the chord joining the points labelled $a$ and $d$ does not intersect the chord joining the points labelled $b$ and $c$. \\\\
Let $M$ be the number of beautiful labelings, and let N be the number of ordered pairs $(x, y)$ of positive integers such that $x + y \le n$ and $\gcd(x, y) = 1$. Prove that
\[ M = N + 1. \] 

IMO 2012 

IMO 2012 problem 1:  Given triangle $ABC$ the point $J$ is the centre of the excircle opposite the vertex $A.$ This excircle is tangent to the side $BC$ at $M$, and to the lines $AB$ and $AC$ at $K$ and $L$, respectively. The lines $LM$ and $BJ$ meet at $F$, and the lines $KM$ and $CJ$ meet at $G.$ Let $S$ be the point of intersection of the lines $AF$ and $BC$, and let $T$ be the point of intersection of the lines $AG$ and $BC.$ Prove that $M$ is the midpoint of $ST.$ \\\\
(The \textit{excircle} of $ABC$ opposite the vertex $A$ is the circle that is tangent to the line segment $BC$, to the ray $AB$ beyond $B$, and to the ray $AC$ beyond $C$.) \\\\
\textit{Proposed by Evangelos Psychas, Greece} 
IMO 2012 problem 2:  Let $n\ge 3$ be an integer, and let $a_2,a_3,\ldots ,a_n$ be positive real numbers such that $a_2a_3\cdots a_n=1$. Prove that
\[ (1 + a_2)^2 (1 + a_3)^3 \dotsm (1 + a_n)^n > n^n. \]
\textit{Proposed by Angelo Di Pasquale, Australia} 
IMO 2012 problem 3:  The \textit{liar's guessing game} is a game played between two players $A$ and $B$. The rules of the game depend on two positive integers $k$ and $n$ which are known to both players. \\\\
At the start of the game $A$ chooses integers $x$ and $N$ with $1 \le x \le N.$ Player $A$ keeps $x$ secret, and truthfully tells $N$ to player $B$. Player $B$ now tries to obtain information about $x$ by asking player $A$ questions as follows: each question consists of $B$ specifying an arbitrary set $S$ of positive integers (possibly one specified in some previous question), and asking $A$ whether $x$ belongs to $S$. Player $B$ may ask as many questions as he wishes. After each question, player $A$ must immediately answer it with \textit{yes} or \textit{no}, but is allowed to lie as many times as she wants; the only restriction is that, among any $k+1$ consecutive answers, at least one answer must be truthful. \\\\
After $B$ has asked as many questions as he wants, he must specify a set $X$ of at most $n$ positive integers. If $x$ belongs to $X$, then $B$ wins; otherwise, he loses. Prove that:
\begin{enumerate}
  \item If $n \ge 2^k,$ then $B$ can guarantee a win.
  \item For all sufficiently large $k$, there exists an integer $n \ge (1.99)^k$ such that $B$ cannot guarantee a win.
\end{enumerate}
\textit{Proposed by David Arthur, Canada} 
IMO 2012 problem 4:  Find all functions $f:\mathbb Z\rightarrow \mathbb Z$ such that, for all integers $a,b,c$ that satisfy $a+b+c=0$, the following equality holds:
\[ f(a)^2+f(b)^2+f(c)^2=2f(a)f(b)+2f(b)f(c)+2f(c)f(a). \]
(Here $\mathbb{Z}$ denotes the set of integers.) \\\\
\textit{Proposed by Liam Baker, South Africa} 
IMO 2012 problem 5:  Let $ABC$ be a triangle with $\angle BCA=90^{\circ}$, and let $D$ be the foot of the altitude from $C$. Let $X$ be a point in the interior of the segment $CD$. Let $K$ be the point on the segment $AX$ such that $BK=BC$. Similarly, let $L$ be the point on the segment $BX$ such that $AL=AC$. Let $M$ be the point of intersection of $AL$ and $BK$. \\\\
Show that $MK=ML$. \\\\
\textit{Proposed by Josef Tkadlec, Czech Republic} 
IMO 2012 problem 6:  Find all positive integers $n$ for which there exist non-negative integers $a_1, a_2, \ldots, a_n$ such that
\[
\frac{1}{2^{a_1}} + \frac{1}{2^{a_2}} + \cdots + \frac{1}{2^{a_n}} = 
\frac{1}{3^{a_1}} + \frac{2}{3^{a_2}} + \cdots + \frac{n}{3^{a_n}} = 1.
\]
\textit{Proposed by Dusan Djukic, Serbia} 

IMO 2011 

IMO 2011 problem 1:  Given any set $A = \{a_1, a_2, a_3, a_4\}$ of four distinct positive integers, we denote the sum $a_1 +a_2 +a_3 +a_4$ by $s_A$. Let $n_A$ denote the number of pairs $(i, j)$ with $1 \leq  i < j \leq 4$ for which $a_i +a_j$ divides $s_A$. Find all sets $A$ of four distinct positive integers which achieve the largest possible value of $n_A$. \\\\
\textit{Proposed by Fernando Campos, Mexico} 
IMO 2011 problem 2:  Let $\mathcal{S}$ be a finite set of at least two points in the plane. Assume that no three points of $\mathcal S$ are collinear. A \textit{windmill} is a process that starts with a line $\ell$ going through a single point $P \in \mathcal S$. The line rotates clockwise about the \textit{pivot} $P$ until the first time that the line meets some other point belonging to $\mathcal S$. This point, $Q$, takes over as the new pivot, and the line now rotates clockwise about $Q$, until it next meets a point of $\mathcal S$. This process continues indefinitely. \\
Show that we can choose a point $P$ in $\mathcal S$ and a line $\ell$ going through $P$ such that the resulting windmill uses each point of $\mathcal S$ as a pivot infinitely many times. \\\\
\textit{Proposed by Geoffrey Smith, United Kingdom} 
IMO 2011 problem 3:  Let $f : \mathbb R \to \mathbb R$ be a real-valued function defined on the set of real numbers that satisfies
\[ f(x + y) \leq yf(x) + f(f(x)) \]
for all real numbers $x$ and $y$. Prove that $f(x) = 0$ for all $x \leq 0$. \\\\
\textit{Proposed by Igor Voronovich, Belarus} 
IMO 2011 problem 4:  Let $n > 0$ be an integer. We are given a balance and $n$ weights of weight $2^0, 2^1, \cdots, 2^{n-1}$. We are to place each of the $n$ weights on the balance, one after another, in such a way that the right pan is never heavier than the left pan. At each step we choose one of the weights that has not yet been placed on the balance, and place it on either the left pan or the right pan, until all of the weights have been placed. \\
Determine the number of ways in which this can be done. \\\\
\textit{Proposed by Morteza Saghafian, Iran} 
IMO 2011 problem 5:  Let $f$ be a function from the set of integers to the set of positive integers. Suppose that, for any two integers $m$ and $n$, the difference $f(m) - f(n)$ is divisible by $f(m- n)$. Prove that, for all integers $m$ and $n$ with $f(m) \leq f(n)$, the number $f(n)$ is divisible by $f(m)$. \\\\
\textit{Proposed by Mahyar Sefidgaran, Iran} 
IMO 2011 problem 6:  Let $ABC$ be an acute triangle with circumcircle $\Gamma$. Let $\ell$ be a tangent line to $\Gamma$, and let $\ell_a, \ell_b$ and $\ell_c$ be the lines obtained by reflecting $\ell$ in the lines $BC$, $CA$ and $AB$, respectively. Show that the circumcircle of the triangle determined by the lines $\ell_a, \ell_b$ and $\ell_c$ is tangent to the circle $\Gamma$. \\\\
\textit{Proposed by Japan} 

IMO 2010 

IMO 2010 problem 1:  Find all function $f:\mathbb{R}\rightarrow\mathbb{R}$ such that for all $x,y\in\mathbb{R}$ the following equality holds
\[ f(\left\lfloor x\right\rfloor y)=f(x)\left\lfloor f(y)\right\rfloor \]
where $\left\lfloor a\right\rfloor $ is greatest integer not greater than $a.$ \\\\
\textit{Proposed by Pierre Bornsztein, France} 
IMO 2010 problem 2:  Given a triangle $ABC$, with $I$ as its incenter and $\Gamma$ as its circumcircle, $AI$ intersects $\Gamma$ again at $D$. Let $E$ be a point on the arc $BDC$, and $F$ a point on the segment $BC$, such that $\angle BAF=\angle CAE < \dfrac12\angle BAC$. If $G$ is the midpoint of $IF$, prove that the meeting point of the lines $EI$ and $DG$ lies on $\Gamma$. \\\\
\textit{Proposed by Tai Wai Ming and Wang Chongli, Hong Kong} 
IMO 2010 problem 3:  Find all functions $g:\mathbb{N}\rightarrow\mathbb{N}$ such that
\[ \left(g(m)+n\right)\left(g(n)+m\right) \]
is a perfect square for all $m,n\in\mathbb{N}.$ \\\\
\textit{Proposed by Gabriel Carroll, USA} 
IMO 2010 problem 4:  Let $P$ be a point interior to triangle $ABC$ (with $CA \neq CB$). The lines $AP$, $BP$ and $CP$ meet again its circumcircle $\Gamma$ at $K$, $L$, respectively $M$. The tangent line at $C$ to $\Gamma$ meets the line $AB$ at $S$. Show that from $SC = SP$ follows $MK = ML$. \\\\
\textit{Proposed by Marcin E. Kuczma, Poland} 
IMO 2010 problem 5:  Each of the six boxes $B_1$, $B_2$, $B_3$, $B_4$, $B_5$, $B_6$ initially contains one coin. The following operations are allowed \\\\
Type 1) Choose a non-empty box $B_j$, $1\leq j \leq 5$, remove one coin from $B_j$ and add two coins to $B_{j+1}$; \\\\
Type 2) Choose a non-empty box $B_k$, $1\leq k \leq 4$, remove one coin from $B_k$ and swap the contents (maybe empty) of the boxes $B_{k+1}$ and $B_{k+2}$. \\\\
Determine if there exists a finite sequence of operations of the allowed types, such that the five boxes $B_1$, $B_2$, $B_3$, $B_4$, $B_5$ become empty, while box $B_6$ contains exactly $2010^{2010^{2010}}$ coins. \\\\
\textit{Proposed by Hans Zantema, Netherlands} 
IMO 2010 problem 6:  Let $a_1, a_2, a_3, \ldots$ be a sequence of positive real numbers, and $s$ be a positive integer, such that
\[ a_n = \max \{ a_k + a_{n-k} \mid 1 \leq k \leq n-1 \} \ \textrm{ for all } \ n > s. \]
Prove there exist positive integers $\ell \leq s$ and $N$, such that
\[ a_n = a_{\ell} + a_{n - \ell} \ \textrm{ for all } \ n \geq N. \]
\textit{Proposed by Morteza Saghafiyan, Iran} 

IMO 2009 

IMO 2009 problem 1:  Let $ n$ be a positive integer and let $ a_1,a_2,a_3,\ldots,a_k$ $ ( k\ge 2)$ be distinct integers in the set $ { 1,2,\ldots,n}$ such that $ n$ divides $ a_i(a_{i + 1} - 1)$ for $ i = 1,2,\ldots,k - 1$. Prove that $ n$ does not divide $ a_k(a_1 - 1).$ \\\\
\textit{Proposed by Ross Atkins, Australia } 
IMO 2009 problem 2:  Let $ ABC$ be a triangle with circumcentre $ O$. The points $ P$ and $ Q$ are interior points of the sides $ CA$ and $ AB$ respectively. Let $ K,L$ and $ M$ be the midpoints of the segments $ BP,CQ$ and $ PQ$. respectively, and let $ \Gamma$ be the circle passing through $ K,L$ and $ M$. Suppose that the line $ PQ$ is tangent to the circle $ \Gamma$. Prove that $ OP = OQ.$ \\\\
\textit{Proposed by Sergei Berlov, Russia } 
IMO 2009 problem 3:  Suppose that $ s_1,s_2,s_3, \ldots$ is a strictly increasing sequence of positive integers such that the sub-sequences
\[
s_{s_1},\, s_{s_2},\, s_{s_3},\, \ldots\qquad\text{and}\qquad s_{s_1+1},\, s_{s_2+1},\, s_{s_3+1},\, \ldots
\]
are both arithmetic progressions. Prove that the sequence $ s_1, s_2, s_3, \ldots$ is itself an arithmetic progression. \\\\
\textit{Proposed by Gabriel Carroll, USA} 
IMO 2009 problem 4:  Let $ ABC$ be a triangle with $ AB = AC$ . The angle bisectors of $ \angle C AB$ and $ \angle AB C$ meet the sides $ B C$ and $ C A$ at $ D$ and $ E$ , respectively. Let $ K$ be the incentre of triangle $ ADC$. Suppose that $ \angle B E K = 45^\circ$ . Find all possible values of $ \angle C AB$ . \\\\
\textit{Jan Vonk, Belgium, Peter Vandendriessche, Belgium and Hojoo Lee, Korea } 
IMO 2009 problem 5:  Determine all functions $ f$ from the set of positive integers to the set of positive integers such that, for all positive integers $ a$ and $ b$, there exists a non-degenerate triangle with sides of lengths
\[ a, f(b) \text{ and } f(b + f(a) - 1). \]
(A triangle is non-degenerate if its vertices are not collinear.) \\\\
\textit{Proposed by Bruno Le Floch, France} 
IMO 2009 problem 6:  Let $ a_1, a_2, \ldots , a_n$ be distinct positive integers and let $ M$ be a set of $ n - 1$ positive integers not containing $ s = a_1 + a_2 + \ldots + a_n.$ A grasshopper is to jump along the real axis, starting at the point $ 0$ and making $ n$ jumps to the right with lengths $ a_1, a_2, \ldots , a_n$ in some order. Prove that the order can be chosen in such a way that the grasshopper never lands on any point in $ M.$ \\\\
\textit{Proposed by Dmitry Khramtsov, Russia} 

IMO 2008 

IMO 2008 problem 1:  Let $ H$ be the orthocenter of an acute-angled triangle $ ABC$. The circle $ \Gamma_A$ centered at the midpoint of $ BC$ and passing through $ H$ intersects the sideline $ BC$ at points  $ A_1$ and $ A_2$. Similarly, define the points $ B_1$, $ B_2$, $ C_1$ and $ C_2$. \\\\
Prove that the six points $ A_1$, $ A_2$, $ B_1$, $ B_2$, $ C_1$ and $ C_2$ are concyclic. \\\\
\textit{Author: Andrey Gavrilyuk, Russia} 
IMO 2008 problem 2:  \textbf{(a)} Prove that
\[
\frac {x^2}{\left(x - 1\right)^2} + \frac {y^2}{\left(y - 1\right)^2} + \frac {z^2}{\left(z - 1\right)^2} \geq 1
\]
for all real numbers $x$, $y$, $z$, each different from $1$, and satisfying $xyz=1$. \\\\
\textbf{(b)} Prove that equality holds above for infinitely many triples of rational numbers $x$, $y$, $z$, each different from $1$, and satisfying $xyz=1$. \\\\
\textit{Author: Walther Janous, Austria} 
IMO 2008 problem 3:  Prove that there are infinitely many positive integers $ n$ such that $ n^2 + 1$ has a prime divisor greater than $ 2n + \sqrt {2n}$. \\\\
\textit{Author: Kestutis Cesnavicius, Lithuania} 
IMO 2008 problem 4:  Find all functions $ f: (0, \infty) \mapsto (0, \infty)$ (so $ f$ is a function from the positive real numbers) such that
\[
\frac {\left( f(w) \right)^2 + \left( f(x) \right)^2}{f(y^2) + f(z^2) } = \frac {w^2 + x^2}{y^2 + z^2}
\]
for all positive real numbers $ w,x,y,z,$ satisfying $ wx = yz.$ \\\\\\
\textit{Author: Hojoo Lee, South Korea} 
IMO 2008 problem 5:  Let $ n$ and $ k$ be positive integers with $ k \geq n$ and $ k - n$ an even number. Let $ 2n$ lamps labelled $ 1$, $ 2$, ..., $ 2n$ be given, each of which can be either \textit{on} or \textit{off}. Initially all the lamps are off. We consider sequences of steps: at each step one of the lamps is switched (from on to off or from off to on). \\\\
Let $ N$ be the number of such sequences consisting of $ k$ steps and resulting in the state where lamps $ 1$ through $ n$ are all on, and lamps $ n + 1$ through $ 2n$ are all off. \\\\
Let $ M$ be number of such sequences consisting of $ k$ steps, resulting in the state where lamps $ 1$ through $ n$ are all on, and lamps $ n + 1$ through $ 2n$ are all off, but where none of the lamps $ n + 1$ through $ 2n$ is ever switched on. \\\\
Determine $ \frac {N}{M}$. \\\\\\
\textit{Author: Bruno Le Floch and Ilia Smilga, France} 
IMO 2008 problem 6:  Let $ ABCD$ be a convex quadrilateral with $ BA\neq BC$. Denote the incircles of triangles $ ABC$ and $ ADC$ by $ \omega_1$ and $ \omega_2$ respectively. Suppose that there exists a circle $ \omega$ tangent to ray $ BA$ beyond $ A$ and to the ray $ BC$ beyond $ C$, which is also tangent to the lines $ AD$ and $ CD$. Prove that the common external tangents to $ \omega_1$ and $\omega_2$ intersect on $ \omega$. \\\\\\
\textit{Author: Vladimir Shmarov, Russia} 

IMO 2007 

IMO 2007 problem 1:  Real numbers $ a_1$, $ a_2$, $ \ldots$, $ a_n$ are given. For each $ i$, $ (1 \leq i \leq n )$, define
\[ d_i = \max \{ a_j\mid 1 \leq j \leq i \} - \min \{ a_j\mid i \leq j \leq n \} \]
and let $ d = \max \{d_i\mid 1 \leq i \leq n \}$.
\begin{enumerate}[(a)]
  \item Prove that, for any real numbers $ x_1\leq x_2\leq \cdots \leq x_n$,
\end{enumerate}
\[ \max \{ |x_i - a_i| \mid 1 \leq i \leq n \}\geq \frac {d}{2}. \quad \quad (*) \]
(b) Show that there are real numbers $ x_1\leq x_2\leq \cdots \leq x_n$ such that the equality holds in (*). \\\\
\textit{Author: Michael Albert, New Zealand} 
IMO 2007 problem 2:  Consider five points $ A$, $ B$, $ C$, $ D$ and $ E$ such that $ ABCD$ is a parallelogram and $ BCED$ is a cyclic quadrilateral. Let $ \ell$ be a line passing through $ A$. Suppose that $ \ell$ intersects the interior of the segment $ DC$ at $ F$ and intersects line $ BC$ at $ G$. Suppose also that $ EF = EG = EC$. Prove that $ \ell$ is the bisector of angle $ DAB$. \\\\
\textit{Author: Charles Leytem, Luxembourg} 
IMO 2007 problem 3:  In a mathematical competition some competitors are friends. Friendship is always mutual. Call a group of competitors a \textit{clique} if each two of them are friends. (In particular, any group of fewer than two competitiors is a clique.) The number of members of a clique is called its \textit{size}. \\\\
Given that, in this competition, the largest size of a clique is even, prove that the competitors can be arranged into two rooms such that the largest size of a clique contained in one room is the same as the largest size of a clique contained in the other room. \\\\
\textit{Author: Vasily Astakhov, Russia} 
IMO 2007 problem 4:  In triangle $ ABC$ the bisector of angle $ BCA$ intersects the circumcircle again at $ R$, the perpendicular bisector of $ BC$ at $ P$, and the perpendicular bisector of $ AC$ at $ Q$. The midpoint of $ BC$ is $ K$ and the midpoint of $ AC$ is $ L$. Prove that the triangles $ RPK$ and $ RQL$ have the same area. \\\\
\textit{Author: Marek Pechal, Czech Republic} 
IMO 2007 problem 5:  Let $a$ and $b$ be positive integers. Show that if $4ab - 1$ divides $(4a^2 - 1)^2$, then $a = b$. \\\\
\textit{Author: Kevin Buzzard and Edward Crane, United Kingdom } 
IMO 2007 problem 6:  Let $ n$ be a positive integer. Consider
\[ S = \left\{ (x,y,z) \mid x,y,z \in \{ 0, 1, \ldots, n\}, x + y + z > 0 \right \} \]
as a set of $ (n + 1)^3 - 1$ points in the three-dimensional space. Determine the smallest possible number of planes, the union of which contains $ S$ but does not include $ (0,0,0)$. \\\\
\textit{Author: Gerhard Wöginger, Netherlands } 

IMO 2006 

IMO 2006 problem 1:  Let $ABC$ be triangle with incenter $I$. A point $P$ in the interior of the triangle satisfies
\[ \angle PBA+\angle PCA = \angle PBC+\angle PCB. \]
Show that $AP \geq AI$, and that equality holds if and only if $P=I$. 
IMO 2006 problem 2:  Let $P$ be a regular $2006$-gon. A diagonal is called \textit{good} if its endpoints divide the boundary of $P$ into two parts, each composed of an odd number of sides of $P$. The sides of $P$ are also called \textit{good}. \\
Suppose $P$ has been dissected into triangles by $2003$ diagonals, no two of which have a common point in the interior of $P$. Find the maximum number of isosceles triangles having two good sides that could appear in such a configuration. 
IMO 2006 problem 3:  Determine the least real number $M$ such that the inequality
\[ |ab(a^2-b^2)+bc(b^2-c^2)+ca(c^2-a^2)| \leq M(a^2+b^2+c^2)^2 \]
holds for all real numbers $a$, $b$ and $c$. 
IMO 2006 problem 4:  Determine all pairs $(x, y)$ of integers such that
\[ 1+2^x+2^{2x+1}= y^2. \] 
IMO 2006 problem 5:  Let $P(x)$ be a polynomial of degree $n > 1$ with integer coefficients and let $k$ be a positive integer. Consider the polynomial $Q(x) = P(P(\ldots P(P(x)) \ldots ))$, where $P$ occurs $k$ times. Prove that there are at most $n$ integers $t$ such that $Q(t) = t$. 
IMO 2006 problem 6:  Assign to each side $b$ of a convex polygon $P$ the maximum area of a triangle that has $b$ as a side and is contained in $P$. Show that the sum of the areas assigned to the sides of $P$ is at least twice the area of $P$. 

IMO 2005 

IMO 2005 problem 1:  Six points are  chosen on the sides of an equilateral triangle $ABC$: $A_1$, $A_2$ on $BC$, $B_1$, $B_2$ on $CA$ and $C_1$, $C_2$ on $AB$, such that they are the vertices of a convex hexagon $A_1A_2B_1B_2C_1C_2$ with equal side lengths. \\\\
Prove that the lines $A_1B_2$, $B_1C_2$ and $C_1A_2$ are concurrent. \\\\
\textit{Bogdan Enescu, Romania} 
IMO 2005 problem 2:  Let $a_1,a_2,\ldots$ be a sequence of integers with infinitely many positive and negative terms. Suppose that for every positive integer $n$ the numbers $a_1,a_2,\ldots,a_n$ leave $n$ different remainders upon division by $n$. \\\\
Prove that every integer occurs exactly once in the sequence $a_1,a_2,\ldots$. 
IMO 2005 problem 3:  Let $x,y,z$ be three positive reals such that $xyz\geq 1$. Prove that
\[
\frac { x^5-x^2 }{x^5+y^2+z^2} + \frac {y^5-y^2}{x^2+y^5+z^2} + \frac {z^5-z^2}{x^2+y^2+z^5} \geq 0 .
\]
\textit{Hojoo Lee, Korea} 
IMO 2005 problem 4:  Determine all positive integers relatively prime to all the terms of the infinite sequence
\[ a_n=2^n+3^n+6^n -1,\ n\geq 1. \] 
IMO 2005 problem 5:  Let $ABCD$ be a fixed convex quadrilateral with $BC=DA$ and $BC$ not parallel with $DA$. Let two variable points $E$ and $F$ lie of the sides $BC$ and $DA$, respectively and satisfy $BE=DF$. The lines $AC$ and $BD$ meet at $P$, the lines $BD$ and $EF$ meet at $Q$, the lines $EF$ and $AC$ meet at $R$. \\\\
Prove that the circumcircles of the triangles $PQR$, as $E$ and $F$ vary, have a common point other than $P$. 
IMO 2005 problem 6:  In a mathematical competition, in which $6$ problems were posed to the participants, every two of these problems were solved by more than $\frac 25$ of the contestants. Moreover, no contestant solved all the $6$ problems. Show that there are at least $2$ contestants who solved exactly $5$ problems each. \\\\
\textit{Radu Gologan and Dan Schwartz} 

IMO 2004 

IMO 2004 problem 1:  \begin{enumerate}
  \item Let $ABC$ be an acute-angled triangle with $AB\neq AC$. The circle with diameter $BC$ intersects the sides $AB$ and $AC$ at $M$ and $N$ respectively. Denote by $O$ the midpoint of the side $BC$. The bisectors of the angles $\angle BAC$ and $\angle MON$ intersect at $R$. Prove that the circumcircles of the triangles $BMR$ and $CNR$ have a common point lying on the side $BC$.
\end{enumerate} 
IMO 2004 problem 2:  Find all polynomials $f$ with real coefficients such that for all reals $a,b,c$ such that $ab+bc+ca = 0$ we have the following relations
\[ f(a-b) + f(b-c) + f(c-a) = 2f(a+b+c). \] 
IMO 2004 problem 3:  Define a ``hook" to be a figure made up of six unit squares as shown below in the picture, or any of the figures obtained by applying rotations and reflections to this figure.
Determine all $ m\times n$ rectangles that can  be covered without gaps and without overlaps with hooks such that \\\\
- the rectangle is covered without gaps and without overlaps \\
- no part of a hook covers area outside the rectangle. 
IMO 2004 problem 4:  Let $n \geq 3$ be an integer. Let $t_1$, $t_2$, ..., $t_n$ be positive real numbers such that
\[
n^2 + 1 > \left( t_1 + t_2 + \cdots + t_n \right) \left( \frac{1}{t_1} + \frac{1}{t_2} + \cdots + \frac{1}{t_n} \right).
\]
Show that $t_i$, $t_j$, $t_k$ are side lengths of a triangle for all $i$, $j$, $k$ with $1 \leq i < j < k \leq n$. 
IMO 2004 problem 5:  In a convex quadrilateral $ABCD$, the diagonal $BD$ bisects neither the angle $ABC$ nor the angle $CDA$. The point $P$ lies inside $ABCD$ and satisfies
\[ \angle PBC=\angle DBA\quad\text{and}\quad \angle PDC=\angle BDA. \]
Prove that $ABCD$ is a cyclic quadrilateral if and only if $AP=CP$. 
IMO 2004 problem 6:  We call a positive integer \textit{alternating} if every two consecutive digits in its decimal representation are of different parity. \\\\
Find all positive integers $n$ such that $n$ has a multiple which is alternating. 

IMO 2003 

IMO 2003 problem 1:  Let $A$ be a $101$-element subset of the set $S=\{1,2,\ldots,1000000\}$. Prove that there exist numbers $t_1$, $t_2, \ldots, t_{100}$ in $S$ such that the sets
\[ A_j=\{x+t_j\mid x\in A\},\qquad j=1,2,\ldots,100 \]
are pairwise disjoint. 
IMO 2003 problem 2:  Determine all pairs of positive integers $(a,b)$ such that
\[ \dfrac{a^2}{2ab^2-b^3+1} \]
is a positive integer. 
IMO 2003 problem 3:  Each pair of opposite sides of a convex hexagon has the following property: the distance between their midpoints is equal to  $\dfrac{\sqrt{3}}{2}$ times the sum of their lengths. Prove that all the angles of the hexagon are equal. 
IMO 2003 problem 4:  Let $ABCD$ be a cyclic quadrilateral. Let $P$, $Q$, $R$ be the feet of the perpendiculars from $D$ to the lines $BC$, $CA$, $AB$, respectively. Show that $PQ=QR$ if and only if the bisectors of $\angle ABC$ and $\angle ADC$ are concurrent with $AC$. 
IMO 2003 problem 5:  Let $n$ be a positive integer and let $x_1\le x_2\le\cdots\le x_n$ be real numbers. \\
Prove that
\[ \left(\sum_{i,j=1}^n|x_i-x_j|\right)^2\le\frac{2(n^2-1)}{3}\sum_{i,j=1}^n(x_i-x_j)^2. \]
Show that the equality holds if and only if $x_1, \ldots, x_n$ is an arithmetic sequence. 
IMO 2003 problem 6:  Let $p$ be a prime number. Prove that there exists a prime number $q$ such that for every integer $n$, the number $n^p-p$ is not divisible by $q$. 

IMO 2002 

IMO 2002 problem 1:  Let $n$ be a positive integer. Each point $(x,y)$ in the plane, where $x$ and $y$ are non-negative integers with $x+y<n$, is coloured red or blue, subject to the following condition: if a point $(x,y)$ is red, then so are all points $(x',y')$ with $x'\leq x$ and $y'\leq y$. Let $A$ be the number of ways to choose $n$ blue points with distinct $x$-coordinates, and let $B$ be the number of ways to choose $n$ blue points with distinct $y$-coordinates. Prove that $A=B$. 
IMO 2002 problem 2:  The circle $S$ has centre $O$, and $BC$ is a diameter of $S$. Let $A$ be a point of $S$ such that $\angle AOB<120{{}^\circ}$.  Let $D$ be the midpoint of the arc $AB$ which does not contain $C$. The line through $O$ parallel to $DA$ meets the line $AC$ at $I$. The perpendicular bisector of $OA$ meets $S$ at $E$ and at $F$. Prove that $I$ is the incentre of the triangle $CEF.$ 
IMO 2002 problem 3:  Find all pairs of positive integers $m,n\geq3$ for which there exist infinitely many positive integers $a$ such that
\[ \frac{a^m+a-1}{a^n+a^2-1} \]
is itself an integer. \\\\
\textit{Laurentiu Panaitopol, Romania} 
IMO 2002 problem 4:  Let $n\geq2$ be a positive integer, with divisors $1=d_1<d_2<\,\ldots<d_k=n$.  Prove that $d_1d_2+d_2d_3+\,\ldots\,+d_{k-1}d_k$ is always less than $n^2$, and determine when it is a divisor of $n^2$. 
IMO 2002 problem 5:  Find all functions $f$ from the reals to the reals such that
\[ \left(f(x)+f(z)\right)\left(f(y)+f(t)\right)=f(xy-zt)+f(xt+yz) \]
for all real $x,y,z,t$. 
IMO 2002 problem 6:  Let $n\geq3$ be a positive integer. Let $C_1,C_2,C_3,\ldots,C_n$ be unit circles in the plane, with centres $O_1,O_2,O_3,\ldots,O_n$ respectively. If no line meets more than two of the circles, prove that
\[ \sum\limits^{}_{1\leq i<j\leq n}{1\over O_iO_j}\leq{(n-1)\pi\over 4}. \] 

IMO 2001 

IMO 2001 problem 1:  Consider an acute-angled triangle $ABC$. Let $P$ be the foot of the altitude of triangle $ABC$ issuing from the vertex $A$, and let $O$ be the circumcenter of triangle $ABC$. Assume that $\angle C \geq \angle B+30^{\circ}$. Prove that $\angle A+\angle COP < 90^{\circ}$. 
IMO 2001 problem 2:  Prove that for all positive real numbers $a,b,c$,
\[ \frac{a}{\sqrt{a^2 + 8bc}} + \frac{b}{\sqrt{b^2 + 8ca}} + \frac{c}{\sqrt{c^2 + 8ab}} \geq 1. \] 
IMO 2001 problem 3:  Twenty-one girls and twenty-one boys took part in a mathematical competition. It turned out that each contestant solved at most six problems, and for each pair of a girl and a boy, there was at least one problem that was solved by both the girl and the boy. Show that there is a problem that was solved by at least three girls and at least three boys. 
IMO 2001 problem 4:  Let $n$ be an odd integer greater than 1 and let $c_1, c_2, \ldots, c_n$ be integers. For each permutation $a = (a_1, a_2, \ldots, a_n)$ of $\{1,2,\ldots,n\}$, define $S(a) = \sum_{i=1}^n c_i a_i$. Prove that there exist permutations $a \neq b$ of $\{1,2,\ldots,n\}$ such that $n!$ is a divisor of $S(a)-S(b)$. 
IMO 2001 problem 5:  Let $ABC$ be a triangle with $\angle BAC = 60^{\circ}$. Let $AP$ bisect $\angle BAC$ and let $BQ$ bisect  $\angle ABC$, with $P$ on $BC$ and $Q$ on $AC$. If $AB + BP = AQ + QB$, what are the angles of the triangle? 
IMO 2001 problem 6:  Let $a > b > c > d$ be positive integers and suppose that
\[ ac + bd = (b+d+a-c)(b+d-a+c). \]
Prove that $ab + cd$ is not prime. 

IMO 2000 

IMO 2000 problem 1:  Two circles $ G_1$ and $ G_2$ intersect at two points $ M$ and $ N$. Let $ AB$ be the line tangent to these circles at $ A$ and $ B$, respectively, so that $ M$ lies closer to $ AB$ than $ N$. Let $ CD$ be the line parallel to $ AB$ and passing through the point $ M$, with $ C$ on $ G_1$ and $ D$ on $ G_2$. Lines $ AC$ and $ BD$ meet at $ E$; lines $ AN$ and $ CD$ meet at $ P$; lines $ BN$ and $ CD$ meet at $ Q$. Show that $ EP = EQ$. 
IMO 2000 problem 2:  Let $ a, b, c$ be positive real numbers so that $ abc = 1$. Prove that
\[
\left( a - 1 + \frac 1b \right) \left( b - 1 + \frac 1c \right) \left( c - 1 + \frac 1a \right) \leq 1.
\] 
IMO 2000 problem 3:  Let $ n \geq 2$ be a positive integer and $ \lambda$ a positive real number. Initially there are $ n$ fleas on a horizontal line, not all at the same point. We define a move as choosing two fleas at some points $ A$ and $ B$, with $ A$ to the left of $ B$, and letting the flea from $ A$ jump over the flea from $ B$ to the point $ C$ so that $ \frac {BC}{AB} = \lambda$. \\\\
Determine all values of $ \lambda$ such that, for any point $ M$ on the line and for any initial position of the $ n$ fleas, there exists a sequence of moves that will take them all to the position right of $ M$. 
IMO 2000 problem 4:  A magician has one hundred cards numbered 1 to 100. He puts them into three boxes, a red one, a white one and a blue one, so that each box contains at least one card. A member of the audience draws two cards from two different boxes and announces the sum of numbers on those cards. Given this information, the magician locates the box from which no card has been drawn. \\\\
How many ways are there to put the cards in the three boxes so that the trick works? 
IMO 2000 problem 5:  Does there exist a positive integer $ n$ such that $ n$ has exactly 2000 prime divisors and $ n$ divides $ 2^n + 1$? 
IMO 2000 problem 6:  Let $ AH_1, BH_2, CH_3$ be the altitudes of an acute angled triangle $ ABC$. Its incircle touches the sides $ BC, AC$ and $ AB$ at $ T_1, T_2$ and $ T_3$ respectively. Consider the symmetric images  of  the  lines $ H_1H_2, H_2H_3$ and $ H_3H_1$ with  respect  to  the  lines $ T_1T_2, T_2T_3$ and $ T_3T_1$.  Prove  that  these  images  form a  triangle  whose  vertices  lie  on  the incircle of $ ABC$. 

IMO 1999 

IMO 1999 problem 1:  A set $ S$ of points from the space will be called \textbf{completely symmetric} if it has at least three elements and fulfills the condition that for every two distinct points $ A$ and $ B$ from $ S$, the perpendicular bisector plane of the segment $ AB$ is a plane of symmetry for $ S$. Prove that if a completely symmetric set is finite, then it consists of the vertices of either a regular polygon, or a regular tetrahedron or a regular octahedron. 
IMO 1999 problem 2:  Let $n \geq 2$ be a fixed integer. Find the least constant $C$ such the inequality
\[ \sum_{i<j} x_ix_j \left(x^2_i+x^2_j \right) \leq C
\left(\sum_ix_i \right)^4 \]
holds for any $x_1, \ldots ,x_n \geq 0$ (the sum on the left consists of $\binom{n}{2}$ summands). For this constant $C$, characterize the instances of equality. 
IMO 1999 problem 3:  Let $n$ be an even positive integer. We say that two different cells of a $n \times n$ board are \textbf{neighboring} if they have a common side. Find the minimal number of cells on the $n \times n$ board that must be marked so that any cell (marked or not marked) has a marked neighboring cell. 
IMO 1999 problem 4:  Find all the pairs of positive integers $(x,p)$ such that p is a prime, $x \leq 2p$ and $x^{p-1}$ is a divisor of $ (p-1)^x+1$. 
IMO 1999 problem 5:  Two circles $\Omega_1$ and $\Omega_2$ touch internally the circle $\Omega$ in M and N and the center of $\Omega_2$ is on $\Omega_1$. The common chord of the circles $\Omega_1$ and $\Omega_2$ intersects $\Omega$ in $A$ and $B$. $MA$ and $MB$ intersects $\Omega_1$ in $C$ and $D$. Prove that $\Omega_2$ is tangent to $CD$. 
IMO 1999 problem 6:  Find all the functions $f: \mathbb{R} \to\mathbb{R}$ such that
\[ f(x-f(y))=f(f(y))+xf(y)+f(x)-1 \]
for all $x,y \in \mathbb{R} $. 

IMO 1998 

IMO 1998 problem 1:  A convex quadrilateral $ABCD$ has perpendicular diagonals. The perpendicular bisectors of the sides $AB$ and $CD$ meet at a unique point $P$ inside $ABCD$. Prove that the quadrilateral $ABCD$ is cyclic if and only if triangles $ABP$ and $CDP$ have equal areas. 
IMO 1998 problem 2:  In a contest, there are $m$ candidates and $n$ judges, where $n\geq 3$ is an odd integer. Each candidate is evaluated by each judge as either pass or fail. Suppose that each pair of judges agrees on at most $k$ candidates. Prove that
\[ {\frac{k}{m}} \geq {\frac{n-1}{2n}}. \] 
IMO 1998 problem 3:  For any positive integer $n$, let $\tau (n)$ denote the number of its positive divisors (including 1 and itself). Determine all positive integers $m$ for which there exists a positive integer $n$ such that $\frac{\tau (n^2)}{\tau (n)}=m$. 
IMO 1998 problem 4:  Determine all pairs $(x,y)$ of positive integers such that $x^2y+x+y$ is divisible by $xy^2+y+7$. 
IMO 1998 problem 5:  Let $I$ be the incenter of triangle $ABC$. Let $K,L$ and $M$ be the points of tangency of the incircle of $ABC$ with $AB,BC$ and $CA$, respectively. The line $t$ passes through $B$ and is parallel to $KL$. The lines $MK$ and $ML$ intersect $t$ at the points $R$ and $S$. Prove that $\angle RIS$ is acute. 
IMO 1998 problem 6:  Determine the least possible value of $f(1998),$ where $f:\Bbb{N}\to \Bbb{N}$ is a function such that for all $m,n\in {\Bbb N}$,
\[ f\left( n^2f(m)\right) =m\left( f(n)\right) ^2. \] 


IMO 1997 

IMO 1997 problem 1:  In the plane the points with integer coordinates are the vertices of unit squares. The squares are coloured alternately black and white (as on a chessboard). For any pair of positive integers $ m$ and $ n$, consider a right-angled triangle whose vertices have integer coordinates and whose legs, of lengths $ m$ and $ n$, lie along edges of the squares. Let $ S_1$ be the total area of the black part of the triangle and $ S_2$ be the total area of the white part. Let $ f(m,n) = | S_1 - S_2 |$.
\begin{enumerate}[a)]
  \item Calculate $ f(m,n)$ for all positive integers $ m$ and $ n$ which are either both even or both odd.
  \item Prove that $ f(m,n) \leq \frac 12 \max \{m,n \}$ for all $ m$ and $ n$.
  \item Show that there is no constant $ C\in\mathbb{R}$ such that $ f(m,n) < C$ for all $ m$ and $ n$.
\end{enumerate} 
IMO 1997 problem 2:  It is known that $ \angle BAC$ is the smallest angle in the triangle $ ABC$. The points $ B$ and $ C$ divide the circumcircle of the triangle into two arcs. Let $ U$ be an interior point of the arc between $ B$ and $ C$ which does not contain $ A$. The perpendicular bisectors of $ AB$ and $ AC$ meet the line $ AU$ at $ V$ and $ W$, respectively. The lines $ BV$ and $ CW$ meet at $ T$. \\\\
Show that $ AU = TB + TC$. \\\\\\
\textit{Alternative formulation:} \\\\
Four different points $ A,B,C,D$ are chosen on a circle $ \Gamma$ such that the triangle $ BCD$ is not right-angled. Prove that:
\begin{enumerate}[(a)]
  \item The perpendicular bisectors of $ AB$ and $ AC$ meet the line $ AD$ at certain points $ W$ and $ V,$ respectively, and that the lines $ CV$ and $ BW$ meet at a certain point $ T.$
  \item The length of one of the line segments $ AD, BT,$ and $ CT$ is the sum of the lengths of the other two.
\end{enumerate} 
IMO 1997 problem 3:  Let $ x_1$, $ x_2$, $ \ldots$, $ x_n$  be real numbers satisfying the conditions:
\[
\left\{
\begin{array}{cccc} |x_1 + x_2 + \cdots + x_n | \& = \& 1 \& \ \\
|x_i| \& \leq \& \displaystyle \frac {n + 1}{2} \& \ \textrm{ for }i = 1, 2, \ldots , n. \end{array}
\right.
\]
Show that there exists a permutation   $ y_1$, $ y_2$, $ \ldots$, $ y_n$  of $ x_1$, $ x_2$, $ \ldots$, $ x_n$  such that
\[ | y_1 + 2 y_2 + \cdots + n y_n | \leq \frac {n + 1}{2}. \] 
IMO 1997 problem 4:  An $ n \times n$ matrix whose entries come from the set $ S = \{1, 2, \ldots , 2n - 1\}$ is called a \textit{silver matrix} if, for each $ i = 1, 2, \ldots , n$, the $ i$-th row and the $ i$-th column together contain all elements of $ S$. Show that:
\begin{enumerate}[(a)]
  \item there is no silver matrix for $ n = 1997$;
  \item silver matrices exist for infinitely many values of $ n$.
\end{enumerate} 
IMO 1997 problem 5:  Find all pairs $ (a,b)$ of positive integers that satisfy the equation: $ a^{b^2} = b^a$. 
IMO 1997 problem 6:  For each positive integer $ n$, let $ f(n)$ denote the number of ways of representing $ n$ as a sum of powers of 2 with nonnegative integer exponents. Representations which differ only in the ordering of their summands are considered to be the same. For instance, $ f(4) = 4$, because the number 4 can be represented in the following four ways: 4; 2+2; 2+1+1; 1+1+1+1. \\\\
Prove that, for any integer $ n \geq 3$ we have $ 2^{\frac {n^2}{4}} < f(2^n) < 2^{\frac {n^2}2}$. 

IMO 1996 

IMO 1996 problem 1:  We are given a positive integer $ r$ and a rectangular board $ ABCD$ with dimensions $ AB = 20, BC = 12$. The rectangle is divided into a grid of $ 20 \times 12$ unit squares. The following moves are permitted on the board: one can move from one square to another only if the distance between the centers of the two squares is $ \sqrt {r}$. The task is to find a sequence of moves leading from the square with $ A$ as a vertex to the square with $ B$ as a vertex.
\begin{enumerate}[(a)]
  \item Show that the task cannot be done if $ r$ is divisible by 2 or 3.
  \item Prove that the task is possible when $ r = 73$.
  \item Can the task be done when $ r = 97$?
\end{enumerate} 
IMO 1996 problem 2:  Let $ P$ be a point inside a triangle $ ABC$ such that
\[ \angle APB - \angle ACB = \angle APC - \angle ABC. \]
Let $ D$, $ E$ be the incenters of triangles $ APB$, $ APC$, respectively. Show that the lines $ AP$, $ BD$, $ CE$ meet at a point. 
IMO 1996 problem 3:  Let $ \mathbb{N}_0$ denote the set of nonnegative integers. Find all  functions $ f$ from $ \mathbb{N}_0$ to itself such that
\[ f(m + f(n)) = f(f(m)) + f(n)\qquad \text{for all} \; m, n \in \mathbb{N}_0. \] 
IMO 1996 problem 4:  The positive integers $ a$ and $ b$ are such that the numbers  $ 15a + 16b$ and $ 16a - 15b$ are both squares of positive integers. What is the least possible value that can be taken on by the smaller of these two squares? 
IMO 1996 problem 5:  Let $ ABCDEF$ be a convex hexagon such that $ AB$ is parallel to  $ DE$, $ BC$ is parallel to $ EF$, and $ CD$ is parallel to $ FA$. Let  $ R_A,R_C,R_E$ denote the circumradii of triangles $ FAB,BCD,DEF$, respectively, and let $ P$ denote the perimeter of the hexagon. Prove that
\[ R_A + R_C + R_E\geq \frac {P}{2}. \] 
IMO 1996 problem 6:  Let $ p,q,n$ be three positive integers with $ p + q < n$. Let $ (x_0,x_1,\cdots ,x_n)$ be an $ (n + 1)$-tuple of integers satisfying the following conditions :
\begin{enumerate}[(a)]
  \item $ x_0 = x_n = 0$, and
  \item For each $ i$ with $ 1\leq i\leq n$, either $ x_i - x_{i - 1} = p$ or $ x_i - x_{i - 1} = - q$.
\end{enumerate}
Show that there exist indices $ i < j$ with $ (i,j)\neq (0,n)$, such that $ x_i = x_j$. 

IMO 1995 

IMO 1995 problem 1:  Let $ A,B,C,D$ be four distinct points on a line, in that order.  The circles with diameters $ AC$ and $ BD$ intersect at $ X$ and $ Y$. The line $ XY$ meets $ BC$ at $ Z$. Let $ P$ be a point on the line $ XY$ other than $ Z$. The line $ CP$ intersects the circle with diameter $ AC$ at $ C$ and $ M$, and the line $ BP$ intersects the circle with diameter $ BD$ at $ B$ and $ N$. Prove that the lines $ AM,DN,XY$ are concurrent. 
IMO 1995 problem 2:  Let $ a$, $ b$, $ c$ be positive real numbers such that $ abc = 1$. Prove that
\[
\frac {1}{a^3\left(b + c\right)} + \frac {1}{b^3\left(c + a\right)} + \frac {1}{c^3\left(a + b\right)}\geq \frac {3}{2}.
\] 
IMO 1995 problem 3:  Determine all integers $ n > 3$ for which there exist $ n$ points $ A_1,\cdots ,A_n$ in the plane, no three collinear, and real numbers $ r_1,\cdots ,r_n$ such that for $ 1\leq i < j < k\leq n$, the area of $ \triangle A_iA_jA_k$ is $ r_i + r_j + r_k$. 
IMO 1995 problem 4:  Find the maximum value of $ x_0$ for which there exists a sequence $ x_0,x_1\cdots ,x_{1995}$ of positive reals with $ x_0 = x_{1995}$, such that
\[ x_{i - 1} + \frac {2}{x_{i - 1}} = 2x_i + \frac {1}{x_i}, \]
for all $ i = 1,\cdots ,1995$. 
IMO 1995 problem 5:  Let $ ABCDEF$ be a convex hexagon with $ AB = BC = CD$ and $ DE = EF = FA$,  such that $ \angle BCD = \angle EFA = \frac {\pi}{3}$. Suppose $ G$ and $ H$ are points in the interior of the hexagon such that $ \angle AGB = \angle DHE = \frac {2\pi}{3}$. Prove that $ AG + GB + GH + DH + HE \geq CF$. 
IMO 1995 problem 6:  Let $ p$ be an odd prime number. How many $ p$-element subsets $ A$ of $ \{1,2,\dots,2p\}$ are there, the sum of whose elements is divisible by $ p$? 


IMO 1994 

IMO 1994 problem 1:  Let $ m$ and $ n$ be two positive integers. Let $ a_1$, $ a_2$, $ \ldots$, $ a_m$ be $ m$ different numbers from the set $ \{1, 2,\ldots, n\}$ such that for any two indices $ i$ and $ j$ with $ 1\leq i \leq j \leq m$ and $ a_i + a_j \leq n$, there exists an index $ k$ such that $ a_i + a_j = a_k$. Show that
\[ \frac {a_1 + a_2 + ... + a_m}{m} \geq \frac {n + 1}{2}. \] 
IMO 1994 problem 2:  Let $ ABC$ be an isosceles triangle with $ AB = AC$. $ M$ is the midpoint of $ BC$ and $ O$ is the point on the line $ AM$ such that $ OB$ is perpendicular to $ AB$. $ Q$ is an arbitrary point on $ BC$ different from $ B$ and $ C$. $ E$ lies on the line $ AB$ and $ F$ lies on the line $ AC$ such that $ E, Q, F$ are distinct and collinear. Prove that $ OQ$ is perpendicular to $ EF$ if and only if $ QE = QF$. 
IMO 1994 problem 3:  For any positive integer $ k$, let $ f_k$ be the number of elements in the set $ \{ k + 1, k + 2, \ldots, 2k\}$ whose base 2 representation contains exactly three 1s.
\begin{enumerate}[(a)]
  \item Prove that for any positive integer $ m$, there exists at least one positive integer $ k$ such that $ f(k) = m$.
  \item Determine all positive integers $ m$ for which there exists \textit{exactly one} $ k$ with $ f(k) = m$.
\end{enumerate} 
IMO 1994 problem 4:  Find all ordered pairs $ (m,n)$ where $ m$ and $ n$ are positive integers such that $ \frac {n^3 + 1}{mn - 1}$ is an integer. 
IMO 1994 problem 5:  Let $ S$ be the set of all real numbers strictly greater than -1. Find all functions $ f: S \to S$ satisfying the two conditions:
\begin{enumerate}[(a)]
  \item $ f(x + f(y) + xf(y)) = y + f(x) + yf(x)$ for all $ x, y$ in $ S$;
  \item $ \frac {f(x)}{x}$ is strictly increasing on each of the two intervals $ - 1 < x < 0$ and $ 0 < x$.
\end{enumerate} 
IMO 1994 problem 6:  Show that there exists a set $ A$ of positive integers with the following property: for any infinite set $ S$ of primes, there exist \textit{two} positive integers $ m$ in $ A$ and $ n$ not in $ A$, each of which is a product of $ k$ distinct elements of $ S$ for some $ k \geq 2$. 

IMO 1993 

IMO 1993 problem 1:  Let $n > 1$ be an integer and let $f(x) = x^n + 5 \cdot x^{n-1} + 3.$ Prove that there do not exist polynomials $g(x),h(x),$ each having integer coefficients and degree at least one, such that $f(x) = g(x) \cdot h(x).$ 
IMO 1993 problem 2:  Let $A$, $B$, $C$, $D$ be four points in the plane, with $C$ and $D$ on the same side of the line $AB$, such that $AC \cdot BD = AD \cdot BC$ and $\angle ADB = 90^{\circ}+\angle ACB$. Find the ratio
\[ \frac{AB \cdot CD}{AC \cdot BD}, \]
and prove that the circumcircles of the triangles $ACD$ and $BCD$ are orthogonal. (Intersecting circles are said to be orthogonal if at either common point their tangents are perpendicuar. Thus, proving that the circumcircles of the triangles $ACD$ and $BCD$ are orthogonal is equivalent to proving that the tangents to the circumcircles of the triangles $ACD$ and $BCD$ at the point $C$ are perpendicular.) 
IMO 1993 problem 3:  On an infinite chessboard, a solitaire game is played as follows: at the start, we have $n^2$ pieces occupying a square of side $n.$ The only allowed move is to jump over an occupied square to an unoccupied one, and the piece which has been jumped over is removed. For which $n$ can the game end with only one piece remaining on the board? 
IMO 1993 problem 4:  For three points $A,B,C$ in the plane, we define $m(ABC)$ to be the smallest length of the three heights of the triangle $ABC$, where in the case $A$, $B$, $C$ are collinear, we set $m(ABC) = 0$. Let $A$, $B$, $C$ be given points in the plane. Prove that for any point $X$ in the plane,
\[ m(ABC) \leq m(ABX) + m(AXC) + m(XBC). \] 
IMO 1993 problem 5:  Let $\mathbb{N} = \{1,2,3, \ldots\}$. Determine if there exists a strictly increasing function $f: \mathbb{N} \mapsto \mathbb{N}$ with the following properties:
\begin{enumerate}[(i)]
  \item $f(1) = 2$;
  \item $f(f(n)) = f(n) + n, (n \in \mathbb{N})$.
\end{enumerate} 
IMO 1993 problem 6:  Let $n > 1$ be an integer. In a circular arrangement of $n$ lamps $L_0, \ldots, L_{n-1},$ each of of which can either ON or OFF, we start with the situation where all lamps are ON, and then carry out a sequence of steps, $Step_0, Step_1, \ldots .$  If $L_{j-1}$ ($j$ is taken mod $n$) is ON then $Step_j$ changes the state of $L_j$ (it goes from ON to OFF or from OFF to ON) but does not change the state of any of the other lamps. If $L_{j-1}$ is OFF then $Step_j$ does not change anything at all. Show that:
\begin{enumerate}[(i)]
  \item There is a positive integer $M(n)$ such that after $M(n)$ steps all lamps are ON again,
  \item If $n$ has the form $2^k$ then all the lamps are ON after $n^2-1$ steps,
  \item If $n$ has the form $2^k + 1$ then all lamps are ON after $n^2 - n + 1$ steps.
\end{enumerate} 

IMO 1992 

IMO 1992 problem 1:  Find all integers $\,a,b,c\,$ with $\,1<a<b<c\,$ such that
\[ (a-1)(b-1)(c-1) \]
is a divisor of $abc-1.$ 
IMO 1992 problem 2:  Let $\,{\mathbb{R}}\,$ denote the set of all real numbers. Find all functions $\,f: {\mathbb{R}}\rightarrow {\mathbb{R}}\,$ such that
\[
f\left( x^2+f(y)\right) =y+\left( f(x)\right) ^2\hspace{0.2in}\text{for all}\,x,y\in \mathbb{R}.
\] 
IMO 1992 problem 3:  Consider $9$ points in space, no four of which are coplanar. Each pair of points is joined by an edge (that is, a line segment) and each edge is either colored blue or red or left uncolored. Find the smallest value of  $\,n\,$ such that whenever exactly $\,n\,$ edges are colored, the set of colored edges necessarily contains a triangle all of whose edges have the same color. 
IMO 1992 problem 4:  In the plane let $\,C\,$ be a circle, $\,L\,$ a line tangent to the circle $\,C,\,$ and $\,M\,$ a point on $\,L$. Find the locus of all points $\,P\,$ with the following property: there exists two points $\,Q,R\,$ on $\,L\,$ such that $\,M\,$ is the midpoint of $\,QR\,$ and $\,C\,$ is the inscribed circle of triangle $\,PQR$.
IMO 1992 problem 5: Let $\,S\,$ be a finite set of points in three-dimensional space. Let $\,S_{x},\,S_{y},\,S_{z}\,$ be the sets consisting of the orthogonal projections of the points of $\,S\,$ onto the $yz$-plane, $zx$-plane, $xy$-plane, respectively. Prove that \[ \vert S\vert^{2}\leq \vert S_{x} \vert \cdot \vert S_{y} \vert \cdot \vert S_{z} \vert, \] where $\vert A \vert$ denotes the number of elements in the finite set $A$. 
IMO 1992 problem 6:  For each positive integer $\,n,\;S(n)\,$ is defined to be the greatest integer such that, for every positive integer $\,k\leq S(n),\;n^2\,$ can be written as the sum of $\,k\,$ positive squares. \\\\
\textbf{a.)} Prove that $\,S(n)\leq n^2-14\,$ for each $\,n\geq 4$. \\
\textbf{b.)} Find an integer $\,n\,$ such that $\,S(n)=n^2-14$. \\
\textbf{c.)} Prove that there are infintely many integers $\,n\,$ such that $S(n)=n^2-14.$ 

IMO 1991 

IMO 1991 problem 1:  Given a triangle $ \,ABC,\,$ let $ \,I\,$ be the center of its inscribed circle. The internal bisectors of the angles $ \,A,B,C\,$ meet the opposite sides in $ \,A^{\prime },B^{\prime },C^{\prime }\,$ respectively. Prove that
\[
\frac {1}{4} < \frac {AI\cdot BI\cdot CI}{AA^{\prime }\cdot BB^{\prime }\cdot CC^{\prime }} \leq \frac {8}{27}.
\] 
IMO 1991 problem 2:  Let $ \,n > 6\,$ be an integer and $ \,a_1,a_2,\cdots ,a_k\,$  be all the natural numbers less than $ n$ and relatively prime to $ n$. If
\[ a_2 - a_1 = a_3 - a_2 = \cdots = a_k - a_{k - 1} > 0, \]
prove that $ \,n\,$ must be either a prime number or a power of $ \,2$. 
IMO 1991 problem 3:  Let $ S = \{1,2,3,\cdots ,280\}$. Find the smallest integer $ n$ such that each $ n$-element subset of $ S$ contains five numbers which are pairwise relatively prime. 
IMO 1991 problem 4:  Suppose $ \,G\,$ is a connected graph with $ \,k\,$ edges. Prove that it is possible to label the edges $ 1,2,\ldots ,k\,$ in such a way that at each vertex which belongs to two or more edges, the greatest common divisor of the integers labeling those edges is equal to 1. \\\\
\textbf{Note: Graph-Definition}. A \textbf{graph} consists of a set of points, called vertices, together with a set of edges joining certain pairs of distinct vertices. Each pair of vertices $ \,u,v\,$ belongs to at most one edge. The graph $ G$ is connected if for each pair of distinct vertices $ \,x,y\,$ there is some sequence of vertices $ \,x = v_0,v_1,v_2,\cdots ,v_m = y\,$ such that each pair $ \,v_i,v_{i + 1}\;(0\leq i < m)\,$ is joined by an edge of $ \,G$. 
IMO 1991 problem 5:  Let $ \,ABC\,$ be a triangle and $ \,P\,$ an interior point of  $ \,ABC\,$. Show that at least one of the angles $ \,\angle PAB,\;\angle PBC,\;\angle PCA\,$ is less than or equal to $ 30^{\circ }$. 
IMO 1991 problem 6:  An infinite sequence $ \,x_0,x_1,x_2,\ldots \,$ of real numbers is said to be \textbf{bounded} if there is a constant $ \,C\,$ such that $ \, \vert x_i \vert \leq C\,$ for every $ \,i\geq 0$. Given any real number $ \,a > 1,\,$ construct a bounded infinite sequence $ x_0,x_1,x_2,\ldots \,$ such that
\[ \vert x_i - x_j \vert \vert i - j \vert^a\geq 1 \]
for every pair of distinct nonnegative integers $ i, j$. 

IMO 1990 

IMO 1990 problem 1:  Chords $ AB$ and $ CD$ of a circle intersect at a point $ E$ inside the circle.  Let $ M$ be an interior point of the segment $ EB$.  The tangent line at $ E$ to the circle through $ D$, $ E$, and $ M$ intersects the lines $ BC$ and $ AC$ at $ F$ and $ G$, respectively.  If
\[ \frac {AM}{AB} = t, \]
find $\frac {EG}{EF}$ in terms of $ t$. 
IMO 1990 problem 2:  Let $ n \geq 3$ and consider a set $ E$ of $ 2n - 1$ distinct points on a circle. Suppose that exactly $ k$ of these points are to be colored black.  Such a coloring is \textbf{good} if there is at least one pair of black points such that the interior of one of the arcs between them contains exactly $ n$ points from $ E$.  Find the smallest value of $ k$ so that every such coloring of $ k$ points of $ E$ is good. 
IMO 1990 problem 3:  Determine all integers $ n > 1$ such that
\[ \frac {2^n + 1}{n^2} \]
is an integer. 
IMO 1990 problem 4:  Let $ {\mathbb Q}^ +$ be the set of positive rational numbers. Construct a function $ f : {\mathbb Q}^ + \rightarrow {\mathbb Q}^ +$ such that
\[ f(xf(y)) = \frac {f(x)}{y} \]
for all $ x$, $ y$ in $ {\mathbb Q}^ +$. 
IMO 1990 problem 5:  Given an initial integer $ n_0 > 1$, two players, $ {\mathcal A}$ and $ {\mathcal B}$, choose integers $ n_1$, $ n_2$, $ n_3$, $ \ldots$ alternately according to the following rules : \\\\
\textbf{I.)} Knowing $ n_{2k}$, $ {\mathcal A}$ chooses any integer $ n_{2k + 1}$ such that
\[ n_{2k} \leq n_{2k + 1} \leq n_{2k}^2. \]
\textbf{II.)} Knowing $ n_{2k + 1}$, $ {\mathcal B}$ chooses any integer $ n_{2k + 2}$ such that
\[ \frac {n_{2k + 1}}{n_{2k + 2}} \]
is a prime raised to a positive integer power. \\\\
Player $ {\mathcal A}$ wins the game by choosing the number 1990; player $ {\mathcal B}$ wins by choosing the number 1.  For which $ n_0$ does : \\\\\\
\textbf{a.)} $ {\mathcal A}$ have a winning strategy? \\
\textbf{b.)} $ {\mathcal B}$ have a winning strategy? \\
\textbf{c.)} Neither player have a winning strategy? 
IMO 1990 problem 6:  Prove that there exists a convex 1990-gon with the following two properties : \\\\
\textbf{a.)} All angles are equal. \\
\textbf{b.)} The lengths of the 1990 sides are the numbers $ 1^2$, $ 2^2$, $ 3^2$, $ \cdots$, $ 1990^2$ in some order. 

IMO 1989 

IMO 1989 problem 1:  Prove that in the set $ \{1,2, \ldots, 1989\}$ can be expressed as the disjoint union of subsets $ A_i, \{i = 1,2, \ldots, 117\}$ such that \\\\
\textbf{i.)} each $ A_i$ contains 17 elements \\\\
\textbf{ii.)} the sum of all the elements in each $ A_i$ is the same. 
IMO 1989 problem 2:  $ ABC$ is a triangle, the bisector of angle $ A$ meets the circumcircle of triangle $ ABC$ in $ A_1$, points $ B_1$ and $ C_1$ are defined similarly. Let $ AA_1$ meet the lines that bisect the two external angles at $ B$ and $ C$ in $ A_0$. Define $ B_0$ and $ C_0$ similarly. Prove that the area of triangle $ A_0B_0C_0 = 2 \cdot$ area of hexagon $ AC_1BA_1CB_1 \geq 4 \cdot$ area of triangle $ ABC$. 
IMO 1989 problem 3:  Let $ n$ and $ k$ be positive integers and let $ S$ be a set of $ n$ points in the plane such that \\\\
\textbf{i.)} no three points of $ S$ are collinear, and \\\\
\textbf{ii.)} for every point $ P$ of $ S$ there are at least $ k$ points of $ S$ equidistant from $ P.$ \\\\
Prove that:
\[ k < \frac {1}{2} + \sqrt {2 \cdot n} \] 
IMO 1989 problem 4:  Let $ ABCD$ be a convex quadrilateral such that the sides $ AB, AD, BC$ satisfy $ AB = AD + BC.$ There exists a point $ P$ inside the quadrilateral at a distance $ h$ from the line $ CD$ such that $ AP = h + AD$  and $ BP = h + BC.$ Show that:
\[ \frac {1}{\sqrt {h}} \geq \frac {1}{\sqrt {AD}} + \frac {1}{\sqrt {BC}} \] 
IMO 1989 problem 5:  Prove that for each positive integer $ n$ there exist $ n$ consecutive positive integers none of which is an integral power of a prime number. 
IMO 1989 problem 6:  A permutation $ \{x_1, x_2, \ldots, x_{2n}\}$ of the set $ \{1,2, \ldots, 2n\}$ where $ n$ is a positive integer, is said to have property $ T$ if $ |x_i - x_{i + 1}| = n$ for at least one $ i$ in $ \{1,2, \ldots, 2n - 1\}.$ Show that, for each $ n$, there are more permutations with property $ T$ than without. 

IMO 1988 

IMO 1988 problem 1:  Consider 2 concentric circle radii $ R$ and  $ r$ ($ R > r$) with centre $ O.$ Fix $ P$ on the small circle and consider the variable chord $ PA$ of the small circle. Points $ B$ and $ C$ lie on the large circle; $ B,P,C$ are collinear and $ BC$ is perpendicular to $ AP.$ \\\\
\textbf{i.)} For which values of $ \angle OPA$ is the sum $ BC^2 + CA^2 + AB^2$ extremal? \\\\
\textbf{ii.)} What are the possible positions of the midpoints $ U$ of $ BA$ and $ V$ of $ AC$ as $ \angle OPA$ varies? 
IMO 1988 problem 2:  Let $ n$ be an even positive integer. Let $ A_1, A_2, \ldots, A_{n + 1}$ be sets having $ n$ elements each such that any two of them have exactly one element in common while every element of their union belongs to at least two of the given sets. For which $ n$ can one assign to every element of the union one of the numbers 0 and 1 in such a manner that each of the sets has exactly $ \frac {n}{2}$ zeros? 
IMO 1988 problem 3:  A function $ f$ defined on the positive integers (and taking positive integers values) is given by: \\\\
$
\begin{matrix} f(1) = 1, f(3) = 3 \\
f(2 \cdot n) = f(n) \\
f(4 \cdot n + 1) = 2 \cdot f(2 \cdot n + 1) - f(n) \\
f(4 \cdot n + 3) = 3 \cdot f(2 \cdot n + 1) - 2 \cdot f(n), \end{matrix}
$ \\\\
for all positive integers $ n.$ Determine with proof the number of positive integers $ \leq 1988$ for which $ f(n) = n.$ 
IMO 1988 problem 4:  Show that the solution set of the inequality
\[ \sum^{70}_{k = 1} \frac {k}{x - k} \geq \frac {5}{4} \]
is a union of disjoint intervals, the sum of whose length is 1988. 
IMO 1988 problem 5:  In a right-angled triangle $ ABC$ let $ AD$ be the altitude drawn to the hypotenuse and let the straight line joining the incentres of the triangles $ ABD, ACD$ intersect the sides $ AB, AC$ at the points $ K,L$ respectively. If $ E$ and $ E_1$ dnote the areas of triangles $ ABC$ and $ AKL$ respectively, show that
\[ \frac {E}{E_1} \geq 2. \] 
IMO 1988 problem 6:  Let $ a$ and $ b$ be two positive integers such that $ a \cdot b + 1$ divides $ a^2 + b^2$. Show that $ \frac {a^2 + b^2}{a \cdot b + 1}$ is a perfect square. 

IMO 1987 

IMO 1987 problem 1:  Let $p_n(k)$ be the number of permutations of the set $\{1,2,3,\ldots,n\}$ which have exactly $k$ fixed points. Prove that $\sum_{k=0}^nk p_n(k)=n!$. 
IMO 1987 problem 2:  In an acute-angled triangle $ABC$ the interior bisector of angle $A$ meets $BC$ at $L$ and meets the circumcircle of $ABC$ again at $N$. From $L$ perpendiculars are drawn to $AB$ and $AC$, with feet $K$ and $M$ respectively. Prove that the quadrilateral $AKNM$ and the triangle $ABC$ have equal areas. 
IMO 1987 problem 3:  Let $x_1,x_2,\ldots,x_n$ be real numbers satisfying $x_1^2+x_2^2+\ldots+x_n^2=1$. Prove that for every integer $k\ge2$ there are integers $a_1,a_2,\ldots,a_n$, not all zero, such that $|a_i|\le k-1$ for all $i$, and $|a_1x_1+a_2x_2+\ldots+a_nx_n|\le{(k-1)\sqrt n\over k^n-1}$. 
IMO 1987 problem 4:  Prove that there is no function $f$ from the set of non-negative integers into itself such that $f(f(n))=n+1987$ for all $n$. 
IMO 1987 problem 5:  Let $n\ge3$ be an integer. Prove that there is a set of $n$ points in the plane such that the distance between any two points is irrational and each set of three points determines a non-degenerate triangle with rational area. 
IMO 1987 problem 6:  Let $n\ge2$ be an integer. Prove that if $k^2+k+n$ is prime for all integers $k$ such that $0\le k\le\sqrt{n\over3}$, then $k^2+k+n$ is prime for all integers $k$ such that $0\le k\le n-2$. 

IMO 1986 

IMO 1986 problem 1:  Let $d$ be any positive integer not equal to $2, 5$ or $13$. Show that one can find distinct $a,b$ in the set $\{2,5,13,d\}$ such that $ab-1$ is not a perfect square. 
IMO 1986 problem 2:  Given a point $P_0$ in the plane of the triangle $A_1A_2A_3$. Define  $A_s=A_{s-3}$ for all $s\ge4$. Construct a set of points $P_1,P_2,P_3,\ldots$ such that $P_{k+1}$ is the image of $P_k$ under a rotation center $A_{k+1}$ through an angle $120^o$ clockwise for $k=0,1,2,\ldots$. Prove that if $P_{1986}=P_0$, then the triangle $A_1A_2A_3$ is equilateral. 
IMO 1986 problem 3:  To each vertex of a regular pentagon an integer is assigned, so that the sum of all five numbers is positive. If three consecutive vertices are assigned the numbers $x,y,z$ respectively, and $y<0$, then the following operation is allowed: $x,y,z$ are replaced by $x+y,-y,z+y$ respectively. Such an operation is performed repeatedly as long as at least one of the five numbers is negative. Determine whether this procedure necessarily comes to an end after a finite number of steps. 
IMO 1986 problem 4:  Let $A,B$ be adjacent vertices of a regular $n$-gon ($n\ge5$) with center $O$. A triangle $XYZ$, which is congruent to and initially coincides with $OAB$, moves in the plane in such a way that $Y$ and $Z$ each trace out the whole boundary of the polygon, with $X$ remaining inside the polygon. Find the locus of $X$. 
IMO 1986 problem 5:  Find all functions $f$ defined on the non-negative reals and taking non-negative real values such that: $f(2)=0,f(x)\ne0$ for $0\le x<2$, and $f(xf(y))f(y)=f(x+y)$ for all $x,y$. 
IMO 1986 problem 6:  Given a finite set of points in the plane, each with integer coordinates, is it always possible to color the points red or white so that for any straight line $L$ parallel to one of the coordinate axes the difference (in absolute value) between the numbers of white and red points on $L$ is not greater than $1$? 

IMO 1985 

IMO 1985 problem 1:  A circle has center on the side $AB$ of the cyclic quadrilateral $ABCD$. The other three sides are tangent to the circle. Prove that $AD+BC=AB$. 
IMO 1985 problem 2:  Let $n$ and $k$ be relatively prime positive integers with $k<n$. Each number in the set $M=\{1,2,3,\ldots,n-1\}$ is colored either blue or white. For each $i$ in $M$, both $i$ and $n-i$ have the same color. For each $i\ne k$ in $M$ both $i$ and $|i-k|$ have the same color. Prove that all numbers in $M$ must have the same color. 
IMO 1985 problem 3:  For any polynomial $P(x)=a_0+a_1x+\ldots+a_kx^k$ with integer coefficients, the number of odd coefficients is denoted by $o(P)$. For $i-0,1,2,\ldots$ let $Q_i(x)=(1+x)^i$. Prove that if $i_1,i_2,\ldots,i_n$ are integers satisfying $0\le i_1<i_2<\ldots<i_n$, then:
\[ o(Q_{i_1}+Q_{i_2}+\ldots+Q_{i_n})\ge o(Q_{i_1}). \] 
IMO 1985 problem 4:  Given a set $M$ of $1985$ distinct positive integers, none of which has a prime divisor greater than $23$, prove that $M$ contains a subset of $4$ elements whose product is the $4$th power of an integer. 
IMO 1985 problem 5:  A circle with center $O$ passes through the vertices $A$ and $C$ of the triangle $ABC$ and intersects the segments $AB$ and $BC$ again at distinct points $K$ and $N$ respectively. Let $M$ be the point of intersection of the circumcircles of triangles $ABC$ and $KBN$ (apart from $B$). Prove that $\angle OMB=90^{\circ}$. 
IMO 1985 problem 6:  For every real number $x_1$, construct the sequence $x_1,x_2,\ldots$ by setting:
\[ x_{n+1}=x_n(x_n+{1\over n}). \]
Prove that there exists exactly one value of $x_1$ which gives $0<x_n<x_{n+1}<1$ for all $n$. 

IMO 1984 

IMO 1984 problem 1:  Prove that $0\le yz+zx+xy-2xyz\le{7\over27}$, where $x,y$ and $z$ are non-negative real numbers satisfying $x+y+z=1$. 
IMO 1984 problem 2:  Find one pair of positive integers $a,b$ such that $ab(a+b)$ is not divisible by $7$, but $(a+b)^7-a^7-b^7$ is divisible by $7^7$. 
IMO 1984 problem 3:  Given points $O$ and $A$ in the plane. Every point in the plane is colored with one of a finite number of colors. Given a point $X$ in the plane, the circle $C(X)$ has center $O$ and radius $OX+{\angle AOX\over OX}$, where $\angle AOX$ is measured in radians in the range $[0,2\pi)$. Prove that we can find a point $X$, not on $OA$, such that its color appears on the circumference of the circle $C(X)$. 
IMO 1984 problem 4:  Let $ABCD$ be a convex quadrilateral with the line $CD$ being tangent to the circle on diameter $AB$. Prove that the line $AB$ is tangent to the circle on diameter $CD$ if and only if the lines $BC$ and $AD$ are parallel. 
IMO 1984 problem 5:  Let $ d$ be the sum of the lengths of all the diagonals of a plane convex polygon with $ n$ vertices (where $ n>3$). Let $ p$ be its perimeter. Prove that:
\[ n-3<{2d\over p}<\Bigl[{n\over2}\Bigr]\cdot\Bigl[{n+1\over 2}\Bigr]-2, \]
where $ [x]$ denotes the greatest integer not exceeding $ x$. 
IMO 1984 problem 6:  Let $a,b,c,d$ be odd integers such that $0<a<b<c<d$ and $ad=bc$. Prove that if $a+d=2^k$ and $b+c=2^m$ for some integers $k$ and $m$, then $a=1$. 

IMO 1983 

IMO 1983 problem 1:  Find all functions $f$ defined on the set of positive reals which take positive real values and satisfy: $f(xf(y))=yf(x)$ for all $x,y$; and $f(x)\to0$ as $x\to\infty$. 
IMO 1983 problem 2:  Let $A$ be one of the two distinct points of intersection of two unequal coplanar circles $C_1$ and $C_2$ with centers $O_1$ and $O_2$ respectively. One of the common tangents to the circles touches $C_1$ at $P_1$ and $C_2$ at $P_2$, while the other touches $C_1$ at $Q_1$ and $C_2$ at $Q_2$. Let $M_1$ be the midpoint of $P_1Q_1$ and $M_2$ the midpoint of $P_2Q_2$. Prove that $\angle O_1AO_2=\angle M_1AM_2$. 
IMO 1983 problem 3:  Let $a,b$ and $c$ be positive integers, no two of which have a common divisor greater than $1$. Show that $2abc-ab-bc-ca$ is the largest integer which cannot be expressed in the form $xbc+yca+zab$, where $x,y,z$ are non-negative integers. 
IMO 1983 problem 4:  Let $ABC$ be an equilateral triangle and $\mathcal{E}$ the set of all points contained in the three segments $AB$, $BC$, and $CA$ (including $A$, $B$, and $C$).  Determine whether, for every partition of $\mathcal{E}$ into two disjoint subsets, at least one of the two subsets contains the vertices of a right-angled triangle. 
IMO 1983 problem 5:  Is it possible to choose $1983$ distinct positive integers, all less than or equal to $10^5$, no three of which are consecutive terms of an arithmetic progression? 
IMO 1983 problem 6:  Let $ a$, $ b$ and $ c$ be the lengths of the sides of a triangle. Prove that
\[ a^2b(a - b) + b^2c(b - c) + c^2a(c - a)\ge 0. \]
Determine when equality occurs. 

IMO 1982 

IMO 1982 problem 1:  The function $f(n)$ is defined on the positive integers and takes non-negative integer values. $f(2)=0,f(3)>0,f(9999)=3333$ and for all $m,n:$
\[ f(m+n)-f(m)-f(n)=0 \text{ or } 1. \]
Determine $f(1982)$. 
IMO 1982 problem 2:  A non-isosceles triangle $A_1A_2A_3$ has sides $a_1$, $a_2$, $a_3$ with the side $a_i$ lying opposite to the vertex $A_i$. Let $M_i$ be the midpoint of the side $a_i$, and let $T_i$ be the point where the inscribed circle of triangle $A_1A_2A_3$ touches the side $a_i$. Denote by $S_i$ the reflection of the point $T_i$ in the interior angle bisector of the angle $A_i$. Prove that the lines $M_1S_1$, $M_2S_2$ and $M_3S_3$ are concurrent. 
IMO 1982 problem 3:  Consider infinite sequences $\{x_n\}$ of positive reals such that $x_0=1$ and $x_0\ge x_1\ge x_2\ge\ldots$. \\\\
\textbf{a)} Prove that for every such sequence there is an $n\ge1$ such that:
\[ {x_0^2\over x_1}+{x_1^2\over x_2}+\ldots+{x_{n-1}^2\over x_n}\ge3.999. \]
\textbf{b)} Find such a sequence such that for all $n$:
\[ {x_0^2\over x_1}+{x_1^2\over x_2}+\ldots+{x_{n-1}^2\over x_n}<4. \] 
IMO 1982 problem 4:  Prove that if $n$ is a positive integer such that the equation
\[ x^3-3xy^2+y^3=n \]
has a solution in integers $x,y$, then it has at least three such solutions. Show that the equation has no solutions in integers for $n=2891$. 
IMO 1982 problem 5:  The diagonals $AC$ and $CE$ of the regular hexagon $ABCDEF$ are divided by inner points $M$ and $N$ respectively, so that
\[ {AM\over AC}={CN\over CE}=r. \]
Determine $r$ if $B,M$ and $N$ are collinear. 
IMO 1982 problem 6:  Let $S$ be a square with sides length $100$. Let $L$ be a path within $S$ which does not meet itself and which is composed of line segments $A_0A_1,A_1A_2,A_2A_3,\ldots,A_{n-1}A_n$ with $A_0=A_n$. Suppose that for every point $P$ on the boundary of $S$ there is a point of $L$ at a distance from $P$ no greater than $\frac {1} {2}$. Prove that there are two points $X$ and $Y$ of $L$ such that the distance between $X$ and $Y$ is not greater than $1$ and the length of the part of $L$ which lies between $X$ and $Y$ is not smaller than $198$. 

IMO 1981 

IMO 1981 problem 1:  Consider a variable point $P$ inside a given triangle $ABC$. Let $D$, $E$, $F$ be the feet of the perpendiculars from the point $P$ to the lines $BC$, $CA$, $AB$, respectively. Find all points $P$ which minimize the sum
\[ {BC\over PD}+{CA\over PE}+{AB\over PF}. \] 
IMO 1981 problem 2:  Take $r$ such that $1\le r\le n$, and consider all subsets of $r$ elements of the set $\{1,2,\ldots,n\}$. Each subset has a smallest element. Let $F(n,r)$ be the arithmetic mean of these smallest elements. Prove that:
\[ F(n,r)={n+1\over r+1}. \] 
IMO 1981 problem 3:  Determine the maximum value of $m^2+n^2$, where $m$ and $n$ are integers in the range $1,2,\ldots,1981$ satisfying $(n^2-mn-m^2)^2=1$. 
IMO 1981 problem 4:  \textbf{a.)} For which $n>2$ is there a set of $n$ consecutive positive integers such that the largest number in the set is a divisor of the least common multiple of the remaining $n-1$ numbers? \\\\
\textbf{b.)} For which $n>2$ is there exactly one set having this property? 
IMO 1981 problem 5:  Three circles of equal radius have a common point $O$ and lie inside a given triangle. Each circle touches a pair of sides of the triangle. Prove that the incenter and the circumcenter of the triangle are collinear with the point $O$. 
IMO 1981 problem 6:  The function $f(x,y)$ satisfies: $f(0,y)=y+1, f(x+1,0) = f(x,1), f(x+1,y+1)=f(x,f(x+1,y))$ for all non-negative integers $x,y$. Find $f(4,1981)$. 


IMO 1979 

IMO 1979 problem 1:  If $p$ and $q$ are natural numbers so that
\[ \frac{p}{q}=1-\frac{1}{2}+\frac{1}{3}-\frac{1}{4}+ \ldots -\frac{1}{1318}+\frac{1}{1319}, \]
prove that $p$ is divisible with $1979$. 
IMO 1979 problem 2:  We consider a prism which has the upper and inferior basis the pentagons: $A_1A_2A_3A_4A_5$ and $B_1B_2B_3B_4B_5$. Each of the sides of the two pentagons and the segments $A_iB_j$ with $i,j=1,\ldots$,5 is colored in red or blue. In every triangle which has all sides colored there exists one red side and one blue side. Prove that all the 10 sides of the two basis are colored in the same color. 
IMO 1979 problem 3:  Two circles in a plane intersect. $A$ is one of the points of intersection. Starting simultaneously from $A$ two points move with constant speed, each travelling along its own circle in the same sense. The two points return to $A$ simultaneously after one revolution. Prove that there is a fixed point $P$ in the plane such that the two points are always equidistant from $P.$ 
IMO 1979 problem 4:  We consider a point $P$ in a plane $p$ and a point $Q \not\in p$. Determine all the points $R$ from $p$ for which
\[ \frac{QP+PR}{QR} \]
is maximum. 
IMO 1979 problem 5:  Determine all real numbers a for which there exists positive reals $x_1, \ldots, x_5$ which satisfy the relations $ \sum_{k=1}^5 kx_k=a,$ $ \sum_{k=1}^5 k^3x_k=a^2,$ $ \sum_{k=1}^5 k^5x_k=a^3.$ 
IMO 1979 problem 6:  Let $A$ and $E$ be opposite vertices of an octagon. A frog starts at vertex $A.$ From any vertex except $E$ it jumps to one of the two adjacent vertices. When it reaches $E$ it stops. Let $a_n$ be the number of distinct paths of exactly $n$ jumps ending at $E$. Prove that:
\[ a_{2n-1}=0, \quad a_{2n}={(2+\sqrt2)^{n-1} - (2-\sqrt2)^{n-1} \over\sqrt2}. \] 

IMO 1978 

IMO 1978 problem 1:  Let $ m$ and $ n$ be positive integers such that $ 1 \le m < n$.  In their decimal representations, the last three digits of $ 1978^m$ are equal, respectively, to the last three digits of $ 1978^n$.  Find $ m$ and $ n$ such that $ m + n$ has its least value. 
IMO 1978 problem 2:  We consider a fixed point $P$ in the interior of a fixed sphere$.$ We construct three segments $PA, PB,PC$, perpendicular two by two$,$ with the vertexes $A, B, C$ on the sphere$.$ We consider the vertex $Q$ which is opposite to $P$ in the parallelepiped (with right angles) with $PA, PB, PC$ as edges$.$ Find the locus of the point $Q$ when $A, B, C$ take all the positions compatible with our problem. 
IMO 1978 problem 3:  Let $0<f(1)<f(2)<f(3)<\ldots$ a sequence with all its terms positive$.$ The $n-th$ positive integer which doesn't belong to the sequence is $f(f(n))+1.$ Find $f(240).$ 
IMO 1978 problem 4:  In a triangle $ABC$ we have $AB = AC.$ A circle which is internally tangent with the circumscribed circle of the triangle is also tangent to the sides $AB, AC$ in the points $P,$ respectively $Q.$ Prove that the midpoint of $PQ$ is the center of the inscribed circle of the triangle $ABC.$ 
IMO 1978 problem 5:  Let $f$ be an injective function from ${1,2,3,\ldots}$ in itself. Prove that for any $n$ we have: $\sum_{k=1}^n f(k)k^{-2} \geq \sum_{k=1}^n k^{-1}.$ 
IMO 1978 problem 6:  An international society has its members from six different countries.  The list of members contain $1978$ names, numbered $1, 2, \dots, 1978$.  Prove that there is at least one member whose number is the sum of the numbers of two members from his own country, or twice as large as the number of one member from his own country. 

IMO 1977 

IMO 1977 problem 1:  In the interior of a square $ABCD$ we construct the equilateral triangles $ABK, BCL, CDM, DAN.$ Prove that the midpoints of the four segments $KL, LM, MN, NK$ and the midpoints of the eight segments $AK, BK, BL, CL, CM, DM, DN, AN$ are the 12 vertices of a regular dodecagon. 
IMO 1977 problem 2:  In a finite sequence of real numbers the sum of any seven successive terms is negative and the sum of any eleven successive terms is positive. Determine the maximum number of terms in the sequence. 
IMO 1977 problem 3:  Let $n$ be a given number greater than 2. We consider the set $V_n$ of all the integers of the form $1 + kn$ with $k = 1, 2, \ldots$ A number $m$ from $V_n$ is called indecomposable in $V_n$ if there are not two numbers $p$ and $q$ from $V_n$ so that $m = pq.$ Prove that there exist a number $r \in V_n$ that can be expressed as the product of elements indecomposable in $V_n$ in more than one way. (Expressions which differ only in order of the elements of $V_n$ will be considered the same.) 
IMO 1977 problem 4:  Let $a,b,A,B$ be given reals. We consider the function defined by
\[ f(x) = 1 - a \cdot \cos(x) - b \cdot \sin(x) - A \cdot \cos(2x) - B \cdot \sin(2x). \]
Prove that if for any real number $x$ we have $f(x) \geq 0$ then $a^2 + b^2 \leq 2$ and $A^2 + B^2 \leq 1.$ 
IMO 1977 problem 5:  Let $a,b$ be two natural numbers. When we divide $a^2+b^2$ by $a+b$, we the the remainder $r$ and the quotient $q.$ Determine all pairs $(a, b)$ for which $q^2 + r = 1977.$ 
IMO 1977 problem 6:  Let $\mathbb{N}$ be the set of positive integers. Let $f$ be a function defined on $\mathbb{N}$, which satisfies the inequality $f(n + 1) > f(f(n))$ for all $n \in \mathbb{N}$. Prove that for any $n$ we have $f(n) = n.$ 

IMO 1976 

IMO 1976 problem 1:  In a convex quadrilateral (in the plane) with the area of $32 \text{ cm}^2$ the sum of two opposite sides and a diagonal is $16 \text{ cm}$. Determine all the possible values that the other diagonal can have. 
IMO 1976 problem 2:  Let $P_1(x)=x^2-2$ and $P_j(x)=P_1(P_{j-1}(x))$ for j$=2,\ldots$ Prove that for any positive integer n the roots of the equation $P_n(x)=x$ are all real and distinct. 
IMO 1976 problem 3:  A box whose shape is a parallelepiped can be completely filled with cubes of side $1.$ If we put in it the maximum possible number of cubes, each of volume $2$, with the sides parallel to those of the box, then exactly $40$ percent of the volume of the box is occupied. Determine the possible dimensions of the box. 
IMO 1976 problem 4:  Determine the greatest number, who is the product of some positive integers, and the sum of these numbers is $1976.$ 
IMO 1976 problem 5:  We consider the following system \\
with $q=2p$:
\[
\begin{matrix} a_{11}x_1+\ldots+a_{1q}x_q=0,\\ a_{21}x_1+\ldots+a_{2q}x_q=0,\\ \ldots ,\\ a_{p1}x_1+\ldots+a_{pq}x_q=0,\\ \end{matrix}
\]
in which every coefficient is an element from the set $\{-1,0,1\}$$.$ Prove that there exists a solution $x_1, \ldots,x_q$ for the system with the properties: \\\\
\textbf{a.)} all $x_j, j=1,\ldots,q$ are integers$;$ \\\\
\textbf{b.)} there exists at least one j for which $x_j \neq 0;$ \\\\
\textbf{c.)} $|x_j| \leq q$ for any $j=1, \ldots ,q.$ 
IMO 1976 problem 6:  A sequence $(u_n)$ is defined by
\[ u_0=2 \quad u_1=\frac{5}{2}, u_{n+1}=u_n(u_{n-1}^2-2)-u_1 \quad \textnormal{for  } n=1,\ldots \]
Prove that for any positive integer $n$ we have
\[ [u_n]=2^{\frac{(2^n-(-1)^n)}{3}} \]
(where $[x]$ denotes the smallest integer $\leq x)$ 

IMO 1975 

IMO 1975 problem 1:  We consider two sequences of real numbers $x_1 \geq x_2 \geq \ldots \geq x_n$ and $\ y_1 \geq y_2 \geq \ldots \geq y_n.$ Let $z_1, z_2, .\ldots, z_n$ be a permutation of the numbers $y_1, y_2, \ldots, y_n.$ Prove that $\sum \limits_{i=1}^n ( x_i -\ y_i )^2 \leq \sum \limits_{i=1}^n$ $( x_i - z_i)^2.$ 
IMO 1975 problem 2:  Let $a_1, \ldots, a_n$ be an infinite sequence of strictly positive integers, so that $a_k < a_{k+1}$ for any $k.$ Prove that there exists an infinity of terms $ a_m,$ which can be written like $a_m = x \cdot a_p + y \cdot a_q$ with $x,y$ strictly positive integers and $p \neq q.$ 
IMO 1975 problem 3:  In the plane of a triangle $ABC,$ in its exterior$,$ we draw the triangles $ABR, BCP, CAQ$ so that $\angle PBC = \angle CAQ = 45^{\circ}$, $\angle BCP = \angle QCA = 30^{\circ}$, $\angle ABR = \angle RAB = 15^{\circ}$. \\\\
Prove that \\\\
\textbf{a.)} $\angle QRP = 90\,^{\circ},$ and \\\\
\textbf{b.)} $QR = RP.$ 
IMO 1975 problem 4:  When $4444^{4444}$ is written in decimal notation, the sum of its digits is $ A.$ Let $B$ be the sum of the digits of $A.$ Find the sum of the digits of $ B.$ ($A$ and $B$ are written in decimal notation.) 
IMO 1975 problem 5:  Can there be drawn on a circle of radius $1$ a number of $1975$ distinct points, so that the distance (measured on the chord) between any two points (from the considered points) is a rational number? 
IMO 1975 problem 6:  Determine the polynomials P of two variables so that: \\\\
\textbf{a.)} for any real numbers $t,x,y$ we have $P(tx,ty) = t^n P(x,y)$ where $n$ is a positive integer, the same for all $t,x,y;$ \\\\
\textbf{b.)} for any real numbers $a,b,c$ we have $P(a + b,c) + P(b + c,a) + P(c + a,b) = 0;$ \\\\
\textbf{c.)} $P(1,0) =1.$ 

IMO 1974 

IMO 1974 problem 1:  Three players $A,B$ and $C$ play a game with three cards and on each of these $3$ cards it is written a positive integer, all $3$ numbers are different. A game consists of shuffling the cards, giving each player a card and each player is attributed a number of points equal to the number written on the card and then they give the cards back. After a number $(\geq 2)$ of games we find out that A has $20$ points, $B$ has $10$ points and $C$ has $9$ points. We also know that in the last game B had the card with the biggest number. Who had in the first game the card with the second value (this means the middle card concerning its value). 
IMO 1974 problem 2: Let $ABC$ be a triangle. Prove that there exists a point $D$ on the side $AB$ of the triangle $ABC$, such that $CD$ is the geometric mean of $AD$ and $DB$, iff the triangle $ABC$ satisfies the inequality $\sin A\sin B\le\sin^2\frac{C}{2}$.
IMO 1974 problem 3:  Prove that for any n natural, the number
\[ \sum \limits_{k=0}^n \binom{2n+1}{2k+1} 2^{3k} \]
cannot be divided by $5$. 
IMO 1974 problem 4:  Consider decompositions of an $8\times 8$ chessboard into $p$ non-overlapping rectangles subject to the following conditions:
\begin{enumerate}[(i)]
  \item Each rectangle has as many white squares as black squares.
  \item If $a_i$ is the number of white squares in the $i$-th rectangle, then $a_1<a_2<\ldots <a_p$.
\end{enumerate}
Find the maximum value of $p$ for which such a decomposition is possible. For this value of $p$, determine all possible sequences $a_1,a_2,\ldots ,a_p$. 
IMO 1974 problem 5:  The variables $a,b,c,d,$ traverse, independently from each other, the set of positive real values. What are the values which the expression
\[ S= \frac{a}{a+b+d} + \frac{b}{a+b+c} + \frac{c}{b+c+d} + \frac{d}{a+c+d} \]
takes? 
IMO 1974 problem 6:  Let $P(x)$ be a polynomial with integer coefficients. We denote $\deg(P)$ its degree which is $\geq 1.$ Let $n(P)$ be the number of all the integers $k$ for which we have $(P(k))^2=1.$ Prove that $n(P)- \deg(P) \leq 2.$ 

IMO 1973 

IMO 1973 problem 1:  Prove that the sum of an odd number of vectors of length 1, of common origin $O$ and all situated in the same semi-plane determined by a straight line which goes through $O,$ is at least 1. 
IMO 1973 problem 2:  Establish if there exists a finite set $M$ of points in space, not all situated in the same plane, so that for any straight line $d$ which contains at least two points from M there exists another straight line $d'$, parallel with $d,$ but distinct from $d$, which also contains at least two points from $M$. 
IMO 1973 problem 3:  Determine the minimum value of $a^2 + b^2$ when $(a,b)$ traverses all the pairs of real numbers for which the equation
\[ x^4 + ax^3 + bx^2 + ax + 1 = 0 \]
has at least one real root. 
IMO 1973 problem 4:  A soldier needs to check if there are any mines in the interior or on the sides of an equilateral triangle $ABC.$ His detector can detect a mine at a maximum distance equal to half the height of the triangle. The soldier leaves from one of the vertices of the triangle. Which is the minimum distance that he needs to traverse so that at the end of it he is sure that he completed successfully his mission? 
IMO 1973 problem 5:  $G$ is a set of non-constant functions $f$. Each $f$ is defined on the real line and has the form $f(x)=ax+b$ for some real $a,b$. If $f$ and $g$ are in $G$, then so is $fg$, where $fg$ is defined by $fg(x)=f(g(x))$. If $f$ is in $G$, then so is the inverse $f^{-1}$. If $f(x)=ax+b$, then $f^{-1}(x)= \frac{x-b}{a}$. Every $f$ in $G$ has a fixed point (in other words we can find $x_f$ such that $f(x_f)=x_f$. Prove that all the functions in $G$ have a common fixed point. 
IMO 1973 problem 6:  Let $a_1, \ldots, a_n$ be $n$ positive numbers and $0 < q < 1.$ Determine $n$ positive numbers $b_1, \ldots, b_n$ so that: \\\\
\textit{a.)} $ a_k < b_k$ for all $k = 1, \ldots, n,$ \\
\textit{b.)} $q < \frac{b_{k+1}}{b_k} < \frac{1}{q}$ for all $k = 1, \ldots, n-1,$ \\
\textit{c.)} $\sum \limits^n_{k=1} b_k < \frac{1+q}{1-q} \cdot \sum \limits^n_{k=1} a_k.$ 

IMO 1972 

IMO 1972 problem 1:  Prove that from a set of ten distinct two-digit numbers, it is always possible to find two disjoint subsets whose members have the same sum. 
IMO 1972 problem 2:  Given $n>4$, prove that every cyclic quadrilateral can be dissected into $n$ cyclic quadrilaterals. 
IMO 1972 problem 3:  Prove that $(2m)!(2n)!$ is a multiple of $m!n!(m+n)!$ for any non-negative integers $m$ and $n$. 
IMO 1972 problem 4:  Find all positive real solutions to:
\begin{eqnarray*} (x_1^2-x_3x_5)(x_2^2-x_3x_5) \&\le\& 0 \\ (x_2^2-x_4x_1)(x_3^2-x_4x_1) \&\le\& 0 \\ (x_3^2-x_5x_2)(x_4^2-x_5x_2) \&\le\& 0 \\ (x_4^2-x_1x_3)(x_5^2-x_1x_3) \&\le \& 0 \\ (x_5^2-x_2x_4)(x_1^2-x_2x_4) \&\le\& 0 \\ \end{eqnarray*} 
IMO 1972 problem 5:  $f$ and $g$ are real-valued functions defined on the real line. For all $x$ and $y, f(x+y)+f(x-y)=2f(x)g(y)$. $f$ is not identically zero and $|f(x)|\le1$ for all $x$. Prove that $|g(x)|\le1$ for all $x$. 
IMO 1972 problem 6:  Given four distinct parallel planes, prove that there exists a regular tetrahedron with a vertex on each plane. 

IMO 1971 

IMO 1971 problem 1:  Let
\[
E_n=(a_1-a_2)(a_1-a_3)\ldots(a_1-a_n)+(a_2-a_1)(a_2-a_3)\ldots(a_2-a_n)+\ldots+(a_n-a_1)(a_n-a_2)\ldots(a_n-a_{n-1}).
\]
Let $S_n$ be the proposition that $E_n\ge0$ for all real $a_i$. Prove that $S_n$ is true for $n=3$ and $5$, but for no other $n>2$. 
IMO 1971 problem 2:  Let $P_1$ be a convex polyhedron with vertices $A_1,A_2,\ldots,A_9$. Let $P_i$ be the polyhedron obtained from $P_1$ by a translation that moves $A_1$ to $A_i$. Prove that at least two of the polyhedra $P_1,P_2,\ldots,P_9$ have an interior point in common. 
IMO 1971 problem 3:  Prove that we can find an infinite set of positive integers of the from $2^n-3$ (where $n$ is a positive integer) every pair of which are relatively prime. 
IMO 1971 problem 4:  All faces of the tetrahedron $ABCD$ are acute-angled. Take a point $X$ in the interior of the segment $AB$, and similarly $Y$ in $BC, Z$ in $CD$ and $T$ in $AD$. \\\\
\textbf{a.)} If $\angle DAB+\angle BCD\ne\angle CDA+\angle ABC$, then prove none of the closed paths $XYZTX$ has minimal length; \\\\
\textbf{b.)} If $\angle DAB+\angle BCD=\angle CDA+\angle ABC$, then there are infinitely many shortest paths $XYZTX$, each with length $2AC\sin k$, where $2k=\angle BAC+\angle CAD+\angle DAB$. 
IMO 1971 problem 5:  Prove that for every positive integer $m$ we can find a finite set $S$ of points in the plane, such that given any point $A$ of $S$, there are exactly $m$ points in $S$ at unit distance from $A$. 
IMO 1971 problem 6:  Let $ A = (a_{ij})$, where $ i,j = 1,2,\ldots,n$, be a square matrix with all $ a_{ij}$ non-negative integers. For each $ i,j$ such that $ a_{ij} = 0$, the sum of the elements in the $ i$th row and the $ j$th column is at least $ n$. Prove that the sum of all the elements in the matrix is at least $ \frac {n^2}{2}$. 

IMO 1970 

IMO 1970 problem 1:  $M$ is any point on the side $AB$ of the triangle $ABC$. $r,r_1,r_2$ are the radii of the circles inscribed in $ABC,AMC,BMC$. $q$ is the radius of the circle on the opposite side of $AB$ to $C$, touching the three sides of $AB$ and the extensions of $CA$ and $CB$. Similarly, $q_1$ and $q_2$. Prove that $r_1r_2q=rq_1q_2$. 
IMO 1970 problem 2:  We have $0\le x_i<b$ for $i=0,1,\ldots,n$ and $x_n>0,x_{n-1}>0$. If $a>b$, and $x_nx_{n-1}\ldots x_0$ represents the number $A$ base $a$ and $B$ base $b$, whilst $x_{n-1}x_{n-2}\ldots x_0$ represents the number $A'$ base $a$ and $B'$ base $b$, prove that $A'B<AB'$. 
IMO 1970 problem 3:  The real numbers $a_0,a_1,a_2,\ldots$ satisfy $1=a_0\le a_1\le a_2\le\ldots. b_1,b_2,b_3,\ldots$ are defined by $b_n=\sum_{k=1}^n{1-{a_{k-1}\over a_k}\over\sqrt a_k}$. \\\\
\textbf{a.)} Prove that $0\le b_n<2$. \\\\
\textbf{b.)} Given $c$ satisfying $0\le c<2$, prove that we can find $a_n$ so that $b_n>c$ for all sufficiently large $n$. 
IMO 1970 problem 4:  Find all positive integers $n$ such that the set $\{n,n+1,n+2,n+3,n+4,n+5\}$ can be partitioned into two subsets so that the product of the numbers in each subset is equal. 
IMO 1970 problem 5:  In the tetrahedron $ABCD,\angle BDC=90^o$ and the foot of the perpendicular from $D$ to $ABC$ is the intersection of the altitudes of $ABC$. Prove that:
\[ (AB+BC+CA)^2\le6(AD^2+BD^2+CD^2). \]
When do we have equality? 
IMO 1970 problem 6:  Given $100$ coplanar points, no three collinear, prove that at most $70\%$ of the triangles formed by the points have all angles acute. 

IMO 1969 

IMO 1969 problem 1:  Prove that there are infinitely many positive integers $m$, such that $n^4+m$ is not prime for any positive integer $n$. 
IMO 1969 problem 2:  Let $f(x)=\cos(a_1+x)+{1\over2}\cos(a_2+x)+{1\over4}\cos(a_3+x)+\ldots+{1\over2^{n-1}}\cos(a_n+x)$, where $a_i$ are real constants and $x$ is a real variable. If $f(x_1)=f(x_2)=0$, prove that $x_1-x_2$ is a multiple of $\pi$. 
IMO 1969 problem 3:  For each of $k=1,2,3,4,5$ find necessary and sufficient conditions on $a>0$ such that there exists a tetrahedron with $k$ edges length $a$ and the remainder length $1$. 
IMO 1969 problem 4:  $C$ is a point on the semicircle diameter $AB$, between $A$ and $B$. $D$ is the foot of the perpendicular from $C$ to $AB$. The circle $K_1$ is the incircle of $ABC$, the circle $K_2$ touches $CD,DA$ and the semicircle, the circle $K_3$ touches $CD,DB$ and the semicircle. Prove that $K_1,K_2$ and $K_3$ have another common tangent apart from $AB$. 
IMO 1969 problem 5:  Given $n>4$ points in the plane, no three collinear. Prove that there are at least $\frac{(n-3)(n-4)}{2}$ convex quadrilaterals with vertices amongst the $n$ points. 
IMO 1969 problem 6:  Given real numbers $x_1,x_2,y_1,y_2,z_1,z_2$ satisfying $x_1>0,x_2>0,x_1y_1>z_1^2$, and $x_2y_2>z_2^2$, prove that:
\[ {8\over(x_1+x_2)(y_1+y_2)-(z_1+z_2)^2}\le{1\over x_1y_1-z_1^2}+{1\over x_2y_2-z_2^2}. \]
Give necessary and sufficient conditions for equality. 

IMO 1968 

IMO 1968 problem 1:  Find all triangles whose side lengths are consecutive integers, and one of whose angles is twice another. 
IMO 1968 problem 2:  Find all natural numbers $n$ the product of whose decimal digits is $n^2-10n-22$. 
IMO 1968 problem 3:  Let $a,b,c$ be real numbers with $a$ non-zero. It is known that the real numbers $x_1,x_2,\ldots,x_n$ satisfy the $n$ equations:
\[ ax_1^2+bx_1+c = x_2 \]
\[ ax_2^2+bx_2 +c = x_3 \]
\[ \ldots \quad \ldots \quad \ldots \quad \ldots \]
\[ ax_n^2+bx_n+c = x_1 \]
Prove that the system has \textbf{zero}, <u>one</u> or \textit{more than one} real solutions if $(b-1)^2-4ac$ is \textbf{negative}, equal to <u>zero</u> or \textit{positive} respectively. 
IMO 1968 problem 4:  Prove that every tetrahedron has a vertex whose three edges have the right lengths to form a triangle. 
IMO 1968 problem 5:  Let $f$ be a real-valued function defined for all real numbers, such that for some $a>0$ we have
\[ f(x+a)={1\over2}+\sqrt{f(x)-f(x)^2} \]
for all $x$. \\
Prove that $f$ is periodic, and give an example of such a non-constant $f$ for $a=1$. 
IMO 1968 problem 6: Let $n$ be a natural number. Prove that \[ \left\lfloor \frac{n+2^0}{2^1} \right\rfloor + \left\lfloor \frac{n+2^1}{2^2} \right\rfloor +\cdots +\left\lfloor \frac{n+2^{n-1}}{2^n}\right\rfloor =n. \]

IMO 1967 

IMO 1967 problem 1:  The parallelogram $ABCD$ has $AB=a,AD=1,$ $\angle BAD=A$, and the triangle $ABD$ has all angles acute. Prove that circles radius $1$ and center $A,B,C,D$ cover the parallelogram if and only
\[ a\le\cos A+\sqrt3\sin A. \] 
IMO 1967 problem 2:  Prove that a tetrahedron with just one edge length greater than $1$ has volume at most $ \frac{1}{8}.$ 
IMO 1967 problem 3:  Let $k,m,n$ be natural numbers such that $m+k+1$ is a prime greater than $n+1$. Let $c_s=s(s+1)$. Prove that
\[ (c_{m+1}-c_k)(c_{m+2}-c_k)\ldots(c_{m+n}-c_k) \]
is divisible by the product $c_1c_2\ldots c_n$. 
IMO 1967 problem 4:  $A_0B_0C_0$ and $A_1B_1C_1$ are acute-angled triangles. Describe, and prove, how to construct the triangle $ABC$ with the largest possible area which is circumscribed about $A_0B_0C_0$ (so $BC$ contains $B_0, CA$ contains $B_0$, and $AB$ contains $C_0$) and similar to $A_1B_1C_1.$ 
IMO 1967 problem 5:  Let $a_1,\ldots,a_8$ be reals, not all equal to zero. Let
\[ c_n = \sum^8_{k=1} a^n_k \]
for $n=1,2,3,\ldots$. Given that among the numbers of the sequence $(c_n)$, there are infinitely many equal to zero, determine all the values of $n$ for which $c_n = 0.$ 
IMO 1967 problem 6:  In a sports meeting a total of $m$ medals were awarded over $n$ days. On the first day one medal and $\frac{1}{7}$ of the remaining medals were awarded. On the second day two medals and $\frac{1}{7}$ of the remaining medals were awarded, and so on. On the last day, the remaining $n$ medals were awarded. How many medals did the meeting last, and what was the total number of medals ? 

IMO 1966 

IMO 1966 problem 1:  In a mathematical contest, three problems, $A,B,C$ were posed. Among the participants ther were 25 students who solved at least one problem each. Of all the contestants who did not solve problem $A$, the number who solved $B$ was twice the number who solved $C$. The number of students who solved only problem $A$ was one more than the number of students who solved $A$ and at least one other problem. Of all students who solved just one problem, half did not solve problem $A$. How many students solved only problem $B$? 
IMO 1966 problem 2:  Let $a,b,c$ be the lengths of the sides of a triangle, and $\alpha, \beta, \gamma$ respectively, the angles opposite these sides. Prove that if
\[ a+b=\tan{\frac{\gamma}{2}}(a\tan{\alpha}+b\tan{\beta}) \]
the triangle is isosceles. 
IMO 1966 problem 3:  Prove that the sum of the distances of the vertices of a regular tetrahedron from the center of its circumscribed sphere is less than the sum of the distances of these vertices from any other point in space. 
IMO 1966 problem 4:  Prove that for every natural number $n$, and for every real number $x \neq \frac{k\pi}{2^t}$ ($t=0,1, \dots, n$; $k$ any integer)
\[ \frac{1}{\sin{2x}}+\frac{1}{\sin{4x}}+\dots+\frac{1}{\sin{2^nx}}=\cot{x}-\cot{2^nx} \] 
IMO 1966 problem 5:  Solve the system of equations
\[ |a_1-a_2|x_2+|a_1-a_3|x_3+|a_1-a_4|x_4=1 \]
\[ |a_2-a_1|x_1+|a_2-a_3|x_3+|a_2-a_4|x_4=1 \]
\[ |a_3-a_1|x_1+|a_3-a_2|x_2+|a_3-a_4|x_4=1 \]
\[ |a_4-a_1|x_1+|a_4-a_2|x_2+|a_4-a_3|x_3=1 \]
where $a_1, a_2, a_3, a_4$ are four different real numbers. 
IMO 1966 problem 6:  Let $ ABC$ be a triangle, and let $ P$, $ Q$, $ R$ be three points in the interiors of the sides $ BC$, $ CA$, $ AB$ of this triangle. Prove that the area of at least one of the three triangles $ AQR$, $ BRP$, $ CPQ$ is less than or equal to one quarter of the area of triangle $ ABC$. \\\\
\textit{Alternative formulation:} Let $ ABC$ be a triangle, and let $ P$, $ Q$, $ R$ be three points on the segments $ BC$, $ CA$, $ AB$, respectively. Prove that \\\\
$ \min\left\{\left|AQR\right|,\left|BRP\right|,\left|CPQ\right|\right\}\leq\frac14\cdot\left|ABC\right|$, \\\\
where the abbreviation $ \left|P_1P_2P_3\right|$ denotes the (non-directed) area of an arbitrary triangle $ P_1P_2P_3$. 

IMO 1965 

IMO 1965 problem 1:  Determine all values of $x$ in the interval $0 \leq x \leq 2\pi$ which satisfy the inequality
\[ 2 \cos{x} \leq \sqrt{1+\sin{2x}}-\sqrt{1-\sin{2x}} \leq \sqrt{2}. \] 
IMO 1965 problem 2:  Consider the sytem of equations
\[ a_{11}x_1+a_{12}x_2+a_{13}x_3 = 0 \]
\[ a_{21}x_1+a_{22}x_2+a_{23}x_3 =0 \]
\[ a_{31}x_1+a_{32}x_2+a_{33}x_3 = 0 \]
with unknowns $x_1, x_2, x_3$. The coefficients satisfy the conditions:
\begin{enumerate}[a)]
  \item $a_{11}, a_{22}, a_{33}$ are positive numbers;
  \item the remaining coefficients are negative numbers;
  \item in each equation, the sum ofthe coefficients is positive.
\end{enumerate}
Prove that the given system has only the solution $x_1=x_2=x_3=0$. 
IMO 1965 problem 3:  Given the tetrahedron $ABCD$ whose edges $AB$ and $CD$ have lengths $a$ and $b$ respectively. The distance between the skew lines $AB$ and $CD$ is $d$, and the angle between them is $\omega$. Tetrahedron $ABCD$ is divided into two solids by plane $\epsilon$, parallel to lines $AB$ and $CD$. The ratio of the distances of $\epsilon$ from $AB$ and $CD$ is equal to $k$. Compute the ratio of the volumes of the two solids obtained. 
IMO 1965 problem 4:  Find all sets of four real numbers $x_1, x_2, x_3, x_4$ such that the sum of any one and the product of the other three is equal to 2. 
IMO 1965 problem 5:  Consider $\triangle OAB$ with acute angle $AOB$. Thorugh a point $M \neq O$ perpendiculars are drawn to $OA$ and $OB$, the feet of which are $P$ and $Q$ respectively. The point of intersection of the altitudes of $\triangle OPQ$ is $H$. What is the locus of $H$ if $M$ is permitted to range over
\begin{enumerate}[a)]
  \item the side $AB$;
  \item the interior of $\triangle OAB$.
\end{enumerate} 
IMO 1965 problem 6:  In a plane a set of $n\geq 3$ points is given. Each pair of points is connected by a segment. Let $d$ be the length of the longest of these segments. We define a diameter of the set to be any connecting segment of length $d$. Prove that the number of diameters of the given set is at most $n$. 

IMO 1964 

IMO 1964 problem 1:  \begin{enumerate}[(a)]
  \item Find all positive integers $ n$ for which $ 2^n-1$ is divisible by $ 7$.
  \item Prove that there is no positive integer $ n$ for which $ 2^n+1$ is divisible by $ 7$.
\end{enumerate} 
IMO 1964 problem 2:  Suppose $a,b,c$ are the sides of a triangle. Prove that
\[ a^2(b+c-a)+b^2(a+c-b)+c^2(a+b-c) \leq 3abc \] 
IMO 1964 problem 3:  A circle is inscribed in a triangle $ABC$ with sides $a,b,c$. Tangents to the circle parallel to the sides of the triangle are contructe. Each of these tangents cuts off a triagnle from $\triangle ABC$. In each of these triangles, a circle is inscribed. Find the sum of the areas of all four inscribed circles (in terms of $a,b,c$). 
IMO 1964 problem 4:  Seventeen people correspond by mail with one another-each one with all the rest. In their letters only three different topics are discussed. each pair of correspondents deals with only one of these topics. Prove that there are at least three people who write to each other about the same topic. 
IMO 1964 problem 5:  Supppose five points in a plane are situated so that no two of the straight lines joining them are parallel, perpendicular, or coincident. From each point perpendiculars are drawn to all the lines joining the other four points. Determine the maxium number of intersections that these perpendiculars can have. 
IMO 1964 problem 6:  In tetrahedron $ABCD$, vertex $D$ is connected with $D_0$, the centrod if $\triangle ABC$. Line parallel to $DD_0$ are drawn through $A,B$ and $C$. These lines intersect the planes $BCD, CAD$ and $ABD$ in points $A_2, B_1,$ and $C_1$, respectively. Prove that the volume of $ABCD$ is one third the volume of $A_1B_1C_1D_0$. Is the result if point $D_o$ is selected anywhere within $\triangle ABC$? 

IMO 1963 

IMO 1963 problem 1:  Find all real roots of the equation
\[ \sqrt{x^2-p}+2\sqrt{x^2-1}=x \]
where $p$ is a real parameter. 
IMO 1963 problem 2:  Point $A$ and segment $BC$ are given. Determine the locus of points in space which are vertices of right angles with one side passing through $A$, and the other side intersecting segment $BC$. 
IMO 1963 problem 3:  In an $n$-gon $A_1A_2\ldots A_n$, all of whose interior angles are equal, the lengths of consecutive sides satisfy the relation
\[ a_1\geq a_2\geq \dots \geq a_n. \]
Prove that $a_1=a_2= \ldots= a_n$. 
IMO 1963 problem 4:  Find all solutions $x_1, x_2, x_3, x_4, x_5$ of the system
\[ x_5+x_2=yx_1 \]
\[ x_1+x_3=yx_2 \]
\[ x_2+x_4=yx_3 \]
\[ x_3+x_5=yx_4 \]
\[ x_4+x_1=yx_5 \]
where $y$ is a parameter. 
IMO 1963 problem 5:  Prove that $\cos{\frac{\pi}{7}}-\cos{\frac{2\pi}{7}}+\cos{\frac{3\pi}{7}}=\frac{1}{2}$ 
IMO 1963 problem 6:  Five students $ A, B, C, D, E$ took part in a contest. One prediction was that the contestants would finish in the order $ ABCDE$. This prediction was very poor. In fact, no contestant finished in the position predicted, and no two contestants predicted to finish consecutively actually did so. A second prediction had the contestants finishing in the order $ DAECB$. This prediction was better. Exactly two of the contestants finished in the places predicted, and two disjoint pairs of students predicted to finish consecutively actually did so. Determine the order in which the contestants finished. 

IMO 1962 

IMO 1962 problem 1:  Find the smallest natural number $n$ which has the following properties:
\begin{enumerate}[a)]
  \item Its decimal representation has a 6 as the last digit.
  \item If the last digit 6 is erased and placed in front of the remaining digits, the resulting number is four times as large as the original number $n$.
\end{enumerate} 
IMO 1962 problem 2:  Determine all real numbers $x$ which satisfy the inequality:
\[ \sqrt{3-x}-\sqrt{x+1}>\dfrac{1}{2} \] 
IMO 1962 problem 3:  Consider the cube $ABCDA'B'C'D'$ ($ABCD$ and $A'B'C'D'$ are the upper and lower bases, repsectively, and edges $AA', BB', CC', DD'$ are parallel). The point $X$ moves at a constant speed along the perimeter of the square $ABCD$ in the direction $ABCDA$, and the point $Y$ moves at the same rate along the perimiter of the square $B'C'CB$ in the direction $B'C'CBB'$. Points $X$ and $Y$ begin their motion at the same instant from the starting positions $A$ and $B'$, respectively. Determine and draw the locus of the midpionts of the segments $XY$. 
IMO 1962 problem 4:  Solve the equation $\cos^2{x}+\cos^2{2x}+\cos^2{3x}=1$ 
IMO 1962 problem 5:  On the circle $K$ there are given three distinct points $A,B,C$. Construct (using only a straightedge and a compass) a fourth point $D$ on $K$ such that a circle can be inscribed in the quadrilateral thus obtained. 
IMO 1962 problem 6:  Consider an isosceles triangle. let $R$ be the radius of its circumscribed circle and $r$ be the radius of its inscribed circle. Prove that the distance $d$ between the centers of these two circle is
\[ d=\sqrt{R(R-2r)} \] 
IMO 1962 problem 7:  The tetrahedron $SABC$ has the following property: there exist five spheres, each tangent to the edges $SA, SB, SC, BC, CA, AB,$ or to their extensions.
\begin{enumerate}[a)]
  \item Prove that the tetrahedron $SABC$ is regular.
  \item Prove conversely that for every regular tetrahedron five such spheres exist.
\end{enumerate} 

IMO 1961 

IMO 1961 problem 1:  Solve the system of equations:
\[ x+y+z=a \]
\[ x^2+y^2+z^2=b^2 \]
\[ xy=z^2 \]
where $a$ and $b$ are constants. Give the conditions that $a$ and $b$ must satisfy so that $x,y,z$ are distinct positive numbers. 
IMO 1961 problem 2:  Let $ a$, $ b$, $ c$ be the sides of a triangle, and $ S$ its area. Prove:
\[ a^2 + b^2 + c^2\geq 4S \sqrt {3} \]
In what case does equality hold? 
IMO 1961 problem 3:  Solve the equation $\cos^n{x}-\sin^n{x}=1$ where $n$ is a natural number. 
IMO 1961 problem 4:  Consider triangle $P_1P_2P_3$ and a point $p$ within the triangle. Lines $P_1P, P_2P, P_3P$ intersect the opposite sides in points $Q_1, Q_2, Q_3$ respectively. Prove that, of the numbers
\[ \dfrac{P_1P}{PQ_1}, \dfrac{P_2P}{PQ_2}, \dfrac{P_3P}{PQ_3} \]
at least one is $\leq 2$ and at least one is $\geq 2$ 
IMO 1961 problem 5:  Construct a triangle $ABC$ if $AC=b$, $AB=c$ and $\angle AMB=w$, where $M$ is the midpoint of the segment $BC$ and $w<90$. Prove that a solution exists if and only if
\[ b \tan{\dfrac{w}{2}} \leq c <b \]
In what case does the equality hold? 
IMO 1961 problem 6:  Consider a plane $\epsilon$ and three non-collinear points $A,B,C$ on the same side of $\epsilon$; suppose the plane determined by these three points is not parallel to $\epsilon$. In plane $\epsilon$ take three arbitrary points $A',B',C'$. Let $L,M,N$ be the midpoints of segments $AA', BB', CC'$; Let $G$ be the centroid of the triangle $LMN$. (We will not consider positions of the points $A', B', C'$ such that the points $L,M,N$ do not form a triangle.) What is the locus of point $G$ as $A', B', C'$ range independently over the plane $\epsilon$? 

IMO 1960 

IMO 1960 problem 1:  Determine all three-digit numbers $N$ having the property that $N$ is divisible by 11, and $\dfrac{N}{11}$ is equal to the sum of the squares of the digits of $N$. 
IMO 1960 problem 2:  For what values of the variable $x$ does the following inequality hold:
\[ \dfrac{4x^2}{(1-\sqrt{2x+1})^2}<2x+9 \ ? \] 
IMO 1960 problem 3:  In a given right triangle $ABC$, the hypotenuse $BC$, of length $a$, is divided into $n$ equal parts ($n$ and odd integer). Let $\alpha$ be the acute angel subtending, from $A$, that segment which contains the mdipoint of the hypotenuse. Let $h$ be the length of the altitude to the hypotenuse fo the triangle. Prove that:
\[ \tan{\alpha}=\dfrac{4nh}{(n^2-1)a}. \] 
IMO 1960 problem 4:  Construct triangle $ABC$, given $h_a$, $h_b$ (the altitudes from $A$ and $B$), and $m_a$, the median from vertex $A$. 
IMO 1960 problem 5:  Consider the cube $ABCDA'B'C'D'$ (with face $ABCD$ directly above face $A'B'C'D'$).
\begin{enumerate}[a)]
  \item Find the locus of the midpoints of the segments $XY$, where $X$ is any point of $AC$ and $Y$ is any piont of $B'D'$;
  \item Find the locus of points $Z$ which lie on the segment $XY$ of part a) with $ZY=2XZ$.
\end{enumerate} 
IMO 1960 problem 6:  Consider a cone of revolution with an inscribed sphere tangent to the base of the cone. A cylinder is circumscribed about this sphere so that one of its bases lies in the base of the cone. let $V_1$ be the volume of the cone and $V_2$ be the volume of the cylinder.
\begin{enumerate}[a)]
  \item Prove that $V_1 \neq V_2$;
  \item Find the smallest number $k$ for which $V_1=kV_2$; for this case, construct the angle subtended by a diamter of the base of the cone at the vertex of the cone.
\end{enumerate} 
IMO 1960 problem 7:  An isosceles trapezoid with bases $a$ and $c$ and altitude $h$ is given.
\begin{enumerate}[a)]
  \item On the axis of symmetry of this trapezoid, find all points $P$ such that both legs of the trapezoid subtend right angles at $P$;
  \item Calculate the distance of $p$ from either base;
  \item Determine under what conditions such points $P$ actually exist. Discuss various cases that might arise.
\end{enumerate} 

IMO 1959 

IMO 1959 problem 1:  Prove that the fraction $ \dfrac{21n + 4}{14n + 3}$ is irreducible for every natural number $ n$. 
IMO 1959 problem 2:  For what real values of $x$ is
\[ \sqrt{x+\sqrt{2x-1}}+\sqrt{x-\sqrt{2x-1}}=A \]
given
\begin{enumerate}[a)]
  \item $A=\sqrt{2}$;
  \item $A=1$;
  \item $A=2$,
\end{enumerate}
where only non-negative real numbers are admitted for square roots? 
IMO 1959 problem 3:  Let $a,b,c$ be real numbers. Consider the quadratic equation in $\cos{x}$
\[ a \cos^2{x}+b \cos{x}+c=0. \]
Using the numbers $a,b,c$ form a quadratic equation in $\cos{2x}$ whose roots are the same as those of the original equation. Compare the equation in $\cos{x}$ and $\cos{2x}$ for $a=4$, $b=2$, $c=-1$. 
IMO 1959 problem 4:  Construct a right triangle with given hypotenuse $c$ such that the median drawn to the hypotenuse is the geometric mean of the two legs of the triangle. 
IMO 1959 problem 5:  An arbitrary point $M$ is selected in the interior of the segment $AB$. The square $AMCD$ and $MBEF$ are constructed on the same side of $AB$, with segments $AM$ and $MB$ as their respective bases. The circles circumscribed about these squares, with centers $P$ and $Q$, intersect at $M$ and also at another point $N$. Let $N'$ denote the point of intersection of the straight lines $AF$ and $BC$.
\begin{enumerate}[a)]
  \item Prove that $N$ and $N'$ coincide;
  \item Prove that the straight lines $MN$ pass through a fixed point $S$ independent of the choice of $M$;
  \item Find the locus of the midpoints of the segments $PQ$ as $M$ varies between $A$ and $B$.
\end{enumerate} 
IMO 1959 problem 6:  Two planes, $P$ and $Q$, intersect along the line $p$. The point $A$ is given in the plane $P$, and the point $C$ in the plane $Q$; neither of these points lies on the straight line $p$. Construct an isosceles trapezoid $ABCD$ (with $AB \parallel CD$) in which a circle can be inscribed, and with vertices $B$ and $D$ lying in planes $P$ and $Q$ respectively. 
