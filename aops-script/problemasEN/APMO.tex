
APMO 2021 

APMO 2021 problem 1:  Prove that for each real number $r>2$, there are exactly two or three positive real numbers $x$ satisfying the equation $x^2=r\lfloor x \rfloor$. 
APMO 2021 problem 2:  For a polynomial $P$ and a positive integer $n$, define $P_n$ as the number of positive integer pairs $(a,b)$ such that $a<b \leq n$ and $|P(a)|-|P(b)|$ is divisible by $n$. Determine all polynomial $P$ with integer coefficients such that $P_n \leq 2021$ for all positive integers $n$. 
APMO 2021 problem 3:  Let $ABCD$ be a cyclic convex quadrilateral and $\Gamma$ be its circumcircle. Let $E$ be the intersection of the diagonals of $AC$ and $BD$. Let $L$ be the center of the circle tangent to sides $AB$, $BC$, and $CD$, and let $M$ be the midpoint of the arc $BC$ of $\Gamma$ not containing $A$ and $D$. Prove that the excenter of triangle $BCE$ opposite $E$ lies on the line $LM$. 
APMO 2021 problem 4:  Given a $32 \times 32$ table, we put a mouse (facing up) at the bottom left cell and a piece of cheese at several other cells.  The mouse then starts moving.  It moves forward except that when it reaches a piece of cheese, it eats a part of it, turns right, and continues moving forward.  We say that a subset of cells containing cheese is good if, during this process, the mouse tastes each piece of cheese exactly once and then falls off the table.  Show that:
\begin{enumerate}[(a)]
  \item No good subset consists of 888 cells.
  \item There exists a good subset consisting of at least 666 cells.
\end{enumerate} 
APMO 2021 problem 5:  Determine all Functions $f:\mathbb{Z} \to \mathbb{Z}$ such that $f(f(a)-b)+bf(2a)$ is a perfect square for all integers $a$ and $b$. 

APMO 2020 

APMO 2020 problem 1:  Let $\Gamma$ be the circumcircle of $\triangle ABC$. Let $D$ be a point on the side $BC$. The tangent to $\Gamma$ at $A$ intersects the parallel line to $BA$ through $D$ at point $E$. The segment $CE$ intersects $\Gamma$ again at $F$. Suppose $B$, $D$, $F$, $E$ are concyclic. Prove that $AC$, $BF$, $DE$ are concurrent. 
APMO 2020 problem 2:  Show that $r = 2$ is the largest real number $r$ which satisfies the following condition: \\\\
If a sequence $a_1$, $a_2$, $\ldots$ of positive integers fulfills the inequalities
\[ a_n \leq a_{n+2} \leq\sqrt{a_n^2+ra_{n+1}} \]
for every positive integer $n$, then there exists a positive integer $M$ such that $a_{n+2} = a_n$ for every $n \geq M$. 
APMO 2020 problem 3:  Determine all positive integers $k$ for which there exist a positive integer $m$ and a set $S$ of positive integers such that any integer $n > m$ can be written as a sum of distinct elements of $S$ in exactly $k$ ways. 
APMO 2020 problem 4:  Let $\mathbb{Z}$ denote the set of all integers. Find all polynomials $P(x)$ with integer coefficients that satisfy the following property: \\\\
For any infinite sequence $a_1$, $a_2$, $\dotsc$ of integers in which each integer in $\mathbb{Z}$ appears exactly once, there exist indices $i < j$ and an integer $k$ such that $a_i +a_{i+1} +\dotsb +a_j = P(k)$. 
APMO 2020 problem 5:  Let $n \geq 3$ be a fixed integer. The number $1$ is written $n$ times on a blackboard. Below the blackboard, there are two buckets that are initially empty. A move consists of erasing two of the numbers $a$ and $b$, replacing them with the numbers $1$ and $a+b$, then adding one stone to the first bucket and $\gcd(a, b)$ stones to the second bucket. After some finite number of moves, there are $s$ stones in the first bucket and $t$ stones in the second bucket, where $s$ and $t$ are positive integers. Find all possible values of the ratio $\frac{t}{s}$. 

APMO 2019 

APMO 2019 problem 1:  Let $\mathbb{Z}^+$ be the set of positive integers. Determine all functions $f : \mathbb{Z}^+\to\mathbb{Z}^+$ such that $a^2+f(a)f(b)$ is divisible by $f(a)+b$ for all positive integers $a,b$. 
APMO 2019 problem 2:  Let $m$ be a fixed positive integer. The infinite sequence $\{a_n\}_{n\geq 1}$ is defined in the following way: $a_1$ is a positive integer, and for every integer $n\geq 1$ we have
\[
a_{n+1} =
\begin{cases}a_n^2+2^m \& \text{if } a_n< 2^m \\ a_n/2 \&\text{if } a_n\geq 2^m\end{cases}
\]
For each $m$, determine all possible values of $a_1$ such that every term in the sequence is an integer. 
APMO 2019 problem 3:  Let $ABC$ be a scalene triangle with circumcircle $\Gamma$. Let $M$ be the midpoint of $BC$. A variable point $P$ is selected in the line segment $AM$. The circumcircles of triangles $BPM$ and $CPM$ intersect $\Gamma$ again at points $D$ and $E$, respectively. The lines $DP$ and $EP$ intersect (a second time) the circumcircles to triangles $CPM$ and $BPM$ at $X$ and $Y$, respectively. Prove that as $P$ varies, the circumcircle of $\triangle AXY$ passes through a fixed point $T$ distinct from $A$. 
APMO 2019 problem 4:  Consider a $2018 \times 2019$ board with integers in each unit square. Two unit squares are said to be neighbours if they share a common edge. In each turn, you choose some unit squares. Then for each chosen unit square the average of all its neighbours is calculated. Finally, after these calculations are done, the number in each chosen unit square is replaced by the corresponding average. \\
Is it always possible to make the numbers in all squares become the same after finitely many turns? 
APMO 2019 problem 5:  Determine all the functions $f : \mathbb{R} \to \mathbb{R}$ such that
\[ f(x^2 + f(y)) = f(f(x)) + f(y^2) + 2f(xy) \]
for all real numbers $x$ and $y$. 

APMO 2018 

APMO 2018 problem 1:  Let $H$ be the orthocenter of the triangle $ABC$. Let $M$ and $N$ be the midpoints of the sides $AB$ and $AC$, respectively. Assume that $H$ lies inside the quadrilateral $BMNC$ and that the circumcircles of triangles $BMH$ and $CNH$ are tangent to each other. The line through $H$ parallel to $BC$ intersects the circumcircles of the triangles $BMH$ and $CNH$ in the points $K$ and $L$, respectively. Let $F$ be the intersection point of $MK$ and $NL$ and let $J$ be the incenter of triangle $MHN$. Prove that $F J = F A$. 
APMO 2018 problem 2:  Let $f(x)$ and $g(x)$ be given by \\
$f(x) = \frac{1}{x} + \frac{1}{x-2} + \frac{1}{x-4} + \cdots + \frac{1}{x-2018}$ \\
$g(x) = \frac{1}{x-1} + \frac{1}{x-3} + \frac{1}{x-5} + \cdots + \frac{1}{x-2017}$. \\\\
Prove that $|f(x)-g(x)| >2$ for any non-integer real number $x$ satisfying $0 < x < 2018$. 
APMO 2018 problem 3:  A collection of $n$ squares on the plane is called tri-connected if the following criteria are satisfied:
\begin{enumerate}[(i)]
  \item All the squares are congruent.
  \item If two squares have a point $P$ in common, then $P$ is a vertex of each of the squares.
  \item Each square touches exactly three other squares.
\end{enumerate}
How many positive integers $n$ are there with $2018\leq n \leq 3018$, such that there exists a collection of $n$ squares that is tri-connected? 
APMO 2018 problem 4:  Let $ABC$ be an equilateral triangle. From the vertex $A$ we draw a ray towards the interior of the triangle such that the ray reaches one of the sides of the triangle. When the ray reaches a side, it then bounces off following the law of reflection, that is, if it arrives with a directed angle $\alpha$, it leaves with a directed angle $180^{\circ}-\alpha$. After $n$ bounces, the ray returns to $A$ without ever landing on any of the other two vertices. Find all possible values of $n$. 
APMO 2018 problem 5:  Find all polynomials $P(x)$ with integer coefficients such that for all real numbers $s$ and $t$, if $P(s)$ and $P(t)$ are both integers, then $P(st)$ is also an integer. 

APMO 2017 

APMO 2017 problem 1:  We call a $5$-tuple of integers \textit{arrangeable} if its elements can be labeled $a, b, c, d, e$ in some order so that $a-b+c-d+e=29$. Determine all $2017$-tuples of integers $n_1, n_2, . . . , n_{2017}$ such that if we place them in a circle in clockwise order, then any $5$-tuple of numbers in consecutive positions on the circle is arrangeable. \\\\
\textit{Warut Suksompong, Thailand} 
APMO 2017 problem 2:  Let $ABC$ be a triangle with $AB < AC$. Let $D$ be the intersection point of the internal bisector of angle $BAC$ and the circumcircle of $ABC$. Let $Z$ be the intersection point of the perpendicular bisector of $AC$ with the external bisector of angle $\angle{BAC}$. Prove that the midpoint of the segment $AB$ lies on the circumcircle of triangle $ADZ$. \\\\
\textit{Olimpiada de Matemáticas, Nicaragua} 
APMO 2017 problem 3:  Let $A(n)$ denote the number of sequences $a_1\ge a_2\ge\cdots{}\ge a_k$ of positive integers for which $a_1+\cdots{}+a_k = n$ and each $a_i +1$ is a power of two $(i = 1,2,\cdots{},k)$. Let $B(n)$ denote the number of sequences $b_1\ge b_2\ge \cdots{}\ge b_m$ of positive integers for which $b_1+\cdots{}+b_m =n$ and each inequality $b_j\ge 2b_{j+1}$ holds $(j=1,2,\cdots{}, m-1)$. Prove that $A(n) = B(n)$ for every positive integer $n$. \\\\
\textit{Senior Problems Committee of the Australian Mathematical Olympiad Committee} 
APMO 2017 problem 4:  Call a rational number $r$ \textit{powerful} if $r$ can be expressed in the form $\dfrac{p^k}{q}$ for some relatively prime positive integers $p, q$ and some integer $k >1$. Let $a, b, c$ be positive rational numbers such that $abc = 1$. Suppose there exist positive integers $x, y, z$ such that $a^x + b^y + c^z$ is an integer. Prove that $a, b, c$ are all \textit{powerful}. \\\\
\textit{Jeck Lim, Singapore} 
APMO 2017 problem 5:  Let $n$ be a positive integer. A pair of $n$-tuples $(a_1,\cdots{}, a_n)$ and $(b_1,\cdots{}, b_n)$ with integer entries is called an \textit{exquisite pair} if
\[ |a_1b_1+\cdots{}+a_nb_n|\le 1. \]
Determine the maximum number of distinct $n$-tuples with integer entries such that any two of them form an exquisite pair. \\\\
\textit{Pakawut Jiradilok and Warut Suksompong, Thailand} 

APMO 2016 

APMO 2016 problem 1:  We say that a triangle $ABC$ is great if the following holds: for any point $D$ on the side $BC$, if $P$ and $Q$ are the feet of the perpendiculars from $D$ to the lines $AB$ and $AC$, respectively, then the reflection of $D$ in the line $PQ$ lies on the circumcircle of the triangle $ABC$. Prove that triangle $ABC$ is great if and only if $\angle A = 90^{\circ}$ and $AB = AC$. \\\\
\textit{Senior Problems Committee of the Australian Mathematical Olympiad Committee} 
APMO 2016 problem 2:  A positive integer is called \textit{fancy} if it can be expressed in the form
\[ 2^{a_1}+2^{a_2}+ \cdots+ 2^{a_{100}}, \]
where $a_1,a_2, \cdots, a_{100}$ are non-negative integers that are not necessarily distinct. Find the smallest positive integer $n$ such that no multiple of $n$ is a \textit{fancy} number. \\\\
\textit{Senior Problems Committee of the Australian Mathematical Olympiad Committee} 
APMO 2016 problem 3:  Let $AB$ and $AC$ be two distinct rays not lying on the same line, and let $\omega$ be a circle with center $O$ that is tangent to ray $AC$ at $E$ and ray $AB$ at $F$. Let $R$ be a point on segment $EF$. The line through $O$ parallel to $EF$ intersects line $AB$ at $P$. Let $N$ be the intersection of lines $PR$ and $AC$, and let $M$ be the intersection of line $AB$ and the line through $R$ parallel to $AC$. Prove that line $MN$ is tangent to $\omega$. \\\\
\textit{Warut Suksompong, Thailand} 
APMO 2016 problem 4:  The country Dreamland consists of $2016$ cities. The airline Starways wants to establish some one-way flights between pairs of cities in such a way that each city has exactly one flight out of it. Find the smallest positive integer $k$ such that no matter how Starways establishes its flights, the cities can always be partitioned into $k$ groups so that from any city it is not possible to reach another city in the same group by using at most $28$ flights. \\\\
\textit{Warut Suksompong, Thailand} 
APMO 2016 problem 5:  Find all functions $f: \mathbb{R}^+ \to \mathbb{R}^+$ such that
\[ (z + 1)f(x + y) = f(xf(z) + y) + f(yf(z) + x), \]
for all positive real numbers $x, y, z$. \\\\
\textit{Fajar Yuliawan, Indonesia} 

APMO 2015 

APMO 2015 problem 1:  Let $ABC$ be a triangle, and let $D$ be a point on side $BC$. A line through $D$ intersects side $AB$ at $X$ and ray $AC$ at $Y$ . The circumcircle of triangle $BXD$ intersects the circumcircle $\omega$ of triangle $ABC$ again at point $Z$ distinct from point $B$. The lines $ZD$ and $ZY$ intersect $\omega$ again at $V$ and $W$ respectively. \\
Prove that $AB = V W$ \\\\
\textit{Proposed by Warut Suksompong, Thailand} 
APMO 2015 problem 2:  Let $S = \{2, 3, 4, \ldots\}$ denote the set of integers that are greater than or equal to $2$. Does there exist a function $f : S \to S$ such that
\[ f (a)f (b) = f (a^2 b^2 )\text{ for all }a, b \in S\text{ with }a \ne b? \]
\textit{Proposed by Angelo Di Pasquale, Australia} 
APMO 2015 problem 3:  A sequence of real numbers $a_0, a_1, . . .$ is said to be good if the following three conditions hold.
\begin{enumerate}[(i)]
  \item The value of $a_0$ is a positive integer.
  \item For each non-negative integer $i$ we have $a_{i+1} = 2a_i + 1 $ or $a_{i+1} =\frac{a_i}{a_i + 2} $
  \item There exists a positive integer $k$ such that $a_k = 2014$.
\end{enumerate}
Find the smallest positive integer $n$ such that there exists a good sequence $a_0, a_1, . . .$ of real numbers with the property that $a_n = 2014$. \\\\
\textit{Proposed by Wang Wei Hua, Hong Kong} 
APMO 2015 problem 4:  Let $n$ be a positive integer. Consider $2n$ distinct lines on the plane, no two of which are parallel. Of the $2n$ lines, $n$ are colored blue, the other $n$ are colored red. Let $\mathcal{B}$ be the set of all points on the plane that lie on at least one blue line, and $\mathcal{R}$ the set of all points on the plane that lie on at least one red line. Prove that there exists a circle that intersects $\mathcal{B}$ in exactly $2n - 1$ points, and also intersects $\mathcal{R}$ in exactly $2n - 1$ points. \\\\
\textit{Proposed by Pakawut Jiradilok and Warut Suksompong, Thailand} 
APMO 2015 problem 5:  Determine all sequences $a_0 , a_1 , a_2 , \ldots$ of positive integers with $a_0 \ge 2015$ such that for all integers $n\ge 1$:
\begin{enumerate}[(i)]
  \item $a_{n+2}$ is divisible by $a_n$ ;
  \item $|s_{n+1} - (n + 1)a_n | = 1$, where $s_{n+1} = a_{n+1} - a_n + a_{n-1} - \cdots + (-1)^{n+1} a_0$ .
\end{enumerate}
\textit{Proposed by Pakawut Jiradilok and Warut Suksompong, Thailand} 

APMO 2014 

APMO 2014 problem 1:  For a positive integer $m$ denote by $S(m)$ and $P(m)$ the sum and product, respectively, of the digits of $m$. Show that for each positive integer $n$, there exist positive integers $a_1, a_2, \ldots, a_n$ satisfying the following conditions:
\[ S(a_1) < S(a_2) < \cdots < S(a_n) \text{ and } S(a_i) = P(a_{i+1}) \quad (i=1,2,\ldots,n). \]
(We let $a_{n+1} = a_1$.) \\\\
\textit{Problem Committee of the Japan Mathematical Olympiad Foundation} 
APMO 2014 problem 2:  Let $S = \{1,2,\dots,2014\}$. For each non-empty subset $T \subseteq S$, one of its members is chosen as its representative. Find the number of ways to assign representatives to all non-empty subsets of $S$ so that if a subset $D \subseteq S$ is a disjoint union of non-empty subsets $A, B, C \subseteq S$, then the representative of $D$ is also the representative of one of $A$, $B$, $C$. \\\\
\textit{Warut Suksompong, Thailand} 
APMO 2014 problem 3:  Find all positive integers $n$ such that for any integer $k$ there exists an integer $a$ for which $a^3+a-k$ is divisible by $n$. \\\\
\textit{Warut Suksompong, Thailand} 
APMO 2014 problem 4:  Let $n$ and $b$ be positive integers. We say $n$ is $b$-discerning if there exists a set consisting of $n$ different positive integers less than $b$ that has no two different subsets $U$ and $V$ such that the sum of all elements in $U$ equals the sum of all elements in $V$.
\begin{enumerate}[(a)]
  \item Prove that $8$ is $100$-discerning.
  \item Prove that $9$ is not $100$-discerning.
\end{enumerate}
\textit{Senior Problems Committee of the Australian Mathematical Olympiad Committee} 
APMO 2014 problem 5:  Circles $\omega$ and $\Omega$ meet at points $A$ and $B$. Let $M$ be the midpoint of the arc $AB$ of circle $\omega$ ($M$ lies inside $\Omega$). A chord $MP$ of circle $\omega$ intersects $\Omega$ at $Q$ ($Q$ lies inside $\omega$). Let $\ell_P$ be the tangent line to $\omega$ at $P$, and let $\ell_Q$ be the tangent line to $\Omega$ at $Q$. Prove that the circumcircle of the triangle formed by the lines $\ell_P$, $\ell_Q$ and $AB$ is tangent to $\Omega$. \\\\
\textit{Ilya Bogdanov, Russia and Medeubek Kungozhin, Kazakhstan} 

APMO 2013 

APMO 2013 problem 1:  Let $ABC$ be an acute triangle with altitudes $AD$, $BE$, and $CF$, and let $O$ be the center of its circumcircle.  Show that the segments $OA$, $OF$, $OB$, $OD$, $OC$, $OE$ dissect the triangle $ABC$ into three pairs of triangles that have equal areas. 
APMO 2013 problem 2:  Determine all positive integers $n$ for which $\dfrac{n^2+1}{[\sqrt{n}]^2+2}$ is an integer.  Here $[r]$ denotes the greatest integer less than or equal to $r$. 
APMO 2013 problem 3:  For $2k$ real numbers $a_1, a_2, ..., a_k$, $b_1, b_2, ..., b_k$ define a sequence of numbers $X_n$ by
\[ X_n = \sum_{i=1}^k [a_in + b_i] \quad (n=1,2,...). \]
If the sequence $X_N$ forms an arithmetic progression, show that $\textstyle\sum_{i=1}^k a_i$ must be an integer.  Here $[r]$ denotes the greatest integer less than or equal to $r$. 
APMO 2013 problem 4:  Let $a$ and $b$ be positive integers, and let $A$ and $B$ be finite sets of integers satisfying
\begin{enumerate}[(i)]
  \item $A$ and $B$ are disjoint;
  \item if an integer $i$ belongs to either to $A$ or to $B$, then either $i+a$ belongs to $A$ or $i-b$ belongs to $B$.
\end{enumerate}
Prove that $a\left\lvert A \right\rvert = b \left\lvert B \right\rvert$.  (Here $\left\lvert X \right\rvert$ denotes the number of elements in the set $X$.) 
APMO 2013 problem 5:  Let $ABCD$ be a quadrilateral inscribed in a circle $\omega$, and let $P$ be a point on the extension of $AC$ such that $PB$ and $PD$ are tangent to $\omega$.  The tangent at $C$ intersects $PD$ at $Q$ and the line $AD$ at $R$.  Let $E$ be the second point of intersection between $AQ$ and $\omega$. Prove that $B$, $E$, $R$ are collinear. 

APMO 2012 

APMO 2012 problem 1:  Let $ P $ be a point in the interior of a triangle $ ABC $, and let $ D, E, F $ be the point of intersection of the line $ AP $ and the side $ BC $ of the triangle, of the line $ BP $ and the side $ CA $, and of the line $ CP $ and the side $ AB $, respectively. Prove that the area of the triangle $ ABC $ must be $ 6  $ if the area of each of the triangles $ PFA, PDB $ and $ PEC $ is $ 1 $. 
APMO 2012 problem 2:  Into each box of a $ 2012 \times 2012 $ square grid, a real number greater than or equal to $ 0 $ and less than or equal to $ 1 $ is inserted. Consider splitting the grid into $2$ non-empty rectangles consisting of boxes of the grid by drawing a line parallel either to the horizontal or the vertical side of the grid. Suppose that for at least one of the resulting rectangles the sum of the numbers in the boxes within the rectangle is less than or equal to $ 1 $, no matter how the grid is split into $2$ such rectangles. Determine the maximum possible value for the sum of all the $ 2012 \times 2012 $ numbers inserted into the boxes. 
APMO 2012 problem 3:  Determine all the pairs $ (p , n )$ of a prime number $ p$ and a positive integer $ n$ for which $ \frac{ n^p + 1 }{p^n + 1} $ is an integer. 
APMO 2012 problem 4:  Let $ ABC $ be an acute triangle. Denote by $ D $ the foot of the perpendicular line drawn from the point $ A $ to the side $ BC $, by $M$ the midpoint of $ BC $, and by $ H $ the orthocenter of $ ABC $. Let $ E $ be the point of intersection of the circumcircle $ \Gamma $ of the triangle $ ABC $ and the half line $ MH $, and $ F $ be the point of intersection (other than  $E$) of the line $ ED $ and the circle $ \Gamma $. Prove that $ \tfrac{BF}{CF} = \tfrac{AB}{AC} $ must hold. \\\\
(Here we denote $XY$ the length of the line segment $XY$.) 
APMO 2012 problem 5:  Let $ n $ be an integer greater than or equal to $ 2 $. Prove that if the real numbers $ a_1 , a_2 , \cdots , a_n $ satisfy $ a_1 ^2 + a_2 ^2 + \cdots + a_n ^ 2 = n $, then
\[ \sum_{1 \le i < j \le n} \frac{1}{n- a_i a_j}  \le \frac{n}{2} \]
must hold. 

APMO 2011 

APMO 2011 problem 1:  Let $a,b,c$ be positive integers. Prove that it is impossible to have all of the three numbers $a^2+b+c,b^2+c+a,c^2+a+b$ to be perfect squares. 
APMO 2011 problem 2:  Five points $A_1,A_2,A_3,A_4,A_5$ lie on a plane in such a way that no three among them lie on a same straight line. Determine the maximum possible value that the minimum value for the angles $\angle A_iA_jA_k$ can take where $i, j, k$ are distinct integers between $1$ and $5$. 
APMO 2011 problem 3:  Let $ABC$ be an acute triangle with $\angle BAC=30^{\circ}$. The internal and external angle bisectors of $\angle ABC$ meet the line $AC$ at $B_1$ and $B_2$, respectively, and the internal and external angle bisectors of $\angle ACB$ meet the line $AB$ at $C_1$ and $C_2$, respectively. Suppose that the circles with diameters $B_1B_2$ and $C_1C_2$ meet inside the triangle $ABC$ at point $P$. Prove that $\angle BPC=90^{\circ}$ . 
APMO 2011 problem 4:  Let $n$ be a fixed positive odd integer. Take $m+2$ \textbf{distinct} points $P_0,P_1,\ldots ,P_{m+1}$ (where $m$ is a non-negative integer) on the coordinate plane in such a way that the following three conditions are satisfied:
\begin{enumerate}[1)]
  \item $P_0=(0,1),P_{m+1}=(n+1,n)$, and for each integer $i,1\le i\le m$, both $x$- and $y$- coordinates of $P_i$ are integers lying in between $1$ and $n$ ($1$ and $n$ inclusive).
  \item For each integer $i,0\le i\le m$, $P_iP_{i+1}$ is parallel to the $x$-axis if $i$ is even, and is parallel to the $y$-axis if $i$ is odd.
  \item For each pair $i,j$ with $0\le i<j\le m$, line segments $P_iP_{i+1}$ and $P_jP_{j+1}$ share at most $1$ point.
\end{enumerate}
Determine the maximum possible value that $m$ can take. 
APMO 2011 problem 5:  Determine all functions $f:\mathbb{R}\to\mathbb{R}$, where $\mathbb{R}$ is the set of all real numbers, satisfying the following two conditions:
\begin{enumerate}[1)]
  \item There exists a real number $M$ such that for every real number $x,f(x)<M$ is satisfied.
  \item For every pair of real numbers $x$ and $y$,
\end{enumerate}
\[ f(xf(y))+yf(x)=xf(y)+f(xy) \]
is satisfied. 

APMO 2010 

APMO 2010 problem 1:  Let $ABC$ be a triangle with $\angle BAC \neq 90^{\circ}.$ Let $O$ be the circumcenter of the triangle $ABC$ and $\Gamma$ be the circumcircle of the triangle $BOC.$ Suppose that $\Gamma$ intersects the line segment $AB$ at $P$ different from $B$, and the line segment $AC$ at $Q$ different from $C.$ Let $ON$ be the diameter of the circle $\Gamma.$ Prove that the quadrilateral $APNQ$ is a parallelogram. 
APMO 2010 problem 2:  For a positive integer $k,$ call an integer a $pure$ $k-th$ $power$ if it can be represented as $m^k$ for some integer $m.$ Show that for every positive integer $n,$ there exists $n$ distinct positive integers such that their sum is a pure $2009-$th power and their product is a pure $2010-$th power. 
APMO 2010 problem 3:  Let $n$ be a positive integer. $n$ people take part in a certain party. For any pair of the participants, either the two are acquainted with each other or they are not. What is the maximum possible number of the pairs for which the two are not acquainted but have a common acquaintance among the participants? 
APMO 2010 problem 4:  Let $ABC$ be an acute angled triangle satisfying the conditions $AB>BC$ and $AC>BC$. Denote by $O$ and $H$ the circumcentre and orthocentre, respectively, of the triangle $ABC.$ Suppose that the circumcircle of the triangle $AHC$ intersects the line $AB$ at $M$ different from $A$, and the circumcircle of the triangle $AHB$ intersects the line $AC$ at $N$ different from $A.$ Prove that the circumcentre of the triangle $MNH$ lies on the line $OH$. 
APMO 2010 problem 5:  Find all functions $f$ from the set $\mathbb{R}$ of real numbers into $\mathbb{R}$ which satisfy for all $x, y, z \in \mathbb{R}$ the identity
\[ f(f(x)+f(y)+f(z))=f(f(x)-f(y))+f(2xy+f(z))+2f(xz-yz). \] 

APMO 2009 

APMO 2009 problem 1:  Consider the following operation on positive real numbers written on a blackboard: \\
Choose a number $ r$ written on the blackboard, erase that number, and then write a pair of positive real numbers $ a$ and $ b$ satisfying the condition $ 2 r^2 = ab$ on the board. \\\\
Assume that you start out with just one positive real number $ r$ on the blackboard, and apply this operation $ k^2 - 1$ times to end up with $ k^2$ positive real numbers, not necessarily distinct. Show that there exists a number on the board which does not exceed kr. 
APMO 2009 problem 2:  Let $ a_1$, $ a_2$, $ a_3$, $ a_4$, $ a_5$ be real numbers satisfying the following equations: \\\\
$ \frac{a_1}{k^2+1}+\frac{a_2}{k^2+2}+\frac{a_3}{k^2+3}+\frac{a_4}{k^2+4}+\frac{a_5}{k^2+5} = \frac{1}{k^2}$ for $ k = 1, 2, 3, 4, 5$ \\\\
Find the value of $ \frac{a_1}{37}+\frac{a_2}{38}+\frac{a_3}{39}+\frac{a_4}{40}+\frac{a_5}{41}$ (Express the value in a single fraction.) 
APMO 2009 problem 3:  Let three circles $ \Gamma_1, \Gamma_2, \Gamma_3$, which are non-overlapping and mutually external, be given in the plane. For each point $ P$ in the plane, outside the three circles, construct six points $ A_1, B_1, A_2, B_2, A_3, B_3$ as follows: For each $ i = 1, 2, 3$, $ A_i, B_i$ are distinct points on the circle $ \Gamma_i$ such that the lines $ PA_i$ and $ PB_i$ are both tangents to $ \Gamma_i$. Call the point $ P$ exceptional if, from the construction, three lines $ A_1B_1, A_2 B_2, A_3 B_3$ are concurrent. Show that every exceptional point of the plane, if exists, lies on the same circle. 
APMO 2009 problem 4:  Prove that for any positive integer $ k$, there exists an arithmetic sequence $ \frac{a_1}{b_1}, \frac{a_2}{b_2}, \frac{a_3}{b_3}, ... ,\frac{a_k}{b_k}$ of rational numbers, where $ a_i, b_i$ are relatively prime positive integers for each $ i = 1,2,...,k$ such that the positive integers $ a_1, b_1, a_2, b_2, ...,  a_k, b_k$ are all distinct. 
APMO 2009 problem 5:  Larry and Rob are two robots travelling in one car from Argovia to Zillis. Both robots have control over the steering and steer according to the following algorithm: Larry makes a 90 degrees left turn after every $ \ell$ kilometer driving from start, Rob makes a 90 degrees right turn after every $ r$ kilometer driving from start, where $ \ell$ and $ r$ are relatively prime positive integers. \\\\
In the event of both turns occurring simultaneously, the car will keep going without changing direction. Assume that the ground is flat and the car can move in any direction. Let the car start from Argovia facing towards Zillis. For which choices of the pair ($ \ell$, $ r$) is the car guaranteed to reach Zillis, regardless of how far it is from Argovia? 

APMO 2008 

APMO 2008 problem 1:  Let $ ABC$ be a triangle with $ \angle A < 60^\circ$. Let $ X$ and $ Y$ be the points on the sides $ AB$ and $ AC$, respectively, such that $ CA + AX = CB + BX$ and $ BA + AY = BC + CY$ . Let $ P$ be the point in the plane such that the lines $ PX$ and $ PY$ are perpendicular to $ AB$ and $ AC$, respectively. Prove that $ \angle BPC < 120^\circ$. 
APMO 2008 problem 2:  Students in a class form groups each of which contains exactly three members such that any two distinct groups have at most one member in common. Prove that, when the class size is $ 46$, there is a set of $ 10$ students in which no group is properly contained. 
APMO 2008 problem 3:  Let $ \Gamma$ be the circumcircle of a triangle $ ABC$. A circle passing through points $ A$ and $ C$ meets the sides $ BC$ and $ BA$ at $ D$ and $ E$, respectively. The lines $ AD$ and $ CE$ meet $ \Gamma$ again at $ G$ and $ H$, respectively. The tangent lines of $ \Gamma$ at $ A$ and $ C$ meet the line $ DE$ at $ L$ and $ M$, respectively. Prove that the lines $ LH$ and $ MG$ meet at $ \Gamma$. 
APMO 2008 problem 4:  Consider the function $ f: \mathbb{N}_0\to\mathbb{N}_0$, where $ \mathbb{N}_0$ is the set of all non-negative \\
integers, defined by the following conditions : \\\\
$ (i)$ $ f(0) = 0$; $ (ii)$ $ f(2n) = 2f(n)$ and $ (iii)$ $ f(2n + 1) = n + 2f(n)$ for all $ n\geq 0$. \\\\
$ (a)$ Determine the three sets $ L = \{ n | f(n) < f(n + 1) \}$, $ E = \{n | f(n) = f(n + 1) \}$, and $ G = \{n | f(n) > f(n + 1) \}$. \\
$ (b)$ For each $ k \geq 0$, find a formula for $ a_k = \max\{f(n) : 0 \leq n \leq 2^k\}$ in terms of $ k$. 
APMO 2008 problem 5:  Let $ a, b, c$ be integers satisfying $ 0 < a < c - 1$ and $ 1 < b < c$. For each $ k$, $ 0\leq k \leq a$, Let $ r_k,0 \leq r_k < c$ \\
be the remainder of $ kb$ when divided by $ c$. Prove that the two sets $ \{r_0, r_1, r_2, \cdots , r_a\}$ and $ \{0, 1, 2, \cdots , a\}$ are different. 

APMO 2007 

APMO 2007 problem 1:  Let $S$ be a set of $9$ distinct integers all of whose prime factors are at most $3.$ Prove that $S$ contains $3$ distinct integers such that their product is a perfect cube. 
APMO 2007 problem 2:  Let $ABC$ be an acute angled triangle with $\angle{BAC}=60^\circ$ and $AB > AC$. Let $I$ be the incenter, and $H$ the orthocenter of the triangle $ABC$ . Prove that $2\angle{AHI}= 3\angle{ABC}$. 
APMO 2007 problem 3:  Consider $n$ disks $C_1; C_2; ... ; C_n$ in a plane such that for each $1 \leq i < n$, the center of $C_i$ is on the circumference of $C_{i+1}$, and the center of $C_n$ is on the circumference of $C_1$. Define the \textit{score} of such an arrangement of $n$ disks to be the number of pairs $(i; j )$ for which $C_i$ properly contains $C_j$ . Determine the maximum possible score. 
APMO 2007 problem 4:  Let $x; y$ and $z$ be positive real numbers such that $\sqrt{x}+\sqrt{y}+\sqrt{z}= 1$. Prove that $\frac{x^2+yz}{\sqrt{2x^2(y+z)}}+\frac{y^2+zx}{\sqrt{2y^2(z+x)}}+\frac{z^2+xy}{\sqrt{2z^2(x+y)}}\geq 1.$ 
APMO 2007 problem 5:  A regular $ (5 \times 5)$-array of lights is defective, so that toggling the switch for one light causes each adjacent light in the same row and in the same column as well as the light itself to change state, from on to off, or from off to on. Initially all the lights are switched off. After a certain number of toggles, exactly one light is switched on. Find all the possible positions of this light. 

APMO 2006 

APMO 2006 problem 1:  Let $n$ be a positive integer. Find the largest nonnegative real number $f(n)$ (depending on $n$) with the following property: whenever $a_1,a_2,...,a_n$ are real numbers such that $a_1+a_2+\cdots +a_n$ is an integer, there exists some $i$ such that  $\left|a_i-\frac{1}{2}\right|\ge f(n)$. 
APMO 2006 problem 2:  Prove that every positive integer can be written as a finite sum of distinct integral powers of the golden ratio. 
APMO 2006 problem 3:  Let $p\ge5$ be a prime and let $r$ be the number of ways of placing $p$ checkers on a $p\times p$ checkerboard so that not all checkers are in the same row (but they may all be in the same column). Show that $r$ is divisible by $p^5$. Here, we assume that all the checkers are identical. 
APMO 2006 problem 4:  Let $A,B$ be two distinct points on a given circle $O$ and let $P$ be the midpoint of the line segment AB. Let $O_1$ be the circle tangent to the line $AB$ at $P$ and tangent to the circle $O$. Let $l$ be the tangent line, different from the line $AB$, to $O_1$ passing through $A$. Let $C$ be the intersection point, different from $A$, of $l$ and $O$. Let $Q$ be the midpoint of the line segment $BC$ and $O_2$ be the circle tangent to the line $BC$ at $Q$ and tangent to the line segment $AC$. Prove that the circle $O_2$ is tangent to the circle $O$. 
APMO 2006 problem 5:  In a circus, there are $n$ clowns who dress and paint themselves up using a selection of 12 distinct colours. Each clown is required to use at least five different colours. One day, the ringmaster of the circus orders that no two clowns have exactly the same set of colours and no more than 20 clowns may use any one particular colour. Find the largest number $n$ of clowns so as to make the ringmaster's order possible. 

APMO 2005 

APMO 2005 problem 1:  Prove that for every irrational real number $a$, there are irrational real numbers $b$ and $b'$ so that $a+b$ and $ab'$ are both rational while $ab$ and $a+b'$ are both irrational. 
APMO 2005 problem 2:  Let $a, b, c$ be positive real numbers such that $abc=8$. Prove that
\[
\frac{a^2}{\sqrt{(1+a^3)(1+b^3)}} +\frac{b^2}{\sqrt{(1+b^3)(1+c^3)}} +\frac{c^2}{\sqrt{(1+c^3)(1+a^3)}} \geq \frac{4}{3}
\] 
APMO 2005 problem 3:  Prove that there exists a triangle which can be cut into 2005 congruent triangles. 
APMO 2005 problem 4:  In a small town, there are $n \times n$ houses indexed by $(i, j)$ for $1 \leq i, j \leq n$ with $(1, 1)$ being the house at the top left corner, where $i$ and $j$ are the row and column indices, respectively. At time 0, a fire breaks out at the house indexed by $(1, c)$, where $c \leq \frac{n}{2}$. During each subsequent time interval $[t, t+1]$, the fire fighters defend a house which is not yet on fire while the fire spreads to all undefended \textit{neighbors} of each house which was on fire at time t. Once a house is defended, it remains so all the time. The process ends when the fire can no longer spread. At most how many houses can be saved by the fire fighters? \\
A house indexed by $(i, j)$ is a \textit{neighbor} of a house indexed by $(k, l)$ if $|i - k| + |j - l|=1$. 
APMO 2005 problem 5:  In a triangle $ABC$, points $M$ and $N$ are on sides $AB$ and $AC$, respectively, such that $MB = BC = CN$. Let $R$ and $r$ denote the circumradius and the inradius of the triangle $ABC$, respectively. Express the ratio $MN/BC$ in terms of $R$ and $r$. 

APMO 2004 

APMO 2004 problem 1:  Determine all finite nonempty sets $S$ of positive integers satisfying
\[ {i+j\over (i,j)}\qquad\mbox{is an element of S for all i,j in S}, \]
where $(i,j)$ is the greatest common divisor of $i$ and $j$. 
APMO 2004 problem 2:  Let $O$ be the circumcenter and $H$ the orthocenter of an acute triangle $ABC$. Prove that the area of one of the triangles $AOH$, $BOH$ and $COH$ is equal to the sum of the areas of the other two. 
APMO 2004 problem 3:  Let a set $S$ of 2004 points in the plane be given, no three of which are collinear. Let ${\cal L}$ denote the set of all lines (extended indefinitely in both directions) determined by pairs of points from the set. Show that it is possible to colour the points of $S$ with at most two colours, such that for any points $p,q$ of $S$, the number of lines in ${\cal L}$ which separate $p$ from $q$ is odd if and only if $p$ and $q$ have the same colour. \\\\
Note: A line $\ell$ separates two points $p$ and $q$ if $p$ and $q$ lie on opposite sides of $\ell$ with neither point on $\ell$. 
APMO 2004 problem 4:  For a real number $x$, let $\lfloor x\rfloor$ stand for the largest integer that is less than or equal to $x$. Prove that
\[ \left\lfloor{(n-1)!\over n(n+1)}\right\rfloor \]
is even for every positive integer $n$. 
APMO 2004 problem 5:  Prove that the inequality
\[ \left(a^2+2\right)\left(b^2+2\right)\left(c^2+2\right) \geq 9\left(ab+bc+ca\right) \]
holds for all positive reals $a$, $b$, $c$. 

APMO 2003 

APMO 2003 problem 1:  Let $a,b,c,d,e,f$ be real numbers such that the polynomial
\[ p(x)=x^8-4x^7+7x^6+ax^5+bx^4+cx^3+dx^2+ex+f \]
factorises into eight linear factors $x-x_i$, with $x_i>0$ for $i=1,2,\ldots,8$. Determine all possible values of $f$. 
APMO 2003 problem 2:  Suppose $ABCD$ is a square piece of cardboard with side length $a$. On a plane are two parallel lines $\ell_1$ and $\ell_2$, which are also $a$ units apart. The square $ABCD$ is placed on the plane so that sides $AB$ and $AD$ intersect $\ell_1$ at $E$ and $F$ respectively. Also, sides $CB$ and $CD$ intersect $\ell_2$ at $G$ and $H$ respectively. Let the perimeters of $\triangle AEF$ and $\triangle CGH$ be $m_1$ and $m_2$ respectively. \\\\
Prove that no matter how the square was placed, $m_1+m_2$ remains constant. 
APMO 2003 problem 3:  Let $k\ge 14$ be an integer, and let $p_k$ be the largest prime number which is strictly less than $k$. You may assume that $p_k\ge 3k/4$. Let $n$ be a composite integer. Prove:
\begin{enumerate}[(a)]
  \item if $n=2p_k$, then $n$ does not divide $(n-k)!$;
  \item if $n>2p_k$, then $n$ divides $(n-k)!$.
\end{enumerate} 
APMO 2003 problem 4:  Let $a,b,c$ be the sides of a triangle, with $a+b+c=1$, and let $n\ge 2$ be an integer. Show that
\[ \sqrt[n]{a^n+b^n}+\sqrt[n]{b^n+c^n}+\sqrt[n]{c^n+a^n}<1+\frac{\sqrt[n]{2}}{2}. \] 
APMO 2003 problem 5:  Given two positive integers $m$ and $n$, find the smallest positive integer $k$ such that among any $k$ people, either there are $2m$ of them who form $m$ pairs of mutually acquainted people or there are $2n$ of them forming $n$ pairs of mutually unacquainted people. 

APMO 2002 

APMO 2002 problem 1:  Let $a_1,a_2,a_3,\ldots,a_n$ be a sequence of non-negative integers, where $n$ is a positive integer. Let
\[ A_n={a_1+a_2+\cdots+a_n\over n}\ . \]
Prove that
\[ a_1!a_2!\ldots a_n!\ge\left(\lfloor A_n\rfloor !\right)^n \]
where $\lfloor A_n\rfloor$ is the greatest integer less than or equal to $A_n$, and $a!=1\times 2\times\cdots\times a$ for $a\ge 1$(and $0!=1$). When does equality hold? 
APMO 2002 problem 2:  Find all positive integers $a$ and $b$ such that
\[ {a^2+b\over b^2-a}\quad\mbox{and}\quad{b^2+a\over a^2-b} \]
are both integers. 
APMO 2002 problem 3:  Let $ABC$ be an equilateral triangle. Let $P$ be a point on the side $AC$ and $Q$ be a point on the side $AB$ so that both triangles $ABP$ and $ACQ$ are acute. Let $R$ be the orthocentre of triangle $ABP$ and $S$ be the orthocentre of triangle $ACQ$. Let $T$ be the point common to the segments $BP$ and $CQ$. Find all possible values of $\angle CBP$ and $\angle BCQ$ such that the triangle $TRS$ is equilateral. 
APMO 2002 problem 4:  Let $x,y,z$ be positive numbers such that
\[ {1\over x}+{1\over y}+{1\over z}=1. \]
Show that
\[ \sqrt{x+yz}+\sqrt{y+zx}+\sqrt{z+xy}\ge\sqrt{xyz}+\sqrt{x}+\sqrt{y}+\sqrt{z} \] 
APMO 2002 problem 5:  Let ${\bf R}$ denote the set of all real numbers. Find all functions $f$ from ${\bf R}$ to ${\bf R}$ satisfying:
\begin{enumerate}[(i)]
  \item there are  only finitely many $s$ in ${\bf R}$ such that $f(s)=0$,
\end{enumerate}
and \\\\
(ii) $f(x^4+y)=x^3f(x)+f(f(y))$ for all $x,y$ in ${\bf R}$. 

APMO 2001 

APMO 2001 problem 1:  For a positive integer $n$ let $S(n)$ be the sum of digits in the decimal representation of $n$.   Any positive integer obtained by removing several (at least one) digits from the right-hand end of the decimal representation of $n$ is called a \textit{stump} of $n$. Let $T(n)$ be the sum of all stumps of $n$. Prove that $n=S(n)+9T(n)$. 
APMO 2001 problem 2:  Find the largest positive integer $N$ so that the number of integers in the set $\{1,2,\dots,N\}$ which are divisible by 3 is equal to the number of integers which are divisible by 5 or 7 (or both). 
APMO 2001 problem 3:  Two equal-sized regular $n$-gons intersect to form a $2n$-gon $C$. Prove that the sum of the sides of $C$ which form part of one $n$-gon equals half the perimeter of $C$. \\\\
\textit{Alternative formulation:} \\\\
Let two equal regular $n$-gons $S$ and $T$ be located in the plane such that their intersection $S\cap T$ is a $2n$-gon (with $n\ge 3$). The sides of the polygon $S$ are coloured in red and the sides of $T$ in blue. \\\\
Prove that the sum of the lengths of the blue sides of the polygon $S\cap T$ is equal to the sum of the lengths of its red sides. 
APMO 2001 problem 4:  A point in the plane with a cartesian coordinate system is called a \textit{mixed  point} if one of its coordinates is rational and the other one is irrational. Find all polynomials with real coefficients such that their graphs do not contain any mixed point. 
APMO 2001 problem 5:  Find the greatest integer $n$, such that there are $n+4$ points $A$, $B$, $C$, $D$, $X_1,\dots,~X_n$ in the plane with $AB\ne CD$ that satisfy the following condition: for each $i=1,2,\dots,n$ triangles $ABX_i$ and $CDX_i$ are equal. 

APMO 2000 

APMO 2000 problem 1:  Compute the sum: $\sum_{i=0}^{101} \frac{x_i^3}{1-3x_i+3x_i^2}$ for $x_i=\frac{i}{101}$. 
APMO 2000 problem 2:  Find all permutations $a_1, a_2, \ldots, a_9$ of $1, 2, \ldots, 9$ such that
\[ a_1+a_2+a_3+a_4=a_4+a_5+a_6+a_7= a_7+a_8+a_9+a_1 \]
and
\[ a_1^2+a_2^2+a_3^2+a_4^2=a_4^2+a_5^2+a_6^2+a_7^2= a_7^2+a_8^2+a_9^2+a_1^2 \] 
APMO 2000 problem 3:  Let $ABC$ be a triangle. Let $M$ and $N$ be the points in which the median and the angle bisector, respectively, at $A$ meet the side $BC$. Let $Q$ and $P$ be the points in which the perpendicular at $N$ to $NA$ meets $MA$ and $BA$, respectively. And $O$ the point in which the perpendicular at $P$ to $BA$ meets $AN$ produced. \\\\
Prove that $QO$ is perpendicular to $BC$. 
APMO 2000 problem 4:  Let $n,k$ be given positive integers with $n>k$. Prove that:
\[
\frac{1}{n+1} \cdot \frac{n^n}{k^k (n-k)^{n-k}} < \frac{n!}{k! (n-k)!} < \frac{n^n}{k^k(n-k)^{n-k}}
\] 
APMO 2000 problem 5:  Given a permutation ($a_0, a_1, \ldots, a_n$) of the sequence $0, 1,\ldots, n$. A transportation of $a_i$ with $a_j$ is called legal if $a_i=0$ for $i>0$, and $a_{i-1}+1=a_j$. The permutation ($a_0, a_1, \ldots, a_n$) is called regular if after a number of legal transportations it becomes ($1,2, \ldots, n,0$). \\
For which numbers $n$ is the permutation ($1, n, n-1, \ldots, 3, 2, 0$) regular? 

APMO 1999 

APMO 1999 problem 1:  Find the smallest positive integer $n$ with the following property: there does not exist an arithmetic progression of $1999$ real numbers containing exactly $n$ integers. 
APMO 1999 problem 2:  Let $a_1, a_2, \dots$ be a sequence of real numbers satisfying $a_{i+j} \leq a_i+a_j$ for all $i,j=1,2,\dots$. Prove that
\[ a_1 + \frac{a_2}{2} + \frac{a_3}{3} + \cdots + \frac{a_n}{n} \geq a_n \]
for each positive integer $n$. 
APMO 1999 problem 3:  Let $\Gamma_1$ and $\Gamma_2$ be two circles intersecting at $P$ and $Q$. The common tangent, closer to $P$, of $\Gamma_1$ and $\Gamma_2$ touches $\Gamma_1$ at $A$ and $\Gamma_2$ at $B$. The tangent of $\Gamma_1$ at $P$ meets $\Gamma_2$ at $C$, which is different from $P$, and the extension of $AP$ meets $BC$ at $R$. \\
Prove that the circumcircle of triangle $PQR$ is tangent to $BP$ and $BR$. 
APMO 1999 problem 4:  Determine all pairs $(a,b)$ of integers with the property that the numbers $a^2+4b$ and $b^2+4a$ are both perfect squares. 
APMO 1999 problem 5:  Let $S$ be a set of $2n+1$ points in the plane such that no three are collinear and no four concyclic. A circle will be called $\text{Good}$ if it has 3 points of $S$ on its circumference, $n-1$ points in its interior and $n-1$ points in its exterior. \\
Prove that the number of good circles has the same parity as $n$. 

APMO 1998 

APMO 1998 problem 1:  Let $F$ be the set of all $n$-tuples $(A_1, \ldots, A_n)$ such that each $A_i$ is a subset of $\{1, 2, \ldots, 1998\}$.  Let $|A|$ denote the number of elements of the set $A$.  Find
\[ \sum_{(A_1, \ldots, A_n)\in F} |A_1\cup A_2\cup \cdots \cup A_n| \] 
APMO 1998 problem 2:  Show that for any positive integers $a$ and $b$, $(36a+b)(a+36b)$ cannot be a power of $2$. 
APMO 1998 problem 3:  Let $a$, $b$, $c$ be positive real numbers.  Prove that
\[
\biggl(1+\frac{a}{b}\biggr) \biggl(1+\frac{b}{c}\biggr) \biggl(1+\frac{c}{a}\biggr) \ge 2 \biggl(1+\frac{a+b+c}{\sqrt[3]{abc}}\biggr).
\] 
APMO 1998 problem 4:  Let $ABC$ be a triangle and $D$ the foot of the altitude from $A$. Let $E$ and $F$ lie on a line passing through $D$ such that $AE$ is perpendicular to $BE$, $AF$ is perpendicular to $CF$, and $E$ and $F$ are different from $D$. Let $M$ and $N$ be the midpoints of the segments $BC$ and $EF$, respectively. Prove that $AN$ is perpendicular to $NM$. 
APMO 1998 problem 5:  Find the largest integer $n$ such that $n$ is divisible by all positive integers less than $\sqrt[3]{n}$. 

APMO 1997 

APMO 1997 problem 1:  Given:
\[
S = 1 + \frac{1}{1 + \frac{1}{3}} + \frac{1}{1 + \frac{1}{3} + \frac{1} {6}} + \cdots + \frac{1}{1 + \frac{1}{3} + \frac{1}{6} + \cdots + \frac{1} {1993006}}
\]
where the denominators contain partial sums of the sequence of reciprocals of triangular numbers (i.e. $k=\frac{n(n+1)}{2}$ for $n = 1$, $2$, $\ldots$,$1996$).  Prove that $S>1001$. 
APMO 1997 problem 2:  Find an integer $n$, where $100 \leq n \leq 1997$, such that
\[ \frac{2^n+2}{n} \]
is also an integer. 
APMO 1997 problem 3:  Let $ABC$ be a triangle inscribed in a circle and let
\[ l_a = \frac{m_a}{M_a} \ , \ \ l_b = \frac{m_b}{M_b} \ , \ \ l_c = \frac{m_c}{M_c} \ , \]
where $m_a$,$m_b$, $m_c$ are the lengths of the angle bisectors (internal to the triangle) and $M_a$, $M_b$, $M_c$ are the lengths of the angle bisectors extended until they meet the circle.  Prove that
\[ \frac{l_a}{\sin^2 A} + \frac{l_b}{\sin^2 B} + \frac{l_c}{\sin^2 C} \geq 3 \]
and that equality holds iff $ABC$ is an equilateral triangle. 
APMO 1997 problem 4:  Triangle $A_1 A_2 A_3$ has a right angle at $A_3$. A sequence of points is now defined by the following iterative process, where $n$ is a positive integer. From $A_n$ ($n \geq 3$), a perpendicular line is drawn to meet $A_{n-2}A_{n-1}$ at $A_{n+1}$.
\begin{enumerate}[(a)]
  \item Prove that if this process is continued indefinitely, then one and only one point $P$ is interior to every triangle $A_{n-2} A_{n-1} A_n$, $n \geq 3$.
  \item Let $A_1$ and $A_3$ be fixed points. By considering all possible locations of $A_2$ on the plane, find the locus of $P$.
\end{enumerate} 
APMO 1997 problem 5:  Suppose that $n$ people $A_1$, $A_2$, $\ldots$, $A_n$, ($n \geq 3$) are seated in a circle and that $A_i$ has $a_i$ objects such that
\[ a_1 + a_2 + \cdots + a_n = nN \]
where $N$ is a positive integer. In order that each person has the same number of objects, each person $A_i$ is to give or to receive a certain number of objects to or from its two neighbours $A_{i-1}$ and $A_{i+1}$. (Here $A_{n+1}$ means $A_1$ and $A_n$ means $A_0$.)  How should this redistribution be performed so that the total number of objects transferred is minimum? 

APMO 1996 

APMO 1996 problem 1:  Let $ABCD$ be a quadrilateral $AB = BC = CD = DA$.  Let $MN$ and $PQ$ be two segments perpendicular to the diagonal $BD$ and such that the distance between them is $d > \frac{BD}{2}$, with $M \in AD$, $N \in DC$, $P \in AB$, and $Q \in BC$. Show that the perimeter of hexagon $AMNCQP$ does not depend on the position of $MN$ and $PQ$ so long as the distance between them remains constant. 
APMO 1996 problem 2:  Let $m$ and $n$ be positive integers such that $n \leq m$.  Prove that
\[ 2^n n! \leq \frac{(m+n)!}{(m-n)!} \leq (m^2 + m)^n \] 
APMO 1996 problem 3:  If $ABCD$ is a cyclic quadrilateral, then prove that the incenters of the triangles $ABC$, $BCD$, $CDA$, $DAB$ are the vertices of a rectangle. 
APMO 1996 problem 4:  The National Marriage Council wishes to invite $n$ couples to form 17 discussion groups under the following conditions:
\begin{enumerate}[(1)]
  \item All members of a group must be of the same sex; i.e. they are either all male or all female.
  \item The difference in the size of any two groups is 0 or 1.
  \item All groups have at least 1 member.
  \item Each person must belong to one and only one group.
\end{enumerate}
Find all values of $n$, $n \leq 1996$, for which this is possible.  Justify your answer. 
APMO 1996 problem 5:  Let $a$, $b$, $c$ be the lengths of the sides of a triangle.  Prove that
\[ \sqrt{a+b-c} + \sqrt{b+c-a} + \sqrt{c+a-b} \leq \sqrt{a} + \sqrt{b} + \sqrt{c} \]
and determine when equality occurs. 

APMO 1995 

APMO 1995 problem 1:  Determine all sequences of real numbers $a_1$, $a_2$, $\ldots$, $a_{1995}$ which satisfy:
\[ 2\sqrt{a_n - (n - 1)} \geq a_{n+1} - (n - 1), \ \mbox{for} \ n = 1, 2, \ldots 1994, \]
and
\[ 2\sqrt{a_{1995} - 1994} \geq a_1 + 1. \] 
APMO 1995 problem 2:  Let $a_1$, $a_2$, $\ldots$, $a_n$ be a sequence of integers with values between 2 and 1995 such that:
\begin{enumerate}[(i)]
  \item Any two of the $a_i$'s are relatively prime,
  \item Each $a_i$ is either a prime or a product of primes.
\end{enumerate}
Determine the smallest possible values of $n$ to make sure that the sequence will contain a prime number. 
APMO 1995 problem 3:  Let $PQRS$ be a cyclic quadrilateral such that the segments $PQ$ and $RS$ are not parallel.  Consider the set of circles through $P$ and $Q$, and the set of circles through $R$ and $S$.  Determine the set $A$ of points of tangency of circles in these two sets. 
APMO 1995 problem 4:  Let $C$ be a circle with radius $R$ and centre $O$, and $S$ a fixed point in the interior of $C$.  Let $AA'$ and $BB'$ be perpendicular chords through $S$. Consider the rectangles $SAMB$, $SBN'A'$, $SA'M'B'$, and $SB'NA$.  Find the set of all points $M$, $N'$, $M'$, and $N$ when $A$ moves around the whole circle. 
APMO 1995 problem 5:  Find the minimum positive integer $k$ such that there exists a function $f$ from the set $\Bbb{Z}$ of all integers to $\{1, 2, \ldots k\}$ with the property that $f(x) \neq f(y)$ whenever $|x-y| \in \{5, 7, 12\}$. 

APMO 1994 

APMO 1994 problem 1:  Let $f: \Bbb{R} \rightarrow \Bbb{R}$ be a function such that
\begin{enumerate}[(i)]
  \item For all $x,y \in \Bbb{R}$,
\end{enumerate}
\[ f(x)+f(y)+1 \geq f(x+y) \geq f(x)+f(y) \]
(ii) For all $x \in [0,1)$, $f(0) \geq f(x)$, \\
(iii) $-f(-1) = f(1) = 1$. \\\\
Find all such functions $f$. 
APMO 1994 problem 2:  Given a nondegenerate triangle $ABC$, with circumcentre $O$, orthocentre $H$, and circumradius $R$, prove that $|OH| < 3R$. 
APMO 1994 problem 3:  Let $n$ be an integer of the form $a^2 + b^2$, where $a$ and $b$ are relatively prime integers and such that if $p$ is a prime, $p \leq \sqrt{n}$, then $p$ divides $ab$.  Determine all such $n$. 
APMO 1994 problem 4:  Is there an infinite set of points in the plane such that no three points are collinear, and the distance between any two points is rational? 
APMO 1994 problem 5:  You are given three lists $A$, $B$, and $C$.  List $A$ contains the numbers of the form $10^k$ in base $10$, with $k$ any integer greater than or equal to $1$.  Lists $B$ and $C$ contain the same numbers translated into base $2$ and $5$ respectively:
\[
\begin{array}{lll} A \& B \& C \\ 10 \& 1010 \& 20 \\ 100 \& 1100100 \& 400 \\ 1000 \& 1111101000 \& 13000 \\ \vdots \& \vdots \& \vdots \end{array}
\]
Prove that for every integer $n > 1$, there is exactly one number in exactly one of the lists $B$ or $C$ that has exactly $n$ digits. 

APMO 1993 

APMO 1993 problem 1:  Let $ABCD$ be a quadrilateral such that all sides have equal length and $\angle{ABC} =60^o$.  Let $l$ be a line passing through $D$ and not intersecting the quadrilateral (except at $D$).  Let $E$ and $F$ be the points of intersection of $l$ with $AB$ and $BC$ respectively.  Let $M$ be the point of intersection of $CE$ and $AF$. \\\\
Prove that $CA^2 = CM \times CE$. 
APMO 1993 problem 2:  Find the total number of different integer values the function
\[ f(x) = [x] + [2x] + [\frac{5x}{3}] + [3x] + [4x] \]
takes for real numbers $x$ with $0 \leq x \leq 100$. 
APMO 1993 problem 3:  Let
\begin{eqnarray*} f(x) \& = \& a_n x^n + a_{n-1} x^{n-1} + \cdots + a_0 \ \ \mbox{and} \\ g(x) \& = \& c_{n+1} x^{n+1} + c_n x^n + \cdots + c_0 \end{eqnarray*}
be non-zero polynomials with real coefficients such that $g(x) = (x+r)f(x)$ for some real number $r$.  If $a = \max(|a_n|, \ldots, |a_0|)$ and $c = \max(|c_{n+1}|, \ldots, |c_0|)$, prove that $\frac{a}{c} \leq n+1$. 
APMO 1993 problem 4:  Determine all positive integers $n$ for which the equation
\[ x^n + (2+x)^n + (2-x)^n = 0 \]
has an integer as a solution. 
APMO 1993 problem 5:  Let $P_1$, $P_2$, $\ldots$, $P_{1993} = P_0$ be distinct points in the $xy$-plane \\
with the following properties:
\begin{enumerate}[(i)]
  \item both coordinates of $P_i$ are integers, for $i = 1, 2, \ldots, 1993$;
  \item there is no point other than $P_i$ and $P_{i+1}$ on the line segment joining $P_i$ with $P_{i+1}$ whose coordinates are both integers, for $i = 0, 1, \ldots, 1992$.
\end{enumerate}
Prove that for some $i$, $0 \leq i \leq 1992$, there exists a point $Q$ with coordinates $(q_x, q_y)$ on the line segment joining $P_i$ with $P_{i+1}$ such that both $2q_x$ and $2q_y$ are odd integers. 

APMO 1992 

APMO 1992 problem 1:  A triangle with sides $a$, $b$, and $c$ is given.  Denote by $s$ the semiperimeter, that is $s = \frac{a + b + c}{2}$.  Construct a triangle with sides $s - a$, $s - b$, and $s - c$.  This process is repeated until a triangle can no longer be constructed with the side lengths given. \\\\
For which original triangles can this process be repeated indefinitely? 
APMO 1992 problem 2:  In a circle $C$ with centre $O$ and radius $r$, let $C_1$, $C_2$ be two circles with centres $O_1$, $O_2$ and radii $r_1$, $r_2$ respectively, so that each circle $C_i$ is internally tangent to $C$ at $A_i$ and so that $C_1$, $C_2$ are externally tangent to each other at $A$. \\\\
Prove that the three lines $OA$, $O_1 A_2$, and $O_2 A_1$ are concurrent. 
APMO 1992 problem 3:  Let $n$ be an integer such that $n > 3$.  Suppose that we choose three numbers from the set $\{1, 2, \ldots, n\}$.  Using each of these three numbers only once and using addition, multiplication, and parenthesis, let us form all \\
possible combinations.
\begin{enumerate}[(a)]
  \item Show that if we choose all three numbers greater than $\frac{n}{2}$, then the values of these combinations are all distinct.
  \item Let $p$ be a prime number such that $p \leq \sqrt{n}$.  Show that the number of ways of choosing three numbers so that the smallest one is $p$ and the values of the combinations are not all distinct is precisely the number of positive divisors of $p - 1$.
\end{enumerate} 
APMO 1992 problem 4:  Determine all pairs $(h,s)$ of positive integers with the following property: \\\\
If one draws $h$ horizontal lines and another $s$ lines which satisfy
\begin{enumerate}[(i)]
  \item they are not horizontal,
  \item no two of them are parallel,
  \item no three of the $h + s$ lines are concurrent,
\end{enumerate}
then the number of regions formed by these $h + s$ lines is 1992. 
APMO 1992 problem 5:  Find a sequence of maximal length consisting of non-zero integers in which the sum of any seven consecutive terms is positive and that of any eleven consecutive terms is negative. 

APMO 1991 

APMO 1991 problem 1:  Let $G$ be the centroid of a triangle $ABC$, and $M$ be the midpoint of $BC$.  Let $X$ be on $AB$ and $Y$ on $AC$ such that the points $X$, $Y$, and $G$ are collinear and $XY$ and $BC$ are parallel.  Suppose that $XC$ and $GB$ intersect at $Q$ and $YB$ and $GC$ intersect at $P$.  Show that triangle $MPQ$ is similar to triangle $ABC$. 
APMO 1991 problem 2:  Suppose there are $997$ points given in a plane.  If every two points are joined by a line segment with its midpoint coloured in red, show that there are at least $1991$ red points in the plane.  Can you find a special case with exactly $1991$ red points? 
APMO 1991 problem 3:  Let $a_1$, $a_2$, $\cdots$, $a_n$, $b_1$, $b_2$, $\cdots$, $b_n$ be positive real numbers such that $a_1 + a_2 + \cdots + a_n = b_1 + b_2 + \cdots + b_n$.  Show that
\[
\frac{a_1^2}{a_1 + b_1} + \frac{a_2^2}{a_2 + b_2} + \cdots + \frac{a_n^2}{a_n + b_n} \geq \frac{a_1 + a_2 + \cdots + a_n}{2}
\] 
APMO 1991 problem 4:  During a break, $n$ children at school sit in a circle around their teacher to play a game.  The teacher walks clockwise close to the children and hands out candies to some of them according to the following rule: \\\\
He selects one child and gives him a candy, then he skips the next child and gives a candy to the next one, then he skips 2 and gives a candy to the next one, then he skips 3, and so on. \\\\
Determine the values of $n$ for which eventually, perhaps after many rounds, all children will have at least one candy each. 
APMO 1991 problem 5:  Given are two tangent circles and a point $P$ on their common tangent perpendicular to the lines joining their centres.  Construct with ruler and compass all the circles that are tangent to these two circles and pass through the point $P$. 

APMO 1990 

APMO 1990 problem 1:  Given triangle $ABC$, let $D$, $E$, $F$ be the midpoints of $BC$, $AC$, $AB$ respectively and let $G$ be the centroid of the triangle. For each value of $\angle BAC$, how many non-similar triangles are there in which $AEGF$ is a cyclic quadrilateral? 
APMO 1990 problem 2:  Let $a_1$, $a_2$, $\cdots$, $a_n$ be positive real numbers, and let $S_k$ be the sum of the products of $a_1$, $a_2$, $\cdots$, $a_n$ taken $k$ at a time.  Show that
\[ S_k S_{n-k} \geq {n \choose k}^2 a_1 a_2 \cdots a_n \]
for $k = 1$, $2$, $\cdots$, $n - 1$. 
APMO 1990 problem 3:  Consider all the triangles $ABC$ which have a fixed base $AB$ and whose altitude from $C$ is a constant $h$.  For which of these triangles is the product of its altitudes a maximum? 
APMO 1990 problem 4:  A set of 1990 persons is divided into non-intersecting subsets in such a way that
\begin{enumerate}
  \item No one in a subset knows all the others in the subset,
  \item Among any three persons in a subset, there are always at least two who do not know each other, and
  \item For any two persons in a subset who do not know each other, there is exactly one person in the same subset knowing both of them.
  \item Prove that within each subset, every person has the same number of acquaintances.
  \item Determine the maximum possible number of subsets.
\end{enumerate}
Note: It is understood that if a person $A$ knows person $B$, then person $B$ will know person $A$; an acquaintance is someone who is known.  Every person is assumed to know one's self. 
APMO 1990 problem 5:  Show that for every integer $n \geq 6$, there exists a convex hexagon which can be dissected into exactly $n$ congruent triangles. 

APMO 1989 

APMO 1989 problem 1:  Let $x_1$, $x_2$, $\cdots$, $x_n$ be positive real numbers, and let
\[ S = x_1 + x_2 + \cdots + x_n. \]
Prove that
\[
(1 + x_1)(1 + x_2) \cdots (1 + x_n) \leq 1 + S + \frac{S^2}{2!} + \frac{S^3}{3!} + \cdots + \frac{S^n}{n!}
\] 
APMO 1989 problem 2:  Prove that the equation
\[ 6(6a^2 + 3b^2 + c^2) = 5n^2 \]
has no solutions in integers except $a = b = c = n = 0$. 
APMO 1989 problem 3:  Let $A_1$, $A_2$, $A_3$ be three points in the plane, and for convenience, let $A_4= A_1$, $A_5 = A_2$.  For $n = 1$, $2$, and $3$, suppose that $B_n$ is the midpoint of $A_n A_{n+1}$, and suppose that $C_n$ is the midpoint of $A_n B_n$. Suppose that $A_n C_{n+1}$ and $B_n A_{n+2}$ meet at $D_n$, and that $A_n B_{n+1}$ and $C_n A_{n+2}$ meet at $E_n$. \\
Calculate the ratio of the area of triangle $D_1 D_2 D_3$ to the area of triangle $E_1 E_2 E_3$. 
APMO 1989 problem 4:  Let $S$ be a set consisting of $m$ pairs $(a,b)$ of positive integers with the property that $1 \leq a < b \leq n$.  Show that there are at least
\[ 4m \cdot \dfrac{(m - \dfrac{n^2}{4})}{3n} \]
triples $(a,b,c)$ such that $(a,b)$, $(a,c)$, and $(b,c)$ belong to $S$. 
APMO 1989 problem 5:  Determine all functions $f$ from the reals to the reals for which
\begin{enumerate}[(1)]
  \item $f(x)$ is strictly increasing and (2) $f(x) + g(x) = 2x$ for all real $x$,
\end{enumerate}
where $g(x)$ is the composition inverse function to $f(x)$.  (Note: $f$ and $g$ are said to be composition inverses if $f(g(x)) = x$ and $g(f(x)) = x$ for all real $x$.) 
